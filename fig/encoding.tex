\lstset{language=[OpenCL]C}
\begin{figure}
	\centering %
	\subfloat[An input OpenCL kernel, taken from Nvidia's \emph{streamcluster}.]{%
		\noindent\mbox{\parbox{\columnwidth}{%
				\lstinputlisting[label=subfig:source_in]{lst/source_in}%
		}}%
	}\\%
	\subfloat[The input kernel after source rewriting. Variable and function names are normalized, comments removed, and code style enforced.]{%
		\noindent\mbox{\parbox{\columnwidth}{%
				\lstinputlisting[label=subfig:source_out]{lst/source_out}%
		}}%
	}\\%
	\subfloat[Derived vocabulary, ordered by their appearance in the input~\protect\subref{subfig:source_out}. The vocabulary maps tokens to indices.]{%
		\footnotesize
		\begin{tabular}{l l | l l | l l}
			\toprule
			\textbf{idx} & \textbf{token} & \textbf{idx} & \textbf{token} & \textbf{idx} & \textbf{token} \\
			\midrule
			\texttt{00} & \texttt{`\_\_kernel'} & \texttt{09} & \texttt{`,'} & \texttt{19} & \texttt{`int'}\\
			\texttt{01} & \texttt{` '} & \texttt{10} & \texttt{`short'} & \texttt{20} & \texttt{`d'}\\
			\texttt{02} & \texttt{`void'} & \texttt{11} & \texttt{`b'} & \texttt{21} & \texttt{`='}\\
			\texttt{03} & \texttt{`A'} & \texttt{13} & \texttt{`c'} & \texttt{22} & \texttt{`get\_global\_id'}\\
			\texttt{04} & \texttt{`('} & \texttt{14} & \texttt{`)'} & \texttt{23} & \texttt{`0'}\\
			\texttt{05} & \texttt{`\_\_global'} & \texttt{15} & \texttt{`\{'} & \texttt{24} & \texttt{`;'}\\
			\texttt{06} & \texttt{`char'} & \texttt{16} & \texttt{`\textbackslash n'} & \texttt{25} & \texttt{`]'}\\
			\texttt{07} & \texttt{`*'} & \texttt{17} & \texttt{`  '} & \texttt{26} & \texttt{`]'}\\
			\texttt{08} & \texttt{`a'} & \texttt{18} & \texttt{`const'} & \texttt{27} & \texttt{`\}'}\\
		\end{tabular}%
		\label{subfig:source_vocab}%
	}\\%
	\subfloat[Indices encoded kernel sequence. Sequences may be padded to a maximum length.]{%
		\rowcolors{2}{white}{gray!25}
		\footnotesize
		\begin{tabular}{l l l l l l l l l l l}
			\toprule
			\texttt{00} & \texttt{01} & \texttt{02} & \texttt{01} & \texttt{03} & \texttt{04} & \texttt{05} & \texttt{01} & \texttt{06} & \texttt{07} & \texttt{01} \\
			\texttt{08} & \texttt{09} & \texttt{01} & \texttt{10} & \texttt{01} & \texttt{11} & \texttt{09} & \texttt{01} & \texttt{19} & \texttt{01} & \texttt{13} \\
			\texttt{14} & \texttt{01} & \texttt{15} & \texttt{16} & \texttt{17} & \texttt{18} & \texttt{01} & \texttt{19} & \texttt{01} & \texttt{20} & \texttt{01} \\
			\texttt{21} & \texttt{01} & \texttt{22} & \texttt{04} & \texttt{23} & \texttt{14} & \texttt{24} & \texttt{16} & \texttt{17} & \texttt{08} & \texttt{25} \\
			\texttt{20} & \texttt{26} & \texttt{01} & \texttt{21} & \texttt{01} & \texttt{11} & \texttt{24} & \texttt{16} & \texttt{27} & \multicolumn{2}{l}{\texttt{<pad\ldots}>} \\
			\bottomrule
		\end{tabular}%
		\label{subfig:source_enc}%
	}%
	\caption{Deriving a tokenized $1$-of-$k$ vocabulary encoding from an OpenCL source code.}%
	\label{fig:encoding}%
\end{figure}
