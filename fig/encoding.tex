\lstset{language=[OpenCL]C}
\begin{figure}
  \centering %
  \subfloat[%
    An example, short OpenCL kernel, taken from Nvidia's \emph{streamcluster}.%
  ]{%
    \noindent\mbox{\parbox{\columnwidth}{%
      \lstinputlisting[nolol,label=subfig:source_in]{lst/source_in}%
    }}%
  }\\%
  \subfloat[%
    The \emph{streamcluster} kernel after source rewriting. Variable and
    function names are normalised, comments removed, and code style enforced.%
  ]{%
    \noindent\mbox{\parbox{\columnwidth}{%
      \lstinputlisting[nolol,label=subfig:source_out]{lst/source_out}%
    }}%
  }\\%
  \subfloat[%
    Derived vocabulary, ordered by their appearance in the
    input~\protect\subref{subfig:source_out}. The vocabulary maps tokens to
    integer indices.%
  ]{%
    \footnotesize
\begin{tabular}{l l | l l | l l}
  \toprule
  \textbf{idx} & \textbf{token} & \textbf{idx} & \textbf{token} & \textbf{idx} & \textbf{token} \\
  \midrule
  \texttt{1} & \texttt{`\_\_kernel'} & \texttt{10} & \texttt{`,'} & \texttt{19} & \texttt{`const'} \\
  \texttt{2} & \texttt{` '} & \texttt{11} & \texttt{`short'} & \texttt{20} & \texttt{`d'} \\
  \texttt{3} & \texttt{`void'} & \texttt{12} & \texttt{`b'} & \texttt{21} & \texttt{`='} \\
  \texttt{4} & \texttt{`A'} & \texttt{13} & \texttt{`int'} & \texttt{22} & \texttt{`get\_global\_id'} \\
  \texttt{5} & \texttt{`('} & \texttt{14} & \texttt{`c'} & \texttt{23} & \texttt{`0'} \\
  \texttt{6} & \texttt{`\_\_global'} & \texttt{15} & \texttt{`)'} & \texttt{24} & \texttt{`;'} \\
  \texttt{7} & \texttt{`char'} & \texttt{16} & \texttt{`\{'} & \texttt{25} & \texttt{`['} \\
  \texttt{8} & \texttt{`*'} & \texttt{17} & \texttt{`\textbackslash n'} & \texttt{26} & \texttt{`]'} \\
  \texttt{9} & \texttt{`a'} & \texttt{18} & \texttt{`\space\space'} & \texttt{27} & \texttt{`\}'} \\
  \bottomrule
\end{tabular}%
%
    \label{subfig:source_vocab}%
  }\\%
  \subfloat[%
    Indices encoded kernel sequence. Sequences may be padded to a fixed length
    by repeating an out-of-vocabulary integer (e.g. -1).%
  ]{%
    \rowcolors{2}{white}{gray!25}
\begin{tabular}{l l l l l l l l l l l}
	\toprule
	\texttt{01} & \texttt{02} & \texttt{03} & \texttt{02} & \texttt{04} & \texttt{05} & \texttt{06} & \texttt{02} & \texttt{07} & \texttt{08} & \texttt{02} \\
	\texttt{09} & \texttt{10} & \texttt{02} & \texttt{11} & \texttt{02} & \texttt{12} & \texttt{10} & \texttt{02} & \texttt{13} & \texttt{02} & \texttt{14} \\
	\texttt{15} & \texttt{02} & \texttt{16} & \texttt{17} & \texttt{18} & \texttt{19} & \texttt{02} & \texttt{13} & \texttt{02} & \texttt{20} & \texttt{02} \\
	\texttt{21} & \texttt{02} & \texttt{22} & \texttt{05} & \texttt{23} & \texttt{15} & \texttt{24} & \texttt{17} & \texttt{18} & \texttt{09} & \texttt{25} \\
	\texttt{20} & \texttt{26} & \texttt{02} & \texttt{21} & \texttt{02} & \texttt{12} & \texttt{24} & \texttt{17} & \texttt{27} & \multicolumn{2}{l}{\texttt{<pad\ldots>}} \\
	\bottomrule
\end{tabular}
%
    \label{subfig:source_enc}%
  }%
  \caption[Deriving a vocabulary encoding from an OpenCL source code]{%
    Deriving a tokenised $1$-of-$k$ vocabulary encoding from an OpenCL source
    code.%
  }%
  \label{fig:encoding}%
\end{figure}
