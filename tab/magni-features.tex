\begin{table}
  \rowcolors{2}{gray!25}{white}
  \centering%
    \begin{tabular}{| l l |}
      \hline
      \rowcolor{gray!50}
      \textbf{Name} & \textbf{Description} \\
      \hline
      \texttt{BasicBlocks} & \#.\ basic blocks \\
      \texttt{Branches} & \#.\ branches \\
      \texttt{DivInsts} & \#.\ divergent instructions \\
      \texttt{DivRegionInsts} & \#.\ instructions in divergent regions \\
      \texttt{DivRegionInstsRatio} & \#.\ instr. in divergent regions / total instructions \\
      \texttt{DivRegions} & \#.\ divergent regions \\
      \texttt{TotInsts} & \#.\ instructions \\
      \texttt{FPInsts} & \#.\ floating point instructions \\
      \texttt{ILP} & average ILP / basic block \\
      \texttt{Int/FP Inst Ratio} & \#.\ branches \\
      \texttt{IntInsts} & \#.\ integer instructions \\
      \texttt{MathFunctions} & \#.\ match builtin functions \\
      \texttt{MLP} & average MLP / basic block \\
      \texttt{Loads} & \#.\ loads \\
      \texttt{Stores} & \#.\ stores \\
      \texttt{UniformLoads} & \#.\ loads unaffected by coarsening direction \\
      \texttt{Barriers} & \#.\ barriers \\
      \hline
    \end{tabular}%
    \label{tab:features-pact14-raw}%
  \caption[\citeauthor{Magni2014}features for predicting thread coarsening]{%
    Candidate features used by \citeauthor{Magni2014}for predicting thread
    coarsening. From these values, they compute relative deltas for each
    iteration of coarsening, then use PCA for selection.%
  }%
  \label{tab:magni-features} %
\end{table}
