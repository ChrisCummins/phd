\begin{table}
	\rowcolors{2}{white}{gray!25}
	\scriptsize%
	\centering%
	%\subfloat[Candidate features]{%
		\begin{tabular}{l L{4.6cm}}
			\toprule
			\textbf{Name} & \textbf{Description} \\
			\midrule
			\texttt{BasicBlocks} & \#.\ basic blocks \\
			\texttt{Branches} & \#.\ branches \\
			\texttt{DivInsts} & \#.\ divergent instructions \\
			\texttt{DivRegionInsts} & \#.\ instructions in divergent regions \\
			\texttt{DivRegionInstsRatio} & \#.\ instr. in divergent regions / total instructions \\
			\texttt{DivRegions} & \#.\ divergent regions \\
			\texttt{TotInsts} & \#.\ instructions \\
			\texttt{FPInsts} & \#.\ floating point instructions \\
			\texttt{ILP} & average ILP / basic block \\
			\texttt{Int/FP Inst Ratio} & \#.\ branches \\
			\texttt{IntInsts} & \#.\ integer instructions \\
			\texttt{MathFunctions} & \#.\ match builtin functions \\
			\texttt{MLP} & average MLP / basic block \\
			\texttt{Loads} & \#.\ loads \\
			\texttt{Stores} & \#.\ stores \\
			\texttt{UniformLoads} & \#.\ loads unaffected by coarsening direction \\
			\texttt{Barriers} & \#.\ barriers \\
			\bottomrule
		\end{tabular}%
		\label{tab:features-pact14-raw}%
%	}\\*
%    \rowcolors{2}{white}{gray!25}
%	\subfloat[Final selected features]{%
%		\rowcolors{2}{white}{gray!25}
%		\begin{tabular}{l L{4.6cm}}
%			\toprule
%			\textbf{Name} & \textbf{Description} \\
%			\midrule
%			\texttt{F1: DeltaInsts} & $\Delta$ instructions / \#.\ instructions \\
%			\texttt{F2: DeltaDiv} & $\Delta$ divergent regions / \#.\ divergent regions \\
%			\texttt{F3: ArithmeticIntensity} & \#.\ math instructions / \#.\ instructions  \\
%			\texttt{F4: TotInsts} & \#.\ instructions \\
%			\texttt{F5: DivRegionInsts} & \#.\ instructions in divergent regions \\
%			\texttt{F6: DeltaArithDensity} & $\Delta$ arithmetic density / arithmetic density \\
%			\texttt{F7: ILP} & average ILP / basic block \\
%			\bottomrule
%		\end{tabular}%
%		\label{tab:features-pact14-final}%
%	} %
	\caption{%
		Candidate features used by Magni \emph{et al.\ }for predicting thread coarsening. From these values, they compute relative deltas for each iteration of coarsening, then use PCA for selection.%
	} %
	\label{tab:magni-features} %
\end{table}
