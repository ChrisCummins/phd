\section{Experimental Setup}

In this section we describe the particular experimental parameters used.

\subsection{Platforms}

\begin{table*}[t!]
  \scriptsize %
  \centering %
  % \rowcolors{2}{white}{gray!25}
  \begin{tabular}{ cllllll | rr }
\toprule
\textbf{\#. } & \textbf{Platform} & \textbf{Device} & \textbf{Driver} & \textbf{OpenCL} & 
\textbf{Operating system} & \textbf{Device Type} & \textbf{Testing time} & \textbf{Bugs Submitted} \\
\midrule
1 & NVIDIA CUDA & GeForce GTX 1080 & 375.39 & 1.2 & Ubuntu 16.04 64bit & GPU & 150 hours & 7 \\
2 & NVIDIA CUDA & GeForce GTX 780 & 361.42 & 1.2 & openSUSE  13.1 64bit & GPU & 109 hours & 0 \\
3 & Intel Gen OCL Driver & Intel HD Haswell GT2 & 1.3 & 1.2 & Ubuntu 16.04 64bit & GPU & 100 hours & 11 \\
4 & Intel OpenCL & Intel E5-2620 v4 & 1.2.0.25 & 2.0 & Ubuntu 16.04 64bit & CPU & 129 hours & 5 \\
5 & Intel OpenCL & Intel E5-2650 v2 & 1.2.0.44 & 1.2 & CentOS 7.1 64bit & CPU & 128 hours & 1 \\
6 & Intel OpenCL & Intel i5-4570 & 1.2.0.25 & 1.2 & Ubuntu 16.04 64bit & CPU & 126 hours & 4 \\
7 & Intel OpenCL & Intel Xeon Phi & 1.2 & 1.2 & CentOS 7.1 64bit & Accelerator & 129 hours & 1 \\
8 & POCL & POCL (Intel E5-2620) & 0.14 & 2.0 & Ubuntu 16.04 64bit & CPU & 121 hours & 22 \\
9 & ComputeAorta & ComputeAorta (Intel E5-2620) & 1.14 & 1.2 & Ubuntu 16.04 64bit & CPU & 122 hours & 0 \\
10 & Oclgrind & Oclgrind Simulator & 16.10 & 1.2 & Ubuntu 16.04 64bit & Emulator & 117 hours & 5 \\

\bottomrule
\end{tabular}


  \caption{%
    OpenCL platforms, the time spent in automated testing, and the number of bug reports submitted to date. \cc{refresh bugs submitted}%
  }
  \label{tab:platforms}
\end{table*}

We conducted testing of 10 OpenCL platforms, summarized in Table~\ref{tab:platforms}. Each testbed consists of a \emph{<device, driver>} pair. We covered a broad range of hardware: 3 GPUs, 4 CPUs, a co-processor, and an emulator. 7 of the compilers tested are commercial products, 3 of them are open source. Our suite of platforms includes both combinations of different drivers for the same device, and different devices using the same driver.

\subsection{Testbeds}
%HJL - Not sure if this is correct - if it is, ain't in the tables. So probably not.}
\hl{For each platform, we create multiple testbeds by various combinations of the following settings: running with or without compiler optimizations; running with a single thread or running with 2048 threads.}
%HJL alt- For each platform, we create two testbeds by either running with or without compiler optimizations.
