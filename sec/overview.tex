\section{Overview of Our Approach and Results}\label{sec:overview}

\paragraph{Test-case generation} We mine GitHub for OpenCL fragments, which we use to construct a corpus of representative programs. We learn a generative model over this corpus. DeepSmith, our deep learning program generator, automatically and rapidly synthesizes an unbounded number of programs for fuzz testing compilers.

\paragraph{OpenCL configurations tested} We conducted testing of 10 different OpenCL configurations, summarized in Table~\ref{tab:platforms}. A \emph{configuration} refers to an OpenCL \emph{<device, driver>} pair. We covered the broadest range of hardware available to us: 3 GPUs, 4 CPUs, an Accelerator, and an Emulator. 8 of the compilers tested commercial products, 2 of them are open source. We tested both different drivers for the same device, and different devices using the same driver.


\begin{table*}[t!]
	\scriptsize %
	\centering %
	% \rowcolors{2}{white}{gray!25}
	\begin{tabular}{ cllllll | rr }
\toprule
\textbf{\#. } & \textbf{Platform} & \textbf{Device} & \textbf{Driver} & \textbf{OpenCL} & 
\textbf{Operating system} & \textbf{Device Type} & \textbf{Testing time} & \textbf{Bugs Submitted} \\
\midrule
1 & NVIDIA CUDA & GeForce GTX 1080 & 375.39 & 1.2 & Ubuntu 16.04 64bit & GPU & 150 hours & 7 \\
2 & NVIDIA CUDA & GeForce GTX 780 & 361.42 & 1.2 & openSUSE  13.1 64bit & GPU & 109 hours & 0 \\
3 & Intel Gen OCL Driver & Intel HD Haswell GT2 & 1.3 & 1.2 & Ubuntu 16.04 64bit & GPU & 100 hours & 11 \\
4 & Intel OpenCL & Intel E5-2620 v4 & 1.2.0.25 & 2.0 & Ubuntu 16.04 64bit & CPU & 129 hours & 5 \\
5 & Intel OpenCL & Intel E5-2650 v2 & 1.2.0.44 & 1.2 & CentOS 7.1 64bit & CPU & 128 hours & 1 \\
6 & Intel OpenCL & Intel i5-4570 & 1.2.0.25 & 1.2 & Ubuntu 16.04 64bit & CPU & 126 hours & 4 \\
7 & Intel OpenCL & Intel Xeon Phi & 1.2 & 1.2 & CentOS 7.1 64bit & Accelerator & 129 hours & 1 \\
8 & POCL & POCL (Intel E5-2620) & 0.14 & 2.0 & Ubuntu 16.04 64bit & CPU & 121 hours & 22 \\
9 & ComputeAorta & ComputeAorta (Intel E5-2620) & 1.14 & 1.2 & Ubuntu 16.04 64bit & CPU & 122 hours & 0 \\
10 & Oclgrind & Oclgrind Simulator & 16.10 & 1.2 & Ubuntu 16.04 64bit & Emulator & 117 hours & 5 \\

\bottomrule
\end{tabular}


	\caption{OpenCL configurations we tested, the time spent in automated testing, and the number of bug reports submitted to date.}
	\label{tab:platforms}
\end{table*}


\paragraph{Bugs found} All configurations yielded bugs. Every compiler crashed, and every compiler yielded anomalous results --- either programs which crash, or programs which silently compute the wrong result. To date, we have submitted XX bug reports to compiler vendors. Of those, 41\% are cases where the compiler crashes, and the remainder are cases where the generated program either silently emits wrong-code, or crashes at runtime. In comparing our approach against the state-of-the-art in OpenCL compiler test-case generation, we found our approach was able to \cc{\ldots}