\section{DeepTune: Learning On Raw Program Code}%
\label{sec:deeptune}

DeepTune is an end-to-end machine learning pipeline for optimisation heuristics. Its primary input is the source code of a program to be optimised, and through a series of neural networks, it directly predicts the optimisation which should be applied. By learning on source code, the approach is not tied to a specific compiler, platform, or optimisation problem. The same design can be reused to build multiple heuristics. The most important innovation of DeepTune is that it forgoes the need for human experts to select and tune appropriate features.


\subsection{Overview}

Figure~\ref{fig:deeptune} provides an overview of the system. A source re-writer removes semantically irrelevant information (such as comments) from the source code of the target program and passes it to a language model. The language model converts the arbitrary length stream of code into a fixed length vector of real values which fully capture the properties and structure of the source, replacing the role of hand designed features. This vector can then optionally be concatenated with auxiliary inputs, which allow passing additional data about runtime or architectural parameters to the model for heuristics which need more than just compile-time information. Finally, a standard feed-forward network is used to predict the best heuristic parameters to optimise the program.

\begin{figure}
  \centering
  \includegraphics[width=\columnwidth]{img/deeptune} %
  \vspace{-2em}%
  \caption{%
    DeepTune architecture. Code properties are extracted from source code by the
    language model. They are fed, together with optional auxiliary inputs, to
    the heuristic model to produce the final prediction.%
  }%
  \label{fig:deeptune}
\end{figure}


DeepTune is open source. The model is implemented in Keras, with TensorFlow~\cite{Abadi} and Theano~\cite{Bergstra2011} back-ends.


\subsection{Language Model}

Learning effective representations of source code is a difficult task. A successful model must be able to:

\begin{itemize}
  \item derive semantic and syntactic patterns of a programming language entirely from sample codes;
  \item identify the patterns and representation in source codes which are relevant to the task at hand; and
  \item discriminate performance characteristics arising from potentially subtle differences in similar codes.
\end{itemize}

To achieve this task, state-of-the-art language modelling techniques are employed, coupled with a series of generic, language agnostic code transformations.

\subsubsection{Source Re-writer}

To begin with, a series of \emph{source normalising} transformations are applied, extending the system described in Chapter~\ref{chap:clgen}. These transformations, implemented as an LLVM pass, parse the AST, removing conditional compilation, then rebuild the input source code using a consistent code style and identifier naming scheme. The role of source normalisation is to simplify the task of modelling source code by ensuring that trivial semantic differences in programs such as the choice of variable names or the insertion of comments do not affect the learned model. Figures~\ref{subfig:source_in} and~\ref{subfig:source_out} show the source rewriting process applied to a simple program.

\subsubsection{Sequence Encoder}

The textual representation of program codes must be encoded as numeric sequences for feeding as input to the machine learning model. This is achieved by extending the encoder of Chapter~\ref{sec:deeptune}, in which a programming language's keywords and common names are treated as individual tokens while the rest of the text is encoded on a character-level basis. This approach hits a balance between compressing the input text and keeping the number of tokens in the vocabulary low. Figure~\ref{subfig:source_vocab} shows the vocabulary derived for a single input source code Figure~\ref{subfig:source_out}.

\subsubsection{Embedding}

During encoding, tokens in the vocabulary are mapped to unique integer values, e.g. \texttt{float} $\rightarrow 0$, \texttt{int} $\rightarrow$ 1. The integer values chosen are arbitrary, and offer a \emph{sparse} data representation, meaning that a language model cannot infer the relationships between tokens based on their mappings. This is in contrast to the \emph{dense} representations of other domains, such as pixels in images, which can be interpolated between to derive the differences in colours.

To mitigate this, an \emph{embedding} is used, which translates tokens in a sparse, integer encoded vocabulary into a lower dimensional vector space, allowing semantically related tokens like \texttt{float} and \texttt{int} to be mapped to nearby points~\cite{Mikolov2013a,Baroni2014}. An embedding layer maps each token in the integer encoded vocabulary to a vector of real values. Given a vocabulary size $|V|$ and embedding dimensionality $D$, an embedding matrix $\bm{W_{E}} \in \mathbb{R}^{|V| \times D}$ is learned during training, so that an integer encoded sequences of tokens $\bm{t} \in \mathbb{N}^{L}$ is mapped to the matrix $\bm{T} \in \mathbb{R}^{L \times D}$. In this work, an embedding dimensionality $D = 64$ is used.

\newsavebox{\NvidiaStreamClusterInput}
\begin{lrbox}{\NvidiaStreamClusterInput}
  \begin{minipage}{\textwidth}
    \begin{minted}{opencl_lexer.py:OpenCLLexer -x}
//#define Elements
kernel void memset_kernel(global char * mem_d, short val, int number_bytes){
    const int thread_id = get_global_id(0);
    mem_d[thread_id] = val;
}
    \end{minted}
  \end{minipage}
\end{lrbox}

\newsavebox{\NvidiaStreamClusterOutput}
\begin{lrbox}{\NvidiaStreamClusterOutput}
  \begin{minipage}{\textwidth}
    \begin{minted}{opencl_lexer.py:OpenCLLexer -x}
kernel void A(global char* a, short b, int c) {
  const int d = get_global_id(0);
  a[d] = b;
}
    \end{minted}
  \end{minipage}
\end{lrbox}

\begin{figure}
  \centering %
  \subfloat[%
    An example, short OpenCL kernel, taken from Nvidia's \emph{streamcluster}.%
  ]{\noindent\mbox{\parbox{\columnwidth}{\usebox{\NvidiaStreamClusterInput}}}}\\%
  \subfloat[%
    The \emph{streamcluster} kernel after source rewriting. Variable and
    function names are normalised, comments removed, and code style enforced.%
  ]{\noindent\mbox{\parbox{\columnwidth}{\usebox{\NvidiaStreamClusterOutput}}}}\\%
  \subfloat[%
    Derived vocabulary, ordered by their appearance in the
    input~\protect\subref{subfig:source_out}. The vocabulary maps tokens to
    integer indices.%
  ]{%
    \footnotesize
\begin{tabular}{l l | l l | l l}
  \toprule
  \textbf{idx} & \textbf{token} & \textbf{idx} & \textbf{token} & \textbf{idx} & \textbf{token} \\
  \midrule
  \texttt{1} & \texttt{`\_\_kernel'} & \texttt{10} & \texttt{`,'} & \texttt{19} & \texttt{`const'} \\
  \texttt{2} & \texttt{` '} & \texttt{11} & \texttt{`short'} & \texttt{20} & \texttt{`d'} \\
  \texttt{3} & \texttt{`void'} & \texttt{12} & \texttt{`b'} & \texttt{21} & \texttt{`='} \\
  \texttt{4} & \texttt{`A'} & \texttt{13} & \texttt{`int'} & \texttt{22} & \texttt{`get\_global\_id'} \\
  \texttt{5} & \texttt{`('} & \texttt{14} & \texttt{`c'} & \texttt{23} & \texttt{`0'} \\
  \texttt{6} & \texttt{`\_\_global'} & \texttt{15} & \texttt{`)'} & \texttt{24} & \texttt{`;'} \\
  \texttt{7} & \texttt{`char'} & \texttt{16} & \texttt{`\{'} & \texttt{25} & \texttt{`['} \\
  \texttt{8} & \texttt{`*'} & \texttt{17} & \texttt{`\textbackslash n'} & \texttt{26} & \texttt{`]'} \\
  \texttt{9} & \texttt{`a'} & \texttt{18} & \texttt{`\space\space'} & \texttt{27} & \texttt{`\}'} \\
  \bottomrule
\end{tabular}%
%
    \label{subfig:source_vocab}%
  }\\%
  \subfloat[%
    Indices encoded kernel sequence. Sequences may be padded to a fixed length
    by repeating an out-of-vocabulary integer (e.g. -1).%
  ]{%
    \rowcolors{2}{white}{gray!25}
\footnotesize
\begin{tabular}{l l l l l l l l l l l}
	\toprule
	\texttt{01} & \texttt{02} & \texttt{03} & \texttt{02} & \texttt{04} & \texttt{05} & \texttt{06} & \texttt{02} & \texttt{07} & \texttt{08} & \texttt{02} \\
	\texttt{09} & \texttt{10} & \texttt{02} & \texttt{11} & \texttt{02} & \texttt{12} & \texttt{10} & \texttt{02} & \texttt{13} & \texttt{02} & \texttt{14} \\
	\texttt{15} & \texttt{02} & \texttt{16} & \texttt{17} & \texttt{18} & \texttt{19} & \texttt{02} & \texttt{13} & \texttt{02} & \texttt{20} & \texttt{02} \\
	\texttt{21} & \texttt{02} & \texttt{22} & \texttt{05} & \texttt{23} & \texttt{15} & \texttt{24} & \texttt{17} & \texttt{18} & \texttt{09} & \texttt{25} \\
	\texttt{20} & \texttt{26} & \texttt{02} & \texttt{21} & \texttt{02} & \texttt{12} & \texttt{24} & \texttt{17} & \texttt{27} & \multicolumn{2}{l}{\texttt{<pad\ldots>}} \\
	\bottomrule
\end{tabular}
%
    \label{subfig:source_enc}%
  }%
  \caption[Deriving a vocabulary encoding from an OpenCL source code]{%
    Deriving a tokenised $1$-of-$k$ vocabulary encoding from an OpenCL source
    code.%
  }%
  \label{fig:encoding}%
\end{figure}


\subsubsection{Sequence Characterisation}

Once source codes have been encoded into sequences of embedding vectors, neural networks are used to extract a fixed size vector which characterises the entire sequence. This is comparable to the hand engineered feature extractors used in prior works, but is a \emph{learned} process that occurs entirely --- and automatically --- within the hidden layers of the network.

The Long Short-Term Memory (LSTM) architecture is used for sequence characterisation~\cite{Hochreiter1997}. LSTMs implements a Recurrent Neural Network in which the activations of neurons are learned with respect not just to their current inputs, but to previous inputs in a sequence. Unlike regular recurrent networks in which the strength of learning decreases over time (a symptom of the \emph{vanishing gradients} problem~\cite{Pacanu2013}), LSTMs employ a \emph{forget gate} with a linear activation function, allowing them to retain activations for arbitrary durations. This makes them effective at learning complex relationships over long sequences~\cite{Lipton2015}, an especially important capability for modelling program code, as dependencies in sequences frequently occur over long ranges (for example, a variable may be declared as an argument to a function and used throughout).

The LSTM network has two layers of cells. The network receives a sequence of embedding vectors, and returns a single output vector, characterising the entire sequence.


\subsection{Auxiliary Inputs}

An arbitrary number of additional real valued \emph{auxiliary inputs} may be optionally used to augment the source code input. These inputs are provided as a means of increasing the flexibility of the system, for example, to support applications in which the optimisation heuristic depends on dynamic values which cannot be statically determined from the program code~\cite{Ding2015,Stephenson2005}. When present, the values of auxiliary inputs are concatenated with the output of the language model, and fed into a heuristic model.


\subsection{Heuristic Model}

The heuristic model takes the learned representations of the source code and auxiliary inputs (if present), and uses these values to make the final optimisation prediction.

First the values are normalised. Normalisation is necessary because the auxiliary inputs can have any values, whereas the language model activations are in the range [0,1]. If we did not normalise, then scaling the auxiliary inputs could affect the training of the heuristic model. Normalisation occurs in batches. The batch normalisation method of~\cite{Ioffe2015a} is used, in which each scalar of the heuristic model's $n$ inputs $\bm{x}^{(1)}, \ldots, \bm{x}^{(n)}$ is independently normalised to a mean 0 and standard deviation of 1:

\begin{equation}
\bm{\hat{x}}^{(i)} = \bm{\upgamma}^{(i)} \frac{\bm{x}^{(i)} - E(\bm{x}^{(i)})}{\sqrt{Var(\bm{x}^{(i)})}} + \bm{\beta}^{(i)}
\end{equation}

Where $\bm{\upgamma}$ and $\bm{\beta}$ are scale and shift parameters, learned during training.

The final component of DeepTune is comprised of two fully connected neural network layers. The first layer consists of 32 neurons. The second layer consists of a single neuron for each possible heuristic decision. Each neuron applies an activation function $f(x)$ over its inputs. Rectifier activation functions $f(x) = \max(0, x)$ are used in the first layer due to their improved performance during training of deep networks~\cite{Nair2010}. For the output layer, sigmoid activation functions $f(x) = \frac{1}{1+e^{-x}}$ are used which provide activations in the range $[0,1]$.

The activation of each neuron in the output layer represents the model's confidence that the corresponding decision is the correct one. Taking the $\argmax$ of the output layer produces the decision with the largest activation. For example, for a binary optimisation heuristic the final layer will consist of two neurons, and the predicted optimisation is the neuron with the largest activation.


\subsection{Training the network}

DeepTune is trained in the same manner as prior supervised machine learning works, the key difference being that instead of having to manually create and extract features from programs, the raw program codes themselves are used.

The model is trained with Stochastic Gradient Descent (SGD), using the Adam optimiser~\cite{Kingma2015}. For training data $X^{(1)}, \ldots, X^{(n)}$, SGD attempts to find the model parameters $\Theta$ that minimise the output of a loss function:

\begin{equation}
\Theta = \argmin_{\Theta} \frac{1}{n} \sum_{i=1}^{n} \mathcal{L} \left( \bm{X}^{(i)}, \Theta \right)
\end{equation}

where loss function $\mathcal{L} \left(x, \Theta \right)$ computes the logarithmic difference between the predicted and expected values given a model constructed using parameters $\Theta$.

To reduce training time, multiple inputs are \emph{batched} together and fed into the neural network simultaneously, reducing the frequency of costly weight updates during back-propagation. This requires that the inputs to the language model be the same length. Sequences are padded up to a fixed length of 1024 tokens using a special out-of-vocabulary padding token $\tau \not\in V$, allowing matrices of \texttt{batch\_size} $\times$ \texttt{max\_seq\_len} tokens to be processed simultaneously. Batching and padding sequences to a maximum length is only to improve training time. In production use, sequences do not need to be padded, allowing classification of arbitrary length codes in linear time.
