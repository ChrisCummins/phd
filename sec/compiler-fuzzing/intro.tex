\section{Introduction}

Compilers should produce correct code for valid inputs, and meaningful errors for invalid inputs. Failure to do so hinders software development and even causes catastrophic runtime errors. Still, properly testing compilers is hard. Modern optimising compilers are large and complex programs, and their input space is huge. Hand-designed suites of test programs, while important, are inadequate for covering such a large space and will not touch all parts of the compiler.

Random test case generation --- \emph{fuzzing} --- is a well-established and effective method for identifying compiler bugs~\cite{Chen2014a,Chen2013,Kossatchev2005}. When fuzzing, randomly generated valid or semi-valid inputs are fed to the compiler. Any kind of unexpected behaviour, including crashes, freezes, or wrong binaries, indicates a compiler bug. While crashes and freezes in the compiler are easy to detect, determining that binaries are correctly compiled is not generally possible without either developer provided validation for the particular program's behaviour or a gold standard compiler from which to create reference outputs. In the absence of those, Differential Testing~\cite{McKeeman1998} can be used. The generated code and a set of inputs form a \emph{test case} which is compiled and executed on multiple \emph{testbeds}. If the test case should have deterministic behaviour, but the output differs between testbeds, then a bug has been discovered.

Compiler fuzzing requires efficiently generating test cases that trigger compiler bugs. The state-of-the-art approach, CSmith~\cite{Yang2011}, generates large random programs by defining and sampling a probabilistic grammar which covers a subset of the C programming language. Through this grammar, CSmith ensures that the generated code easily passes the compiler front-end and stresses the most complex part of the compiler, the middle-end. Complex static and dynamic analyses make sure that programs are free from undefined behaviour. The programs are then differentially tested.

While CSmith has been successfully used to identify hundreds of bugs in otherwise-robust  compilers, it and similar approaches have a significant drawback. They represent a huge undertaking and require a thorough understanding of the target programming language. CSmith was developed over the course of years and consists of over 41k lines of handwritten C++ code. By tightly coupling the generation logic with the target programming language, each feature of the grammar must be painstakingly and expertly engineered for each new target language. For example, lifting CSmith from C to OpenCL~\cite{Lidbury2015a} --- a superficially simple task --- took 9 months and an additional 8k lines of code. Given the difficulty of defining a new grammar, typically only a subset of the language is implemented.

This chapter introduces \emph{DeepSmith}, a novel machine learning approach to accelerating compiler validation through the inference of generative models for compiler inputs. DeepSmith is a fast, effective, and low effort approach to the generation of random programs for compiler fuzzing. The methodology, extending the technique developed in Chapter~\ref{chap:clgen}, uses recent advances in deep learning to automatically \emph{infer} probabilistic models of how humans write code, instead of painstakingly defining a grammar to the same end. By training a deep neural network on a corpus of handwritten code, it is able to infer both the syntax and semantics of the programming language and the common constructs and patterns. The approach essentially frames the generation of random programs as a language modelling problem. This greatly simplifies and accelerates the process. The expressiveness of the generated programs is limited only by what is contained in the corpus, not the developer's expertise or available time. Such a corpus can readily be assembled from open source repositories. Once trained, the model is used to automatically generate tens of thousands of realistic programs. Finally, established differential testing methodologies are used on them to expose bugs in compilers.

In this chapter, the approach is applied to test compilers for the OpenCL programming language. In 48 hours of automated testing of commercial and open source compilers, bugs are discovered in all of them, and 67 bug reports are submitted. The generated test cases are on average two orders of magnitude smaller than the state-of-the-art, require $3.03\times$ less time to generate and evaluate, and expose bugs which the state-of-the-art cannot. The random program generator, comprising only 500 lines of code, took 12 hours to train for OpenCL versus the state-of-the-art taking 9 man-months to port from a generator for C and 50,000 lines of code.  This work primarily targets OpenCL, an open standard for programming heterogeneous systems, though the approach is largely language-agnostic. OpenCL is chosen for three reasons: it is an emerging standard with the challenging promise of functional portability across a diverse range of heterogeneous hardware; OpenCL is compiled ``online'', meaning that even compiler crashes and freezes may not be discovered until a product is deployed to customers; and there is already a hand written random program generator for the language to compare against. With 18 lines of code, the program generator is extended to a second language, uncovering crashes in Solidity compilers in 12 hours of automated testing.

This chapter is organised as follows:  Section~\ref{sec:deepsmith} presents DeepSmith, a novel approach to compiler validation. Section~\ref{sec:deepsmith-experimental-setup} describes the experimental setup of an extensive evaluation of OpenCL compilers using DeepSmith. Section~\ref{sec:deepsmith-eval} evaluates the results of the experiment, with Section~\ref{subsec:deepsmith-solidity-extensibility} containing preliminary results supporting DeepSmith's potential for multi-lingual compiler fuzzing. Section~\ref{sec:deepsmith-conclusion} provides concluding remarks for this chapter.
