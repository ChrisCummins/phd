\section{Experimental Setup}
\label{sec:deepsmith-experimental-setup}

In this section we describe the experimental parameters used.

\subsection{OpenCL Systems}

We conducted testing of 10 OpenCL systems, summarised in
Table~\ref{tab:deepsmith-platforms}. We covered a broad range of hardware: 3 GPUs, 4
CPUs, a co-processor, and an emulator. 7 of the compilers tested are commercial
products, 3 of them are open source. Our suite of systems includes both
combinations of different drivers for the same device, and different devices
using the same driver.

\begin{table}
  \centering %
  \subfloat[][]{\rowcolors{2}{gray!25}{white}
\begin{tabular}{ | c l l l l | }
	\hline
	\rowcolor{gray!50}
	\textbf{\#. } & \textbf{Platform} & \textbf{Device} & \textbf{Driver} & \textbf{OpenCL} \\
	\hline
	1 & NVIDIA CUDA & GeForce GTX 1080 & 375.39 & 1.2 \\
	2 & NVIDIA CUDA & GeForce GTX 780 & 361.42 & 1.2 \\
	3 & Beignet & Intel HD Haswell GT2 & 1.3 & 1.2 \\
	4 & Intel OpenCL & Intel E5-2620 v4 & 1.2.0.25 & 2.0 \\
	5 & Intel OpenCL & Intel E5-2650 v2 & 1.2.0.44 & 1.2 \\
	6 & Intel OpenCL & Intel i5-4570 & 1.2.0.25 & 1.2 \\
	7 & Intel OpenCL & Intel Xeon Phi & 1.2 & 1.2 \\
	8 & POCL & POCL (Intel E5-2620) & 0.14 & 1.2 \\
	9 & Codeplay & ComputeAorta (Intel E5-2620) & 1.14 & 1.2 \\
	10 & Oclgrind & Oclgrind Simulator & 16.10 & 1.2 \\
	\hline
\end{tabular}
}%
  \\%
  \subfloat[][]{\rowcolors{2}{gray!25}{white}
\begin{tabular}{ | c l l R{2.2cm} | R{2.75cm} | }
  \hline
  \rowcolor{gray!50}
  \textbf{\#. } & \textbf{Operating system} & \textbf{Device Type} & \textbf{Open Source?} & \textbf{Bug Reports Submitted} \\
  \hline
  1 & Ubuntu 16.04 64bit & GPU & & 8 \\
  2 & openSUSE 13.1 64bit & GPU & & 1 \\
  3 & Ubuntu 16.04 64bit & GPU & Yes & 13 \\
  4 & Ubuntu 16.04 64bit & CPU & & 6 \\
  5 & CentOS 7.1 64bit & CPU & & 1 \\
  6 & Ubuntu 16.04 64bit & CPU & & 5 \\
  7 & CentOS 7.1 64bit & Accelerator & & 3 \\
  8 & Ubuntu 16.04 64bit & CPU & Yes & 22 \\
  9 & Ubuntu 16.04 64bit & CPU & & 1 \\
  10 & Ubuntu 16.04 64bit & Emulator & Yes & 7 \\
  \hline
\end{tabular}
} %
  \caption[OpenCL systems and the number of bug reports submitted to date]{%
    OpenCL systems and the number of bug reports submitted to date (22\% of which have been fixed, the remainder are pending). For each system, two testbeds are created, one with compiler optimisations, the other without.%
  }
  \label{tab:deepsmith-platforms}
\end{table}

\subsection{Testbeds}

For each OpenCL system, we create two testbeds. In the first, the compiler is
run with optimisations disabled. In the second, optimisations are enabled. Each
testbed is then a triple, consisting of \emph{<device, driver, is\_optimised>}
settings. This mechanism gives 20 testbeds to evaluate.


\subsection{Test Cases}

For each generated program we create inputs as described in
Section~\ref{sec:test-harness}. In addition, we need to choose the number of
threads to use. We generate two test cases, one using one thread, the other
using 2048 threads. A test case is then a triple, consisting of \emph{<program,
inputs, threads>} settings.

\subsection{Bug Search Time Allowance}

DeepSmith and CLsmith are compared by allowing both to run for 48 hours on
each of the 20 testbeds. CLSmith used its default configuration. The total
runtime for a test case consists of the generation and execution time.
