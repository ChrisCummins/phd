\section{Summary}
\label{sec:deepsmith-conclusion}

This chapter presents a novel framework for compiler fuzzing. By posing the generation of random programs as an unsupervised machine learning problem, the cost and human effort required to engineer a compiler fuzzer are drastically lowered. This aims to address the \emph{adoption challenge} of machine learning (Section~\ref{subsec:challenge-adoption}). Large parts of the stack are programming language-agnostic, requiring only a corpus of example programs, an encoder, and a test harness to target a new language.

The approach is demonstrated by targeting the challenging many-core domain of OpenCL. The implementation, DeepSmith, has uncovered dozens of bugs in both commercial and open-source OpenCL compilers. DeepSmith exposed bugs in parts of the compiler where current approaches have not, for example in missing error handling. A preliminary exploration of the extensibility of our approach to other languages has been performed. DeepSmith test cases are small, two orders of magnitude shorter than the state-of-the-art, and easily interpretable.
