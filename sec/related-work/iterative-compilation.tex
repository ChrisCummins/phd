\section{Iterative Compilation \& Machine Learning}
\label{sec:related-work-iterative-compilation}

Modern compilers are complex, containing dozens or hundreds of optimization passes. Determining which optimisation passes should be applied, and in what order, is nontrivial, and depends on many factors from the properties of the program being compiled to the the target hardware. Current state-of-practise is for compilers to use a fixed ordering of optimisations, and for each optimization to contain a heuristic to determine when it should be used, and with what parameters. Such a heuristic is typically developed by hand at the expense of compiler expertise and manhours.

The result is that performance

\todo[inline]{The motivation for iterative compilation is that optimization heuristics are imperfect.}

% Georgiou, K., Blackmore, C., Xavier-de-Souza, S., & Eder, K. (2018). Less is More: Exploiting the Standard Compiler Optimization Levels for Better Performance and Energy Consumption. In SCOPES.
\todo[inline]{Using \emph{less} flags that -O2 can improve performance~\cite{Georgiou2018}.}

Iterative compilation is a method of performance tuning in which a
program is compiled and profiled using multiple different
configurations of optimisations in order to find the configuration
which maximises performance. One of the first formalised
publications of the technique appeared in \citeyear{Bodin1998} by
\citeauthor{Bodin1998}~\cite{Bodin1998}.  Iterative compilation has
since been demonstrated to be a highly effective form of empirical
performance tuning for selecting compiler optimisations.

% Kraska, T., Beutel, A., Chi, E. H., Dean, J., & Polyzotis, N. (2017). The Case for Learned Index Structures. ArXiv:1712.01208.
\todo[inline]{The Case for Learned Index Structures~\cite{Kraska2017}.}

Given the huge number of possible compiler optimisations (there are
207 flags and parameters to control optimisations in GCC v4.9), it is
often unfeasible to perform an exhaustive search of the entire
optimisation space, leading to the development of methods for reducing
the cost of evaluating configurations. These methods reduce evaluation
costs either by shrinking the dimensionality or size of the
optimisation space, or by guiding a directed search to traverse a
subset of the space.

Machine learning has been successful applied to this problem,
in~\cite{Stephenson2003}, using ``meta optimisation'' to tune compiler
heuristics through an evolutionary algorithm to automate the search of
the optimisation space. \citeauthor{Fursin2011} continued this with
Milepost GCC, the first machine learning-enabled self-tuning
compiler~\cite{Fursin2011}. A recent survey of the use of machine
learning to improve heuristics quality by \citeauthor{Burke2013}
concludes that the automatic \emph{generation} of these self-tuning
heuristics but is an ongoing research challenge that offers the
greatest generalisation benefits~\cite{Burke2013}.

% Ashouri, A. H., Killian, W., Cavazos, J., Palermo, G., & Silvano, C. (2018). A Survey on Compiler Autotuning using Machine Learning. CSUR, 51(5). https://doi.org/10.1145/3197978
\todo[inline]{Autotuning through ML survey~\cite{Ashouri2018}.}

An approach to online tuning of parallel programs is presented in~\cite{Ansel2012} which partitions the available parallel resources of a device in to two partitions and then executes two different configurations simultaneously using each partition. The configuration used for one of the configuration is guaranteed to be ``safe'', and the performance

 % Eastep, J., Wingate, D., & Agarwal, A. (2011). Smart Data
 % Structures: An Online Machine Learning Approach to Multicore Data
 % Structures. In Proceedings of the 8th ACM International Conference
 % on Autonomic Computing (pp. 11–20). New York, NY, USA:
 % ACM. doi:10.1145/1998582.1998587
Online reinforcement learning for optimising data structures online, \cite{Tesauro2005}.

 % Tesauro, G. (2005). Online Resource Allocation Using Decompositional
 % Reinforcement Learning. In AAAI (pp. 886–891).
Reinforcement learning for resource allocation~\cite{Eastep2011}.

% W. F. Ogilvie, P. Petoumenos, Z. Wang, and H. Leather, “Intelligent
% Heuristic Construction with Active Learning,” in 18th International
% Workshop on Compilers for Parallel Computing, 2015.
\todo[inline]{Using Active Learning to speed up the learning of compiler heuristics~\cite{Ogilvie2015}. Towards online autotuning, albeit only with binary optimisation parameter.}

% Ashouri, A. H., Milano, G., Palermo, G., Park, E., Cavazos, J., & Silvano, C. (2016). COBAYN: Compiler Autotuning Framework Using Bayesian Networks. TACO, 13(2).
\todo[inline]{Amir's ML autotuning framework~\cite{Ashouri2016}.}

%
% SOME EXAMPLES OF ML IN THE WILD:
%

In~\cite{Saclay2010,Memon2013,Fursin2014}, \citeauthor{Fursin2014} advocate a collaborative and ``big data'' driven approach to autotuning, arguing that the challenges facing the widespread adoption of autotuning and machine learning methodologies can be attributed to: a lack of common, diverse benchmarks and data sets; a lack of common experimental methodology; problems with continuously changing hardware and software stacks; and the difficulty to validate techniques due to a lack of sharing in publications. They propose a system for addressing these concerns, the Collective Mind knowledge system, which, while in early stages of ongoing development, is intended to provide a modular infrastructure for sharing autotuning performance data and related artefacts across the internet. In addition to sharing performance data, the approach taken in this thesis emphasises the collective \emph{exploitation} of such performance data, so that data gathered from one device may be used to inform the autotuning decisions of another. This requires each device to maintain local caches of shared data to remove the network overhead that would be present from querying a single centralised data store during execution of a hot path. The current implementation of Collective Mind uses a NoSQL JSON format for storing performance data. The relational schema used in this thesis offers greater scaling performance and lower storage overhead as the amount of performance data grows.

Whereas iterative compilation requires an expensive offline training
phase to search an optimisation space, dynamic optimisers perform this
optimisation space exploration at runtime, allowing programs to
respond to dynamic features ``online''. This is a challenging task, as
a random search of an optimisation space may result in configurations
with vastly sub-optimal performance. In a real world system, evaluating
many sub-optimal configurations will cause a significant slowdown of
the program. Thus a requirement of dynamic optimisers is that
convergence time towards optimal parameters is minimised.

Existing dynamic optimisation research has typically taken a low level
approach to performing optimisations. Dynamo is a dynamic optimiser
which performs binary level transformations of programs using
information gathered from runtime profiling and
tracing~\cite{Bala2000}. While this provides the ability to respond to
dynamic features, it restricts the range of optimisations that can be
applied to binary transformations. These low level transformations
cannot match the performance gains that higher level parameter tuning
produces.

An interesting related tangent to iterative compilation is the
development of so-called ``super-optimisers''. In~\cite{Massalin1987},
the smallest possible program which performs a specific function is
found through a brute force enumeration of the entire instruction
set. Starting with a program of a single instruction, the
super-optimiser tests to see if any possible instruction passes a set
of conformity tests. If not, the program length is increased by a
single instruction and the process repeats. The exponential growth in
the size of the search space is far too expensive for all but the
smallest of hot paths, typically less than 13 instructions. The
optimiser is limited to register to register memory transfers, with no
support for pointers, a limitation which is addressed
in~\cite{Joshi2002}. Denali is a super-optimiser which uses constraint
satisfaction and rewrite rules to generate programs which are
\emph{provably} optimal, instead of searching for the optimal
configuration through empirical testing. Denali first translates a low
level machine code into guarded multi-assignment form, then uses a
matching algorithm to build a graph of all of a set of logical axioms
which match parts of the graph, before using boolean satisfiability to
disprove the conjecture that a program cannot be written in $n$
instructions. If the conjecture cannot be disproved, the size of $n$
is increased and the process repeats.


% \section{Performance Tuning for Heterogeneous Parallelism}

As briefly discussed in Section~\ref{sec:gpgpu}, the complex
interactions between optimisations and heterogeneous hardware makes
performance tuning for heterogeneous parallelism a difficult
task. Performant GPGPU programs require careful attention from the
developer to properly manage data layout in DRAM, caching, diverging
control flow, and thread communication. The performance of programs
depends heavily on fully utilising zero-overhead thread scheduling,
memory bandwidth, and thread grouping. \citeauthor{Ryoo2008a}
illustrate the importance of these factors by demonstrating speedups
of up to $432\times$ for matrix multiplication in CUDA by appropriate
use of tiling and loop unrolling~\cite{Ryoo2008a}. The importance of
proper exploitation of local shared memory and synchronisation costs
is explored in~\cite{Lee2010}.

In~\cite{Chen2014}, data locality optimisations are automated using a
description of the hardware and a memory-placement-agnostic
compiler. The authors demonstrate impressive speedups of up to
$2.08\times$, although at the cost of requiring accurate memory
hierarchy descriptor files for all targeted hardware. The descriptor
files must be hand generated, requiring expert knowledge of the
underlying hardware in order to properly exploit memory locality.

Data locality for nested parallel patterns is explored in~\cite{Lee}.
The authors use an automatic mapping strategy for nested parallel
skeletons on GPUs, which uses a custom intermediate representation and
a CUDA code generator, achieving $1.24\times$ speedup over hand
optimised code on 7 of 8 Rodinia benchmarks.

Reduction of the GPGPU optimisation space is demonstrated
in~\cite{Ryoo2008}, using the common subset of optimal configurations
across a set of training examples. This technique reduces the search
space by 98\%, although it does not guarantee that for a new program,
the reduced search space will include the optimal configuration.

\citeauthor{Magni2014} demonstrated that thread coarsening of OpenCL
kernels can lead to speedups in program performance between
$1.11\times$ and $1.33\times$ in~\cite{Magni2014}. The authors achieve
this using a machine learning model to predict optimal thread
coarsening factors based on the static features of kernels, and an
LLVM function pass to perform the required code transformations.

A framework for the automatic generation of OpenCL kernels from
high-level programming concepts is described in~\cite{Steuwer2015}. A
set of rewrite rules is used to transform high-level expressions to
OpenCL code, creating a space of possible implementations. This
approach is ideologically similar to that of PetaBricks, in that
optimisations are made through algorithmic choice, although in this
case the transformations are performed automatically at the compiler
level. The authors report performance on a par with that of hand
written OpenCL kernels.


% \section{Autotuning Algorithmic Skeletons}

An enumeration of the optimisation space of Intel Thread Building
Blocks in~\cite{Contreras2008} shows that runtime knowledge of the
available parallel hardware can have a significant impact on program
performance. \citeauthor{Collins2012} exploit this
in~\cite{Collins2012}, first using Principle Components Analysis to
reduce the dimensionality of the space of possible optimisation
parameters, followed by a search of parameter values to optimise
program performance by a factor of $1.6\times$ over values chosen by a
human expert. In~\cite{Collins2013}, they extend this using static
feature extraction and nearest neighbour classification to further
prune the search space, achieving an average 89\% of the oracle
performance after evaluating 45 parameters.

\citeauthor{Dastgeer2011} developed a machine learning based auto-tuner
for the SkePU skeleton library in~\cite{Dastgeer2011}. Training data
is used to predict the optimal execution device (i.e.\ CPU, GPU) for a
given program by predicting execution time and memory copy overhead
based on problem size. The auto-tuner only supports vector operations,
and there is limited cross-architecture
evaluation. In~\cite{Dastgeer2015a}, the authors extend SkePU to
improve the data consistency and transfer overhead of container types,
reporting up to a $33.4\times$ speedup over the previous
implementation.


% \section{Code Generation and Autotuning for Stencils}

Stencil codes have a variety of computationally expensive uses from
fluid dynamics to quantum mechanics. Efficient, tuned stencil kernels
are highly sought after, with early work in \citeyear{Bolz2003} by
\citeauthor{Bolz2003} demonstrating the capability of GPUs for
massively parallel stencil operations~\cite{Bolz2003}. In the
resulting years, stencil codes have received much attention from the
performance tuning research community.

\citeauthor{Ganapathi2009} demonstrated early attempts at autotuning
multi-core stencil codes in~\cite{Ganapathi2009}, drawing upon the
successes of statistical machine learning techniques in the compiler
community, as discussed in
Section~\ref{sec:iterative-compilation}. They present an auto-tuner
which can achieve performance up to 18\% better than that of a human
expert. From a space of 10 million configurations, they evaluate the
performance of a randomly selected 1500 combinations, using Kernel
Canonical Correlation Analysis to build correlations between tunable
parameter values and measured performance targets. Performance targets
are L1 cache misses, TLB misses, cycles per thread, and power
consumption. The use of KCAA restricts the scalability of their system
as the complexity of model building grows exponentially with the
number of features. In their evaluation, the system requires two hours
of compute time to build the KCAA model for only 400 seconds of
benchmark data. They present a compelling argument for the use of
energy efficiency as an optimisation target in addition to runtime,
citing that it was the power wall that lead to the multi-core
revolution in the first place. Their choice of only 2 benchmarks and 2
platforms makes the evaluation of their auto-tuner somewhat limited.

\citeauthor{Berkeley2009} targeted 3D stencils code performance
in~\cite{Berkeley2009}. Stencils are decomposed into core blocks,
sufficiently small to avoid last level cache capacity misses. These
are then further decomposed to thread blocks, designed to exploit
common locality threads may have within a shared cache or local
memory. Thread blocks are divided into register blocks in order to
take advantage of data level parallelism provided by the available
registers. Data allocation is optimised on NUMA systems. The
performance evaluation considers speedups of various optimisations
with and without consideration for host/device transfer overhead.

\citeauthor{Kamil2010} present an autotuning framework
in~\cite{Kamil2010} which accepts as input a Fortran 95 stencil
expression and generates tuned shared-memory parallel implementations
in Fortan, C, or CUDA. The system uses an IR to explore autotuning
transformations, enumerating a subset of the optimisation space and
recording only a single execution time for each configuration,
reporting the fastest. They demonstrate their system on 4
architectures using 3 benchmarks, with speedups of up to $22\times$
compared to serial implementations. The CUDA code generator does not
optimise for the GPU memory hierarchy, using only global memory. As
demonstrated in this thesis, improper utilisation of local memory can
hinder program performance by two orders of magnitude. There is no
directed search or cross-program learning.

In~\cite{Zhang2013a}, \citeauthor{Zhang2013a} present a code generator
and auto-tuner for 3D Jacobi stencil codes. Using a DSL to express
kernel functions, the code generator performs substitution from one of
two CUDA templates to create programs for execution on GPUs. GPU
programs are parameterised and tuned for block size, block dimensions,
and whether input data is stored in read only texture memory. This
creates an optimisation space of up to 200 configurations. In an
evaluation of 4 benchmarks, the authors report impressive performance
that is comparable with previous implementations of iterative Jacobi
stencils on GPUs~\cite{Holewinski2012, Phillips2010}. The dominating
parameter is shown to be block dimensions, followed by block size,
then read only memory. The DSL presented in the paper is limited to
expressing only Jacobi Stencils applications. Critically, their
auto-tuner requires a full enumeration of the parameter space for each
program. Since there is no indication of the compute time required to
gather this data, it gives the impression that the system would be
impractical for the needs of general purpose stencil computing. The
auto-tuner presented in this thesis overcomes this drawback by learning
parameter values which transfer to unseen stencils, without the need
for an expensive tuning phase for each program and architecture.

In~\cite{Christen2011}, \citeauthor{Christen2011} present a DSL for
expressing stencil codes, a C code generator, and an auto-tuner for
exploring the optimisation space of blocking and vectorisation
strategies. The DSL supports stencil operations on arbitrarily
high-dimensional grids. The auto-tuner performs either an exhaustive,
multi-run Powell search, Nelder Mead, or evolutionary search to find
optimal parameter values. They evaluate their system on two CPUS and
one GPU using 6 benchmarks. A comparison of tuning results between
different GPU architectures would have been welcome, as the results
presented in this thesis show that devices have different responses to
optimisation parameters. The authors do not present a ratio of the
available performance that their system achieves, or how the
performance of optimisations vary across benchmarks or devices.

A stencil grid can be decomposed into smaller subsections so that
multiple GPUs can operate on each subsection independently. This
requires a small overlapping region where each subsection meets ---
the halo region --- to be shared between devices. For iterative
stencils, values in the halo region must be synchronised between
devices after each iteration, leading to costly communication
overheads. One possible optimisation is to deliberately increase the
size of the halo region, allowing each device to compute updated
values for the halo region, instead of requiring a synchronisation of
shared state. This reduces the communication costs between GPUs, at
the expense of introducing redundant computation. Tuning the size of
this halo region is the goal of PARTANS~\cite{Lutz2013}, an autotuning
framework for multi-GPU stencil computations. \citeauthor{Lutz2013}
explore the effect of varying the size of the halo regions using six
benchmark applications, finding that the optimal halo size depends on
the size of the grid, the number of partitions, and the connection
mechanism (i.e.\ PCI express). The authors present an auto-tuner which
determines problem decomposition and swapping strategy offline, and
performs an online search for the optimal halo size. The selection of
overlapping halo region size compliments the selection of work-group
size which is the subject of this thesis. However, PARTANS uses a
custom DSL rather than the generic interface provided by SkelCL, and
PARTANS does not learn the results of tuning across programs, or
across multiple runs of the same program.

% \section{Compilers}

Using active learning to minimise the cost of iterative compilation~\cite{Ogilvie2017}.
Continuous learning of compiler heuristics~\cite{Tartara2013,Fursin2011}.

Section~\ref{subsec:training-with-synthetic-benchmarks} describes training with synthetic benchmarks.

\subsection{Feature Selection}

\todo[inline]{Automatic feature \emph{selection} has been done before. The technique proposed in this thesis is for feature \emph{generation}, i.e. extracting a novel representation from text.}

Machine learning has emerged as a viable means in automatically constructing heuristics for code optimisation~\cite{Stephenson2005,Agakov,Wang2010,Kulkarni2012,Ding2015,Muralidharan2016}. Its great advantage is that it can adapt to changing hardware platforms as it has no a priori assumptions about their behaviour. The success of machine learning based code optimisation has required having a set of high-quality features that can capture the important characteristics of the target program. Given that there is an infinite number of these potential features, finding the right set of features is a non-trivial, time-consuming task.

Various forms of program features have been used in compiler-based machine learning. These include static code structures~\cite{Jiang2010} and runtime information such as system load~\cite{Wen2015} and performance counters~\cite{Dubach2009}. In compiler research, the feature sets used for predictive models are often provided without explanation and rarely is the quality of those features evaluated. More commonly, an initial large, high dimensional candidate feature space is pruned via feature selection~\cite{Stephenson2005}, or projected into a lower dimensional space~\cite{Collins2013,Dubach2007}. FEAST employs a range of existing feature selection methods to select useful candidate features~\cite{Ting2016}. Unlike these approaches, DeepTune extracts features and reduces the dimensionality of the feature space completely internally and without expert guidance.

Park \emph{et al.} present a unique graph-based approach for feature representations~\cite{Park2012}. They use a Support Vector Machine where the kernel is based on graph similarity metric. Their technique still requires hand coded features at the basic block level, but thereafter, graph similarity against each of the training programs takes the place of global features. Being a kernel method, it requires that training data graphs be shipped with the compiler, which may not scale as the size of the training data grows with the number of instances, and some training programs may be very large. Finally, their graph matching metric is expensive, requiring $O(n^3)$ to compare against each training example. By contrast, our method does not need any hand built static code features, and the deployment memory footprint is constant and prediction time is linear in the length of the program, regardless of the size of the training set.

A few methods have been proposed to automatically generate features from the compiler's intermediate representation~\cite{Namolaru2010a,Leather2014}. These approaches closely tie the implementation of the predictive model to the compiler IR, which means changes to the IR will require modifications to the model. The work of \cite{Leather2014} uses genetic programming to search for features, and required a huge grammar to be written, some 160kB in length. Although much of this can be created from templates, selecting the right range of capabilities and search space bias is non trivial and up to the expert. The work of \cite{Namolaru2010a} expresses the space of features via logic programming over relations that represent information from the IRs. It greedily searches for expressions that represent good features. However, their approach relies on expert selected relations, combinators and constraints to work. For both approaches, the search time may be significant.

Cavazos \emph{et al.\ }present a reaction-based predictive model for software-hardware co-design~\cite{Cavazos2006}. Their approach profiles the target program using several carefully selected compiler options to see how program runtime changes under these options for a given micro-architecture setting. They then use the program ``reactions'' to predict the best available application speedup. While their approach does not use static code features, developers must carefully select a few settings from a large number of candidate options for profiling, because poorly chosen options can significantly affect the quality of the model. Moreover, the program must be run several times before optimisation, while our technique does not require the program to be profiled.


% Moren, K., & Gohringer, D. (2018). Automatic Mapping for OpenCL-Programs on CPU/GPU Heterogeneous Platforms. In ICCS. https://doi.org/10.1007/978-3-319-93701-4
\todo[inline]{CPU/GPU mapping~\cite{Shi2018}.}