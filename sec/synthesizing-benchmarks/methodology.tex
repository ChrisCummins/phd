\section{Experimental Methodology}
\label{sec:clgen-eval-methodology}

This section describes the methodology for a quantitative evaluation of CLgen-generated benchmarks.

\subsection{Experimental Setup}

\paragraph*{Predictive Model}

To evaluate the efficacy of synthetic benchmarks for training, the predictive model of \citeauthor{Grewe2013} is used~\cite{Grewe2013}. The predictive model is used to determine the optimal mapping of a given OpenCL kernel to either a GPU or CPU. It uses supervised learning to construct a decision tree with a combination of static and dynamic kernel features extracted from source code and the OpenCL runtime, detailed in Table~\ref{tab:clgen-cgo13-features}.

\begin{table}
	\centering%
	\subfloat[Individual code features]{%
  \rowcolors{2}{gray!25}{white}
  \begin{tabular}{| l c l |}
    \hline
    \rowcolor{gray!50}
    \textbf{Name} & \textbf{Type} & \textbf{Description} \\
    \hline
    \texttt{comp} & static & \#. compute operations \\
    \texttt{mem} & static & \#. accesses to global memory \\
    \texttt{localmem} & static & \#. accesses to local memory \\
    \texttt{coalesced} & static & \#. coalesced memory accesses \\
    \texttt{transfer} & dynamic & size of data transfers \\
    \texttt{wgsize} & dynamic & \#. work-items per kernel \\
    \hline
  \end{tabular}%
  \label{tab:clgen-cgo13-features-raw}%
}\\*
\subfloat[Combinations of raw features]{%
  \rowcolors{2}{gray!25}{white}
  \begin{tabular}{| l l l |}
    \hline
    \rowcolor{gray!50}
    \textbf{Name} & \textbf{Formulation} & \textbf{Description} \\
    \hline
    \texttt{F1} & \texttt{transfer/(comp+mem)} & Communication-computation ratio \\
    \texttt{F2} & \texttt{coalesced/mem} & \% Coalesced memory accesses \\
    \texttt{F3} & \texttt{(localmem/mem)$\times$wgsize} & Memory access ratio $\times$ \#.\ work-items \\
    \texttt{F4} & \texttt{comp/mem} & Computation-memory ratio\\
    \hline
  \end{tabular}%
  \label{tab:clgen-cgo13-features-combined}%
}
%
	\caption[\citeauthor{Grewe2013} features for heterogeneous device mapping]{%
		Features used by \citeauthor{Grewe2013} to predict CPU/GPU mapping of OpenCL kernels. The features are extracted using a custom analysis pass based using LLVM.
	}%
	\label{tab:clgen-cgo13-features} %
\end{table}

\paragraph*{Benchmarks}

As in~\cite{Grewe2013}, the model is tested on the NAS Parallel Benchmarks (NPB)~\cite{Bailey1991a}. The hand-optimised OpenCL implementation of \citeauthor{Seo2011}~\cite{Seo2011} is used. In~\cite{Grewe2013} the authors augment the training set of the predictive model with 47 additional kernels taken from 4 GPGPU benchmark suites. To more fully sample the program space, a much larger collection of 142 kernels is used, summarised in Table~\ref{tab:cgo17-benchmarks}. These additional programs are taken from all 7 of the most frequently used benchmark suites identified in Section~\ref{sec:the-case-for-benchmark-generators}. None of these programs were used to train CLgen. This brings the total number of OpenCL benchmark kernels used in the evaluation to 256.

1,000 kernels were synthesised with CLgen to use as additional benchmarks.

\begin{table}% tab:benchmarks
  \centering%
  \rowcolors{2}{gray!25}{white}
\begin{tabular}{| l r r r |}
  \hline
  \rowcolor{gray!50}
  & \textbf{Version} & \textbf{\#. benchmarks} & \textbf{\#. kernels}\\
  \hline
  \textbf{NPB (SNU~\cite{Seo2011})} & 1.0.3 & 7 & 114 \\
  \textbf{Rodinia~\cite{Che2009}} & 3.1 & 14 & 31 \\
  \textbf{NVIDIA SDK} & 4.2 & 6 & 12 \\
  \textbf{AMD SDK} & 3.0 & 12 & 16 \\
  \textbf{Parboil~\cite{Stratton2012}} & 0.2 & 6 & 8 \\
  \textbf{PolyBench~\cite{Grauer-Gray2012}} & 1.0 & 14 & 27 \\
  \textbf{SHOC~\cite{Danalis2010}} & 1.1.5 & 12 & 48 \\
  \textbf{Total} & - & 71 & 256 \\
  \hline
\end{tabular}

  \caption[Benchmarks used in evaluation]{List of benchmarks} %
  \label{tab:cgo17-benchmarks} %
\end{table}

\paragraph*{Experimental Platforms}

Two 64-bit CPU-GPU systems are used to evaluate the approach, detailed in Table~\ref{tab:cgo17-platforms}. One system has an AMD GPU and uses OpenSUSE 12.3; the other is equipped with an NVIDIA GPU and uses Ubuntu 16.04. Both platforms were unloaded.

\begin{table}% tab:platforms
  \centering %
	\rowcolors{2}{gray!25}{white}
\begin{tabular}{| l l l l |}
  \hline
  \rowcolor{gray!50}
  & \textbf{Intel CPU} & \textbf{AMD GPU} & \textbf{NVIDIA GPU} \\
  \hline
  \textbf{Model} & Core i7-3820 & Tahiti 7970 & GTX 970 \\
  \textbf{Frequency} & 3.6 GHz & 1000 MHz & 1050 MHz \\
  \textbf{\#. Cores} & 4 & 2048 & 1664 \\
  \textbf{Memory} & 8 GB & 3 GB & 4 GB \\
  \textbf{Throughput} & 105 GFLOPS & 3.79 TFLOPS & 3.90 TFLOPS \\
  \textbf{Driver} & AMD 1526.3 & AMD 1526.3 & NVIDIA 361.42 \\
  \textbf{Compiler} & GCC 4.7.2 & GCC 4.7.2 & GCC 5.4.0 \\
  \hline
\end{tabular}

  \caption[Experimental platforms used in evaluation]{Experimental platforms.}
  \label{tab:cgo17-platforms}
\end{table}

\paragraph*{Data sets}

The NPB and Parboil benchmark suites are packaged with multiple data sets. We use all of the packaged data sets (5 per program in NPB, 1-4 per program in Parboil). For all other benchmarks, the default data sets are used. The CLgen host driver was configured to synthesise payloads between 128B-130MB, approximating that of the dataset sizes found in the benchmark programs.


\subsection{Methodology}

The same methodology is used as in~\cite{Grewe2013}. Each experiment is repeated five times and the average execution time is recorded. The execution time includes both device compute time and the data transfer overheads.

Models are evaluated using \emph{leave-one-out cross-validation}. For each benchmark, a model is trained on data from all other benchmarks and used to predict the mapping for each kernel and dataset in the excluded program. The process is repeated with and without the addition of synthetic benchmarks in the training data. Only the real handwritten benchmarks are used to test model predictions, the synthetic benchmarks are not used.
