\section{CLdrive: A System for Driving Arbitrary OpenCL Kernels}
\label{sec:cldrive}

This section presents \emph{CLdrive}, a host driver developed to gather performance data from synthesised CLgen code. The driver accepts as input an OpenCL kernel, generates \emph{payloads} of user-configurable sizes, then executes the kernel using the generated payloads so as to collect kernel runtimes, and to provide dynamic checking of kernel behaviour.

\subsection{Generating Data Payloads}

A \emph{payload} encapsulates all of the arguments of an OpenCL compute kernel. After parsing the input kernel to derive argument types, a rule-based approach is used to generate synthetic payloads. For a given global size $S_g$: host buffers of $S_g$ elements are allocated and populated with random values for global pointer arguments, device-only buffers of $S_g$ elements are allocated for local pointer arguments, integral arguments are given the value $S_g$, and all other scalar arguments are given random values. Host to device data transfers are enqueued for all non-write-only global buffers, and all non-read-only global buffers are transferred back to the host after kernel execution.

\subsection{Dynamic Checker}

A class of programs are defined as performing \emph{useful work} if they predictably compute some result. For the purpose of performance benchmarking the values computed by a program are of little interest, so a low-overhead runtime behaviour check is used to validate that a synthesised program does useful work based on the outcome of four executions of a tested program:

\begin{enumerate}
\item Create 4 equal size payloads $A_{1in}$, $B_{1in}$, $A_{2in}$,
  $B_{2in}$, subject to restrictions: $A_{1in}=A_{2in}$,
  $B_{1in}=B_{2in}$, $A_{1in} \ne B_{1in}$.
\item Execute kernel $k$ 4 times: $k(A_{1in}) \rightarrow A_{1out}$,
  $k(B_{1in}) \rightarrow B_{1out}$,
  $k(A_{2in}) \rightarrow A_{2out}$,
  $k(B_{2in}) \rightarrow B_{2out}$.
\item Assert that:
  \begin{itemize}
  \item $A_{1out} \ne A_{1in}$ and $B_{1out} \ne B_{1in}$, else $k$ has no
  output (for these inputs).%
  \item $A_{1out} \ne B_{1out}$ and $A_{2out} \ne B_{2out}$, else $k$ is input insensitive t (for these inputs).%
  \item $A_{1out}=A_{2out}$ and $B_{1out}=B_{2out}$, else $k$ is
  non-deterministic.
  \end{itemize}
\end{enumerate}

Equality checks for floating point values are performed with an appropriate epsilon to accommodate rounding errors, and a timeout threshold is also used to catch kernels which are non-terminating. The method is based on random differential testing~\cite{McKeeman1998}, though I emphasise that this is not a general purpose approach and is tailored specifically for benchmarking for performance characterisation. For example, one would anticipate a false positive rate for kernels with subtle sources of non-determinism which more thorough methods may expose~\cite{Betts2012,Price2015,Sorensen2016}, however for the purpose of performance modelling, such methods were deemed unnecessary.
