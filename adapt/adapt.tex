% 2-8 pages, 10pt font
%
% Topics:
%
% * Machine learning based autotuning.
% * Representative benchmarking.
% * Automatic fault tolerance.
% * Run-time adaption.
%

% The following \documentclass options may be useful:
%
% preprint      Remove this option only once the paper is in final form.
% 10pt          To set in 10-point type instead of 9-point.
% 11pt          To set in 11-point type instead of 9-point.
% authoryear    To obtain author/year citation style instead of numeric.
\documentclass[preprint,nonatbib,10pt]{sigplanconf}

\documentclass[prodmode,acmtaco]{acmsmall}

% Package to generate and customize Algorithm as per ACM style
\usepackage[ruled]{algorithm2e}
\renewcommand{\algorithmcfname}{ALGORITHM}
\SetAlFnt{\small}
\SetAlCapFnt{\small}
\SetAlCapNameFnt{\small}
\SetAlCapHSkip{0pt}
\IncMargin{-\parindent}

% Metadata Information
\acmVolume{9}
\acmNumber{4}
\acmArticle{39}
\acmYear{2016}
\acmMonth{3}

% Copyright
%\setcopyright{acmcopyright}
\setcopyright{acmlicensed}
%\setcopyright{rightsretained}
%\setcopyright{usgov}
%\setcopyright{usgovmixed}
%\setcopyright{cagov}
%\setcopyright{cagovmixed}

% DOI
\doi{0000001.0000001}

%ISSN
\issn{1234-56789}


%%%%%%%%%%%%%%%%%%%%%%%%%
%% Document and Layout %%
%%%%%%%%%%%%%%%%%%%%%%%%%

% Fix for multiple "No room for a new \dimen" errors.
%
% See: http://tex.stackexchange.com/questions/38607/no-room-for-a-new-dimen
%
\usepackage{etex}

\usepackage[utf8]{inputenc}

% Fix for "'babel/polyglossia' detected but 'csquotes' missing"
% warning. NOTE: Include after inputenc.
%
\usepackage{csquotes}

% Make internal macro definitions accessible,
% e.g. \@title, \@date \@author.
\makeatletter

% Multi-column support.
\usepackage{multicol}

% A useful package which includes macros like \ifdef{}{}{}:
%
\usepackage{etoolbox}

% Uncomment the following line to remove column separation:
%
%\setlength{\columnsep}{5mm}

% Allow user-defined warning and error filters.
%
\usepackage{silence}

\usepackage{adjustbox}


%%%%%%%%%%%%%%%%%%%%%
% Table of Contents %
%%%%%%%%%%%%%%%%%%%%%

% % Set chapter and section numbering depth:
% %
% \setcounter{secnumdepth}{2}


%%%%%%%%%%%%%%%%
% Bibliography %
%%%%%%%%%%%%%%%%
% \usepackage[%
%     backend=biber,
%     style=ieee,
%     % style=numeric-comp,
%     % style=numeric-comp,  % numerical-compressed
%     sorting=none,        % nty,nyt,nyvt,anyt,anyvt,ynt,ydnt,none
%     sortcites=true,      % sort \cite{b a d c}: true,false
%     block=none,          % space between blocks: none,space,par,nbpar,ragged
%     indexing=false,      % indexing options: true,false,cite,bib
%     citereset=none,      % don't reset cites
%     isbn=false,          % print ISBN?
%     url=true,            % print URL?
%     doi=false,           % print DOI?
%     natbib=true,         % natbib compatability
%    ]{biblatex}

% \usepackage{natbib}

% % Filter annoying and unavoidable biblatex warning:
% \WarningFilter{biblatex}{Patching footnotes failed}

% Reduce the font size of the bibliography:
% \renewcommand{\bibfont}{\normalfont\scriptsize}

% Determine which BibTeX file to use:
%
% If available, use my Mendeley BibTex library, located in the home
% directory. Note that this is a relative path and will break if
% either this file or the BibTex library are moved. If the library is
% not present, use the local refs.bib file.
% \newcommand{\BibResourceGlobal}{../../../library.bib}
% \newcommand{\BibResourceLocal}{refs.bib}

% \IfFileExists{\BibResourceGlobal}
%   {\newcommand{\BibResource}{\BibResourceGlobal}}
%   {\newcommand{\BibResource}{\BibResourceLocal}}

% \addbibresource{\BibResource}


%%%%%%%%%%%%%%
% Appendices %
%%%%%%%%%%%%%%

% Appendix package. Documentation:
%
%  http://mirror.ox.ac.uk/sites/ctan.org/macros/latex/contrib/appendix/appendix.pdf
%
% Package options:
%
% toc      - Put a header (e.g., `Appendices') into the Table of Contents
%            (the ToC) before listing the appendices. (This is done by
%            calling the \addappheadtotoc command.)
% page     - Puts a title (e.g., `Appendices') into the document at the
%            point where the appendices environment is begun. (This is
%            done by calling the \appendixpage command.)
% title    - Adds a name (e.g., `Appendix') before each appendix title in
%            the body of the document. The name is given by the value
%            of \appendixname. Note that this is the default behaviour
%            for classes that have chapters.
% titletoc - Adds a name (e.g., `Appendix') before each appendix listed
%            in the ToC. The name is given by the value
%            of \appendixname.
% header   - Adds a name (e.g., `Appendix') before each appendix in page
%            headers.  The name is given by the value
%            of \appendixname. Note that this is the default behaviour
%            for classes that have chapters.
\usepackage[title, titletoc]{appendix}

% pre-requisites for rendering upquotes in listings package.
\usepackage[T1]{fontenc}
\usepackage{lmodern}
\usepackage{textcomp}

% code listings.
\usepackage{listings}

% set \ttfamily to use courier fonts.
%
% See: http://tex.stackexchange.com/a/33686
%
\usepackage{courier}

\lstset{frame=bt,                    % Add top and bottom frame lines
breaklines=true,             % Force line wrapping
captionpos=b,                % Place caption below listing
numbers=left,                % Add left-side line numbers
basicstyle=\scriptsize\ttfamily, % Set font size and type
showstringspaces=false,      % Don't show visible whitespace
numberstyle=\tiny,
upquote=true,                % Use upright quotes, not curly
commentstyle=\bfseries}      % Embolden comments

% Use (*@ @*) to escape LaTeX commands within listings.
\lstset{escapeinside={(*@}{@*)}}

% Add 10pt space between chapters in TOC listings entries:
%\let\Chapter\chapter
%\def\chapter{\addtocontents{lol}{\protect\addvspace{10pt}}\Chapter}


%%%%%%%%%%%%%%%%%%%%%%%%
%% Graphics and maths %%
%%%%%%%%%%%%%%%%%%%%%%%%
\usepackage{amsmath}

% Vector notation, e.g. \vv{x}:
%
\usepackage{esvect}

% Additional amsmath symbols, see:
%
% http://texblog.org/2007/08/27/number-sets-prime-natural-integer-rational-real-and-complex-in-latex/
%
\usepackage{amsfonts}
\usepackage{amssymb}

\usepackage{graphicx}
\usepackage{mathtools}
\usepackage{tikz}
\usepackage{tikz-qtree}

% Provide bold font face in maths.
\usepackage{bm}

\usepackage{subcaption}
\expandafter\def\csname ver@subfig.sty\endcsname{}

% Define an 'myalignat' command which behave as 'alignat' without the
% vertical top and bottom padding. See:
%     http://www.latex-community.org/forum/viewtopic.php?f=5&t=1890
\newenvironment{myalignat}[1]{%
\setlength{\abovedisplayskip}{-.7\baselineskip}%
\setlength{\abovedisplayshortskip}{\abovedisplayskip}%
\start@align\z@\st@rredtrue#1
}%
{\endalign}

% Define additional operators:
\DeclareMathOperator*{\argmin}{arg\,min}
\DeclareMathOperator*{\argmax}{arg\,max}

\DeclareMathOperator*{\gain}{Gain}

% Skeleton operators.
\DeclareMathOperator*{\map}{Map}
\DeclareMathOperator*{\reduce}{Reduce}
\DeclareMathOperator*{\scan}{Scan}
\DeclareMathOperator*{\stencil}{Stencil}
\DeclareMathOperator*{\zip}{Zip}
\DeclareMathOperator*{\allpairs}{All\,Pairs}

% Maths plots using pgfplots, see:
%
%     http://pgfplots.sourceforge.net/pgfplots.pdf
%
\usepackage{pgfplots}

% Disable compatability mode.
%
\pgfplotsset{compat=1.12}

% Gantt charts using pgfgantt, see:
%
%     http://www.ctan.org/pkg/pgfgantt
%
\usepackage{pgfgantt}

% Fix milestone aspect ratio by defining a custom element.
\newganttchartelement*{mymilestone}{
mymilestone/.style={
shape=diamond,
inner sep=2pt,
draw=black,
top color=black,
bottom color=black,
}
}

% Tikz flowchart configuration.
\usetikzlibrary{shapes,arrows,shadows,fit,backgrounds}
\tikzstyle{decision} = [diamond,
draw,
text width=4.5em,
text badly centered,
node distance=3cm,
inner sep=0pt]
\tikzstyle{block}    = [rectangle,
draw,
text width=5em,
text centered,
node distance=3cm,
minimum height=4em,
inner sep=.2cm]
\tikzstyle{line}     = [draw, -latex']

% Add dirtree picture style, see:
%
%     http://tex.stackexchange.com/a/34268
%
\newcount\dirtree@lvl
\newcount\dirtree@plvl
\newcount\dirtree@clvl
\def\dirtree@growth{%
\ifnum\tikznumberofcurrentchild=1\relax
\global\advance\dirtree@plvl by 1
\expandafter\xdef\csname dirtree@p@\the\dirtree@plvl\endcsname{\the\dirtree@lvl}
\fi
\global\advance\dirtree@lvl by 1\relax
\dirtree@clvl=\dirtree@lvl
\advance\dirtree@clvl by -\csname dirtree@p@\the\dirtree@plvl\endcsname
\pgf@xa=0.33cm\relax
\pgf@ya=-\baselineskip\relax
\pgf@ya=\dirtree@clvl\pgf@ya
\pgftransformshift{\pgfqpoint{\the\pgf@xa}{\the\pgf@ya}}%
\ifnum\tikznumberofcurrentchild=\tikznumberofchildren
\global\advance\dirtree@plvl by -1
\fi
}
\tikzset{
dirtree/.style={
growth function=\dirtree@growth,
every node/.style={anchor=north},
every child node/.style={anchor=west},
edge from parent path={(\tikzparentnode\tikzparentanchor) |- (\tikzchildnode\tikzchildanchor)}
}
}

% UML sequence diagram macros, see:
%
%     https://code.google.com/p/pgf-umlsd/
%
% Options:
%
%     underline - Underline object names
%
\usepackage[underline=false]{pgf-umlsd}

% Support for SVG graphics.
%
% NOTE that you must pass the "--shell-escape" argument to pdflatex to
% compile. NOTE also that images *MUST* be placed within the graphics
% path.
\usepackage{svg}
\graphicspath{{img/}}

%%%%%%%%%%%%%%%%%%%%%%
%% Tables and lists %%
%%%%%%%%%%%%%%%%%%%%%%

% Required to use labm8 exported tables.
%
\usepackage{booktabs}

% Required for full page-width tables.
\usepackage{tabularx}

%\usepackage{enumitem}
%\setenumerate{itemsep=0pt}

% Use no left margin for lists:
%\setlist{leftmargin=*}

\usepackage{longtable}

% Define column types L, C, R with known text justification and fixed
% widths:
\usepackage{array}
\newcolumntype{L}[1]{>{\raggedright\let\newline\\\arraybackslash\hspace{0pt}}m{#1}}
\newcolumntype{C}[1]{>{\centering\let\newline\\\arraybackslash\hspace{0pt}}m{#1}}
\newcolumntype{R}[1]{>{\raggedleft\let\newline\\\arraybackslash\hspace{0pt}}m{#1}}


%%%%%%%%%%%%%%%%%%%%%%%%%%%%%
%% Typesetting and symbols %%
%%%%%%%%%%%%%%%%%%%%%%%%%%%%%

% Adjustable font sizes in \Verbatim{}
\usepackage{fancyvrb}

%\usepackage{titlesec}
% Set section and paragraph heading fonts:
%\titleformat*{\section}{\Large\bfseries}
%\titleformat*{\subsection}{\normalsize\bfseries}
%\titleformat*{\subsubsection}{\normalsize}
%\titleformat*{\paragraph}{\large\bfseries}
%\titleformat*{\subparagraph}{\large\bfseries}

% Set section heading margins. Usage:
% \titlespacing*{<command>}{<left>}{<before>}{<after>}
%\titlespacing*{\section}{0pt}{.6em}{.3em}
%\titlespacing*{\subsection}{0pt}{.6em}{.2em}

% Set paragraph indentation size. Default is 15pt.
%\setlength{\parindent}{10pt}

% The line spacing can be globally set using \linespread:
%
% \linespread{1.2}

% Add a command \hr{} which will draw a horizontal rule the width of
% the text.
%
\newcommand{\hr}{\noindent\makebox[\linewidth]{\rule{\textwidth}{0.2pt}}}

% Add a command \br{} which will create a horizontal space of exactly
% one line height.
%
\newcommand{\br}{\hspace{\baselineskip}}

% Define a command to allow word breaking.
\newcommand*\wrapletters[1]{\wr@pletters#1\@nil}
\def\wr@pletters#1#2\@nil{#1\allowbreak\if&#2&\else\wr@pletters#2\@nil\fi}

% Define a command to create centred page titles.
\newcommand{\centredtitle}[1]{
\begin{center}
  \large
  \vspace{0.9cm}
  \textbf{#1}
\end{center}}

% Support hyperlinks using the \hyperref, \url and \href
% macros. Usage:
%
%    \hyperref[label_name]{''link text''}
%
%    \url{<my_url>}
%
%    \href{<my_url>}{<description>}
%
\usepackage{hyperref}

% Disable colored borders of links, cross-references etc in PDF output
\hypersetup{pdfborder={0 0 0}}

% Provide generic commands \degree, \celsius, \perthousand, \micro
% and \ohm which work both in text and maths mode.
\usepackage{gensymb}

%%%%%%%%%%%%%%%%%%%%%%%%%%%%%%%%%
%% Placeholder text generation %%
%%%%%%%%%%%%%%%%%%%%%%%%%%%%%%%%%

% Use either \blindtext or \libpsum to generate placeholder text. Also
% note the macros \blinditemize, \blindenumerate, \blinddescription.
\usepackage[english]{babel}
\usepackage{blindtext}
\usepackage{lipsum}


\begin{document}

\special{papersize=8.5in,11in}
\setlength{\pdfpageheight}{\paperheight}
\setlength{\pdfpagewidth}{\paperwidth}

\conferenceinfo{ADAPT '16}{Month d--d, 20yy, City, ST, Country}
\copyrightyear{2016}
\copyrightdata{978-1-nnnn-nnnn-n/yy/mm}
\doi{nnnnnnn.nnnnnnn}

% Uncomment one of the following two, if you are not going for the
% traditional copyright transfer agreement.

%\exclusivelicense                % ACM gets exclusive license to publish,
                                  % you retain copyright

%\permissiontopublish             % ACM gets nonexclusive license to publish
                                  % (paid open-access papers,
                                  % short abstracts)

% \titlebanner{banner above paper title}        % These are ignored unless
\preprintfooter{ADAPT workshop '16}   % 'preprint' option specified.

% \title{Autotuning Stencil Codes using Synthetic Benchmarks}
% \title{Autotuning OpenCL Workgroup Sizes for Stencil Codes}
% \title{Machine learning for OpenCL Workgroup Sizes of Stencil Codes}
% \title{Machine learning for OpenCL Workgroup Sizes of Stencil Codes}
% \title{Methods for Autotuning Workgroup Size of OpenCL Stencil Codes}
\title{Autotuning OpenCL Workgroup Size for Stencil Patterns}

% Comparison of multiple approaches to autotuning stencil patterns:
%
% * Compare classifiers with regressors
% * Compare synthetic vs real training

% \subtitle{Subtitle Text, if any}

\authorinfo{Chris Cummins\and Pavlos Petoumenos\and Hugh Leather}
           {University of Edinburgh}
           {c.cummins@ed.ac.uk,\{ppetoume,hleather\}@inf.ed.ac.uk}

\maketitle

\begin{abstract}
  Selecting the appropriate workgroup size for OpenCL kernels requires
  knowledge of the underlying hardware, the data being operated on,
  and properties of the kernel. This makes portable performance tuning
  a difficult task, and simple heuristics and statically chosen values
  fail to exploit the available performance. To address this, we
  propose the use of machine learning-enabled autotuning to predict
  workgroup sizes for stencil patterns on CPUs and multi-GPUs.

  We present three methodologies for predicting workgroup sizes. The
  first, using classifiers to select the optimal workgroup size. The
  second and third proposed methodologies employ the novel use of
  regressors for performing classification by predicting the runtime
  of kernels and relative performance of different workgroup sizes,
  respectively. We evaluate the effectiveness of each technique in an
  empirical study of 429 combinations of architecture, kernel, and
  dataset, comparing an average of 629 unique workgroup sizes for
  each. We find that auotuning provides a median $3.79\times$ speedup
  over the best possible performance which can be achieved statically,
  achieving 94\% of the available performance.
\end{abstract}

% \category{CR-number}{subcategory}{third-level}

% % general terms are not compulsory anymore,
% % you may leave them out
% \terms
% term1, term2

% \keywords
% keyword1, keyword2

\section{Introduction}\label{sec:introduction}

Stencil codes have a variety of computationally demanding uses from
fluid dynamics to quantum mechanics. Efficient, tuned stencil
implementations are highly sought after, with early work in
\citeyear{Bolz2003} by \citeauthor{Bolz2003} demonstrating the
capability of GPUs for massively parallel stencil
operations~\cite{Bolz2003}. Since then, the introduction of the OpenCL
standard has introduced greater programmability of heterogeneous
devices by providing a vendor-independent layer of abstraction for
data parallel programming of CPUs, GPUs, DSPs, and other
devices~\cite{Stone2010}. However, achieving portable performance of
OpenCL programs is a hard task --- such programs are sensitive to
properties of the underlying hardware, to the program being executed,
and even to the \emph{dataset} that is operated upon. This forces
developers to laboriously hand tune performance on a per-program
basis, since simple heuristics fail to exploit the available
performance. (TODO: reference)

In this paper, we implement machine learning-enabled autotuning for
one such optimisation parameter of OpenCL programs --- that of
workgroup size selection. The 2D optimisation space of OpenCL kernel
workgroup sizes is large, complex and non-linear. Successfully
applying machine learning to such a space requires plentiful training
data, the careful selection of features, and an appropriate
classification approach. The main contributions of this paper are:
%
\begin{itemize}
\item TODO
\end{itemize}

\section{The SkelCL Stencil Pattern}

\begin{figure}
\centering
\includegraphics[width=.75\columnwidth]{img/stencil}
\caption[Stencil border region]{%
  The components of a stencil: an input matrix is decomposed into
  workgroups, consisting of $w_r \times w_c$ elements. Each element is
  mapped to a work-item. Each work-item operates its corresponding
  element and a surrounding border region $S$, consisting of the four
  independent components describing the number of elements north
  $S_n$, east $S_e$, west $S_w$, and south $S_s$ (in this example, 1
  element to the south, and 2 elements in all other directions). Each
  tile is allocated in local memory for fast access of repeated read
  operations.%
}
\label{fig:stencil-shape}
\end{figure}

Introduced in~\cite{Steuwer2011}, SkelCL is an Algorithmic Skeleton
library which provides OpenCL implementations of data parallel
patterns for heterogeneous parallelism using CPUs and
multi-GPUs. SkelCL provides a stencil pattern~\cite{Breuer2014a} in
which a user-provided \emph{customising function} is applied to each
element of a 2D matrix. The value of each element is updated based on
its current value and the value of one or more neighbouring elements,
called the \emph{border region}. The border region is described by a
\emph{stencil shape}, which defines an $i \times j$ rectangular region
about each cell which is used to update the cell value. Stencil shapes
may be asymmetrical, and are defined in terms of the number of cells
in the border region to the north, east, south, and west of each cell,
as shown in Figure~\ref{fig:stencil-shape}. Where elements of a border
region fall outside of the matrix bounds, values are substituted from
either a predefined padding value, or the value of the nearest cell
within the matrix, determined by the user.

When a SkelCL stencil pattern is executed, each of the elements in the
matrix are mapped to OpenCL work-items; and this collection of
work-items is divided into \emph{workgroups} for execution on the
target hardware. A work-item reads the value of its corresponding
matrix element and the surrounding elements defined by the border
region. Since the border regions of neighbouring elements overlap, the
value of all elements within a workgroup are stored in a \emph{tile},
allocated as a contiguous block of local memory. This greatly reduces
the latency of the repeated memory accessed performed by
work-items. Changing the workgroup size thus affects the amount of
local memory required for each workgroup, which in turn affects the
number of workgroups which may be simultaneously active. While the
user defines the size, type, and border region of the matrix being
operated upon, it is the responsibility of the SkelCL stencil
implementation to select an appropriate workgroup size to use.


\section{Autotuning Workgroup Size}

Selecting the appropriate workgroup size for an OpenCL kernel depends
on the properties of the kernel itself, underlying architecture, and
dataset. For a given \emph{scenario} (that is, a combination of
kernel, architecture, and dataset), the goal of this work is to
harness machine learning to \emph{predict} a performant workgroup size
to use, based on some prior knowledge of the performance of workgroup
sizes for other scenarios. In this section, we describe the
optimisation space and the steps required to apply machine
learning. The autotuning algorithms are described in
Section~\ref{sec:ml}.

\subsection{Constraints}

The space of possible workgroup sizes $W$ is constrained by properties
of both the architecture and kernel. Each OpenCL device imposes a
maximum workgroup size which can be statically checked through the
OpenCL Device API. This constraint reflects architectural limitations
of how code is mapped to the underlying execution hardware. Typical
values are powers of two, e.g.\ 1024, 4096, 8192. Additionally,
kernels enforce a maximum workgroup size. This value can be queries at
runtime once a program has been compiled for a specific execution
device. Factors which affect a kernel's maximum workgroup size include
the number of registers required, and the available number of SIMD
execution units for each type of executable instruction.

While in theory, any workgroup size which satisfies the device and
kernel workgroup size constraints should provide a functioning
program, in practise we find that some combinations of scenario and
workgroup size cause an \texttt{CL\_OUT\_OF\_RESOURCES} error to be
thrown when the kernel is enqueued. Note that in many OpenCL
implementations, this error type acts as a generic placeholder and may
not necessarily indicate that the underlying cause of the error was
due to finite resources constraints. Further discussion on the
possible causes and effects of refused parameters is contained in
Section~\ref{sec:results}, but for the purposes of autotuning we
define \emph{refused parameters} as workgroup sizes which satisfy the
kernel and architectural constraints, yet cause a
\texttt{CL\_OUT\_OF\_RESOURCES} error to be thrown when the kernel is
enqueued. We define the space of \emph{legal} workgroup sizes for a
given scenario $s$ as those which satisfy the architectural and kernel
constraints, and are not refused:
%
\begin{equation}
  \footnotesize
  W_{legal}(s) = \left\{w | w \in W, w < W_{\max}(s) \right\} - W_{refused}(s)
\end{equation}
%
Where $W_{\max}(s)$ can be determined at runtime prior to the kernels
execution, but the set $W_{refused}(s)$ can only be discovered
emergently. The set of \emph{safe} parameters are those which are
legal for all scenarios:
%
\begin{equation}
  % \footnotesize
  W_{safe} = \cap \left\{ W_{legal}(s) | s \in S \right\}
\end{equation}


\subsection{Stencil Features}

Since properties of the architecture, program, and dataset all
contribute to the performance of a workgroup size, the success of a
machine learning system depends on the ability to translate these
properties into meaningful explanatory variables ---
\emph{features}. For each scenario, 102 features are extracted
describing the architecture, kernel, and dataset.

Architecture features are extracted using the OpenCL Device API to
query properties such as the size of local memory, maximum work group
size, and number of compute units. Kernel features are extracted from
the source code stencil kernels by compiling first to LLVM IR bitcode,
and using statistics passes to obtain static instruction counts for
each type of instruction present in the kernel, as well as the total
number of instructions. These instruction counts are divided by the
total number of instructions to produce instruction
\emph{densities}. Dataset features include the input and output data
types, and the 2D matrix dimensions.


\subsection{Training Data}\label{subsec:training}

Training data are collected by measuring the runtimes of stencil
programs using different workgroup sizes. These stencil programs are
generated synthetically using a parameterised template substitution
engine. A stencil template is parameterised first by stencil shape
(one parameter for each of the four directions), input and output data
types (either integers, or single or double precision floating
points), and \emph{complexity} --- a simple boolean metric for
indicating the desired number of memory accesses and instructions per
iteration, reflecting the relatively bi-modal nature of stencil codes,
either compute intensive (e.g.\ finite difference time domain and
other PDE solvers), or lightweight (e.g.\ Game of Life and Gaussian
blur).


\section{Machine Learning Methods}\label{sec:ml}

The aim of this work is to design a system which, given a set of prior
observations of the empirical performance of stencil codes with
different workgroup sizes, predicts workgroup sizes for \emph{unseen}
scenarios which maximise the performance. This section presents three
contrasting methods for achieving this goal.


\subsection{Predicting Oracle Workgroup Sizes}

\begin{algorithm}[t]
\begin{algorithmic}[1]
\Require scenario $s$
\Ensure workgroup size $w$

\Procedure{Baseline}{s}
% \Comment Select the best $w$ from $W_{safe}$.
\State $w \leftarrow \text{classify}(f(s))$
\If{$w \in W_{legal}(s)$}
    \State \textbf{return} $w$
\Else
  \State \textbf{return} $\underset{w \in W_{safe}}{\argmax}
\left(
  \prod_{s \in S_{training}} p(s, w)
\right)^{1/|S_{training}|}$
\EndIf
\EndProcedure
\item[] % line break

\Procedure{Random}{s}
% \Comment Select a random workgroup size.
\State $w \leftarrow \text{classify}(f(s))$
\While{$w \not\in W_{legal}(s)$}
  \State $W \leftarrow \left\{ w | w < W_{max}(s), w \not\in W_{refused}(s) \right\}$
  \State $w \leftarrow $ random selection $w \in W$
\EndWhile
\State \textbf{return} $w$
\EndProcedure
\item[] % line break

\Procedure{NearestNeighbour}{s}
% \Comment Select the closest workgroup size to prediction.
\State $w \leftarrow \text{classify}(f(s))$
\While{$w \not\in W_{legal}(s)$}
  \State $d_{min} \leftarrow \infty$
  \State $w_{closest} \leftarrow \text{null}$
  \For{$c \in \left\{ w | w < W_{\max}(s), w \not\in W_{refused}(s) \right\}$}
    \State $d \leftarrow \sqrt{\left(c_r - w_r\right)^2 + \left(c_c - w_c\right)^2}$
    \If{$d < d_{min}$}
      \State $d_{min} \leftarrow d$
      \State $w_{closest} \leftarrow c$
    \EndIf
  \EndFor
  \State $w \leftarrow w_{closest}$
\EndWhile
\State \textbf{return} $w$
\EndProcedure
\end{algorithmic}
\caption{Prediction using classifiers}
\label{alg:autotune-classification}
\end{algorithm}

The first approach to predicting workgroup sizes is to consider the
set of possible workgroup sizes as a hypothesis space and to use a
classifier to predict, for a given set of features, the \emph{oracle}
workgroup size. The oracle workgroup size $\Omega(s)$ is the workgroup
size which provides the lowest mean runtime $t(s,w)$:
%
\begin{equation}
  \Omega(s) = \argmin_{w \in W_{legal}(s)} t(s,w)
\end{equation}
%
Training a classifier for this purpose requires pairs of stencil
features $f(s)$ labelled with their oracle workgroup size for a set of
training scenarios $S_{training}$:
%
\begin{equation}
  D_{training} = \left\{ \left(f(s), \Omega(s)\right) | s \in S_{training} \right\}
\end{equation}
%
After training, the classifier predicts workgroup sizes for unseen
scenarios from the set of oracle workgroup sizes from the training
set. This is a common and intuitive approach to autotuning, in that a
classifier predicts the best parameter value based on the best
parameter values for previous scenarios. However, given the
constrained space of workgroup sizes, this presents the problem that
future scenarios may have different sets of legal workgroup sizes to
that of the training data:
%
\begin{equation}
  \bigcup_{\forall s \in S_{future}} W_{legal}(s) \nsubseteq \left\{ \Omega(s) | s \in S_{training} \right\}
\end{equation}
%
This results in an autotuner which may predict workgroup sizes that
are not legal for scenarios, either because they exceed $W_{\max}(s)$,
or because parameters are refused, $w \in W_{refused}(s)$. For these
cases, we evaluate the effectiveness of three \emph{fallback
  strategies}:
%
\begin{enumerate}
\item \emph{Baseline} --- select the workgroup size which provides the
  highest average case performance from the set of safe workgroup sizes.
\item \emph{Random} --- select a random workgroup size which is
  expected from prior observations to be legal.
\item \emph{Nearest Neighbour} --- select the workgroup size which
  from prior observations is expected to be legal, and has the lowest
  Euclidian distance to the prediction.
\end{enumerate}
%
See Algorithm~\ref{alg:autotune-classification} for definitions.


\subsection{Predicting Kernel Runtimes}

\begin{algorithm}[t]
\begin{algorithmic}[1]
\Require scenario $s$, regressor $R(x, w)$, fitness function $\Delta(x)$
\Ensure workgroup size $w$

\State $W \leftarrow \left\{ w | w < W_{\max}(s) \right\} -
W_{refused}(s)$
\Comment Candidates.
\State $w \leftarrow \underset{w \in W}{\argmax} \Delta(R(f(s), w))$
\Comment Select best candidate.
\While{$w \not\in W_{legal}(s)$}
  \State $W_{refused}(s) = W_{refused}(s) + \{w\}$
  \State $W \leftarrow W - \left\{ w \right\}$
  \Comment Remove candidate from selection.
  \State $w \leftarrow \underset{w \in W}{\argmax} \Delta(R(f(s), w))$
  \Comment Select best candidate.
\EndWhile
\State \textbf{return} $w$
\end{algorithmic}
\caption{Prediction using regressors}
\label{alg:autotune-regression}
\end{algorithm}

A problem of predicting oracle workgroup sizes is that, for each
training instance, an exhaustive search of the optimisation space must
be performed in order to find the oracle workgroup size. An
alternative approach is to instead predict the expected \emph{runtime}
of a kernel given a specific workgroup size. Given training data
consisting of $(f(s),w,t)$ tuples, where $f(s)$ are scenario features,
$w$ is the workgroup size, and $t$ is the observed runtime, we train a
regressor $R(f(s), w)$ which predicts the runtime of scenario and
workgroup size combinations. The selected workgroup size
$\bar{\Omega}(s)$ is then the workgroup size from a pool of candidates
which minimises the output of the regressor, as shown in
Algorithm~\ref{alg:autotune-regression}. The fitness function
$\Delta(x)$ is computes the reciprocal of the predicted runtime, so as
to favour shorter over longer runtimes. Note that the algorithm is
self correcting in the presence of refused parameters --- if a
workgroup size is refused, it is removed from the candidate pool, and
the next best candidate is chosen. This removes the need for fallback
handlers, and the technique allows for training on data for which the
oracle workgroup size is unknown, and has the secondary advantage that
this allows for an additional training instance to be gathered every
time a kernel is evaluated.


\subsection{Predicting Relative Performance}

Accurately predicting the runtime of arbitrary stencil codes is a
difficult problem due to the impacts of flow control. In such cases,
it may be more effective to instead predict the \emph{relative}
performance of two different workgroup sizes for the same kernel. To
do this, we predict the \emph{speedup} of a workgroup size over a
baseline. This baseline is the workgroup which provides the best
average case performance across all scenarios and is known to be
safe. Such a baseline value represents the \emph{best} possible
performance which can be achieved using a single, statically chosen
workgroup size. We train a regressor $R(f(s), w)$ to predict the
relative performance of workgroup sizes over this baseline parameter,
and apply the same algorithm as for predicting runtimes. The fitness
function returns the predicted speedup of a workgroup size over the
baseline, so the selected workgroup size $\bar{\Omega}(s)$ is the
workgroup size from a pool of candidates which maximises the output of
the regressor. This has the same advantageous properties as predicting
runtimes, but by training using relative performance, we minimise the
risk of control flow leading to inaccurate predictions.


\section{Experimental Setup}

\begin{table*}
\scriptsize
\centering
\begin{tabular}{l l | l l l l l l}
\toprule
Host & Host Memory &  OpenCL Device &  Compute units & Frequency & Local Memory & Global Cache & Global Memory \\
\midrule
Intel i5-2430M & 8 GB  & CPU              &              4 &   2400 Hz &        32 KB &       256 KB &       7937 MB \\
Intel i5-4570  & 8 GB  & CPU              &              4 &   3200 Hz &        32 KB &       256 KB &       7901 MB \\
Intel i7-3820  & 8 GB  & CPU              &              8 &   1200 Hz &        32 KB &       256 KB &       7944 MB \\
Intel i7-3820  & 8 GB  & AMD Tahiti 7970  &             32 &   1000 Hz &        32 KB &        16 KB &       2959 MB \\
Intel i7-3820  & 8 GB  & Nvidia GTX 590   &              1 &   1215 Hz &        48 KB &       256 KB &       1536 MB \\
Intel i7-2600K & 16 GB & Nvidia GTX 690   &              8 &   1019 Hz &        48 KB &       128 KB &       2048 MB \\
Intel i7-2600  & 8 GB  & Nvidia GTX TITAN &             14 &    980 Hz &        48 KB &       224 KB &       6144 MB \\
\bottomrule
\end{tabular}
\caption{Specification of experimental platforms and OpenCL devices.}
\label{tab:hw}
\end{table*}

To evaluate the performance of the presented autotuning techniques, an
exhaustive enumeration of the workgroup size optimisation space for
429 combinations of architecture, program, and dataset was performed.

Table~\ref{tab:hw} describes the experimental platforms and OpenCL
devices used. Each platform was unloaded, frequency governors were
disabled, and the benchmark processes were set to the highest priority
available to the task scheduler. Datasets and programs were stored in
an in-memory file system. All runtimes were recorded with millisecond
precision using OpenCL's Profiling API to record the kernel execution
time. A minimum of 30 samples were recorded for each workgroup size in
multiples of 2 up to the maximum allowed for each scenario.

\begin{table}
\scriptsize
\centering
\begin{tabular}{lrrrrp{1.3cm}}
\toprule
      Name &  North &  South &  East &  West &  Instruction Count \\
\midrule
   synthetic-a & 1--30 & 1--30 & 1--30 & 1--30 & 67--137\\
   synthetic-b & 1--30 & 1--30 & 1--30 & 1--30 & 592--706\\
   gaussian    & 1--10 & 1--10 & 1--10 & 1--10 & 82--83 \\
   gol         &      1 &      1 &     1 &     1 &                190 \\
   he          &      1 &      1 &     1 &     1 &                113 \\
   nms         &      1 &      1 &     1 &     1 &                224 \\
   sobel       &      1 &      1 &     1 &     1 &                246 \\
   threshold   &      0 &      0 &     0 &     0 &                 46 \\
\bottomrule
\end{tabular}
\caption{%
  Stencil kernels, border sizes (north, south, east, and west),
  and static instruction counts.
}
\label{tab:kernels}
\end{table}

In addition to the synthetic stencil benchmarks described in
Section~\ref{subsec:training}, six stencil kernels taken from four
reference implementations of standard stencil applications from the
fields of image processing, cellular automata, and partial
differential equation solvers are used: Canny Edge Detection, Conway's
Game of Life, Heat Equation, and Gaussian
Blur. Table~\ref{tab:kernels} shows details of the stencil kernels for
these reference applications and the synthetic training benchmarks
used. For each program, dataset sizes of size $512\times512$,
$1024\times1024$, $2048\times2048$, and $4096\times4096$ were used.

Program behaviour is validated by comparing program output against a
gold standard output collected by executing each of the real-world
benchmarks programs using the baseline workgroup size. The output of
real-world benchmarks with other workgroup sizes is compared to this
gold standard output to test for correct program execution.

Five different classification algorithms are used to predict oracle
workgroup sizes, chosen for their contrasting properties: Naive Bayes,
SMO, Logistic Regression, J48 Decision tree, and Random Forest. For
regression, a Random Forest with regression trees is used, chosen
because of its efficient handling of large feature sets compared to
linear models. The autotuning system is implemented as system daemon
in Python. SkelCL stencil programs request workgroup sizes from this
daemon, which performs feature extraction and classification.

% Feature extraction (particularly compilation to LLVM IR) introduces a
% runtime overhead to the classification process. To minimise this, a
% relational database stores lookup tables for device and dataset
% features, indexed by device IDs and checksums of kernel source codes,
% respectively. During autotuning, before feature extraction for either
% occurs a lookup is performed in the relevant table, meaning that the
% cost of feature extraction is amortised over time.


\section{Results}\label{sec:results}

This section describes the results of enumerating the workgroup size
optimisation space. The effectiveness of autotuning techniques for
exploiting this space are examined in
Section~\ref{sec:evaluation}. The experimental results consist of
measured runtimes for a set of \emph{test cases}, where a test case
$\tau_i$ consists of a scenario, workgroup size pair
$\tau_i = (s_i,w_i)$, and is associated with a \emph{sample} of
observed runtimes of the program. A total of 269813 test cases were
evaluated, which represents an exhaustive enumeration of the workgroup
size optimisation space for 429 scenarios. For each scenario, runtimes
for an average of 629 (max 7260) unique workgroup sizes were
measured. The average sample size for each test case is 83 (min 33,
total 16917118).

\begin{figure}
  \centering
  \includegraphics[width=\columnwidth]{img/oracle_param_space.pdf}
  \caption{%
    Oracle frequency counts for a subset of the workgroup sizes,
    $w_c \le 100, w_r \le 100$. There are 135 unique oracle workgroup
    sizes. The most common oracle workgroup size is
    $w_{(64 \times 4)}$, optimal for 15\% of scenarios.%
  }
\label{fig:oracle-wgsizes}
\end{figure}

The workgroup size optimisation space is non-linear and
complex. Across the 429 different scenarios, there are 135 unique
oracle workgroup sizes, with 31.5\% of scenarios having a unique
oracle workgroup size. Figure~\ref{fig:oracle-wgsizes} shows the
distribution of these oracle workgroup sizes. The average speedup of
the oracle workgroup size over the worst workgroup size for each
scenario is $15.14\times$ (min $1.03\times$, max $207.72\times$).

Of the 8504 unique workgroup sizes tested, 11.4\% were refused in one
or more test cases, with an average of 5.5\% test cases leading to
refused parameters. While there are certain patterns (for example,
workgroup sizes which contain $w_c$ and $w_r$ values which are
multiples of eight are less frequently refused, which is a common
width of SIMD vector operations~\cite{IntelCorporation2012}), a
refused parameter is an obvious inconvenience to the user, as one
would expect that any workgroup size within the specified maximum
should behave \emph{correctly}, if not efficiently.

\begin{figure}
  \centering
  \centering
  \includegraphics[width=.6\columnwidth]{img/refused_params_by_device}
  \caption{%
    The ratio of test cases with refused workgroup sizes, grouped by
    OpenCL device ID. No parameters were refused by the AMD device.%
  }
\label{fig:refused-params}
\end{figure}

Experimental results suggest that the problem is vendor --- or at
least device --- specific. Figure~\ref{fig:refused-params} shows the
ratio of refused test cases, grouped by device. We see a much greater
quantity of refused parameters for test cases on Intel CPU devices
than any other type, while no workgroup sizes were refused by the AMD
GPU. The exact underlying cause for these refused parameters is
unknown, but can likely by explained by inconsistencies or errors in
specific OpenCL driver implementations. As these OpenCL
implementations are still in active development, it is anticipated
that errors caused by unexpected behaviour will become more infrequent
as the technology matures. Note that the ratio of refused parameters
decreases across the three generations of Nvidia GPUs: GTX 590 (2011),
GTX 690 (2012), and GTX TITAN (2013). For now, it is imperative that
any autotuning system is capable of adapting to these refused
parameters by suggesting alternatives when they occur.

The baseline parameter $\bar{w}$ is the workgroup size which provides
the best overall performance while being legal for all
scenarios. Because of refused parameters, only a \emph{single}
workgroup size $w_{(4 \times 4)}$ from the set of experimental results
is found to have a legality of 100\%, suggesting that an adaptive
approach to setting workgroup size is necessary not just for the sake
of maximising performance, but also for guaranteeing program
correctness. The utility of the baseline parameter is that it
represents the best performance that can be achieved through static
tuning of the workgroup size parameter; however, compared to the
oracle workgroup size for each scenario, the baseline parameter
achieves only 24\% of the available performance.


\section{Evaluation}\label{sec:evaluation}

In this section we evaluate the effectiveness of the three proposed
autotuning techniques for predicting performant workgroup sizes. For
each autotuning technique, we partition the experimental data into
training and testing sets. Three strategies for partitioning the data
are used: the first is a 10-fold cross-validation; the second is to
divide the data such that only data collected from synthetic
benchmarks are used for training and only data collected from the
real-world benchmarks are used for testing; the third approach is to
create leave-one-out partitions for each unique device, kernel, and
dataset. For each combination of autotuning technique and testing
dataset, we evaluate each of the workgroup sizes predicted for the
testing data using the following metrics:
%
\begin{itemize}
\item time (real) --- the time taken to make the autotuning
  prediction. This includes classification time and any communication
  overheads.
\item accuracy (binary) --- whether the predicted workgroup size is
  the true oracle, $w = \Omega(s)$.
\item validity (binary) --- whether the predicted workgroup size
  satisfies the workgroup size constraints constraints,
  $w < W_{\max}(s)$.
\item refused (binary) --- whether the predicted workgroup size is
  refused, $w \in W_{refused}(s)$.
\item performance (real) --- the performance of the predicted
  workgroup size relative to the oracle for that scenario.
\item speedups (real) --- the relative performance of the predicted
  workgroup size relative to the baseline workgroup size
  $w_{(4 \times 4)}$, and human expert workgroup size
  $w_{(32 \times 4)}$ (where applicable).
\end{itemize}
%
The \emph{validty} and \emph{refused} metrics measure how often
fallback strategies are required to select a legal workgroup size
$w \in W_{legal}(s)$. This is only required for the classification
approach to autotuning, since the process of selecting workgroup sizes
using regressors respects workgroup size constraints.

\begin{figure}
\centering
\includegraphics[width=\columnwidth]{img/classification-syn-real}
\caption{%
  Autotuning performance using classifiers and synthetic benchmarks. Each
  classifier is trained on data collected from synthetic stencil
  applications, and tested for prediction quality using data from 6
  real-world benchmarks. Each of the different values correspond to a
  different data partitioning strategy, e.g.\ cross-kernel
  partitioning, 10-fold validation, etc. 95\% confidence intervals are
  shown where appropriate.%
}
\label{fig:class-syn}
\end{figure}


\begin{figure}
\centering
\begin{subfigure}[h]{.48\columnwidth}
\centering
\includegraphics[width=\columnwidth]{img/runtime-class-xval}
\caption{}
\label{fig:runtime-class-xval}
\end{subfigure}
\begin{subfigure}[h]{.48\columnwidth}
\centering
\includegraphics[width=\columnwidth]{img/speedup-class-xval}
\caption{}
\label{fig:speedup-class-xval}
\end{subfigure}
\caption{%
  Autotuning performance for each type of test dataset using
  regressors to predict: (\subref{fig:runtime-class-xval}) kernel
  runtimes, and (\subref{fig:speedup-class-xval}) relative performance
  of workgroup sizes.%
}
\label{fig:regression-class}
\end{figure}

\subsection{Predicting Oracle Workgroup Size}

With the exception of the ZeroR, which is a classifier that predicts
\emph{only} the baseline workgroup size
$w_{\left( 4 \times 4 \right)}$, the classifiers achieve good speedups
over the baseline, ranging from $4.61\times$ to $5.05\times$ when
averaged across all test sets. Figure~\ref{fig:class-syn} shows the
results when classifiers are trained using data from synthetic
benchmarks and tested using real-world benchmarks. The highest average
speedup is achieved by the SMO classifier, and the lowest by Naive
Bayes. The difference between average speedups is not significant
between the types of classifier, with the exception of SimpleLogistic,
which performs poorly when trained with synthetic benchmarks and
tested against real-world programs. This suggests the model
over-fitting to features of the synthetic benchmarks which are not
shared by the real-world tests.

Of the three approaches to handling invalid predictions, the fallback
handler with the best average case performance is
\textsc{NearestNeighbour}, achieving an average speedup across all
classifiers and validation sets of $5.26\times$. The speedup of
\textsc{Random} fallback handler is $3.69\times$, and $1.0\times$ for
\textsc{Baseline}. Interestingly, both the lowest and highest speedups
are achieved by the \textsc{Random} fallback handler, since it
essentially performs a random exploration of the optimisation
space. However, the \textsc{NearestNeighbour} fallback handler
provides consistently greater speedups for the majority of test cases,
indicating that it successfully exploits structure in the optimisation
spaces.


\subsection{Predicting Runtimes and Speedups}

Figures~\ref{fig:runtime-class-xval} and~\ref{fig:speedup-class-xval}
show a summary of results for autotuning using regressors to predict
program runtimes and speedups, respectively. Of the two regression
techniques, predicting the \emph{speedup} of workgroup sizes is much
more successful than predicting the \emph{runtime}. This is most
likely caused by the inherent difficulty in predicting the runtime of
arbitrary code, where dynamic factors such as flow control and loop
bounds are not captured by static instruction counts and densities
which are used as features by the machine learning models. The average
speedup achieved by predicting runtimes is $4.14\times$. For
predicting speedups, the average is $5.57\times$.

The prediction times using regressors are significantly greater than
using classifiers. This is because, while a classifier makes a single
prediction, the number of predictions required of a regressor grows
with the size of $W_{\max}(s)$, since classification with regression
requires making predictions for all
$w \in \left\{ w | w < W_{\max}(s) \right\}$. The fastest classifier
is J48, due to the it's simplicity --- it can be implemented as a
sequence of nested \texttt{if} and \texttt{else} statements.


\subsection{Comparison with Human Expert}

\begin{figure}
\centering
\includegraphics[width=\columnwidth]{img/speedup-distributions}
\caption[Speedup results over human expert]{%
  Distributions of speedups over \emph{human expert}, ignoring cases
  where the workgroup size selected by human experts is
  invalid. Classifiers are using \textsc{NearestNeighbour} fallback
  handlers. The speedup axis is fixed to the range 0--2.5 to highlight
  the IQRs, which results in some outlier speedups > 2.5 being
  clipped.%
}
\label{fig:speedup-distributions}
\end{figure}

In the original implementation of the SkelCL stencil
pattern~\cite{Breuer2014a}, \citeauthor{Breuer2014a} selected a
workgroup size of $w_{(32 \times 4)}$ in an evaluation of 4 stencil
operations on a Tesla S1070 system. In our evaluation of 429
combinations of kernel, architecture, and dataset, we found that this
workgroup size is refused by 2.6\% of scenarios, making it unsuitable
for use as a baseline. However, if we remove the scenarios for which
$w_{(32 \times 4)}$ is \emph{not} a legal workgroup size, we can
directly compare the performance against the autotuning predictions.

Figure~\ref{fig:speedup-distributions} plots the distribution of
speedups of all test instances over the human expert parameter for
each autotuning technique. The speedup distributions show consistent
classification results for the five classification techniques, with
the speedup at the lower quartile for all classifiers being
$\ge 1.0\times$. The IQR for all classifiers is $< 0.5$, but there are
outliers with speedups both well below $1.0\times$ and well above
$2.0\times$. In contrast, the speedups achieved using regressors to
predict runtimes have a lower range, but also a lower median and a
larger IQR. Clearly, this approach is the least effective of the
evaluated autotuning techniques. Using regressors to predict relative
performance is more successful, with the highest median speedup of all
the techniques. However, it also has a large IQR and the lower
quartile has a speedup value well below 1, meaning that for more than
25\% of test instances, the workgroup size selected did not perform as
well as the human expert selected workgroup size.


% \begin{figure}
% \centering
% \begin{subfigure}[t]{0.48\columnwidth}
% \centering
% \includegraphics[width=\columnwidth]{img/heatmap_1}
% \vspace{-1.5em} % Shrink vertical padding
% \caption{}
% \label{fig:class-hmaps-1}
% \end{subfigure}
% \begin{subfigure}[t]{0.48\columnwidth}
% \centering
% \includegraphics[width=\columnwidth]{img/heatmap_2}
% \vspace{-1.5em} % Shrink vertical padding
% \caption{}
% \label{fig:class-hmaps-2}
% \end{subfigure}
% \\
% \begin{subfigure}[t]{0.48\columnwidth}
% \centering
% \includegraphics[width=\columnwidth]{img/heatmap_3}
% \vspace{-1.5em} % Shrink vertical padding
% \caption{}
% \label{fig:class-hmaps-3}
% \end{subfigure}
% \begin{subfigure}[t]{0.48\columnwidth}
% \centering
% \includegraphics[width=\columnwidth]{img/heatmap_5}
% \vspace{-1.5em} % Shrink vertical padding
% \caption{}
% \label{fig:class-hmaps-4}
% \end{subfigure}
% \\
% \begin{subfigure}[t]{0.48\columnwidth}
% \centering
% \includegraphics[width=\columnwidth]{img/reg_runtime_heatmap}
% \vspace{-1.5em} % Shrink vertical padding
% \caption{}
% \label{fig:class-hmaps-5}
% \end{subfigure}
% \begin{subfigure}[t]{0.48\columnwidth}
% \centering
% \includegraphics[width=\columnwidth]{img/reg_speedup_heatmap}
% \vspace{-1.5em} % Shrink vertical padding
% \caption{}
% \label{fig:class-hmaps-6}
% \end{subfigure}
% \caption[Classification error heatmaps]{%
%   Heatmaps of classification errors for 10-fold cross-validation,
%   showing a subset of the optimisation space. The shading in each
%   cells indicates if it is predicted less frequently (blue), ore more
%   frequently (red) than it is optimal. Colour gradients are normalised
%   across plots.%
% }
% \label{fig:class-hmaps}
% \end{figure}

% Figure~\ref{fig:class-hmaps} visualises the classification errors of
% each of the autotuning techniques. It shows that while the performance
% of all of the classifiers is comparable, the distribution of
% predictions is not. Only the NaiveBayes and RandomForest classifiers
% predicted the human expert selected workgroup size of
% $w_{(32 \times 4)}$ as frequently, or more frequently, than it was
% optimal. The two regression techniques were the least accurate of all
% of the autotuning techniques.


\section{Related Work}\label{sec:related}

% Iterative compilation is the method of performance tuning in which a
% program is compiled and profiled using multiple different
% configurations of optimisations in order to find the configuration
% which maximises performance. One of the the first formalised
% publications of the technique appeared in \citeyear{Bodin1998} by
% \citeauthor{Bodin1998}~\cite{Bodin1998}.  Iterative compilation has
% since been demonstrated to be a highly effective form of empirical
% performance tuning for selecting compiler optimisations.

% Given the huge number of possible compiler optimisations (there are
% 207 flags and parameters to control optimisations in GCC v4.9), it
% is often unfeasible to perform an exhaustive search of the entire
% optimisation space, leading to the development of methods for
% reducing the cost of evaluating configurations. These methods reduce
% evaluation costs either by shrinking the dimensionality or size of
% the optimisation space, or by guiding a directed search to traverse
% a subset of the space.

% Machine learning has been successful applied to this problem,
% in~\cite{Stephenson2003}, using ``meta optimisation'' to tune
% compiler heuristics through an evolutionary algorithm to automate
% the search of the optimisation space. \citeauthor{Fursin2011}
% continued this with Milepost GCC, the first machine learning-enabled
% self-tuning compiler~\cite{Fursin2011}. A recent survey of the use
% of machine learning to improve heuristics quality by
% \citeauthor{Burke2013} concludes that the automatic
% \emph{generation} of these self-tuning heuristics but is an ongoing
% research challenge that offers the greatest generalisation
% benefits~\cite{Burke2013}.

\citeauthor{Ganapathi2009} demonstrated early attempts at autotuning
multicore stencil codes in~\cite{Ganapathi2009}, presenting an
autotuner which evaluates a random 1500 selections from an space of 10
million optimisations, achieving 18\% better than that of a human
expert. The Kernel Canonical Correlation Analysis used in their
autotuner restricts the scalability of their system, as the complexity
of model building grows exponentially with the number of features. In
their evaluation, the system requires two hours of compute time to
build the model for only 400 seconds of benchmark data.

% \citeauthor{Berkeley2009} targeted 3D stencils code performance
% in~\cite{Berkeley2009}. Stencils are decomposed into core blocks,
% sufficiently small to avoid last level cache capacity misses. These
% are then further decomposed to thread blocks, designed to exploit
% common locality threads may have within a shared cache or local
% memory. Thread blocks are divided into register blocks in order to
% take advantage of data level parallelism provided by the available
% registers. Data allocation is optimised on NUMA systems. The
% performance evaluation considers speedups of various optimisations
% with and without consideration for host/device transfer overhead.

\citeauthor{Kamil2010} present an autotuning framework
in~\cite{Kamil2010} which accepts as input a Fortran 95 stencil
expression and generates tuned shared-memory parallel implementations
in Fortan, C, or CUDA. The system uses an IR to explore autotuning
transformations, enumerating a subset of the optimisation space and
recording only a single execution time for each configuration,
reporting the fastest. They demonstrate their system on 4
architectures using 3 benchmarks, with speedups of up to $22\times$
compared to serial implementations. The CUDA code generator does not
optimise for the GPU memory hierarchy, using only global memory. As
demonstrated in this work, improper utilisation of local memory can
hinder program performance by two orders of magnitude.
% There is no directed search or cross-program learning.

% In~\cite{Zhang2013a}, \citeauthor{Zhang2013a} present a code generator
% and autotuner for 3D Jacobi stencil codes. Using a DSL to express
% kernel functions, the code generator performs substitution from one of
% two CUDA templates to create programs for execution on GPUs. GPU
% programs are parameterised and tuned for block size, block dimensions,
% and whether input data is stored in read only texture memory. This
% creates an optimisation space of up to 200 configurations. In an
% evaluation of 4 benchmarks, the authors report impressive performance
% that is comparable with previous implementations of iterative Jacobi
% stencils on GPUs~\cite{Holewinski2012, Phillips2010}. The dominating
% parameter is shown to be block dimensions, followed by block size,
% then read only memory. The DSL presented in the paper is limited to
% expressing only Jacobi Stencils applications. Critically, their
% autotuner requires a full enumeration of the parameter space for each
% program. Since there is no indication of the compute time required to
% gather this data, it gives the impression that the system would be
% impractical for the needs of general purpose stencil computing. The
% autotuner presented in this thesis overcomes this drawback by learning
% parameter values which transfer to unseen stencils, without the need
% for an expensive tuning phase for each program and architecture.

% In~\cite{Christen2011}, \citeauthor{Christen2011} presentf a DSL for
% expressing stencil codes, a C code generator, and an autotuner for
% exploring the optimisation space of blocking and vectorisation
% strategies. The DSL supports stencil operations on arbitrarily
% high-dimensional grids. The autotuner performs either an exhaustive,
% multi-run Powell search, Nelder Mead, or evolutionary search to find
% optimal parameter values. They evaluate their system on two CPUS and
% one GPU using 6 benchmarks. A comparison of tuning results between
% different GPU architectures would have been welcome, as the results
% presented in this thesis show that devices have different responses to
% optimisation parameters. The authors do not present a ratio of the
% available performance that their system achieves, or how the
% performance of optimisations vary across benchmarks or devices.

PARTANS is an autotuning framework which targets the size of the halo
region for multi-GPU stencils~\cite{Lutz2013}. \citeauthor{Lutz2013}
explore the effect of varying the optimisation space using six
benchmark applications, finding that the optimal halo size depends on
the size of the grid, the number of partitions, and the connection
mechanism. The authors present an autotuner which determines problem
decomposition and swapping strategy offline, and performs an online
search for the optimal halo size. The selection of overlapping halo
region size compliments the selection of workgroup size which is the
subject of this thesis. However, PARTANS uses a custom DSL rather than
the generic interface provided by SkelCL, and PARTANS does not learn
the results of tuning across programs, or across multiple runs of the
same program.

\citeauthor{Collins2012} autotune Algorithmic Skeletons
in~\cite{Collins2012}, first using Principle Components Analysis to
reduce the dimensionality of the optimisation space, followed by a
search of parameter values to optimise program performance by a factor
of $1.6\times$ over values chosen by a human
expert. In~\cite{Collins2013}, they extend this using static feature
extraction and nearest neighbour classification to further prune the
search space, achieving an average 89\% of the oracle performance
after evaluating 45 parameters.

A method for the automatic generation of synthetic benchmarks for the
purpose of performance tuning is presented in~\cite{Chiu2015}, using
parameterised template substitution over a user-defined range of
values to generate training programs. The authors describe an
application of their tool for generating OpenCL stencil kernels, but
do not report any performance results.

% Performant GPGPU programs require careful attention from the developer
% to properly manage data layout in DRAM, caching, diverging control
% flow, and thread communication. The importance of proper exploitation
% of local shared memory and synchronisation costs is explored
% in~\cite{Lee2010}. In~\cite{Chen2014}, data locality optimisations are
% automated using a description of the hardware and a
% memory-placement-agnostic compiler. The authors demonstrate impressive
% speedups of up to $2.08\times$, although at the cost of requiring
% accurate memory hierarchy descriptor files for all targeted
% hardware. The descriptor files must be hand generated, requiring
% expert knowledge of the underlying hardware in order to properly
% exploit memory locality.


\section{Conclusions}\label{sec:conclusions}

To the best of our knowledge, the autotuning methodologies presented
in this work constitute the first attempt to autotune the workgroup
size of high-level stencil patterns. The autotuning techniques
proposed in this paper achieve up to 94\% of the available
performance, providing speedups of $5.57\times$ over static tuning,
while providing robust fallbacks in the presence of unexpected
behaviour of OpenCL driver implementations. Of the techniques
proposed, predicting the relative performances of workgroup sizes
using regressors provides the highest median speedup, whilst
predicting the oracle workgroup size using decision tree classifiers
requires the lowest overhead. This presents a trade-off between
classification time and training time, which could be explored in
future work, possibly using a hybrid combination of the techniques
presented in this paper.

In the future, we will expand the autotuner to accommodate additional
optimisation parameters, as well as further exploring the transition
towards online machine learning which is enabled by using regressors
to predict relative performance or kernel runtimes. This could be
combined with the use of adaptive sampling plans to minimise the
number of observations required to distinguish bad from good parameter
values, such as presented in~\cite{Leather2009}. The use of dynamic
profiling could be used to increase the prediction accuracy of kernel
runtimes.


\acks

This work was supported by grant EP/L01503X/1 for the University of
Edinburgh School of Informatics Centre for Doctoral Training in
Pervasive Parallelism
(\url{http://pervasiveparallelism.inf.ed.ac.uk/}) from the UK
Engineering and Physical Sciences Research Council (EPSRC).

% We recommend abbrvnat bibliography style.

\label{bibliography}
\printbibliography


\end{document}
