Compilers are a fundamental yet complex technology. They comprise thousands of interacting components which must be expertly engineered and tuned to both maximise performance and ensure correctness. The rising demand for data-intensive workloads along with the rapid transition to heterogeneous parallelism has left compiler developers struggling to keep up. The cost of these shortcomings is wasted energy, poor performance, and buggy software. What is needed is ways to lower the cost of constructing compilers.

This thesis presents three techniques to simplify and accelerate compiler construction. First, a tool for automatic performance characterisation through benchmark generation; second, a low cost and effective fuzzer for validating correctness; third, a simple technique to address the labour intensive process of optimisation heuristic construction. The key insight of this work is the use of \emph{recurrent neural networks} for modelling the syntax and semantics of programming languages.

First, a generative model for benchmarks is developed. Unlike prior approaches, the model is inferred entirely from corpora of readily available open source programs, requiring no grammar or prior knowledge of programming languages. This greatly reduces the cost of development, yet the generator produces output of such quality that professional software developers cannot distinguish generated from handwritten code. The generated programs are used as benchmarks to supplement the training data of state-of-the-art predictive models for compiler optimisations. The additional fine-grained exploration of the feature space attributable to the generated programs yields an automatic $1.27\times$ improvement in performance. In addition, the extra information automatically exposes weaknesses in the feature design which, when corrected, yields a further $4.30\times$ improvement in performance.

Second, the technique is extended to the domain of compiler validation. A compiler fuzzer is developed which is far simpler than the state-of-the-art, yet is effective. The fuzzer is capable of generating an expressive range of tests that expose bugs that the state-of-the-art cannot. It is implemented in $100\times$ fewer lines of code than the state-of-the-art. An extensive testing campaign of OpenCL compilers reveals 67 new bugs, many of which have now been fixed.

Finally, this thesis explores a methodology for constructing compiler heuristics without code features. Unlike state-of-the-art approaches in which code features have to be expertly engineered and selected, the proposed approach uses recurrent neural networks which learn directly over the textual representation of program code. Doing so outperforms state-of-the-art predictive models in two challenging optimisation domains. Additionally, by using the same neural network structure for different tasks, this enables the novel transfer of information between optimisation problems.
