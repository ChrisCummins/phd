\section{Introduction}

In this chapter I apply the OmniTune framework to SkelCL. The publicly
available implementation
\footnote{\url{https://github.com/ChrisCummins/omnitune}} predicts
workgroup sizes for OpenCL stencil skeleton kernels in order to
minimise their runtime on CPUs and multi-GPU systems. The optimisation
space presented by the workgroup size of OpenCL kernels is large,
complex, and non-linear. Successfully applying machine learning to
such a space requires plentiful training data, the careful selection
of features, and classification approach. The following sections
address these challenges.


% Anyone downloading a copy of OmniTune will instantly have access to
% the global database of training data, including the
% \checkme{16917118} runtimes which were collected to write this
% thesis.

% \texttt{cec.chlox1mra3iz.us-west-2.rds.amazonaws.com:3306}

\section{Training}\label{sec:training}

One challenge of performing empirical performance evaluations is
gathering enough applications to ensure meaningful
comparisons. Synthetic benchmarks are one technique for circumventing
this problem. The automatic generation of such benchmarks has clear
benefits for reducing evaluation costs; however, creating meaningful
benchmark programs is a difficult problem if we are to avoid the
problems of redundant computation and produce provable halting
benchmarks.

In practise, stencil codes exhibit many common traits: they have a
tightly constrained interface, predictable memory access patterns, and
well defined numerical input and output data types. This can be used
to create a confined space of possible stencil codes by enforcing
upper and lower bounds on properties of the codes which can not
normally be guaranteed for general-purpose programs, e.g.\ the number
of floating point operations. In doing so, it is possible to
programatically generate stencil workloads which share similar
properties to those which we intend to target.

Based on observations of real world stencil codes from the fields of
cellular automata, image processing, and PDE solvers, I implemented a
stencil generator which uses parameterised kernel templates to produce
source codes for collecting training data. The stencil codes are
parameterised by stencil shape (one parameter for each of the four
directions), input and output data types (either integers, or single
or double precision floating points), and \emph{complexity} --- a
simple boolean metric for indicating the desired number of memory
accesses and instructions per iteration, reflecting the relatively
bi-modal nature of the reference stencil codes, either compute
intensive (e.g. FDTD simulations), or lightweight (e.g. Game of Life).

Using a large number of synthetic benchmarks helps adress the ``small
$n$, large $P$'' problem, which describes the difficulty of
statistical inference in spaces for which the set of possible
hypotheses $P$ is significantly larger than the number of observations
$n$. By creating parameterised, synthetic benchmarks, it is possible
to explore a much larger set of the space of possible stencil codes
than if relying solely on reference applications, reducing the risk of
overfitting to particular program features.


\section{Stencil Features}


\begin{table}
  \begin{multicols}{2}
    \scriptsize
    \centering
    \rowcolors{2}{white}{gray!25}
    \begin{tabular}{p{4.5cm}p{1.3cm}}
      \toprule
      \textbf{Dataset Features} &         \textbf{Type} \\
      \midrule
      Number of columns in matrix & numeric \\
      Number of rows in matrix & numeric \\
      Input data type & categorical \\
      Output data type & categorical \\
      \bottomrule
    \end{tabular}
    \vskip 2.75em
    \begin{tabular}{p{4.5cm}p{1.3cm}}
      \toprule
      \textbf{Kernel Features} &         \textbf{Type} \\
      \midrule
      Border region north & numeric \\
      Border region south & numeric \\
      Border region east & numeric \\
      Border region west & numeric \\
      Static instruction count & numeric \\
      \texttt{AShr} instruction density & numeric \\
      \texttt{Add} instruction density & numeric \\
      \texttt{Alloca} instruction density & numeric \\
      \texttt{And} instruction density & numeric \\
      \texttt{Br} instruction density & numeric \\
      \texttt{Call} instruction density & numeric \\
      \texttt{FAdd} instruction density & numeric \\
      \texttt{FCmp} instruction density & numeric \\
      \texttt{FDiv} instruction density & numeric \\
      \texttt{FMul} instruction density & numeric \\
      \texttt{FPExt} instruction density & numeric \\
      \texttt{FPToSI} instruction density & numeric \\
      \texttt{FSub} instruction density & numeric \\
      \texttt{GetElementPtr} instruction density & numeric \\
      \texttt{ICmp} instruction density & numeric \\
      \texttt{InsertValue} instruction density & numeric \\
      \texttt{Load} instruction density & numeric \\
      \texttt{Mul} instruction density & numeric \\
      \texttt{Or} instruction density & numeric \\
      \texttt{PHI} instruction density & numeric \\
      \texttt{Ret} instruction density & numeric \\
      \texttt{SDiv} instruction density & numeric \\
      \texttt{SExt} instruction density & numeric \\
      \texttt{SIToFP} instruction density & numeric \\
      \texttt{SRem} instruction density & numeric \\
      \texttt{Select} instruction density & numeric \\
      \texttt{Shl} instruction density & numeric \\
      \texttt{Store} instruction density & numeric \\
      \texttt{Sub} instruction density & numeric \\
      \texttt{Trunc} instruction density & numeric \\
      \texttt{UDiv} instruction density & numeric \\
      \texttt{Xor} instruction density & numeric \\
      \texttt{ZExt} instruction density & numeric \\
      Basic block density & numeric \\
      Memory instruction density & numeric \\
      Non external functions density & numeric \\
      Kernel max workgroup size & numeric \\
      \bottomrule
    \end{tabular}
    \vfill
    \columnbreak
    \rowcolors{2}{white}{gray!25}
    \begin{tabular}{p{4.5cm}p{1.3cm}}
      \toprule
      \textbf{Device Features} &         \textbf{Type} \\
      \midrule
      SkelCL device count & numeric \\
      Device address width & categorical \\
      Double precision fp.\ configuration & categorical \\
      Big endian? & categorical \\
      Execution capabilities & categorical \\
      Supported extensions & categorical \\
      Global memory cache size & numeric \\
      Global memory cache size & categorical \\
      Global memory cacheline size & numeric \\
      Global memory size & numeric \\
      Host unified memory? & categorical \\
      2D image max height & numeric \\
      2D image max width & numeric \\
      3D image max depth & numeric \\
      3D image max height & numeric \\
      3D image max width & numeric \\
      Image support & categorical \\
      Local memory size & numeric \\
      Local memory type & categorical \\
      Max clock frequency & numeric \\
      Number of compute units & numeric \\
      Max kernel constant args & numeric \\
      Max constant buffer size & numeric \\
      Max memory allocation size & numeric \\
      Max parameter size & numeric \\
      Max read image arguments & numeric \\
      Max samplers & numeric \\
      Max device workgroup size & numeric \\
      Max workitem dimensions & numeric \\
      Max work item sizes width & numeric \\
      Max work item sizes height & numeric \\
      Max work item sizes depth & numeric \\
      Max write image arguments & numeric \\
      Mem base address align & numeric \\
      Min data type align size & numeric \\
      Native vector width \texttt{char} & numeric \\
      Native vector width \texttt{double} & numeric \\
      Native vector width \texttt{float} & numeric \\
      Native vector width \texttt{half} & numeric \\
      Native vector width \texttt{int} & numeric \\
      Native vector width \texttt{long} & numeric \\
      Native vector width \texttt{short} & numeric \\
      Preferred vector width \texttt{char} & numeric \\
      Preferred vector width \texttt{double} & numeric \\
      Preferred vector width \texttt{float} & numeric \\
      Preferred vector width \texttt{half} & numeric \\
      Preferred vector width \texttt{int} & numeric \\
      Preferred vector width \texttt{long} & numeric \\
      Preferred vector width \texttt{short} & categorical \\
      Queue properties & categorical \\
      Single precision fp.\ configuration & categorical \\
      Device type & categorical \\
      OpenCL vendor & categorical \\
      OpenCL vendor ID & categorical \\
      OpenCL version & categorical \\
      \bottomrule
    \end{tabular}
  \end{multicols}
  \caption[OmniTune SkelCL Stencil features]{%
  OmniTune SkelCL Stencil features for dataset, kernel, and device.%
  }
  \label{tab:features}
\end{table}


Properties of the architecture, program, and dataset all contribute to
the performance of a workgroup size. The success of a machine learning
system depends on the ability to translate these properties into
meaningful explanatory variables --- \emph{features}. To capture this
in OmniTune, parameter requests are packed with a copy of the OpenCL
kernel and attributes of the dataset and device. The OmniTune server
extracts 102 features describing hte architecture, kernel, and dataset
from the message:
%
\begin{itemize}
  \item \textbf{Device} --- OmniTune uses the OpenCL
  \texttt{clGetDeviceInfo()} API to query a number of properties about
  the target execution device. Examples include the size of local
  memory, maximum work group size, number of compute units, etc.
  \item \textbf{Kernel} --- The user code for a stencil is passed to the
  OmniTune server, which compiles the OpenCL kernel to LLVM IR
  bitcode. The \texttt{opt} \texttt{InstCount} statistics pass is used
  to obtain static counts for each type of instruction present in the
  kernel, as well as the total number of instructions. The instruction
  counts for each type are divided by the total number of instructions
  to produce a \emph{density} of instruction for that type. Examples
  include total static instruction count, ratio of instructions per
  type, ratio of basic blocks per instruction, etc.
  \item \textbf{Dataset} --- The SkelCL container type is used to
  extract the input and output data types, and the 2D grid size.
\end{itemize}
%
See Table~\ref{tab:features} for a list of feature names and types.


\subsection{Reducing Feature Extraction Overhead}


Feature extraction (particularlly compilation to LLVM IR) introduces a
runtime overhead to the classification process. To minimise this,
lookup tables for device and dataset features are used, and cached
locally in the OmniTune server and pushed to the remote data
store. The device ID is used to index the devices table, and the
checksum of an OpenCL source is used to index the kernel features
table. Before feature extraction for either occurs, a lookup is
performed in the relevant table, meaning that the cost of feature
extraction is amortised over time.


\section{Optimisation Parameters}\label{sec:op-params}

SkelCL stencil kernels are parameterised by a workgroup size $w$,
which consists of two integer values to denote the number of rows and
columns (where we need to distinguish the individual components, we
will use symbols $w_r$ and $w_c$ to denote rows and columns,
respectively).


\subsection{Constraints}

Unlike in many autotuning applications, the space of optimisation
parameter values is subject to hard constraints, and these constraints
cannot conviently be statically determined. Contributing factors are
architectural limitations, kernel constraints, and refused parameters.


\subsubsection{Architectural constraints}

Each OpenCL device imposes a maximum workgroup size which can be
statically checked by querying the \texttt{clGetDeviceInfo()} API for
that device. These are defined by archiectural limitations of how code
is mapped to the underlying execution hardware. Typical values are
powers of two, e.g.\ 1024, 4096, 8192.


\subsubsection{Kernel constraints}

At runtime, once an OpenCL program has been compiled to a kernel,
users can query the maximum workgroup size supported by that kernel
using the \texttt{clGetKernelInfo()} API. This value cannot easily be
obtained statically as there is no mechanism to determine the maximum
workgroup size for a given source code and device without first
compiling it, which in OpenCL does not occur until runtime. Factors
which affect a kernel's maximum workgroup size include the number
registers required for a kernel, and the available number of SIMD
execution units for each type of instructions in a kernel.


\subsubsection{Refused parameters}

In addition to satisfying the constraints of the device and kernel,
not all points in the workgroup size optimisation space are guaranteed
to provide working programs. A refused parameter is a workgroup size
which satisfies the kernel and architectural constraints, yet causes a
\texttt{CL\_OUT\_OF\_RESOURCES} error to be thrown when the kernel is
enqueued. Note that in many OpenCL implementations, this error type
acts as a generic placeholder and may not necessarily indicate that
the underlying cause of the error was due to finite resources
constraints.


\subsubsection{Legality}

We define a \emph{legal} workgroup size as one which, for a given
\emph{scenario} (a combination of program, device, and dataset),
satisfies the architectural and kernel constraints, and is not
refused. The subset of all possible workgroup sizes
$W_{legal}(s) \subset W$ that are legal for a given sceanario $s$ is
then:
%
\begin{equation}
  W_{legal}(s) = \left\{w | w \in W, w < W_{\max}(s) \right\} - W_{refused}(s)
\end{equation}
%
Where $W_{\max}(s)$ can be determined at runtime prior to the kernels
execution, but the set $W_{refused}(s)$ can only be determined
experimentally.


\subsection{Assessing Relative Performance}

Given a set of observations, where an observation is a scenario,
workgroup size tuple $(s,w)$; a function $t(s,w)$ which returns the
arithmetic mean of the runtimes for a set of observations; we can
calculate the speedup $r(s, w_1, w_2)$ of competing workgroup sizes
$w_1$ over $w_2$ using:
%
\begin{equation}
  r(s, w_1, w_2) = \frac{t(s,w_2)}{t(s,w_1)}\\
\end{equation}
%

\subsubsection{Oracle Workgroup Size}

The \emph{oracle} workgroup size $\Omega(s) \in W_{legal}(s)$ of a
sceanrio $s$ is the $w$ value which provides the lowest mean
runtime. This allows for comparing the performance $p(s,w)$ of a
particular workgroup against the maximum available performance for
that scenario, within the range $0 \le p(s,w) \le 1$:

\begin{align}
  \Omega(s) &= \argmin_{w \in W_{legal}(s)} t(s,w)\\
  p(s,w) &= r(s, w, \Omega(s))
\end{align}


\subsubsection{Establishing a Baseline}

The geometric mean is used to aggregate normalised relative
performances due to its multiplicative
property~\cite{Fleming1986}. For a given workgroup size, the average
performance $\bar{p}(w)$ across the set of all scenarios $S$ can be
found using the geometric mean of performance relative to the oracle:
%
\begin{equation}
  \bar{p}(w) =
  \left(
  \prod_{s \in S} r(s, w, \Omega(s))
  \right)^{1/|S|}
\end{equation}
%
The \emph{baseline} workgroup size $\bar{w}$ is the value which
provides the best average case performance across all scenarios. Such
a baseline value represents the \emph{best} possible performance which
can be achieved using a single, statically chosen workgroup size. By
defining $W_{safe} \in W$ as the intersection of legal workgroup
sizes, the baseline can be found using:

\begin{align}
  W_{safe} &= \cap \left\{ W_{legal}(s) | s \in S \right\}\\
  \bar{w} &= \argmax_{w \in W_{safe}} \bar{p}(w)
\end{align}


\section{Machine Learning}\label{sec:omnitune-ml}

The challenge is to design a system which, given a set of prior
observations of the empirical performance of stencil codes with
different workgroup sizes, predict workgroup sizes for \emph{unseen}
stencils which will maximise the performance. The OmniTune server
supports three methods for achieving this.


\subsection{Predicting Oracle Workgroup Size}\label{subsec:omnitune-ml-class}

The first approach to predicting workgroup sizes is to consider the
set of possible workgroup sizes as a hypothesis space, and to use a
classifier to predict, for a given set of features, the workgroup size
which will provide the best performance. The classifier takes a set of
training scenarios $S_{training}$, and generates labelled training
data as pairs of scenario features $f(s)$ and observed oracle
workgroup sizes:
%
\begin{equation}
  T = \left\{ \left(f(s), \Omega(s)\right) | s \in S_{training} \right\}
\end{equation}
%
During testing, the classifier predicts workgroup sizes from the set
of oracle workgroup sizes from the training set:
%
\begin{equation}
  W_{training} = \left\{ \Omega(s) | s \in S_{training} \right\}
\end{equation}
%
This approach presents the problem that after training, there is no
guarantee that the set of workgroup sizes which may be predicted is
within the set of legal workgroup sizes for future scenarios:
%
\begin{equation}
  \bigcup_{\forall s \in S_{testing}} W_{legal}(s) \nsubseteq W_{training}
\end{equation}
%
This may result in a classifier predicting a workgroup size which is
not legal for a scenario, $w \not\in W_{legal}(s)$, either because it
exceeds $W_{\max}(s)$, or because the parameter is refused. For these
cases, I evaluate the effectiveness of three fallback strategies to
select a legal workgroup size:
%
\begin{enumerate}
  \item \emph{Baseline} --- select the workgroup size which is known to
  be safe $w < W_{safe}$, and provides the highest average case
  performance on training data.
  \item \emph{Random} --- select a random workgroup size which is known
  from prior observations to be legal $w \in W_{legal}(s)$.
  \item \emph{Nearest Neighbour} --- select the workgroup size which
  from prior observations is known to be legal, and has the lowest
  Euclidian distance to the prediction.
\end{enumerate}
%
See Algorithm~\ref{alg:autotune-classification} for definitions.

\begin{algorithm}
  \begin{algorithmic}[1]
\Require scenario $s$
$d$.
\Ensure workgroup size $w$

\Procedure{Baseline}{s}
\Comment Select the best $w$ from $W_{safe}$.
\State $w \leftarrow \text{classify}(f(s))$
\If{$w \in W_{legal}(s)$}
    \State \textbf{return} $w$
\Else
  \State \textbf{return} $\underset{w \in W_{safe}}{\argmax}
\left(
  \prod_{s \in S_{training}} p(s, w)
\right)^{1/|S_{training}|}$
\EndIf
\EndProcedure
\item[] % line break

\Procedure{Random}{s}
\Comment Select a random workgroup size.
\State $w \leftarrow \text{classify}(f(s))$
\While{$w \not\in W_{legal}(s)$}
  \State $w \leftarrow $ random choice $w \in \left\{ w | w < W_{max}(s), w \not\in W_{refused}(s) \right\}$
\EndWhile
\State \textbf{return} $w$
\EndProcedure
\item[] % line break

\Procedure{NearestNeighbour}{s}
\Comment Select the closest workgroup size to prediction.
\State $w \leftarrow \text{classify}(f(s))$
\While{$w \not\in W_{legal}(s)$}
  \State $d_{min} \leftarrow \infty$
  \State $w_{closest} \leftarrow \text{null}$
  \For{$c \in \left\{ w | w < W_{\max}(s), w \not\in W_{refused}(s) \right\}$}
    \State $d \leftarrow \sqrt{\left(c_r - w_r\right)^2 + \left(c_c - w_c\right)^2}$
    \If{$d < d_{min}$}
      \State $d_{min} \leftarrow d$
      \State $w_{closest} \leftarrow c$
    \EndIf
  \EndFor
  \State $w \leftarrow w_{closest}$
\EndWhile
\State \textbf{return} $w$
\EndProcedure
\end{algorithmic}

  \caption{Selecting optimal workgroup sizes using classification}
  \label{alg:autotune-classification}
\end{algorithm}

\subsection{Predicting Stencil Code Runtime}\label{subsec:omnitune-ml-runtime}

\begin{algorithm}
  \begin{algorithmic}[1]
    \Require kernel features $k$, hardware features $h$, dataset features
    $d$.
    \Ensure workgroup size $w$

    \State $W \leftarrow \left\{ w | w < W_{\max}(s) \right\} - W_{refused}(s)$
    \Comment Set of possible workgroup sizes.
    \State $w \leftarrow \underset{w \in W}{\argmin} g(f(s), w)$
    \Comment Predict candidate workgroup size.
    \While{$w \not\in W_{legal}(s)$}
    \State $W \leftarrow W - \left\{ w \right\}$
    \Comment Remove candidate from set.
    \State $w \leftarrow \underset{w \in W}{\argmin} g(f(s), w)$
    \Comment Select next candidate workgroup size.
    \EndWhile
    \State \textbf{return} $w$
  \end{algorithmic}
  \caption{Selecting workgroup sizes by predicting program runtimes}
  \label{alg:autotune-runtime-regression}
\end{algorithm}

A problem of predicting oracle workgroup sizes is that it requires
each training point to be labelled with the oracle workgroup size
which can be only be evaluated using an exhaustive search. An
alternative approach is to build a model to attempt to predict the
\emph{runtime} of a stencil given a specific workgroup size. This
allows for training on data for which the oracle workgroup size is
unknown, and has the secondary advantage that this allows for an
additional training data point to be gathered each time a stencil is
evaluated. Given training data consisting of $(f(s),w,t)$ tuples,
where $s$ is a scenario, $w$ is the workgroup size, and $t$ is the
observed mean runtime, we can train a regressor $g(f(s), w)$ which
predicts the mean runtime of an unseen scenario. The predicted oracle
workgroup size $\bar{\Omega}(s)$ is then the $w$ value which minimises
the output of the regressor:
%
\begin{equation}
  \bar{\Omega}(s) = \underset{w \in W_{legal}(s)}{\argmin} g(f(s), w)
\end{equation}
%
Note that since we cannot know in advance which workgroup sizes will
be refused, that is, $W_{refused}(s)$ cannot be statically determined,
this process must be iterated until a workgroup size which not refused
is selected. Algorithm~\ref{alg:autotune-runtime-regression} shows
this process.


\subsection{Predicting Relative Performance of Workgroup Sizes}\label{subsec:omnitune-ml-speedup}

\begin{algorithm}
  \begin{algorithmic}[1]
    \Require kernel features $k$, hardware features $h$, dataset features
    $d$, baseline $w_b$.
    \Ensure workgroup size $w$

    \State $W \leftarrow \left\{ w | w < W_{\max}(s) \right\} - W_{refused}(s)$
    \Comment Set of possible workgroup sizes.
    \State $w \leftarrow \underset{w \in W}{\argmax} g(f(s), w,w_b)$
    \Comment Predict candidate workgroup size.
    \While{$w \not\in W_{legal}(s)$}
    \State $W \leftarrow W - \left\{ w \right\}$
    \Comment Remove candidate from set.
    \State $w \leftarrow \underset{w \in W}{\argmax} g(f(s), w,w_b)$
    \Comment Select next candidate workgroup size.
    \EndWhile
    \State \textbf{return} $w$
  \end{algorithmic}
  \caption{Selecting workgroup sizes by predicting relative performance}
  \label{alg:autotune-speedup-regression}
\end{algorithm}

Accurately predicting the runtime of an arbitrary program is a
difficult problem due to the impacts of flow control. In such cases,
it may be more effective to instead predict the \emph{relative}
performance of two different workgroup sizes for the same program. To
do this, we select a baseline workgroup size $w_b \in W_{safe}$, and
train a regressor $g(f(s),w,w_b)$ with training data labelled with the
relative performance over the baseline $r(w, w_b)$. Predicting the
optimal workgroup requires maximising the output of the regressor:
%
\begin{equation}
  \bar{\Omega}(s) = \underset{w \in W_{legal}(s)}{\argmax} g(f(s),w,w_b)
\end{equation}
%
As with predicting runtimes, this process must be iterated to
accommodate for the emergent properties of $W_{legal}(s)$. See
Algorithm~\ref{alg:autotune-speedup-regression} for a description of
this process.


\section{Implementation}

The OmniTune framework is implemented as a set of Python classes and
interfaces, which are inherited from or implemented to target specific
autotuning cases. The entire framework consists of 8987 lines of
Python code, of which 976 is dedicated to the SkelCL frontend. By
design, the client-server model minimises the impact of number of
modifications that are required to enable autotuning in client
applications. The only modification required is to replace the
hardcoded values for workgroup size with a subroutine to request a
workgroup size from the OmniTune server over a DBUS connection. The
server is implemented as a standalone Python program, and uses sqlite
to maintain local data caches. The OmniTune remote is an Amazon Web
Services virtual machine instance, using MySQL as the relational data
store. Figure~\ref{fig:omnitune-system-flow} shows the relational
database schema used to store stencil runtime information. Additional
tables store data in coalesced forms for use as machine learning
training data.

For classification, five classifiers are supported, chosen for their
contrasting properties: Naive Bayes, SMO, Logistic Regression, J48
Decision tree, and Random Forest. For regression, a Random Forest with
regression trees is used, chosen because of its efficient handling of
large feature sets, compared to linear models.

\begin{figure}
  \centering
  \includegraphics[width=.95\textwidth]{img/omnitune-data-schema.pdf}
  \caption[Database schema for storing performance results]{%
  Database schema for storing SkelCL stencil runtimes. Feature lookup
  tables and normalisation are used to provide extremely compact
  storage, requiring only 56 bytes for each additional runtime of a
  known stencil program.%
  }
  \label{fig:omnitune-system-flow}
\end{figure}


\section{Summary}

This section has described has the application of OmniTune for
predicting workgroup sizes of SkelCL stencil programs, using three
different machine learning approaches. The first approach is to
predict the optimal workgroup size for a given program based on
training data which included the optimal workgroup sizes for other
stencils. The second approach is to select a workgroup sizex by
sweeping the space of possible workgroup sizes, predicting the runtime
a program with each. The third approach is to select a workgroup size
by sweeping the space of possible workgroup sizes, predicting the
relative gain of each compared to a known baseline. In the next
section, we will describe the process for collecting empirical
performance data.
