As the trend towards higher core counts and increasing parallelism
continues, the need for high level, accessible abstractions to manage
such parallelism will continue to go. Autotuning proves a valuable aid
for achieving these goals, providing the benefits of low level
performance tuning while maintaining ease of use, without burdening
developers with optimisation concerns. As the need for autotuned
parallelism rises, the desire for collaborative techniques for sharing
performance data must be met with systems capable of supporting this
cross-platform learning.

In this thesis, I have presented my attempt at providing such a
system, by designing a novel framework which has the benefits of fast,
``always-on'' autotuning, while being able to synchronise data with
global repositories of knowledge which others may contribute to. The
framework provides an interface for autotuning which is sufficiently
generic to be easily re-purposed to target a range of optimisation
parameters.

To demonstrate the utility of this framework, I implemented a frontend
for predicting the workgroup size of OpenCL kernels for SkelCL stencil
codes. This optimisation space is complex, non linear, and critical
for the performance of stencil kernels, with up to a $207.72\times$
slowdown if an improper value is picked. Selecting the correct
workgroup size is difficult --- requiring a knowledge of the kernel,
dataset, and underlying architecture. The challenge is increased even
more so by inconsistencies in the underlying system which cause some
workgroup sizes to fail completely. Of the 269813 combinations of
workgroup size, device, program, and dataset tested; only a
\emph{single} workgroup size was valid for all test cases, and
achieved only 24\% of the available performance. The value selected by
human experts was invalid for 2.6\% of test cases. Autotuning in this
space requires a system which is resilient these challenges, and
several techniques were implemented to address them.

Runtime performance of autotuned stencil kernels is very promising,
achieving an average 90\% of the available performance with only a 3ms
autotuning overhead. Even ignoring the cases for which the human
expert selected workgroup size is invalid, this provides a
$1.33\times$ speedup, or a $5.57\times$ speedup over the best
performance that can be achieved using static tuning. Classification
performance is comparable when predicting workgroup sizes for both
unseen programs and unseen devices. I believe that the combination of
performance improvements and the collaborative nature of OmniTune
makes for a compelling case for the use of autotuning as a key
component for enabling performant, high level parallel programming.


\section{Critical Analysis}

This section contains a critical analysis of the work presented in
previous chapters.

\subsubsection{OmniTune Framework}

The purpose of the OmniTune framework is to provide a generic
interface for runtime autotuning. This is demonstrated through the
implementation of a SkelCL frontend; however, to truly evaluate the
ease of use of this framework, it would have been preferable to
implement one or more additional autotuning frontends, to target
different optimisation spaces. This could expose any leakages in the
abstractions between the SkelCL-specific and generic autotuning
components.


\subsubsection{Synthetic Benchmarks}

The OmniTune SkelCL frontend provides a template substitution engine
for generating synthetic stencil benchmarks. The implementation of
this generator is rigidly tied to the SkelCL stencil format. It would
be preferred if this template engine was made more flexible, to
support generation of arbitrary test programs. Additionally, due to
time constraints, I did not have the opportunity to explore how the
number of synthetic benchmarks in machine learning test data sets
affects classification performance.

One possible use of the synthetic stencil benchmark generator could be
for creating minimal test cases of refused OpenCL parameters so that
bug reports could be filed with the relevant implementation
vendor. However, this would have added a great level of complexity to
the the generator, as it would have to isolate and remove the
dependency on SkelCL to generate minimal programs, requiring
significant implementation work.


\subsubsection{Use of Machine Learning}

The evaluation of OmniTune in this thesis uses multiple classifiers
and regressors to predict workgroup sizes. The behaviour of these
classifiers and regressors is provided by the Weka data mining
suite. Many of these classifiers have parameters which affect their
prediction behaviour. The quality of the evaluation could have been
improved by exploring the effects that changing the values of these
parameters has on the OmniTune classification performance. It would
also have been informative to dedicate a portion of the evaluation to
feature engineering, evaluating the information gain of each feature
and exploring the effects of feature transformations on classification
performance.


\subsubsection{Evaluation Methodology}

The evaluation compares autotuning performance against the best
possible performance that can be achieved using static tuning, a
simple heuristic to tune workgroup size on a per-device basis, and
against the workgroup size chosen by human experts. It would have been
beneficial to also include a comparison of the performance of these
autotuned stencils against hand-crafted equivalent programs in pure
OpenCL, without using the SkelCL framework. This would allow a direct
comparison between the performance of stencil kernels using high level
and low level abstractions, but could not be completed due to time
constraints and difficulties in acquiring suitable comparison
benchmarks and datasets.


\section{Future Work}

Future work can be divided into two categories: continued development
of OmniTune, and extending the behaviour of the SkelCL autotuner.

The cost of offline training with OmniTune could be reduced by
exploring the use of adaptive sampling plans, such as presented
in~\cite{Leather2009}. This could reduce the number of runtime samples
required to distinguish good from bad optimisation parameter values.

Algorithm~\ref{alg:autotune-hybrid} proposes the behaviour of a hybrid
approach to selecting the workgroup size of iterative SkelCL
stencils. This approach attempts to exploit the advantages of all of
the techniques presented in this thesis. First, runtime regression is
used to predict the minimum runtime and a candidate workgroup
size. If, after evaluating this workgroup size, the predicted runtime
turned out to be inaccurate, then a prediction is made using speedup
regression. Such a hybrid approach would enable online tuning through
the continued acquisition of runtime and speedup performance, which
would compliment the collaborative aspirations of OmniTune, and the
existing server-remote infrastructure.

Other skeleton optimisation parameters could be autotuned by SkelCL,
including higher level optimisations such as the selection of border
region loading strategy, or selecting the optimal execution device(s)
for multi-device systems. Optimisation parameters of additional
skeletons could be autotuned, or the interaction of multiple related
optimisation parameters could be explored. Power consumption could be
used as an additional optimisation cotarget.

\begin{algorithm}
  \begin{algorithmic}[1]
\Require kernel features $k$, hardware features $h$, dataset features $d$.
\Ensure workgroup size $w$

\State $r \leftarrow \underset{w \in W_{legal}(s)}{\min} f(k,h,d,w)$
\Comment Predict minimum runtime.
\State $w \leftarrow \underset{w \in W_{legal}(s)}{\argmin} f(k,h,d,w)$
\Comment Workgroup size for $r$.
\State $t_r \leftarrow$ measure runtime of program with $w$
\State \textsc{Submit}$\left( f(s), w, t_r \right)$
\Comment Submit observed runtime
\If{$t_r \approx r$}
  \State \textbf{return} $w$
\Comment Predicted runtime is accurate.
\Else
   \State $W \leftarrow \left\{ w | w < W_{\max}(s) \right\}$
   \State converged $\leftarrow$ false
   \State $w_b \leftarrow$ baseline parameter
   \State $t_b \leftarrow$ measure runtime of runtime of program with
   $w_b$
   \State \textsc{Submit}$\left( f(s), w_b, t_b \right)$
   \Comment Submit observed runtime
   \While{not converged}
     \State $s \leftarrow \underset{w \in W}{\max} g(k,h,d,w)$
     \Comment Predict best speedup.
     \State $w \leftarrow \underset{w \in W}{\argmax} g(k,h,d,w)$
     \Comment Workgroup size for $s$.
     \State $t \leftarrow$ measure runtime of program with $s$
     \State \textsc{Submit}$\left( f(s), w, t \right)$
     \Comment Submit observed runtime
     \State $s_w \leftarrow t_t / t$
     \State \textsc{Submit}$\left( f(s), w, s_w \right)$
     \Comment Submit observed speedup
     \If{$s_w \approx s$}
       \State converged = true
       \Comment Predicted speedup is accurate.
     \Else
       \State $W = W - \{ w \}$
     \EndIf
   \EndWhile
   \State \textbf{return} $w$
\EndIf
\end{algorithmic}

  \caption{%
  Selecting workgroup size using a combination of classifiers and
  regressors.%
  }
  \label{alg:autotune-hybrid}
\end{algorithm}
