% DEADLINE: 12 NOON, FRIDAY, 21ST AUGUST 2015
%
% The project is only assessed on the basis of a final written
% thesis. Additional material, such as the code you submit, may be taken
% into account in case of doubt, but you should make sure that all the
% work you have done is carefully described in the thesis. Theses will
% typically conform to the following format:
%
% * The length should be 40 -- 70 pages in total and no shorter than 35
%   pages.
%
% * Title page with abstract.
%
% * Introduction: an introduction to the document, clearing stating
%   the hypothesis or objective of the project, motivation for the work
%   and the results achieved. The structure of the remainder of the
%   document should also be outlined.
%
% * Background: background to the project, previous work, exposition of
%   relevant literature, setting of the work in the proper context. This
%   should contain sufficient information to allow the reader to
%   appreciate the contribution you have made.
%
% * Description of the work undertaken: this may be divided into
%   chapters describing the conceptual design work and the actual
%   implementation separately. Any problems or difficulties and the
%   suggested solutions should be mentioned. Alternative solutions and
%   their evaluation should also be included.
%
% * Analysis or Evaluation: results and their critical analysis should
%   be reported, whether the results conform to expectations or
%   otherwise and how they compare with other related work. Where
%   appropriate evaluation of the work against the original objectives
%   should be presented.
%
% * Conclusion: concluding remarks and observations, unsolved problems,
%   suggestions for further work.  Bibliography.
%
% In addition, the thesis must be accompanied by a statement declaring
% that the student has read and understood the University's plagiarism
% guidelines. Students should budget at least six weeks for the final
% thesis writing-up phase. Where appropriate the thesis may additionally
% contain appendices in which relevant program listings, experimental
% data, circuit diagrams, formal proofs, etc. may be included. However,
% students should keep in mind that they are marked on the quality of
% the thesis, not its length. The thesis must be word-processed using
% either LaTeX or a system with similar capabilities. The LaTeX thesis
% template can be found via the local packages web page. You don't have
% to use these packages, but your thesis must match the style (i.e.,
% font size, text width etc) shown in the sample output for an
% Informatics thesis. Technical problems during project work are only
% considered for resources we provide; no technical support,
% compensation for lost data, extensions for time lost due to technical
% problems with external hard- and software as provided will be given,
% except where this is explicitly stated as part of a project
% specification and adequately resourced at the start of the project.


%%%%%%%%%%%%%%%%%%%%%%
%% Document details %%
%%%%%%%%%%%%%%%%%%%%%%
% A two page document describing lessons learned from the project, the
% revised PhD vision and insights, and a plan of activities for the
% next three months
%
\documentclass[11pt]{article}

\author{Chris Cummins}
\date{September 2015}
\title{PhD Progression Review: Autotuning\\and Skeleton-aware Compilation}

\documentclass[prodmode,acmtaco]{acmsmall}

% Package to generate and customize Algorithm as per ACM style
\usepackage[ruled]{algorithm2e}
\renewcommand{\algorithmcfname}{ALGORITHM}
\SetAlFnt{\small}
\SetAlCapFnt{\small}
\SetAlCapNameFnt{\small}
\SetAlCapHSkip{0pt}
\IncMargin{-\parindent}

% Metadata Information
\acmVolume{9}
\acmNumber{4}
\acmArticle{39}
\acmYear{2016}
\acmMonth{3}

% Copyright
%\setcopyright{acmcopyright}
\setcopyright{acmlicensed}
%\setcopyright{rightsretained}
%\setcopyright{usgov}
%\setcopyright{usgovmixed}
%\setcopyright{cagov}
%\setcopyright{cagovmixed}

% DOI
\doi{0000001.0000001}

%ISSN
\issn{1234-56789}


%%%%%%%%%%%%%%%%%%%%%%%%%
%% Document and Layout %%
%%%%%%%%%%%%%%%%%%%%%%%%%

% Fix for multiple "No room for a new \dimen" errors.
%
% See: http://tex.stackexchange.com/questions/38607/no-room-for-a-new-dimen
%
\usepackage{etex}

\usepackage[utf8]{inputenc}

% Fix for "'babel/polyglossia' detected but 'csquotes' missing"
% warning. NOTE: Include after inputenc.
%
\usepackage{csquotes}

% Make internal macro definitions accessible,
% e.g. \@title, \@date \@author.
\makeatletter

% Multi-column support.
\usepackage{multicol}

% A useful package which includes macros like \ifdef{}{}{}:
%
\usepackage{etoolbox}

% Uncomment the following line to remove column separation:
%
%\setlength{\columnsep}{5mm}

% Allow user-defined warning and error filters.
%
\usepackage{silence}

\usepackage{adjustbox}


%%%%%%%%%%%%%%%%%%%%%
% Table of Contents %
%%%%%%%%%%%%%%%%%%%%%

% % Set chapter and section numbering depth:
% %
% \setcounter{secnumdepth}{2}


%%%%%%%%%%%%%%%%
% Bibliography %
%%%%%%%%%%%%%%%%
% \usepackage[%
%     backend=biber,
%     style=ieee,
%     % style=numeric-comp,
%     % style=numeric-comp,  % numerical-compressed
%     sorting=none,        % nty,nyt,nyvt,anyt,anyvt,ynt,ydnt,none
%     sortcites=true,      % sort \cite{b a d c}: true,false
%     block=none,          % space between blocks: none,space,par,nbpar,ragged
%     indexing=false,      % indexing options: true,false,cite,bib
%     citereset=none,      % don't reset cites
%     isbn=false,          % print ISBN?
%     url=true,            % print URL?
%     doi=false,           % print DOI?
%     natbib=true,         % natbib compatability
%    ]{biblatex}

% \usepackage{natbib}

% % Filter annoying and unavoidable biblatex warning:
% \WarningFilter{biblatex}{Patching footnotes failed}

% Reduce the font size of the bibliography:
% \renewcommand{\bibfont}{\normalfont\scriptsize}

% Determine which BibTeX file to use:
%
% If available, use my Mendeley BibTex library, located in the home
% directory. Note that this is a relative path and will break if
% either this file or the BibTex library are moved. If the library is
% not present, use the local refs.bib file.
% \newcommand{\BibResourceGlobal}{../../../library.bib}
% \newcommand{\BibResourceLocal}{refs.bib}

% \IfFileExists{\BibResourceGlobal}
%   {\newcommand{\BibResource}{\BibResourceGlobal}}
%   {\newcommand{\BibResource}{\BibResourceLocal}}

% \addbibresource{\BibResource}


%%%%%%%%%%%%%%
% Appendices %
%%%%%%%%%%%%%%

% Appendix package. Documentation:
%
%  http://mirror.ox.ac.uk/sites/ctan.org/macros/latex/contrib/appendix/appendix.pdf
%
% Package options:
%
% toc      - Put a header (e.g., `Appendices') into the Table of Contents
%            (the ToC) before listing the appendices. (This is done by
%            calling the \addappheadtotoc command.)
% page     - Puts a title (e.g., `Appendices') into the document at the
%            point where the appendices environment is begun. (This is
%            done by calling the \appendixpage command.)
% title    - Adds a name (e.g., `Appendix') before each appendix title in
%            the body of the document. The name is given by the value
%            of \appendixname. Note that this is the default behaviour
%            for classes that have chapters.
% titletoc - Adds a name (e.g., `Appendix') before each appendix listed
%            in the ToC. The name is given by the value
%            of \appendixname.
% header   - Adds a name (e.g., `Appendix') before each appendix in page
%            headers.  The name is given by the value
%            of \appendixname. Note that this is the default behaviour
%            for classes that have chapters.
\usepackage[title, titletoc]{appendix}

% pre-requisites for rendering upquotes in listings package.
\usepackage[T1]{fontenc}
\usepackage{lmodern}
\usepackage{textcomp}

% code listings.
\usepackage{listings}

% set \ttfamily to use courier fonts.
%
% See: http://tex.stackexchange.com/a/33686
%
\usepackage{courier}

\lstset{frame=bt,                    % Add top and bottom frame lines
breaklines=true,             % Force line wrapping
captionpos=b,                % Place caption below listing
numbers=left,                % Add left-side line numbers
basicstyle=\scriptsize\ttfamily, % Set font size and type
showstringspaces=false,      % Don't show visible whitespace
numberstyle=\tiny,
upquote=true,                % Use upright quotes, not curly
commentstyle=\bfseries}      % Embolden comments

% Use (*@ @*) to escape LaTeX commands within listings.
\lstset{escapeinside={(*@}{@*)}}

% Add 10pt space between chapters in TOC listings entries:
%\let\Chapter\chapter
%\def\chapter{\addtocontents{lol}{\protect\addvspace{10pt}}\Chapter}


%%%%%%%%%%%%%%%%%%%%%%%%
%% Graphics and maths %%
%%%%%%%%%%%%%%%%%%%%%%%%
\usepackage{amsmath}

% Vector notation, e.g. \vv{x}:
%
\usepackage{esvect}

% Additional amsmath symbols, see:
%
% http://texblog.org/2007/08/27/number-sets-prime-natural-integer-rational-real-and-complex-in-latex/
%
\usepackage{amsfonts}
\usepackage{amssymb}

\usepackage{graphicx}
\usepackage{mathtools}
\usepackage{tikz}
\usepackage{tikz-qtree}

% Provide bold font face in maths.
\usepackage{bm}

\usepackage{subcaption}
\expandafter\def\csname ver@subfig.sty\endcsname{}

% Define an 'myalignat' command which behave as 'alignat' without the
% vertical top and bottom padding. See:
%     http://www.latex-community.org/forum/viewtopic.php?f=5&t=1890
\newenvironment{myalignat}[1]{%
\setlength{\abovedisplayskip}{-.7\baselineskip}%
\setlength{\abovedisplayshortskip}{\abovedisplayskip}%
\start@align\z@\st@rredtrue#1
}%
{\endalign}

% Define additional operators:
\DeclareMathOperator*{\argmin}{arg\,min}
\DeclareMathOperator*{\argmax}{arg\,max}

\DeclareMathOperator*{\gain}{Gain}

% Skeleton operators.
\DeclareMathOperator*{\map}{Map}
\DeclareMathOperator*{\reduce}{Reduce}
\DeclareMathOperator*{\scan}{Scan}
\DeclareMathOperator*{\stencil}{Stencil}
\DeclareMathOperator*{\zip}{Zip}
\DeclareMathOperator*{\allpairs}{All\,Pairs}

% Maths plots using pgfplots, see:
%
%     http://pgfplots.sourceforge.net/pgfplots.pdf
%
\usepackage{pgfplots}

% Disable compatability mode.
%
\pgfplotsset{compat=1.12}

% Gantt charts using pgfgantt, see:
%
%     http://www.ctan.org/pkg/pgfgantt
%
\usepackage{pgfgantt}

% Fix milestone aspect ratio by defining a custom element.
\newganttchartelement*{mymilestone}{
mymilestone/.style={
shape=diamond,
inner sep=2pt,
draw=black,
top color=black,
bottom color=black,
}
}

% Tikz flowchart configuration.
\usetikzlibrary{shapes,arrows,shadows,fit,backgrounds}
\tikzstyle{decision} = [diamond,
draw,
text width=4.5em,
text badly centered,
node distance=3cm,
inner sep=0pt]
\tikzstyle{block}    = [rectangle,
draw,
text width=5em,
text centered,
node distance=3cm,
minimum height=4em,
inner sep=.2cm]
\tikzstyle{line}     = [draw, -latex']

% Add dirtree picture style, see:
%
%     http://tex.stackexchange.com/a/34268
%
\newcount\dirtree@lvl
\newcount\dirtree@plvl
\newcount\dirtree@clvl
\def\dirtree@growth{%
\ifnum\tikznumberofcurrentchild=1\relax
\global\advance\dirtree@plvl by 1
\expandafter\xdef\csname dirtree@p@\the\dirtree@plvl\endcsname{\the\dirtree@lvl}
\fi
\global\advance\dirtree@lvl by 1\relax
\dirtree@clvl=\dirtree@lvl
\advance\dirtree@clvl by -\csname dirtree@p@\the\dirtree@plvl\endcsname
\pgf@xa=0.33cm\relax
\pgf@ya=-\baselineskip\relax
\pgf@ya=\dirtree@clvl\pgf@ya
\pgftransformshift{\pgfqpoint{\the\pgf@xa}{\the\pgf@ya}}%
\ifnum\tikznumberofcurrentchild=\tikznumberofchildren
\global\advance\dirtree@plvl by -1
\fi
}
\tikzset{
dirtree/.style={
growth function=\dirtree@growth,
every node/.style={anchor=north},
every child node/.style={anchor=west},
edge from parent path={(\tikzparentnode\tikzparentanchor) |- (\tikzchildnode\tikzchildanchor)}
}
}

% UML sequence diagram macros, see:
%
%     https://code.google.com/p/pgf-umlsd/
%
% Options:
%
%     underline - Underline object names
%
\usepackage[underline=false]{pgf-umlsd}

% Support for SVG graphics.
%
% NOTE that you must pass the "--shell-escape" argument to pdflatex to
% compile. NOTE also that images *MUST* be placed within the graphics
% path.
\usepackage{svg}
\graphicspath{{img/}}

%%%%%%%%%%%%%%%%%%%%%%
%% Tables and lists %%
%%%%%%%%%%%%%%%%%%%%%%

% Required to use labm8 exported tables.
%
\usepackage{booktabs}

% Required for full page-width tables.
\usepackage{tabularx}

%\usepackage{enumitem}
%\setenumerate{itemsep=0pt}

% Use no left margin for lists:
%\setlist{leftmargin=*}

\usepackage{longtable}

% Define column types L, C, R with known text justification and fixed
% widths:
\usepackage{array}
\newcolumntype{L}[1]{>{\raggedright\let\newline\\\arraybackslash\hspace{0pt}}m{#1}}
\newcolumntype{C}[1]{>{\centering\let\newline\\\arraybackslash\hspace{0pt}}m{#1}}
\newcolumntype{R}[1]{>{\raggedleft\let\newline\\\arraybackslash\hspace{0pt}}m{#1}}


%%%%%%%%%%%%%%%%%%%%%%%%%%%%%
%% Typesetting and symbols %%
%%%%%%%%%%%%%%%%%%%%%%%%%%%%%

% Adjustable font sizes in \Verbatim{}
\usepackage{fancyvrb}

%\usepackage{titlesec}
% Set section and paragraph heading fonts:
%\titleformat*{\section}{\Large\bfseries}
%\titleformat*{\subsection}{\normalsize\bfseries}
%\titleformat*{\subsubsection}{\normalsize}
%\titleformat*{\paragraph}{\large\bfseries}
%\titleformat*{\subparagraph}{\large\bfseries}

% Set section heading margins. Usage:
% \titlespacing*{<command>}{<left>}{<before>}{<after>}
%\titlespacing*{\section}{0pt}{.6em}{.3em}
%\titlespacing*{\subsection}{0pt}{.6em}{.2em}

% Set paragraph indentation size. Default is 15pt.
%\setlength{\parindent}{10pt}

% The line spacing can be globally set using \linespread:
%
% \linespread{1.2}

% Add a command \hr{} which will draw a horizontal rule the width of
% the text.
%
\newcommand{\hr}{\noindent\makebox[\linewidth]{\rule{\textwidth}{0.2pt}}}

% Add a command \br{} which will create a horizontal space of exactly
% one line height.
%
\newcommand{\br}{\hspace{\baselineskip}}

% Define a command to allow word breaking.
\newcommand*\wrapletters[1]{\wr@pletters#1\@nil}
\def\wr@pletters#1#2\@nil{#1\allowbreak\if&#2&\else\wr@pletters#2\@nil\fi}

% Define a command to create centred page titles.
\newcommand{\centredtitle}[1]{
\begin{center}
  \large
  \vspace{0.9cm}
  \textbf{#1}
\end{center}}

% Support hyperlinks using the \hyperref, \url and \href
% macros. Usage:
%
%    \hyperref[label_name]{''link text''}
%
%    \url{<my_url>}
%
%    \href{<my_url>}{<description>}
%
\usepackage{hyperref}

% Disable colored borders of links, cross-references etc in PDF output
\hypersetup{pdfborder={0 0 0}}

% Provide generic commands \degree, \celsius, \perthousand, \micro
% and \ohm which work both in text and maths mode.
\usepackage{gensymb}

%%%%%%%%%%%%%%%%%%%%%%%%%%%%%%%%%
%% Placeholder text generation %%
%%%%%%%%%%%%%%%%%%%%%%%%%%%%%%%%%

% Use either \blindtext or \libpsum to generate placeholder text. Also
% note the macros \blinditemize, \blindenumerate, \blinddescription.
\usepackage[english]{babel}
\usepackage{blindtext}
\usepackage{lipsum}


\begin{document}

\begin{center}
\Large
\textbf{PhD Progression Review: Autotuning\\and Skeleton-aware Compilation}
\vspace{0.2cm}

\normalsize
Chris Cummins\\
September 2015
\vspace{0.2cm}
\end{center}
%\maketitle

\begin{abstract}
\noindent
I present a research proposal to develop an optimising compiler
specifically tailored for algorithmic skeletons. Such a compiler would
be capable of performing optimisations which are not possible by
library-level skeleton frameworks: optimisations across skeleton calls
(e.g.\ to re-order the execution of nested skeletons), and across
muscle functions (e.g.\ to perform load balancing of pipeline
stages). This document briefly outlines the proposed project, and
summarises the findings from the MSc phase of my CDT.
\end{abstract}

\section{Introduction}

In my initial CDT application, I postulated that in order to address
the under-utilisation of parallel resources that is a symptom of the
modern hardware/software ecosystem, the goal should be to move towards
\emph{automated} parallelism. My MSc thesis worked towards this goal
by developing a system for adaptive tuning of patterns (algorithmic
skeletons) for high level parallel programming. Skeletons provide
abstractions that free developers from the concerns of managing
parallel resources. The need for adaptive tuning of skeletons was
demonstrated by the significant performance improvements which were
achieved through autotuning of only a single two-dimensional parameter
space for stencil skeletons. I demonstrated a median geometric speedup
of $3.79\times$ across benchmarks from both real world and
synthetically generated sources. My intention over the coming months
is to broaden the scope of this work to consider the interaction
between algorithmic skeletons and optimising compilers.

% The first year of this CDT was spent investigating autotuning
% algorithmic skeletons for stencil codes on CPUs and GPUs. This was a
% challenging but rewarding project which providing me with a
% grounding in research approaches to high level parallel programming,
% GPU parallelism, autotuning, and machine learning. In this document
% I critically evaluate my work during this MSc phase, and propose a
% direction for future research which builds upon the knowledge and
% skills acquired in the first year, while addressing the challenges I
% faced.

% The broad scope of the proposal for my initial application was to
% address the parallel programmability challenge, by enabling a shift
% towards automatic parallelism. During my first three months the scope
% of this proposal was adjusted to focus on improving the high level
% parallel programming by adaptive tuning. To this end, my MSc thesis
% investigated adaptive tuning of a parameter space for skeletal
% programming. It is my intention to continue working towards this goal
% by focusing next on the interaction between compilers and algorithmic
% skeletons.

% I’ve shown that the optimisation space of a fixed implementation can
% be successfully tuned using machine learning. The next step is to
% combine these techniques with an understanding of optimising
% compilers in order to generate adaptive implementations.


\section{MSc Phase Review}

My MSc thesis introduced \textit{OmniTune}, a framework for runtime
autotuning of parameters. The effectiveness of this approach was
demonstrated using the workgroup size of stencil skeletons for CPUs
and GPUs. Machine learning models trained using data collected from
synthetically generated benchmarks were evaluated for prediction
quality in this two-dimensional parameter space.

The plan for the project consisted of three sequential phases:
identifying a profitable optimisation space, probing the space to
gather empirical performance data, and developing a system to exploit
this optimisation space. In reality, there was a great overlap between
the progress of each phase: identifying the optimisation space was
achieved using empirical performance data, and in turn gathering
performance data required the partial development of the autotuning
system. Broadly, the stencil optimisations space had been identified
by early March, the collection of empirical performance data began in
mid April, and the development of the final OmniTune autotuning system
began in mid May. For future work, a methodology based upon shorter,
iterative cycles of testing, implementing, and evaluating should be
used. This should also be coupled with regular write-ups of current
findings and results, in addition to the logbook I kept this
year. Another insight which can be transferred to future work is the
importance of clear, testable hypotheses, which should be identified
during early planning
stages. % All source code and notes were hosted publicly on GitHub.

% - short development cycles, with milestones and reviews at each stage
%
% - a more clearly testable hypothesis - i.e. a less vague project plan


\section{%Beginning the
PhD Phase}

While the tuning of parameter values demonstrably has a significant
impact on the performance of skeleton programs, there is great scope
for optimisation at the compilation stage. The complexity of parallel
programs restricts the number of optimisations which compilers may
apply, and furthermore there are higher-level restructurings and
implementation choices which could be made by compilers equipped with
an understanding of the semantics of skeleton operations.

\subsection{Objectives}

The objective of compiler research for algorithmic skeletons would be
to enable ``skeleton-aware compilation''. This can be considered both
as the enabling of existing compiler optimisations to algorithmic
skeletons which cannot currently be applied due to limitations in the
static analysis of parallel programs, and the development of novel
optimisations which are specific to algorithmic skeletons. In both
cases, skeleton-aware compilation will be able to provide measurable
performance improvements of programs compiled using the existing state
of the art in optimising compilers.


\subsection{Methodology}

This research project will begin with an exposition of the relevant
literature to help enumerate the range of possible optimisations which
could be applied, and to identify what has already been achieved.
When identifying a range of optimisations which could be developed,
low cost proof-of-concept tests can be performed for each by manually
performing an optimisation by hand for a specific benchmark and
gathering empirical performance data. If the optimisation leads to
measurable performance improvements, a transformation pass in a
relevant compiler (e.g.\ LLVM) can be implemented to automatically
apply this optimisation. Test programs may be selected from existing
benchmark suites (e.g. the Intel Thread Building Blocks port of
PARSEC), or by using synthetically generated programs, perhaps by
extending the benchmark generator developed for my MSc project. In
case of optimisations for which the potential benefit cannot
statically be determined (e.g.\ load balancing between stages of a
pipeline), the OmniTune framework may be extended to provide
autotuning capabilities for the space of possible optimisations.


\section{Summary}

In my first year I have demonstrated the importance of adaptive tuning
for maximising the performance of high level parallel patterns. I
intend to continue improving the performance of high level parallel
patterns, but taking an approach from the level of the compiler, with
the aim of developing skeleton-aware compilation. The skills I
acquired this year in skeletal programming, heterogeneous parallelism,
machine learning, and autotuning will all be required for achieving
this aim. The success of the project can be quantifiably measured in
terms of performance improvements to real world skeleton programs.


\label{bibliography}
\printbibliography

\end{document}


%%%%%%%%%%%%%%%%%%%%%%%%%%%%%%%
% Draft Version Specification %
%%%%%%%%%%%%%%%%%%%%%%%%%%%%%%%
% Comment out the following line to disable all draft features:
\def\draft{}
\def\version{2015-08-14}

%%%%%%%%%%%%%%%%%%%%%
%% Draft versions  %%
%%%%%%%%%%%%%%%%%%%%%

% A set of macros for adding "todo" and rough notes to
% documents. Usage: \todo{Todo note}.

% Pretty colours plz.
\usepackage{color}

% \ifdraft{} and \ifndef{} commands.
\newcommand{\ifdraft}[1]{\ifdef{\draft}{#1}{}}
\newcommand{\ifndraft}[1]{\ifdef{\draft}{}{#1}}

% \todo{} command.
\newcommand{\todo}[1]{\ifdraft{\noindent\textcolor{red}{\em\footnotesize#1}}}

% \TODO{} command.
\newcommand{\TODO}[1]{\ifdraft{\todo{\textbf{TODO:} #1}}}

% \note{} command.
\newcommand{\note}[1]{%
  \ifdraft{\noindent\textcolor{blue}{\em\footnotesize note: #1}}%
}

% \checkme{} command.
\newcommand{\checkme}[1]{%
  \ifdraft{\textcolor{blue}{#1}}%
  \ifndraft{#1}%
}

% \fixme{} command.
\newcommand{\fixme}[1]{%
  \noindent\textcolor{red}{\em\footnotesize#1}%
  \ifndraft{\PackageWarning{draft}{fixme: #1}}%
}

% \FIXME{} command.
\newcommand{\FIXME}[1]{%
  \todo{\textbf{FIXME:} #1}%
  \ifndraft{\PackageWarning{draft}{FIXME: #1}}%
}

% \xref{} command.
\newcommand{\xref}[1]{%
  \ifdraft{\textcolor{red}{\em (xref: #1)}}%
  \ifndraft{\PackageWarning{draft}{Missing xref: #1}\textbf{???}}%
}

% \CitationNeeded{} placeholder.
\newcommand{\CitationNeeded}{%
  \ifdraft{\textsuperscript{\textcolor{red}{\em (Citation needed)}}}%
  \ifndraft{\PackageWarning{draft}{Citation needed}}%
}


% Paper title
\title{Autotuning Stencils Codes\\with Algorithmic Skeletons}

% Author
\author{Chris Cummins}

\documentclass[prodmode,acmtaco]{acmsmall}

% Package to generate and customize Algorithm as per ACM style
\usepackage[ruled]{algorithm2e}
\renewcommand{\algorithmcfname}{ALGORITHM}
\SetAlFnt{\small}
\SetAlCapFnt{\small}
\SetAlCapNameFnt{\small}
\SetAlCapHSkip{0pt}
\IncMargin{-\parindent}

% Metadata Information
\acmVolume{9}
\acmNumber{4}
\acmArticle{39}
\acmYear{2016}
\acmMonth{3}

% Copyright
%\setcopyright{acmcopyright}
\setcopyright{acmlicensed}
%\setcopyright{rightsretained}
%\setcopyright{usgov}
%\setcopyright{usgovmixed}
%\setcopyright{cagov}
%\setcopyright{cagovmixed}

% DOI
\doi{0000001.0000001}

%ISSN
\issn{1234-56789}


%%%%%%%%%%%%%%%%%%%%%%%%%
%% Document and Layout %%
%%%%%%%%%%%%%%%%%%%%%%%%%

% Fix for multiple "No room for a new \dimen" errors.
%
% See: http://tex.stackexchange.com/questions/38607/no-room-for-a-new-dimen
%
\usepackage{etex}

\usepackage[utf8]{inputenc}

% Fix for "'babel/polyglossia' detected but 'csquotes' missing"
% warning. NOTE: Include after inputenc.
%
\usepackage{csquotes}

% Make internal macro definitions accessible,
% e.g. \@title, \@date \@author.
\makeatletter

% Multi-column support.
\usepackage{multicol}

% A useful package which includes macros like \ifdef{}{}{}:
%
\usepackage{etoolbox}

% Uncomment the following line to remove column separation:
%
%\setlength{\columnsep}{5mm}

% Allow user-defined warning and error filters.
%
\usepackage{silence}

\usepackage{adjustbox}


%%%%%%%%%%%%%%%%%%%%%
% Table of Contents %
%%%%%%%%%%%%%%%%%%%%%

% % Set chapter and section numbering depth:
% %
% \setcounter{secnumdepth}{2}


%%%%%%%%%%%%%%%%
% Bibliography %
%%%%%%%%%%%%%%%%
% \usepackage[%
%     backend=biber,
%     style=ieee,
%     % style=numeric-comp,
%     % style=numeric-comp,  % numerical-compressed
%     sorting=none,        % nty,nyt,nyvt,anyt,anyvt,ynt,ydnt,none
%     sortcites=true,      % sort \cite{b a d c}: true,false
%     block=none,          % space between blocks: none,space,par,nbpar,ragged
%     indexing=false,      % indexing options: true,false,cite,bib
%     citereset=none,      % don't reset cites
%     isbn=false,          % print ISBN?
%     url=true,            % print URL?
%     doi=false,           % print DOI?
%     natbib=true,         % natbib compatability
%    ]{biblatex}

% \usepackage{natbib}

% % Filter annoying and unavoidable biblatex warning:
% \WarningFilter{biblatex}{Patching footnotes failed}

% Reduce the font size of the bibliography:
% \renewcommand{\bibfont}{\normalfont\scriptsize}

% Determine which BibTeX file to use:
%
% If available, use my Mendeley BibTex library, located in the home
% directory. Note that this is a relative path and will break if
% either this file or the BibTex library are moved. If the library is
% not present, use the local refs.bib file.
% \newcommand{\BibResourceGlobal}{../../../library.bib}
% \newcommand{\BibResourceLocal}{refs.bib}

% \IfFileExists{\BibResourceGlobal}
%   {\newcommand{\BibResource}{\BibResourceGlobal}}
%   {\newcommand{\BibResource}{\BibResourceLocal}}

% \addbibresource{\BibResource}


%%%%%%%%%%%%%%
% Appendices %
%%%%%%%%%%%%%%

% Appendix package. Documentation:
%
%  http://mirror.ox.ac.uk/sites/ctan.org/macros/latex/contrib/appendix/appendix.pdf
%
% Package options:
%
% toc      - Put a header (e.g., `Appendices') into the Table of Contents
%            (the ToC) before listing the appendices. (This is done by
%            calling the \addappheadtotoc command.)
% page     - Puts a title (e.g., `Appendices') into the document at the
%            point where the appendices environment is begun. (This is
%            done by calling the \appendixpage command.)
% title    - Adds a name (e.g., `Appendix') before each appendix title in
%            the body of the document. The name is given by the value
%            of \appendixname. Note that this is the default behaviour
%            for classes that have chapters.
% titletoc - Adds a name (e.g., `Appendix') before each appendix listed
%            in the ToC. The name is given by the value
%            of \appendixname.
% header   - Adds a name (e.g., `Appendix') before each appendix in page
%            headers.  The name is given by the value
%            of \appendixname. Note that this is the default behaviour
%            for classes that have chapters.
\usepackage[title, titletoc]{appendix}

% pre-requisites for rendering upquotes in listings package.
\usepackage[T1]{fontenc}
\usepackage{lmodern}
\usepackage{textcomp}

% code listings.
\usepackage{listings}

% set \ttfamily to use courier fonts.
%
% See: http://tex.stackexchange.com/a/33686
%
\usepackage{courier}

\lstset{frame=bt,                    % Add top and bottom frame lines
breaklines=true,             % Force line wrapping
captionpos=b,                % Place caption below listing
numbers=left,                % Add left-side line numbers
basicstyle=\scriptsize\ttfamily, % Set font size and type
showstringspaces=false,      % Don't show visible whitespace
numberstyle=\tiny,
upquote=true,                % Use upright quotes, not curly
commentstyle=\bfseries}      % Embolden comments

% Use (*@ @*) to escape LaTeX commands within listings.
\lstset{escapeinside={(*@}{@*)}}

% Add 10pt space between chapters in TOC listings entries:
%\let\Chapter\chapter
%\def\chapter{\addtocontents{lol}{\protect\addvspace{10pt}}\Chapter}


%%%%%%%%%%%%%%%%%%%%%%%%
%% Graphics and maths %%
%%%%%%%%%%%%%%%%%%%%%%%%
\usepackage{amsmath}

% Vector notation, e.g. \vv{x}:
%
\usepackage{esvect}

% Additional amsmath symbols, see:
%
% http://texblog.org/2007/08/27/number-sets-prime-natural-integer-rational-real-and-complex-in-latex/
%
\usepackage{amsfonts}
\usepackage{amssymb}

\usepackage{graphicx}
\usepackage{mathtools}
\usepackage{tikz}
\usepackage{tikz-qtree}

% Provide bold font face in maths.
\usepackage{bm}

\usepackage{subcaption}
\expandafter\def\csname ver@subfig.sty\endcsname{}

% Define an 'myalignat' command which behave as 'alignat' without the
% vertical top and bottom padding. See:
%     http://www.latex-community.org/forum/viewtopic.php?f=5&t=1890
\newenvironment{myalignat}[1]{%
\setlength{\abovedisplayskip}{-.7\baselineskip}%
\setlength{\abovedisplayshortskip}{\abovedisplayskip}%
\start@align\z@\st@rredtrue#1
}%
{\endalign}

% Define additional operators:
\DeclareMathOperator*{\argmin}{arg\,min}
\DeclareMathOperator*{\argmax}{arg\,max}

\DeclareMathOperator*{\gain}{Gain}

% Skeleton operators.
\DeclareMathOperator*{\map}{Map}
\DeclareMathOperator*{\reduce}{Reduce}
\DeclareMathOperator*{\scan}{Scan}
\DeclareMathOperator*{\stencil}{Stencil}
\DeclareMathOperator*{\zip}{Zip}
\DeclareMathOperator*{\allpairs}{All\,Pairs}

% Maths plots using pgfplots, see:
%
%     http://pgfplots.sourceforge.net/pgfplots.pdf
%
\usepackage{pgfplots}

% Disable compatability mode.
%
\pgfplotsset{compat=1.12}

% Gantt charts using pgfgantt, see:
%
%     http://www.ctan.org/pkg/pgfgantt
%
\usepackage{pgfgantt}

% Fix milestone aspect ratio by defining a custom element.
\newganttchartelement*{mymilestone}{
mymilestone/.style={
shape=diamond,
inner sep=2pt,
draw=black,
top color=black,
bottom color=black,
}
}

% Tikz flowchart configuration.
\usetikzlibrary{shapes,arrows,shadows,fit,backgrounds}
\tikzstyle{decision} = [diamond,
draw,
text width=4.5em,
text badly centered,
node distance=3cm,
inner sep=0pt]
\tikzstyle{block}    = [rectangle,
draw,
text width=5em,
text centered,
node distance=3cm,
minimum height=4em,
inner sep=.2cm]
\tikzstyle{line}     = [draw, -latex']

% Add dirtree picture style, see:
%
%     http://tex.stackexchange.com/a/34268
%
\newcount\dirtree@lvl
\newcount\dirtree@plvl
\newcount\dirtree@clvl
\def\dirtree@growth{%
\ifnum\tikznumberofcurrentchild=1\relax
\global\advance\dirtree@plvl by 1
\expandafter\xdef\csname dirtree@p@\the\dirtree@plvl\endcsname{\the\dirtree@lvl}
\fi
\global\advance\dirtree@lvl by 1\relax
\dirtree@clvl=\dirtree@lvl
\advance\dirtree@clvl by -\csname dirtree@p@\the\dirtree@plvl\endcsname
\pgf@xa=0.33cm\relax
\pgf@ya=-\baselineskip\relax
\pgf@ya=\dirtree@clvl\pgf@ya
\pgftransformshift{\pgfqpoint{\the\pgf@xa}{\the\pgf@ya}}%
\ifnum\tikznumberofcurrentchild=\tikznumberofchildren
\global\advance\dirtree@plvl by -1
\fi
}
\tikzset{
dirtree/.style={
growth function=\dirtree@growth,
every node/.style={anchor=north},
every child node/.style={anchor=west},
edge from parent path={(\tikzparentnode\tikzparentanchor) |- (\tikzchildnode\tikzchildanchor)}
}
}

% UML sequence diagram macros, see:
%
%     https://code.google.com/p/pgf-umlsd/
%
% Options:
%
%     underline - Underline object names
%
\usepackage[underline=false]{pgf-umlsd}

% Support for SVG graphics.
%
% NOTE that you must pass the "--shell-escape" argument to pdflatex to
% compile. NOTE also that images *MUST* be placed within the graphics
% path.
\usepackage{svg}
\graphicspath{{img/}}

%%%%%%%%%%%%%%%%%%%%%%
%% Tables and lists %%
%%%%%%%%%%%%%%%%%%%%%%

% Required to use labm8 exported tables.
%
\usepackage{booktabs}

% Required for full page-width tables.
\usepackage{tabularx}

%\usepackage{enumitem}
%\setenumerate{itemsep=0pt}

% Use no left margin for lists:
%\setlist{leftmargin=*}

\usepackage{longtable}

% Define column types L, C, R with known text justification and fixed
% widths:
\usepackage{array}
\newcolumntype{L}[1]{>{\raggedright\let\newline\\\arraybackslash\hspace{0pt}}m{#1}}
\newcolumntype{C}[1]{>{\centering\let\newline\\\arraybackslash\hspace{0pt}}m{#1}}
\newcolumntype{R}[1]{>{\raggedleft\let\newline\\\arraybackslash\hspace{0pt}}m{#1}}


%%%%%%%%%%%%%%%%%%%%%%%%%%%%%
%% Typesetting and symbols %%
%%%%%%%%%%%%%%%%%%%%%%%%%%%%%

% Adjustable font sizes in \Verbatim{}
\usepackage{fancyvrb}

%\usepackage{titlesec}
% Set section and paragraph heading fonts:
%\titleformat*{\section}{\Large\bfseries}
%\titleformat*{\subsection}{\normalsize\bfseries}
%\titleformat*{\subsubsection}{\normalsize}
%\titleformat*{\paragraph}{\large\bfseries}
%\titleformat*{\subparagraph}{\large\bfseries}

% Set section heading margins. Usage:
% \titlespacing*{<command>}{<left>}{<before>}{<after>}
%\titlespacing*{\section}{0pt}{.6em}{.3em}
%\titlespacing*{\subsection}{0pt}{.6em}{.2em}

% Set paragraph indentation size. Default is 15pt.
%\setlength{\parindent}{10pt}

% The line spacing can be globally set using \linespread:
%
% \linespread{1.2}

% Add a command \hr{} which will draw a horizontal rule the width of
% the text.
%
\newcommand{\hr}{\noindent\makebox[\linewidth]{\rule{\textwidth}{0.2pt}}}

% Add a command \br{} which will create a horizontal space of exactly
% one line height.
%
\newcommand{\br}{\hspace{\baselineskip}}

% Define a command to allow word breaking.
\newcommand*\wrapletters[1]{\wr@pletters#1\@nil}
\def\wr@pletters#1#2\@nil{#1\allowbreak\if&#2&\else\wr@pletters#2\@nil\fi}

% Define a command to create centred page titles.
\newcommand{\centredtitle}[1]{
\begin{center}
  \large
  \vspace{0.9cm}
  \textbf{#1}
\end{center}}

% Support hyperlinks using the \hyperref, \url and \href
% macros. Usage:
%
%    \hyperref[label_name]{''link text''}
%
%    \url{<my_url>}
%
%    \href{<my_url>}{<description>}
%
\usepackage{hyperref}

% Disable colored borders of links, cross-references etc in PDF output
\hypersetup{pdfborder={0 0 0}}

% Provide generic commands \degree, \celsius, \perthousand, \micro
% and \ohm which work both in text and maths mode.
\usepackage{gensymb}

%%%%%%%%%%%%%%%%%%%%%%%%%%%%%%%%%
%% Placeholder text generation %%
%%%%%%%%%%%%%%%%%%%%%%%%%%%%%%%%%

% Use either \blindtext or \libpsum to generate placeholder text. Also
% note the macros \blinditemize, \blindenumerate, \blinddescription.
\usepackage[english]{babel}
\usepackage{blindtext}
\usepackage{lipsum}


\abstract{\diff{Constructing an optimising compiler is an enormous undertaking. Modern compilers are multi-million dollar projects taking many years of development.
%
There are more devices + heterogeneity. Each device requires a new compilers.
%
% LLVM 8.0.1 SLOCount: 
% 1206 unique developers have contributed 307,817 commits, with
% 35,497,808 line additions	and 20,909,608 line deletions. There
% are 5,356,816 lines of code which 103035

% sloccount . --personcost 103035
% average US software developer salary from glassdor
%  https://www.glassdoor.co.uk/Salaries/us-software-engineer-salary-SRCH_IL.0,2_IN1_KO3,20.htm?countryRedirect=true
%
Demand outstrips supply.
%
What is need is better tools to lower the cost of compiler construction.}

% Compilers are a fundamental technology. Their role in translating software to machine code must be performed without error, while maximising the performance and efficiency of the generated code. The precedent for more rigorous validation and improved performance is well established, yet progress is challenging. Compilers comprise thousands of interacting components which must be expertly engineered and tuned, and much of the work of compiler construction has eluded automation. \diff{Furthermore, the rapid transition to heterogeneous parallelism has driven development of broad new range of accelerators which require aggressively-optimising compilers to obtain good performance. For the trend towards heterogeneity to continue, compiler construction must be made cheap.}

% The cost of these shortcomings is wasted energy, poor performance, and buggy software. What is needed is ways to lower the cost of constructing compilers.

This thesis presents new techniques that dramatically lower the cost of compiler construction, while improving robustness and performance. The enabling insight for this research is the leveraging of \emph{recurrent neural networks} to \diff{model the correlations between source code and program behaviour}, enabling tasks which previously required enormous engineering effort to be automated. This is demonstrated in three domains:

% This thesis presents three techniques to simplify and accelerate compiler construction.
% First, a tool for automatic performance characterisation through benchmark generation; second, a low-cost and effective fuzzer for validating correctness; third, a simple technique to address the labour intensive process of optimisation heuristic construction.

First, a generative model for compiler benchmarks is developed. This model is inferred automatically from corpora of readily available open source programs, requiring no grammar or prior knowledge of the programming language. This greatly reduces the cost of development compared to prior approaches, yet the generator produces output of such quality that professional software developers cannot distinguish generated from handwritten code. The efficacy of the generator is demonstrated by supplementing the training data of state-of-the-art predictive models for compiler optimisations. The additional fine-grained exploration of the feature space yields both an automatic improvement in heuristic performance and exposes weaknesses in the prior art which, when corrected, yields further improvements in performance.

Second, this thesis presents techniques that extend the prior approach to the domain of compiler validation. A compiler fuzzer is developed which is far simpler than the state-of-the-art, yet is effective. By learning a generative model rather than engineering a generator from scratch using a grammar, it is implemented in $100\times$ fewer lines of code than the state-of-the-art, and is capable of generating an expressive range of tests that expose bugs that prior techniques cannot. An extensive testing campaign of OpenCL compilers reveals 67 new bugs, many of which have now been fixed.

Finally, this thesis addresses the challenges of machine learning for compiler optimisations, developing methodologies for learning compiler heuristics without the need for code features. Contrasting prior approaches that require features to be expertly engineered and selected, the proposed approach learns directly over the raw textual representation of program code. Doing so outperforms state-of-the-art heuristics in two challenging optimisation domains. Additionally, the methodology permits the novel transfer of information between optimisation problems, enabling a model trained for one task to be adapted to perform another, further improving performance.

\diff{TODO: Conclusions}}

%%%%%%%%%%
%% Body %%
%%%%%%%%%%
\begin{document}

  \begin{preliminary}
  \maketitle
  % \begin{laysummary}\todo[inline]{Crisis, solution, happiness}
\end{laysummary}
  \begin{acknowledgements}\section*{Acknowledgments}

This work was supported by the UK Engineering and Physical Sciences Research
Council under grants EP/L01503X/1 (CDT in Pervasive Parallelism), EP/M01567X/1
(SANDeRs), EP/M015793/1 (DIVIDEND), and EP/P003915/1 (SUMMER). The code and data
for this paper are available at: \url{https://chriscummins.cc/pact17}.
\end{acknowledgements}
  \begin{declaration}I declare that this thesis was composed by myself, that the work contained herein is my own except where explicitly stated otherwise in the text, and that this work has not been submitted for any other degree or professional qualification except as specified. Some of the material used in this thesis has been published in the following papers:
\begin{itemize}
	\item Chris Cummins, Pavlos Petoumenos, Zheng Wang, and Hugh Leather ``Synthesizing Benchmarks for Predictive Modeling''. In \emph{Proceedings of the International Symposium on Code Generation and Optimization (CGO)}, 2017~\cite{Cummins2017a}.
	\item Chris Cummins, Pavlos Petoumenos, Zheng Wang, and Hugh Leather ``End-to-end Deep Learning of Optimization Heuristics''. In \emph{Proceedings of the International Conference on Parallel Architectures and Compilation Techniques (PACT)}, 2017~\cite{Cummins2017b}.
	\item Chris Cummins, Pavlos Petoumenos, Alastair Murray, and Hugh Leather ``Compiler Fuzzing through Deep Learning''. In \emph{Proceedings of the ACM SIGSOFT International Symposium on Software Testing and Analysis (ISSTA)}, 2018~\cite{Cummins2018}.
\end{itemize}
\vspace{1in}\raggedleft({\em Chris Cummins\/})
\end{declaration}
  \dedication{\input{dedication}}
  \tableofcontents
  \listoffigures
  \listoftables
  % \listofalgorithms
  \listoflistings
  % \listoftodos
\end{preliminary}



  % INTRODUCTION
  % ============
  %
  % An introduction to the document, clearing stating the hypothesis or
  % objective of the project, motivation for the work and the results
  % achieved. The structure of the remainder of the document should also
  % be outlined.
  \chapter{Introduction}\label{chap:introduction}
  \section{Introduction}

This chapter reviews research in areas relevant to this thesis.

The chapter is structured as follows: Section~\ref{sec:related-work-iterative-compilation} reviews the literature of iterative compilation and machine learning for compilers. Section~\ref{sec:related-work-machine-learning-for-pl} describes related work in machine learning over programming languages. Section~\ref{sec:related-work-program-generation} reviews the relevant literature of program generation. Finally Section~\ref{sec:related-work-summary} concludes.


  % BACKGROUND
  % ==========
  %
  % Background to the project, previous work, exposition of relevant
  % literature, setting of the work in the proper context. This should
  % contain sufficient information to allow the reader to appreciate the
  % contribution you have made.
  %
  %   Background - Necessary knowledge to understand thesis. Demonstration
  %     of competence.
  %
  %   Related Work - Description of competitors. How am I better?
  \chapter{Background}\label{chap:background}
  \chapter{Deep Learning and Methodologies}
\label{chap:background}

\section{Introduction}

\section{Terminology}

Notation: $\odot$ point-wise multiplication of tensors.

\section{Machine Learning}

\subsection{Principal Component Analysis}

\subsection{Decision Trees}

\section{Deep Learning}

\subsection{Recurrent Neural Networks}

\subsubsection{Long Short-Term Memory}

LSTM variants review~\cite{Greff2015}.

$\bm{x}^t$ is the input vector at time $t$; $\bm{W}$ are input weight matrices; $\bm{R}$ are recurrent weight matrices; $\bm{p}$ are peephole weight vectors; $\bm{b}$ are bias vectors; functions $g$, $h$, and $\sigma$ are point-wise nonlinear activation functions.

block input:
\[ \bm{z}^{t} = g \left( \bm{W}_z \bm{x}^t + \bm{R}_z \bm{y}^{t - 1} + \bm{b}_z \right) \]

input gate:
\[ \bm{i}^{t} = \sigma \left( \bm{W}_i \bm{x}^t + \bm{R}_i \bm{y}^{t-1} + \bm{p}_i \odot c^{t-1} + \bm{b}_i \right) \]

forget gate:
\[ \bm{f}^{t} = \sigma \left( \bm{W}_f \bm{x}^t + \bm{R}_f \bm{y}^{t-1} + \bm{p}_f \odot c^{t-1} + \bm{b}_f \right) \]

cell state:
\[ \bm{c}^t = \bm{i}^t \odot \bm{z}^t + \bm{f}^t \odot \bm{c}^{t-1} \]

output gate:
\[ \bm{o}^{t} = \sigma \left( \bm{W}_i \bm{x}^t + \bm{R}_o \bm{y}^{t - 1} + \bm{p}_o \odot c^{t-1} + \bm{b}_o \right) \]

block output:
\[ \bm{y}^t = \bm{o}^t \odot h(\bm{c}^t) \]

Number of params = \todo[inline]{\ldots}


% TODO: \subsection{Generative Adversarial Networks}

The Generative Adversarial Network (GAN) is a means to estimate a generative model~\cite{Goodfellow2014}. It uses an adversarial process in which two models are simultaneously trained: a generator model $G$ that captures the data distribution, and a discriminative model $D$ that estimates the probability that a sample came from the training data rather than $G$. The training procedure for $G$ is to maximize the probability of $D$ making a mistake.

If both models are neural networks: learn a generator's distribution $p_g$ over data $\bm{x}$. Define a prior on input noise variables $p_z(\bm{z})$. Generator $G(\bm{z}; \Theta_g)$, using parameters $\Theta_g$. Discriminator $D(\bm{x}; \Theta_d)$ outputs a scalar, the probability that $\bm{x}$ came from the data rather than $p_g$. 

Simultaneously train $D$ to maximize the probability of assigning the correct label to both training examples and samples from $G$, and train $G$ to minimize $\log (1 - D(G(\bm{z})))$. $D$ and $G$ play the two-player minimax game with value function $V(G, D)$:

\[ \min_G \max_D V(D, G) = \mathbb{E}_{\bm{x} \sim  p_{data}(\bm{x})} [ \log D(\bm{x}) ] + \mathbb{E}_{\bm{z} \sim p_z(\bm{z})} [ \log (1 - D(G(\bm{z}))) ] \]

Challenge: there may not be sufficient gradient for $G$ to learn well. Early in learning, when $G$ is poor, $D$ can reject samples with high confidence because they are clearly different from the training data.

\section{Evaluation Techniques}

\section{Summary}


  \chapter{Related Work}\label{chap:related}
  This chapter begins with a brief survey of the broad field of
literature that is relevant to algorithmic skeletons. This is followed
by a review of the current state of the art in autotuning research,
focusing on heterogeneous parallelism, algorithmic skeletons, and
stencil codes. It presents the context and rationale for the research
undertaken for this thesis.


\section{Automating Parallelism}

It is widely accepted that parallel programming is difficult, and the
continued repetition of this claim has become something of a trite
mantra for the parallelism research community. An interesting
digression is to discuss some of the ways in which researchers have
attempted to tackle this difficult problem, and why, despite years of
research, it remains an ongoing challenge.

The most ambitious and perhaps daring field of parallelism research is
that of automatic parallelisation, where the goal is to develop
methods and systems to transform arbitrary sequential code into
parallelised code. This is a well studied subject, with the typical
approach being to perform these code transformations at the
compilation stage. In \citeauthor{Banerjee1993}'s thorough
review~\cite{Banerjee1993} on the subject, they outline the key
challenges of automatic parallelisation:
%
\begin{itemize}
  \item determining whether sequential code can be legally transformed
  for parallel execution; and
  \item identifying the transformation which will provide the highest
  performance improvement for a given piece of code.
\end{itemize}
%
Both of these challenges are extremely hard to tackle. For the former,
the difficulties lie in performing accurate analysis of code
behaviour. Obtaining accurate dynamic dependency analysis at compile
time is an unsolved problem, as is resolving pointers and points-to
analysis~\cite{Atkin-granville2013, Hind2001,Ghiya2001}.

The result of these challenges is that reliably performant, automatic
parallelisation of arbitrary programs remains a far from reached goal;
however, there are many note worthy variations on the theme which have
been able to achieve some measure of success.

One such example is speculative parallelism, which circumvents the
issue of having incomplete dependency information by speculatively
executing code regions in parallel while performing dependency tests
at runtime, with the possibility to fall back to ``safe'' sequential
execution if correctness guarantees are not
met~\cite{Prabhu2010,Trachsel2010}. In~\cite{Jimborean2014},
\citeauthor{Jimborean2014} present a system which combines polyhedral
transformations of user code with binary algorithmic skeleton
implementations for speculative parallelisation, reporting speedups
over sequential code of up to $15.62\times$ on a 24 core processor.

Another example is PetaBricks, which is a language and compiler
enabling parallelism through ``algorithmic choice''~\cite{Ansel2009,
Ansel2010}. With PetaBricks, users provide multiple implementations
of algorithms, optimised for different parameters or use cases. This
creates a search space of possible execution paths for a given
program. This has been combined with autotuning techniques for
enabling optimised multigrid programs~\cite{Chan2009}, with the wider
ambition that these autotuning techniques may be applied to all
algorithmic choice programs~\cite{Ansel2014}. While this helps produce
efficient parallel programs, it places a great burden on the
developer, requiring them to provide enough contrasting
implementations to make a search of the optimisation space fruitful.

Annotation-driven parallelism takes a similar approach. The user
annotates their code to provide ``hints'' to the compiler, which can
then perform parallelising transformations. A popular example of this
is OpenMP, which uses compiler pragmas to mark code sections for
parallel or vectorised execution~\cite{Dagum1998}. Previous work has
demonstrated code generators for translating OpenMP to
OpenCL~\cite{Grewe2013} and CUDA~\cite{Lee2009}. Again,
annotation-driven parallelism suffers from placing a burden on the
developer to identify the potential areas for parallelism, and lacks
the structure that algorithmic skeletons provide.

Algorithmic skeletons contrast the goals of automatic parallelisation
by removing the challenge of identifying potential parallelism from
compilers or users, instead allowing users to frame their problems in
terms of well defined patterns of computation. This places the
responsibility of providing performant, well tuned implementations for
these patterns on the skeleton author.


\section{Iterative Compilation \& Machine
Learning}\label{sec:iterative-compilation}

Iterative compilation is the method of performance tuning in which a
program is compiled and profiled using multiple different
configurations of optimisations in order to find the configuration
which maximises performance. One of the the first formalised
publications of the technique appeared in \citeyear{Bodin1998} by
\citeauthor{Bodin1998}~\cite{Bodin1998}. Iterative compilation has
since been demonstrated to be a highly effective form of empirical
performance tuning for selecting compiler optimisations.

Given the huge number of possible compiler optimisations (there are
207 flags and parameters to control optimisations in GCC v4.9), it is
often unfeasible to perform an exhaustive search of the entire
optimisation space, leading to the development of methods for reducing
the cost of evaluating configurations. These methods reduce evaluation
costs either by shrinking the dimensionality or size of the
optimisation space, or by guiding a directed search to traverse a
subset of the space.

Machine learning has been successful applied to this problem,
in~\cite{Stephenson2003}, using ``meta optimisation'' to tune compiler
heuristics through an evolutionary algorithm to automate the search of
the optimisation space. \citeauthor{Fursin2011} continued this with
Milepost GCC, the first machine learning-enabled self-tuning
compiler~\cite{Fursin2011}. A recent survey of the use of machine
learning to improve heuristics quality by \citeauthor{Burke2013}
concludes that the automatic \emph{generation} of these self-tuning
heuristics but is an ongoing research challenge that offers the
greatest generalisation benefits~\cite{Burke2013}.

% An approach to online tuning of parallel programs is presented
% in~\cite{Ansel2012} which partitions the available parallel resources
% of a device in to two partitions and then executes two different
% configurations simultaneously using each partition. The configuration
% used for one of the configuration is guaranteed to be ``safe'', and
% the performance

% % Eastep, J., Wingate, D., & Agarwal, A. (2011). Smart Data
% % Structures: An Online Machine Learning Approach to Multicore Data
% % Structures. In Proceedings of the 8th ACM International Conference
% % on Autonomic Computing (pp. 11–20). New York, NY, USA:
% % ACM. doi:10.1145/1998582.1998587
% \TODO{Online reinforcement learning for optimising data structures
%   online, \cite{Tesauro2005}}

% % Tesauro, G. (2005). Online Resource Allocation Using Decompositional
% % Reinforcement Learning. In AAAI (pp. 886–891).
% \TODO{Reinforcement learning for resource allocation~\cite{Eastep2011}}

% % W. F. Ogilvie, P. Petoumenos, Z. Wang, and H. Leather, “Intelligent
% % Heuristic Construction with Active Learning,” in 18th International
% % Workshop on Compilers for Parallel Computing, 2015.
% \TODO{Using Active Learning to speed up the learning of compiler
%   heuristics~\cite{Ogilvie2015}. Towards online autotuning, albeit
%   only with binary optimisation parameter.}

%
% SOME EXAMPLES OF ML IN THE WILD:
%

% % R. Bitirgen, E. Ipek, and J. F. Martinez, “Coordinated Management of
% % Multiple Interacting Resources in Chip Multiprocessors: A Machine
% % Learning Approach,” in 2008 41st IEEE/ACM International Symposium on
% % Microarchitecture, 2008, pp. 318–329.
% \TODO{Artificial Neural Networks for resource allocation of CMPS:
% \cite{Bitirgen2008}}

% % Z. Wang and M. F. P. O. Boyle, “Partitioning Streaming Parallelism
% % for Multi-cores: A Machine Learning Based Approach,” in Proceedings
% % of the 19th international conference on Parallel architectures and
% % compilation techniques, 2010, pp. 307–318.
% \TODO{Offline ML for partitioning streaming applications:
% \cite{Wang2010}}

In~\cite{Saclay2010,Memon2013,Fursin2014}, \citeauthor{Fursin2014}
advocate a collaborative and ``big data'' driven approach to
autotuning, arguing that the challenges facing the widespread adoption
of autotuning and machine learning methodologies can be attributed to:
a lack of common, diverse benchmarks and datasets; a lack of common
experimental methodology; problems with continuously changing hardware
and software stacks; and the difficulty to validate techniques due to
a lack of sharing in publications. They propose a system for
addressing these concerns, the Collective Mind knowledge system,
which, while in early stages of ongoing development, is intended to
provide a modular infrastructure for sharing autotuning performance
data and related artifacts across the internet. In addition to sharing
performance data, the approach taken in this thesis emphasises the
collective \emph{exploitation} of such performance data, so that data
gathered from one device may be used to inform the autotuning
decisions of another. This requires each device to maintain local
caches of shared data to remove the network overhead that would be
present from querying a single centralised data store during execution
of a hot path. The current implementation of Collective Mind uses a
NoSQL JSON format for storing performance data. The relational schema
used in this thesis offers greater scaling performance and lower
storage overhead as the amount of performance data grows.

Whereas iterative compilation requires an expensive offline training
phase to search an optimisation space, dynamic optimisers perform this
optimisation space exploration at runtime, allowing programs to
respond to dynamic features ``online''. This is a challenging task, as
a random search of an optimisation space may result in configurations
with vastly suboptimal performance. In a real world system, evaluating
many suboptimal configurations will cause a significant slowdown of
the program. Thus a requirement of dynamic optimisers is that
convergence time towards optimal parameters is minimised.

Existing dynamic optimisation research has typically taken a low level
approach to performing optimisations. Dynamo is a dynamic optimiser
which performs binary level transformations of programs using
information gathered from runtime profiling and
tracing~\cite{Bala2000}. While this provides the ability to respond to
dynamic features, it restricts the range of optimisations that can be
applied to binary transformations. These low level transformations
cannot match the performance gains that higher level parameter tuning
produces.

An interesting related tangent to iterative compilation is the
development of so-called ``superoptimisers''. In~\cite{Massalin1987},
the smallest possible program which performs a specific function is
found through a brute force enumeration of the entire instruction
set. Starting with a program of a single instruction, the
superoptimiser tests to see if any possible instruction passes a set
of conformity tests. If not, the program length is increased by a
single instruction and the process repeats. The exponential growth in
the size of the search space is far too expensive for all but the
smallest of hot paths, typically less than 13 instructions. The
optimiser is limited to register to register memory transfers, with no
support for pointers, a limitation which is addressed
in~\cite{Joshi2002}. Denali is a superoptimiser which uses constraint
satisfaction and rewrite rules to generate programs which are
\emph{provably} optimal, instead of searching for the optimal
configuration through empirical testing. Denali first translates a low
level machine code into guarded multi-assignment form, then uses a
matching algorithm to build a graph of all of a set of logical axioms
which match parts of the graph, before using boolean satisfiability to
disprove the conjecture that a program cannot be written in $n$
instructions. If the conjecture cannot be disproved, the size of $n$
is increased and the process repeats.


\subsection{Training with Synthetic Benchmarks}

The use of synthetic benchmarks for providing empirical performance
evaluations dates back to as early as 1974~\cite{Curnow1976}. The
\emph{automatic generation} of such synthetic benchmarks is a more
recent innovation, serving the purpose initially of stress-testing
increasingly complex software systems for behaviour validation and
automatic bug detection~\cite{Verplaetse2000,Godefroid2008}. A range
of techniques have been developed for these purposes, ranging from
applying random mutations to a known dataset to generate test stimuli,
to so-called ``whitebox fuzz testing'' which analyses program traces
to explore the space of a program's control flow. Csmith is one such
tool which generates randomised C source programs for the purpose of
automatically detecting compiler bugs~\cite{Yang2012}.

A method for the automatic generation of synthetic benchmarks for the
purpose of \emph{performance} tuning is presented
in~\cite{Chiu2015}. \citeauthor{Chiu2015} use template substitution
over a user-defined range of values to generate training programs with
a statistically controlled range of features. A Perl preprocessor
generates output source codes from an input description using a custom
input language Genesis. Genesis is more flexible than the system
presented in this thesis, supporting substitution of arbitrary
sources. The authors describe an application of their tool for
generating OpenCL stencil kernels, but do not report any performance
results.


\section{Performance Tuning for Heterogeneous Parallelism}

As briefly discussed in Section~\ref{sec:gpgpu}, the complex
interactions between optimisations and heterogeneous hardware makes
performance tuning for heterogeneous parallelism a difficult
task. Performant GPGPU programs require careful attention from the
developer to properly manage data layout in DRAM, caching, diverging
control flow, and thread communication. The performance of programs
depends heavily on fully utilising zero-overhead thread scheduling,
memory bandwidth, and thread grouping. \citeauthor{Ryoo2008a}
illustrate the importance of these factors by demonstrating speedups
of up to $432\times$ for matrix multiplication in CUDA by appropriate
use of tiling and loop unrolling~\cite{Ryoo2008a}. The importance of
proper exploitation of local shared memory and synchronisation costs
is explored in~\cite{Lee2010}.

In~\cite{Chen2014}, data locality optimisations are automated using a
description of the hardware and a memory-placement-agnostic
compiler. The authors demonstrate impressive speedups of up to
$2.08\times$, although at the cost of requiring accurate memory
hierarchy descriptor files for all targeted hardware. The descriptor
files must be hand generated, requiring expert knowledge of the
underlying hardware in order to properly exploit memory locality.

Data locality for nested parallel patterns is explored in~\cite{Lee}.
The authors use an automatic mapping strategy for nested parallel
skeletons on GPUs, which uses a custom intermediate representation and
a CUDA code generator, achieving $1.24\times$ speedup over hand
optimised code on 7 of 8 Rodinia benchmarks.

Reduction of the GPGPU optimisation space is demonstrated
in~\cite{Ryoo2008}, using the common subset of optimal configurations
across a set of training examples. This technique reduces the search
space by 98\%, although it does not guarantee that for a new program,
the reduced search space will include the optimal configuration.

\citeauthor{Magni2014} demonstrated that thread coarsening of OpenCL
kernels can lead to speedups in program performance between
$1.11\times$ and $1.33\times$ in~\cite{Magni2014}. The authors achieve
this using a machine learning model to predict optimal thread
coarsening factors based on the static features of kernels, and an
LLVM function pass to perform the required code transformations.

A framework for the automatic generation of OpenCL kernels from
high-level programming concepts is described in~\cite{Steuwer2015}. A
set of rewrite rules is used to transform high-level expressions to
OpenCL code, creating a space of possible implementations. This
approach is ideologically similar to that of PetaBricks, in that
optimisations are made through algorithmic choice, although in this
case the transformations are performed automatically at the compiler
level. The authors report performance on a par with that of hand
written OpenCL kernels.


\section{Autotuning Algorithmic Skeletons}

An enumeration of the optimisation space of Intel Thread Building
Blocks in~\cite{Contreras2008} shows that runtime knowledge of the
available parallel hardware can have a significant impact on program
performance. \citeauthor{Collins2012} exploit this
in~\cite{Collins2012}, first using Principle Components Analysis to
reduce the dimensionality of the space of possible optimisation
parameters, followed by a search of parameter values to optimise
program performance by a factor of $1.6\times$ over values chosen by a
human expert. In~\cite{Collins2013}, they extend this using static
feature extraction and nearest neighbour classification to further
prune the search space, achieving an average 89\% of the oracle
performance after evaluating 45 parameters.

\citeauthor{Dastgeer2011} developed a machine learning based autotuner
for the SkePU skeleton library in~\cite{Dastgeer2011}. Training data
is used to predict the optimal execution device (i.e.\ CPU, GPU) for a
given program by predicting execution time and memory copy overhead
based on problem size. The autotuner only supports vector operations,
and there is limited cross-architecture
evaluation. In~\cite{Dastgeer2015a}, the authors extend SkePU to
improve the data consistency and transfer overhead of container types,
reporting up to a $33.4\times$ speedup over the previous
implementation.


\section{Code Generation and Autotuning for Stencils}

Stencil codes have a variety of computationally expensive uses from
fluid dynamics to quantum mechanics. Efficient, tuned stencil kernels
are highly sought after, with early work in \citeyear{Bolz2003} by
\citeauthor{Bolz2003} demonstrating the capability of GPUs for
massively parallel stencil operations~\cite{Bolz2003}. In the
resulting years, stencil codes have received much attention from the
performance tuning research community.

\citeauthor{Ganapathi2009} demonstrated early attempts at autotuning
multicore stencil codes in~\cite{Ganapathi2009}, drawing upon the
successes of statistical machine learning techniques in the compiler
community, as discussed in
Section~\ref{sec:iterative-compilation}. They present an autotuner
which can achieve performance up to 18\% better than that of a human
expert. From a space of 10 million configurations, they evaluate the
performance of a randomly selected 1500 combinations, using Kernel
Canonical Correlation Analysis to build correlations between tunable
parameter values and measured performance targets. Performance targets
are L1 cache misses, TLB misses, cycles per thread, and power
consumption. The use of KCAA restricts the scalability of their system
as the complexity of model building grows exponentially with the
number of features. In their evaluation, the system requires two hours
of compute time to build the KCAA model for only 400 seconds of
benchmark data. They present a compelling argument for the use of
energy efficiency as an optimisation target in addition to runtime,
citing that it was the power wall that lead to the multicore
revolution in the first place. Their choice of only 2 benchmarks and 2
platforms makes the evaluation of their autotuner somewhat limited.

\citeauthor{Berkeley2009} targeted 3D stencils code performance
in~\cite{Berkeley2009}. Stencils are decomposed into core blocks,
sufficiently small to avoid last level cache capacity misses. These
are then further decomposed to thread blocks, designed to exploit
common locality threads may have within a shared cache or local
memory. Thread blocks are divided into register blocks in order to
take advantage of data level parallelism provided by the available
registers. Data allocation is optimised on NUMA systems. The
performance evaluation considers speedups of various optimisations
with and without consideration for host/device transfer overhead.

\citeauthor{Kamil2010} present an autotuning framework
in~\cite{Kamil2010} which accepts as input a Fortran 95 stencil
expression and generates tuned shared-memory parallel implementations
in Fortan, C, or CUDA. The system uses an IR to explore autotuning
transformations, enumerating a subset of the optimisation space and
recording only a single execution time for each configuration,
reporting the fastest. They demonstrate their system on 4
architectures using 3 benchmarks, with speedups of up to $22\times$
compared to serial implementations. The CUDA code generator does not
optimise for the GPU memory hierarchy, using only global memory. As
demonstrated in this thesis, improper utilisation of local memory can
hinder program performance by two orders of magnitude. There is no
directed search or cross-program learning.

In~\cite{Zhang2013a}, \citeauthor{Zhang2013a} present a code generator
and autotuner for 3D Jacobi stencil codes. Using a DSL to express
kernel functions, the code generator performs substitution from one of
two CUDA templates to create programs for execution on GPUs. GPU
programs are parameterised and tuned for block size, block dimensions,
and whether input data is stored in read only texture memory. This
creates an optimisation space of up to 200 configurations. In an
evaluation of 4 benchmarks, the authors report impressive performance
that is comparable with previous implementations of iterative Jacobi
stencils on GPUs~\cite{Holewinski2012, Phillips2010}. The dominating
parameter is shown to be block dimensions, followed by block size,
then read only memory. The DSL presented in the paper is limited to
expressing only Jacobi Stencils applications. Critically, their
autotuner requires a full enumeration of the parameter space for each
program. Since there is no indication of the compute time required to
gather this data, it gives the impression that the system would be
impractical for the needs of general purpose stencil computing. The
autotuner presented in this thesis overcomes this drawback by learning
parameter values which transfer to unseen stencils, without the need
for an expensive tuning phase for each program and architecture.
% TODO: Depending on results of cross-architecture validation, this
% last claim may not hold up.
%
% The majority of applications tested are memory bound. Does this
% transfer to computer bound?

In~\cite{Christen2011}, \citeauthor{Christen2011} presentf a DSL for
expressing stencil codes, a C code generator, and an autotuner for
exploring the optimisation space of blocking and vectorisation
strategies. The DSL supports stencil operations on arbitrarily
high-dimensional grids. The autotuner performs either an exhaustive,
multi-run Powell search, Nelder Mead, or evolutionary search to find
optimal parameter values. They evaluate their system on two CPUS and
one GPU using 6 benchmarks. A comparison of tuning results between
different GPU architectures would have been welcome, as the results
presented in this thesis show that devices have different responses to
optimisation parameters. The authors do not present a ratio of the
available performance that their system achieves, or how the
performance of optimisations vary across benchmarks or devices.

A stencil grid can be decomposed into smaller subsections so that
multiple GPUs can operate on each subsection independently. This
requires a small overlapping region where each subsection meets ---
the halo region --- to be shared between devices. For iterative
stencils, values in the halo region must be synchronised between
devices after each iteration, leading to costly communication
overheads. One possible optimisation is to deliberately increase the
size of the halo region, allowing each device to compute updated
values for the halo region, instead of requiring a synchronisation of
shared state. This reduces the communication costs between GPUs, at
the expense of introducing redundant computation. Tuning the size of
this halo region is the goal of PARTANS~\cite{Lutz2013}, an autotuning
framework for multi-GPU stencil computations. \citeauthor{Lutz2013}
explore the effect of varying the size of the halo regions using six
benchmark applications, finding that the optimal halo size depends on
the size of the grid, the number of partitions, and the connection
mechanism (i.e.\ PCI express). The authors present an autotuner which
determines problem decomposition and swapping strategy offline, and
performs an online search for the optimal halo size. The selection of
overlapping halo region size compliments the selection of workgroup
size which is the subject of this thesis. However, PARTANS uses a
custom DSL rather than the generic interface provided by SkelCL, and
PARTANS does not learn the results of tuning across programs, or
across multiple runs of the same program.


\section{Summary}

There is already a wealth of research literature on the topic
autotuning which begs the question, why isn't the majority of software
autotuned? In this chapter I attempted to answer the question by
reviewing the state of the art in the autotuning literature, with
specific reference to auotuning for GPUs and stencil codes. The bulk
of this research falls prey of one of two shortcomings. Either they
identify and develop a methodology for tuning a particular
optimisation space but then fail to develop a system which can
properly exploit this (for example, by using machine learning to
predict optimal values across programs), or they present an autotuner
which targets too specific of a class of optimisations to be widely
applicable. This project attempts to address both of those
shortcomings by expending great effort to deliver a working
implementation which users can download and use without any setup
costs, and by providing a modular and extensible framework which
allows rapid targeting of new autotuning platforms, enabled by a
shared autotuning logic and distributed training data. The following
chapter outlines the design of this system.


  \chapter{OmniTune - an Extensible, Dynamic Autotuner}\label{chap:autotune}
  \section{Introduction}

In previous chapters I have advocated the use of machine learning for
autotuning. In this chapter, I present OmniTune, a framework for
extensible, distributed autotuning using machine learning. OmniTune
provides a replacement for the kinds of ad-hoc tuning typically
performed by expert programmers by providing runtime prediction of
tunable parameters using collaborative, online learning of
optimisation spaces. First I describe the high level overview of the
approach to autotuning, then I describe the system architecture and
set of interfaces exposed by Omnitune.


\section{Predicting Optimisation Parameter Values}

\begin{figure}[b]
\centering
\includegraphics[width=\textwidth]{img/omnitune-system-flow.pdf}
\caption[Optimisation parameter selection with OmniTune]{%
  The process of selecting optimisation parameter values for a given
  user program with OmniTune.%
}
\label{fig:omnitune-system-flow}
\end{figure}

The goal of machine learning-enabled autotuning is to \emph{predict}
the values for optimisation parameters to maximise some measure of
profit. These predictions are based on models built from prior
observations. The prediction quality is influenced by the number of
prior observations. OmniTune supports both prediction of parameters
based on prior observations, and a method for collecting these
observations. When a client program requests parameter values, it
indicates whether the request is for training or performance purposes,
and uses a different backend to select parameter values for each. New
observations can then be added once parameters have been evaluated.
Figure~\ref{fig:omnitune-system-flow} shows this process.


\section{System Architecture and Interface}

\begin{figure}
\centering
\includegraphics[width=.9\textwidth]{img/omnitune-system-overview.pdf}
\caption[OmniTune system diagram]{%
  High level overview of OmniTune components.%
}
\label{fig:omnitune-system-overview}
\end{figure}

Common implementations of autotuning in the literature either: embed
the autotuning logic within the each target application, or take a
standalone approach in which the autotuner is a program which must be
invoked by the user to tune a target application. Embedding the
autotuner within each target application has the advantage of
providing ``always-on'' behaviour, but is infeasible for complex
systems in which the cost of building machine learning models must be
added to each program run. The standalone approach separates the
autotuning logic, at the expense of adding one additional step to the
build process. The approach taken in OmniTune aims to capture the
advantages of both techniques by implementing a autotuning \emph{as a
  service}, with only the lightweight communication logic embedded in
the target applications.

Omnitune is built around a three tier client-server model. The
applications which are to be autotuned are the \emph{clients}. These
clients commmunicate with a system-wide \emph{server}, which handles
autotuning requests. The server communicates and caches data sourced
from a \emph{remote}, which maintains a global store of all autotuning
data. Figure~\ref{fig:omnitune-system-overview} shows this structure.

There is a many to one relationship between clients, servers, and
remotes, such that a single remote may handle connections to multiple
servers, which in turn may accept connections from multiple
clients. This design has two primary advantages: the first is that it
decouples the autotuning logic from that of the client program,
allowing developers to easily repurpose the autotuning framework to
target additional optimisation parameters without a significant
development overhead for the target applications; the second advantage
is that this enables collective tuning, in which training data
gathered from a range of devices can be accessed and added to by any
OmniTune server.

OmniTune supports autotuning using a separate offline training phase,
online training, or a mixture of both. Figure~\ref{fig:omnitune-comms}
shows an example pattern of communication between the three tiers of
an OmniTune system, showing a distinct training phase. Note that this
training phase is enforced only by the client. The following sections
describe the interfaces between the three components.


\begin{figure}[b]
\centering
\includegraphics[width=.7\textwidth]{img/omnitune-comms}
\caption[Communication pattern between OmniTune components]{%
  An example communication pattern between OmniTune components,
  showing an offline training phase.%
}
\label{fig:omnitune-comms}
\end{figure}


\subsection{Client Interface: Lightweight Communication}

Client applications communicate with an OmniTune server through four
operations:
%
\begin{itemize}
\item \textsc{Request}$(x) \to p$ Given a set of explanatory variables
  $x$, request a set of parameter values $p$ to maximise performance.
\item \textsc{RequestTraining}$(x) \to p$ Given a set of explanatory
  variables $x$, allow the server to select a set of parameter values
  $p$ for evaluating their fitness.
\item \textsc{Submit}$(x, p, y)$ Submit an observed measurement of
  fitness $y$ of parameters $p$, given explantory variables $x$.
\item \textsc{Refuse}$(x, p)$ Refuse a set of parameters $p$ given a
  set of explanatory variables $x$. Once refused, those parameters
  will not be returned by any subsequent calls to \textsc{Request()}
  or \textsc{RequestTraining()}.
\end{itemize}
%
This set of operations enables the core functionality of an autotuner,
while providing flexibility for the client to control how and when
training data is collected.


\subsection{Server: Autotuning Engine}

For each autotuning-capable machine, a system-level daemon hosts a
DBus session bus which client processes communicate with. This daemon
acts as an intermediate between the training data and the client
applications, \emph{serving} requests for optimisation parameter
values. Servers operations are application-specific, so there is a set
of operations to implement autotuning of each supported optimisation
target.

The server is implemented as a standalone Python program, and contains
a library of machine learning tools to perform parameter prediction,
interfacing with Weka using the JNI. Weka is a suite of data mining
software developed by the University of Waikato, freely available
under the GNU GPL
license~\footnote{http://www.cs.waikato.ac.nz/ml/weka/}. OmniTune
servers may perform additional feature extraction of explanatory
variables supplied by incoming client requests. The reason for
performing feature extraction on the server as opposed to on the
client side is that this allows the results of expensive operations
(for example, analysing source code of target applications) to be
cached for use across the lifespan of client applications. The
contents of these local caches are periodically and asynchronously
synced with the remote, to maintain a global store of lookup tables
for expensive operations.

On launch, the server requests the latest training data from the
remote, which it uses to build the relevant models for performing
prediction of optimisation parameter values. Servers communicate with
remotes by submitting or requesting training data in batches, using
two operations:
%
\begin{itemize}
\item \textsc{Push}$(\bf{f}, \bf{c})$ Submit a set of labelled training
  points as pairs $(f,c)$.
\item \textsc{Pull}$() \to (\bf{f}, \bf{c})$ Request training data as a
  set of labelled $(f,c)$ pairs.
\end{itemize}


\subsection{Remote: Distributed Training Data}

The role of the remote is to provide bookkeeping of training data for
machine learning. Using the interface described in the previous
section, remotes allow shared access to data from multiple servers
using a transactional communication pattern.


\section{Summary}

This chapter has described the architecture of of OmniTune, a
distributed autotuner which is capable of performing runtime
prediction of optimal workgroup sizes using a variety of machine
learning approaches. OmniTune uses a client-server model to decouple
the autotuning logic from target programs and to maintain separation
of concerns. It uses lightweight inter-process communication to
achieve low latency autotuning, and uses caches and lookup tables to
minimise the one-off costs of feature extraction.



  \chapter{Autotuning SkelCL Stencils}\label{chap:omnitune-skelcl}
  \section{Introduction}

In this chapter I apply the OmniTune framework to SkelCL. The publicly
available implementation
\footnote{\url{https://github.com/ChrisCummins/omnitune}} predicts
workgroup sizes for OpenCL stencil skeleton kernels in order to
minimise their runtime on CPUs and multi-GPU systems. The optimisation
space presented by the workgroup size of OpenCL kernels is large,
complex, and non-linear. Successfully applying machine learning to
such a space requires plentiful training data, the careful selection
of features, and classification approach. The following sections
address these challenges.


% Anyone downloading a copy of OmniTune will instantly have access to
% the global database of training data, including the
% \checkme{16917118} runtimes which were collected to write this
% thesis.

% \texttt{cec.chlox1mra3iz.us-west-2.rds.amazonaws.com:3306}

\section{Training}\label{sec:training}

One challenge of performing empirical performance evaluations is
gathering enough applications to ensure meaningful
comparisons. Synthetic benchmarks are one technique for circumventing
this problem. The automatic generation of such benchmarks has clear
benefits for reducing evaluation costs; however, creating meaningful
benchmark programs is a difficult problem if we are to avoid the
problems of redundant computation and produce provable halting
benchmarks.

In practise, stencil codes exhibit many common traits: they have a
tightly constrained interface, predictable memory access patterns, and
well defined numerical input and output data types. This can be used
to create a confined space of possible stencil codes by enforcing
upper and lower bounds on properties of the codes which can not
normally be guaranteed for general-purpose programs, e.g.\ the number
of floating point operations. In doing so, it is possible to
programatically generate stencil workloads which share similar
properties to those which we intend to target.

Based on observations of real world stencil codes from the fields of
cellular automata, image processing, and PDE solvers, I implemented a
stencil generator which uses parameterised kernel templates to produce
source codes for collecting training data. The stencil codes are
parameterised by stencil shape (one parameter for each of the four
directions), input and output data types (either integers, or single
or double precision floating points), and \emph{complexity} --- a
simple boolean metric for indicating the desired number of memory
accesses and instructions per iteration, reflecting the relatively
bi-modal nature of the reference stencil codes, either compute
intensive (e.g. FDTD simulations), or lightweight (e.g. Game of Life).

Using a large number of synthetic benchmarks helps adress the ``small
$n$, large $P$'' problem, which describes the difficulty of
statistical inference in spaces for which the set of possible
hypotheses $P$ is significantly larger than the number of observations
$n$. By creating parameterised, synthetic benchmarks, it is possible
to explore a much larger set of the space of possible stencil codes
than if relying solely on reference applications, reducing the risk of
overfitting to particular program features.


\section{Stencil Features}


\begin{table}
  \begin{multicols}{2}
    \scriptsize
    \centering
    \rowcolors{2}{white}{gray!25}
    \begin{tabular}{p{4.5cm}p{1.3cm}}
      \toprule
      \textbf{Dataset Features} &         \textbf{Type} \\
      \midrule
      Number of columns in matrix & numeric \\
      Number of rows in matrix & numeric \\
      Input data type & categorical \\
      Output data type & categorical \\
      \bottomrule
    \end{tabular}
    \vskip 2.75em
    \begin{tabular}{p{4.5cm}p{1.3cm}}
      \toprule
      \textbf{Kernel Features} &         \textbf{Type} \\
      \midrule
      Border region north & numeric \\
      Border region south & numeric \\
      Border region east & numeric \\
      Border region west & numeric \\
      Static instruction count & numeric \\
      \texttt{AShr} instruction density & numeric \\
      \texttt{Add} instruction density & numeric \\
      \texttt{Alloca} instruction density & numeric \\
      \texttt{And} instruction density & numeric \\
      \texttt{Br} instruction density & numeric \\
      \texttt{Call} instruction density & numeric \\
      \texttt{FAdd} instruction density & numeric \\
      \texttt{FCmp} instruction density & numeric \\
      \texttt{FDiv} instruction density & numeric \\
      \texttt{FMul} instruction density & numeric \\
      \texttt{FPExt} instruction density & numeric \\
      \texttt{FPToSI} instruction density & numeric \\
      \texttt{FSub} instruction density & numeric \\
      \texttt{GetElementPtr} instruction density & numeric \\
      \texttt{ICmp} instruction density & numeric \\
      \texttt{InsertValue} instruction density & numeric \\
      \texttt{Load} instruction density & numeric \\
      \texttt{Mul} instruction density & numeric \\
      \texttt{Or} instruction density & numeric \\
      \texttt{PHI} instruction density & numeric \\
      \texttt{Ret} instruction density & numeric \\
      \texttt{SDiv} instruction density & numeric \\
      \texttt{SExt} instruction density & numeric \\
      \texttt{SIToFP} instruction density & numeric \\
      \texttt{SRem} instruction density & numeric \\
      \texttt{Select} instruction density & numeric \\
      \texttt{Shl} instruction density & numeric \\
      \texttt{Store} instruction density & numeric \\
      \texttt{Sub} instruction density & numeric \\
      \texttt{Trunc} instruction density & numeric \\
      \texttt{UDiv} instruction density & numeric \\
      \texttt{Xor} instruction density & numeric \\
      \texttt{ZExt} instruction density & numeric \\
      Basic block density & numeric \\
      Memory instruction density & numeric \\
      Non external functions density & numeric \\
      Kernel max workgroup size & numeric \\
      \bottomrule
    \end{tabular}
    \vfill
    \columnbreak
    \rowcolors{2}{white}{gray!25}
    \begin{tabular}{p{4.5cm}p{1.3cm}}
      \toprule
      \textbf{Device Features} &         \textbf{Type} \\
      \midrule
      SkelCL device count & numeric \\
      Device address width & categorical \\
      Double precision fp.\ configuration & categorical \\
      Big endian? & categorical \\
      Execution capabilities & categorical \\
      Supported extensions & categorical \\
      Global memory cache size & numeric \\
      Global memory cache size & categorical \\
      Global memory cacheline size & numeric \\
      Global memory size & numeric \\
      Host unified memory? & categorical \\
      2D image max height & numeric \\
      2D image max width & numeric \\
      3D image max depth & numeric \\
      3D image max height & numeric \\
      3D image max width & numeric \\
      Image support & categorical \\
      Local memory size & numeric \\
      Local memory type & categorical \\
      Max clock frequency & numeric \\
      Number of compute units & numeric \\
      Max kernel constant args & numeric \\
      Max constant buffer size & numeric \\
      Max memory allocation size & numeric \\
      Max parameter size & numeric \\
      Max read image arguments & numeric \\
      Max samplers & numeric \\
      Max device workgroup size & numeric \\
      Max workitem dimensions & numeric \\
      Max work item sizes width & numeric \\
      Max work item sizes height & numeric \\
      Max work item sizes depth & numeric \\
      Max write image arguments & numeric \\
      Mem base address align & numeric \\
      Min data type align size & numeric \\
      Native vector width \texttt{char} & numeric \\
      Native vector width \texttt{double} & numeric \\
      Native vector width \texttt{float} & numeric \\
      Native vector width \texttt{half} & numeric \\
      Native vector width \texttt{int} & numeric \\
      Native vector width \texttt{long} & numeric \\
      Native vector width \texttt{short} & numeric \\
      Preferred vector width \texttt{char} & numeric \\
      Preferred vector width \texttt{double} & numeric \\
      Preferred vector width \texttt{float} & numeric \\
      Preferred vector width \texttt{half} & numeric \\
      Preferred vector width \texttt{int} & numeric \\
      Preferred vector width \texttt{long} & numeric \\
      Preferred vector width \texttt{short} & categorical \\
      Queue properties & categorical \\
      Single precision fp.\ configuration & categorical \\
      Device type & categorical \\
      OpenCL vendor & categorical \\
      OpenCL vendor ID & categorical \\
      OpenCL version & categorical \\
      \bottomrule
    \end{tabular}
  \end{multicols}
  \caption[OmniTune SkelCL Stencil features]{%
  OmniTune SkelCL Stencil features for dataset, kernel, and device.%
  }
  \label{tab:features}
\end{table}


Properties of the architecture, program, and dataset all contribute to
the performance of a workgroup size. The success of a machine learning
system depends on the ability to translate these properties into
meaningful explanatory variables --- \emph{features}. To capture this
in OmniTune, parameter requests are packed with a copy of the OpenCL
kernel and attributes of the dataset and device. The OmniTune server
extracts 102 features describing hte architecture, kernel, and dataset
from the message:
%
\begin{itemize}
  \item \textbf{Device} --- OmniTune uses the OpenCL
  \texttt{clGetDeviceInfo()} API to query a number of properties about
  the target execution device. Examples include the size of local
  memory, maximum work group size, number of compute units, etc.
  \item \textbf{Kernel} --- The user code for a stencil is passed to the
  OmniTune server, which compiles the OpenCL kernel to LLVM IR
  bitcode. The \texttt{opt} \texttt{InstCount} statistics pass is used
  to obtain static counts for each type of instruction present in the
  kernel, as well as the total number of instructions. The instruction
  counts for each type are divided by the total number of instructions
  to produce a \emph{density} of instruction for that type. Examples
  include total static instruction count, ratio of instructions per
  type, ratio of basic blocks per instruction, etc.
  \item \textbf{Dataset} --- The SkelCL container type is used to
  extract the input and output data types, and the 2D grid size.
\end{itemize}
%
See Table~\ref{tab:features} for a list of feature names and types.


\subsection{Reducing Feature Extraction Overhead}


Feature extraction (particularlly compilation to LLVM IR) introduces a
runtime overhead to the classification process. To minimise this,
lookup tables for device and dataset features are used, and cached
locally in the OmniTune server and pushed to the remote data
store. The device ID is used to index the devices table, and the
checksum of an OpenCL source is used to index the kernel features
table. Before feature extraction for either occurs, a lookup is
performed in the relevant table, meaning that the cost of feature
extraction is amortised over time.


\section{Optimisation Parameters}\label{sec:op-params}

SkelCL stencil kernels are parameterised by a workgroup size $w$,
which consists of two integer values to denote the number of rows and
columns (where we need to distinguish the individual components, we
will use symbols $w_r$ and $w_c$ to denote rows and columns,
respectively).


\subsection{Constraints}

Unlike in many autotuning applications, the space of optimisation
parameter values is subject to hard constraints, and these constraints
cannot conviently be statically determined. Contributing factors are
architectural limitations, kernel constraints, and refused parameters.


\subsubsection{Architectural constraints}

Each OpenCL device imposes a maximum workgroup size which can be
statically checked by querying the \texttt{clGetDeviceInfo()} API for
that device. These are defined by archiectural limitations of how code
is mapped to the underlying execution hardware. Typical values are
powers of two, e.g.\ 1024, 4096, 8192.


\subsubsection{Kernel constraints}

At runtime, once an OpenCL program has been compiled to a kernel,
users can query the maximum workgroup size supported by that kernel
using the \texttt{clGetKernelInfo()} API. This value cannot easily be
obtained statically as there is no mechanism to determine the maximum
workgroup size for a given source code and device without first
compiling it, which in OpenCL does not occur until runtime. Factors
which affect a kernel's maximum workgroup size include the number
registers required for a kernel, and the available number of SIMD
execution units for each type of instructions in a kernel.


\subsubsection{Refused parameters}

In addition to satisfying the constraints of the device and kernel,
not all points in the workgroup size optimisation space are guaranteed
to provide working programs. A refused parameter is a workgroup size
which satisfies the kernel and architectural constraints, yet causes a
\texttt{CL\_OUT\_OF\_RESOURCES} error to be thrown when the kernel is
enqueued. Note that in many OpenCL implementations, this error type
acts as a generic placeholder and may not necessarily indicate that
the underlying cause of the error was due to finite resources
constraints.


\subsubsection{Legality}

We define a \emph{legal} workgroup size as one which, for a given
\emph{scenario} (a combination of program, device, and dataset),
satisfies the architectural and kernel constraints, and is not
refused. The subset of all possible workgroup sizes
$W_{legal}(s) \subset W$ that are legal for a given sceanario $s$ is
then:
%
\begin{equation}
  W_{legal}(s) = \left\{w | w \in W, w < W_{\max}(s) \right\} - W_{refused}(s)
\end{equation}
%
Where $W_{\max}(s)$ can be determined at runtime prior to the kernels
execution, but the set $W_{refused}(s)$ can only be determined
experimentally.


\subsection{Assessing Relative Performance}

Given a set of observations, where an observation is a scenario,
workgroup size tuple $(s,w)$; a function $t(s,w)$ which returns the
arithmetic mean of the runtimes for a set of observations; we can
calculate the speedup $r(s, w_1, w_2)$ of competing workgroup sizes
$w_1$ over $w_2$ using:
%
\begin{equation}
  r(s, w_1, w_2) = \frac{t(s,w_2)}{t(s,w_1)}\\
\end{equation}
%

\subsubsection{Oracle Workgroup Size}

The \emph{oracle} workgroup size $\Omega(s) \in W_{legal}(s)$ of a
sceanrio $s$ is the $w$ value which provides the lowest mean
runtime. This allows for comparing the performance $p(s,w)$ of a
particular workgroup against the maximum available performance for
that scenario, within the range $0 \le p(s,w) \le 1$:

\begin{align}
  \Omega(s) &= \argmin_{w \in W_{legal}(s)} t(s,w)\\
  p(s,w) &= r(s, w, \Omega(s))
\end{align}


\subsubsection{Establishing a Baseline}

The geometric mean is used to aggregate normalised relative
performances due to its multiplicative
property~\cite{Fleming1986}. For a given workgroup size, the average
performance $\bar{p}(w)$ across the set of all scenarios $S$ can be
found using the geometric mean of performance relative to the oracle:
%
\begin{equation}
  \bar{p}(w) =
  \left(
  \prod_{s \in S} r(s, w, \Omega(s))
  \right)^{1/|S|}
\end{equation}
%
The \emph{baseline} workgroup size $\bar{w}$ is the value which
provides the best average case performance across all scenarios. Such
a baseline value represents the \emph{best} possible performance which
can be achieved using a single, statically chosen workgroup size. By
defining $W_{safe} \in W$ as the intersection of legal workgroup
sizes, the baseline can be found using:

\begin{align}
  W_{safe} &= \cap \left\{ W_{legal}(s) | s \in S \right\}\\
  \bar{w} &= \argmax_{w \in W_{safe}} \bar{p}(w)
\end{align}


\section{Machine Learning}\label{sec:omnitune-ml}

The challenge is to design a system which, given a set of prior
observations of the empirical performance of stencil codes with
different workgroup sizes, predict workgroup sizes for \emph{unseen}
stencils which will maximise the performance. The OmniTune server
supports three methods for achieving this.


\subsection{Predicting Oracle Workgroup Size}\label{subsec:omnitune-ml-class}

The first approach to predicting workgroup sizes is to consider the
set of possible workgroup sizes as a hypothesis space, and to use a
classifier to predict, for a given set of features, the workgroup size
which will provide the best performance. The classifier takes a set of
training scenarios $S_{training}$, and generates labelled training
data as pairs of scenario features $f(s)$ and observed oracle
workgroup sizes:
%
\begin{equation}
  T = \left\{ \left(f(s), \Omega(s)\right) | s \in S_{training} \right\}
\end{equation}
%
During testing, the classifier predicts workgroup sizes from the set
of oracle workgroup sizes from the training set:
%
\begin{equation}
  W_{training} = \left\{ \Omega(s) | s \in S_{training} \right\}
\end{equation}
%
This approach presents the problem that after training, there is no
guarantee that the set of workgroup sizes which may be predicted is
within the set of legal workgroup sizes for future scenarios:
%
\begin{equation}
  \bigcup_{\forall s \in S_{testing}} W_{legal}(s) \nsubseteq W_{training}
\end{equation}
%
This may result in a classifier predicting a workgroup size which is
not legal for a scenario, $w \not\in W_{legal}(s)$, either because it
exceeds $W_{\max}(s)$, or because the parameter is refused. For these
cases, I evaluate the effectiveness of three fallback strategies to
select a legal workgroup size:
%
\begin{enumerate}
  \item \emph{Baseline} --- select the workgroup size which is known to
  be safe $w < W_{safe}$, and provides the highest average case
  performance on training data.
  \item \emph{Random} --- select a random workgroup size which is known
  from prior observations to be legal $w \in W_{legal}(s)$.
  \item \emph{Nearest Neighbour} --- select the workgroup size which
  from prior observations is known to be legal, and has the lowest
  Euclidian distance to the prediction.
\end{enumerate}
%
See Algorithm~\ref{alg:autotune-classification} for definitions.

\begin{algorithm}
  \begin{algorithmic}[1]
\Require scenario $s$
$d$.
\Ensure workgroup size $w$

\Procedure{Baseline}{s}
\Comment Select the best $w$ from $W_{safe}$.
\State $w \leftarrow \text{classify}(f(s))$
\If{$w \in W_{legal}(s)$}
    \State \textbf{return} $w$
\Else
  \State \textbf{return} $\underset{w \in W_{safe}}{\argmax}
\left(
  \prod_{s \in S_{training}} p(s, w)
\right)^{1/|S_{training}|}$
\EndIf
\EndProcedure
\item[] % line break

\Procedure{Random}{s}
\Comment Select a random workgroup size.
\State $w \leftarrow \text{classify}(f(s))$
\While{$w \not\in W_{legal}(s)$}
  \State $w \leftarrow $ random choice $w \in \left\{ w | w < W_{max}(s), w \not\in W_{refused}(s) \right\}$
\EndWhile
\State \textbf{return} $w$
\EndProcedure
\item[] % line break

\Procedure{NearestNeighbour}{s}
\Comment Select the closest workgroup size to prediction.
\State $w \leftarrow \text{classify}(f(s))$
\While{$w \not\in W_{legal}(s)$}
  \State $d_{min} \leftarrow \infty$
  \State $w_{closest} \leftarrow \text{null}$
  \For{$c \in \left\{ w | w < W_{\max}(s), w \not\in W_{refused}(s) \right\}$}
    \State $d \leftarrow \sqrt{\left(c_r - w_r\right)^2 + \left(c_c - w_c\right)^2}$
    \If{$d < d_{min}$}
      \State $d_{min} \leftarrow d$
      \State $w_{closest} \leftarrow c$
    \EndIf
  \EndFor
  \State $w \leftarrow w_{closest}$
\EndWhile
\State \textbf{return} $w$
\EndProcedure
\end{algorithmic}

  \caption{Selecting optimal workgroup sizes using classification}
  \label{alg:autotune-classification}
\end{algorithm}

\subsection{Predicting Stencil Code Runtime}\label{subsec:omnitune-ml-runtime}

\begin{algorithm}
  \begin{algorithmic}[1]
    \Require kernel features $k$, hardware features $h$, dataset features
    $d$.
    \Ensure workgroup size $w$

    \State $W \leftarrow \left\{ w | w < W_{\max}(s) \right\} - W_{refused}(s)$
    \Comment Set of possible workgroup sizes.
    \State $w \leftarrow \underset{w \in W}{\argmin} g(f(s), w)$
    \Comment Predict candidate workgroup size.
    \While{$w \not\in W_{legal}(s)$}
    \State $W \leftarrow W - \left\{ w \right\}$
    \Comment Remove candidate from set.
    \State $w \leftarrow \underset{w \in W}{\argmin} g(f(s), w)$
    \Comment Select next candidate workgroup size.
    \EndWhile
    \State \textbf{return} $w$
  \end{algorithmic}
  \caption{Selecting workgroup sizes by predicting program runtimes}
  \label{alg:autotune-runtime-regression}
\end{algorithm}

A problem of predicting oracle workgroup sizes is that it requires
each training point to be labelled with the oracle workgroup size
which can be only be evaluated using an exhaustive search. An
alternative approach is to build a model to attempt to predict the
\emph{runtime} of a stencil given a specific workgroup size. This
allows for training on data for which the oracle workgroup size is
unknown, and has the secondary advantage that this allows for an
additional training data point to be gathered each time a stencil is
evaluated. Given training data consisting of $(f(s),w,t)$ tuples,
where $s$ is a scenario, $w$ is the workgroup size, and $t$ is the
observed mean runtime, we can train a regressor $g(f(s), w)$ which
predicts the mean runtime of an unseen scenario. The predicted oracle
workgroup size $\bar{\Omega}(s)$ is then the $w$ value which minimises
the output of the regressor:
%
\begin{equation}
  \bar{\Omega}(s) = \underset{w \in W_{legal}(s)}{\argmin} g(f(s), w)
\end{equation}
%
Note that since we cannot know in advance which workgroup sizes will
be refused, that is, $W_{refused}(s)$ cannot be statically determined,
this process must be iterated until a workgroup size which not refused
is selected. Algorithm~\ref{alg:autotune-runtime-regression} shows
this process.


\subsection{Predicting Relative Performance of Workgroup Sizes}\label{subsec:omnitune-ml-speedup}

\begin{algorithm}
  \begin{algorithmic}[1]
    \Require kernel features $k$, hardware features $h$, dataset features
    $d$, baseline $w_b$.
    \Ensure workgroup size $w$

    \State $W \leftarrow \left\{ w | w < W_{\max}(s) \right\} - W_{refused}(s)$
    \Comment Set of possible workgroup sizes.
    \State $w \leftarrow \underset{w \in W}{\argmax} g(f(s), w,w_b)$
    \Comment Predict candidate workgroup size.
    \While{$w \not\in W_{legal}(s)$}
    \State $W \leftarrow W - \left\{ w \right\}$
    \Comment Remove candidate from set.
    \State $w \leftarrow \underset{w \in W}{\argmax} g(f(s), w,w_b)$
    \Comment Select next candidate workgroup size.
    \EndWhile
    \State \textbf{return} $w$
  \end{algorithmic}
  \caption{Selecting workgroup sizes by predicting relative performance}
  \label{alg:autotune-speedup-regression}
\end{algorithm}

Accurately predicting the runtime of an arbitrary program is a
difficult problem due to the impacts of flow control. In such cases,
it may be more effective to instead predict the \emph{relative}
performance of two different workgroup sizes for the same program. To
do this, we select a baseline workgroup size $w_b \in W_{safe}$, and
train a regressor $g(f(s),w,w_b)$ with training data labelled with the
relative performance over the baseline $r(w, w_b)$. Predicting the
optimal workgroup requires maximising the output of the regressor:
%
\begin{equation}
  \bar{\Omega}(s) = \underset{w \in W_{legal}(s)}{\argmax} g(f(s),w,w_b)
\end{equation}
%
As with predicting runtimes, this process must be iterated to
accommodate for the emergent properties of $W_{legal}(s)$. See
Algorithm~\ref{alg:autotune-speedup-regression} for a description of
this process.


\section{Implementation}

The OmniTune framework is implemented as a set of Python classes and
interfaces, which are inherited from or implemented to target specific
autotuning cases. The entire framework consists of 8987 lines of
Python code, of which 976 is dedicated to the SkelCL frontend. By
design, the client-server model minimises the impact of number of
modifications that are required to enable autotuning in client
applications. The only modification required is to replace the
hardcoded values for workgroup size with a subroutine to request a
workgroup size from the OmniTune server over a DBUS connection. The
server is implemented as a standalone Python program, and uses sqlite
to maintain local data caches. The OmniTune remote is an Amazon Web
Services virtual machine instance, using MySQL as the relational data
store. Figure~\ref{fig:omnitune-system-flow} shows the relational
database schema used to store stencil runtime information. Additional
tables store data in coalesced forms for use as machine learning
training data.

For classification, five classifiers are supported, chosen for their
contrasting properties: Naive Bayes, SMO, Logistic Regression, J48
Decision tree, and Random Forest. For regression, a Random Forest with
regression trees is used, chosen because of its efficient handling of
large feature sets, compared to linear models.

\begin{figure}
  \centering
  \includegraphics[width=.95\textwidth]{img/omnitune-data-schema.pdf}
  \caption[Database schema for storing performance results]{%
  Database schema for storing SkelCL stencil runtimes. Feature lookup
  tables and normalisation are used to provide extremely compact
  storage, requiring only 56 bytes for each additional runtime of a
  known stencil program.%
  }
  \label{fig:omnitune-system-flow}
\end{figure}


\section{Summary}

This section has described has the application of OmniTune for
predicting workgroup sizes of SkelCL stencil programs, using three
different machine learning approaches. The first approach is to
predict the optimal workgroup size for a given program based on
training data which included the optimal workgroup sizes for other
stencils. The second approach is to select a workgroup sizex by
sweeping the space of possible workgroup sizes, predicting the runtime
a program with each. The third approach is to select a workgroup size
by sweeping the space of possible workgroup sizes, predicting the
relative gain of each compared to a known baseline. In the next
section, we will describe the process for collecting empirical
performance data.



  % METHODOLOGY
  % ===========
  %
  % Description of the work undertaken: this may be divided into
  % chapters describing the conceptual design work and the actual
  % implementation separately. Any problems or difficulties and the
  % suggested solutions should be mentioned. Alternative solutions and
  % their evaluation should also be included.

  \chapter{Exploring the Workgroup Size Space}\label{chap:methodology}
  \section{Methodology}

Use geometric mean for speedups~\cite{Fleming1986}. Statistical
rigour\cite{Georges2007}. Execution times and
variance~\cite{Box}. Benchmarking parallel computing
systems~\cite{Belli2015}.


\subsection{Skeleton-aware compiler optimizations}

First steps:
%
\begin{enumerate}
\item Write a test case program
\item Compile to bytecode and check optimisation is not applied
\item Implement optimisation by hand to source code
\item Measure performance improvement
\item Implement compiler pass to apply optimisation to bytecode
\item Test on real world benchmarks
\end{enumerate}
%
Example: constant propagation across muscle functions in pipe:
%
\begin{enumerate}
\item Write a pipeline of 100 "+1” functions.
\item Compile to bytecode and check that pipeline isn’t collapsed.
\item Collapse pipeline to single stage.
\item Measure performance improvement.
\item Implement compiler pass to apply optimisation to bytecode.
\item Test on real world benchmarks.
\end{enumerate}
%
Next step: load balancing across pipelines - stream partitioning?

Related work: dataflow analysis across skeletons for scheduling, what
kind of optimisations are applied in TLB/actor frameworks, actors in
scala


\subsection{Skeleton-aware debugging}
debugging+profiling for checkpointing

TBB debugging

exoskeleton

SkePU related workx

  % EVALUATION
  % ==========
  %
  % Results and their critical analysis should be reported, whether the
  % results conform to expectations or otherwise and how they compare
  % with other related work. Where appropriate evaluation of the work
  % against the original objectives should be presented.

  \chapter{Evaluation}\label{chap:evaluation}
  \section{Qualitative Evaluation of Generated Programs}\label{sec:eval}

In this section we evaluate the quality of programs synthesized by CLgen by their likeness to hand-written code.

% \subsection{Likeness to Hand-written Code}

Judging whether a piece of code has been written by a human is a challenging task for a machine, so we adopt a methodology from machine learning research based on the \emph{Turing Test}~\cite{Gao2015a,Zhang2016,Vinyals}. We reason that if the output of CLgen is human like code, then a human judge will be unable to distinguish it from hand-written code.

We devised a double blind test in which 15 volunteer OpenCL developers from industry and academia were shown 10 OpenCL kernels each. Participants were tasked with judging whether, for each kernel, they believed it to have been written by hand or by machine. Kernels were randomly selected for each participant from two equal sized pools of synthetically generated and hand-written code from GitHub. We applied the code rewriting process to all kernels to remove comments and ensure uniform identifier naming. The participants were divided into two groups, with 10 of them receiving code generated by CLgen, and 5 of them acting as a control group, receiving code generated by CLSmith~\cite{Lidbury2015a}, a program generator for differential testing\footnote{An online version of this test is available at \emph{http://humanorrobot.uk/}.}.

We scored each participant's answers, finding the average score of the control group to be 96\% (stdev.\ 9\%), an unsurprising outcome as generated programs for testing have multiple ``tells'', for example, their only input is a single \texttt{ulong} pointer. There were no false positives (synthetic code labeled human) for CLSmith, only false negatives (human code labeled synthetic). With CLgen synthesized programs, the average score was 52\% (stdev.\ 17\%), and the ratio of errors was even. This suggests that CLgen code is indistinguishable from hand-written programs, with human judges scoring no better than random chance.



  % CONCLUSION
  % ==========
  %
  % Concluding remarks and observations, unsolved problems, suggestions
  % for further work.
  \chapter{Conclusions}\label{chap:conclusions}
  \chapter{Conclusions}
\label{chap:conclusions}

\section{Contributions}

This section summarises the main contributions of this thesis for XXX.

\subsection{Workload Characterisation}

\subsection{Compiler Optimisations}

\subsection{Compiler Testing}

\section{Critical Analysis}

\subsection{Limitations of Generative Models}

Extensibility to other languages largely untested.

Generating multi-function programs.

% TODO(cec): CGO'17 limitations

Our new approach enables the synthesis of more human-like programs than current state-of-the-art program generators, and without the expert guidance required by template based generators, but it has limitations. Our method of seeding the language models with the start of a function means that we cannot support user defined types, or calls to user-defined functions. This means that we only consider scalars and arrays as inputs; while 6 (2.3\%) of the benchmark kernels from Table~\ref{tab:benchmarks} use irregular data types as inputs. We will address this limitation through recursive program synthesis, whereby a call to a user-defined function or unrecognised type will trigger candidate functions and type definitions to be synthesised. Currently we only run single-kernel benchmarks. We will extend the host driver to explore multi-kernel schedules and interleaving of kernel executions. Our host driver generates data sets from uniform random distributions, as do many of the benchmark suites. For cases where non-uniform inputs are required (e.g. profile-directed feedback), an alternate methodology for generating inputs must be adopted.


\subsection{Limitations of Sequential Classification}

\section{Future Work}

\todo[inline]{Graph-level representations with RNNS~\cite{Jin2018}. Graph surveys~\cite{Li2018a,Wu2018}.}

\todo[inline]{Table of GitHub corpus size by programming language. There's room for machine learning in lots of languages! E.g. Haskell, Java, C/C++, Solidity, Python, OpenCL}


  \clearpage
  \label{bibliography}
  \printbibliography

\end{document}
