\documentclass[prodmode,acmtaco]{acmsmall}

% Package to generate and customize Algorithm as per ACM style
\usepackage[ruled]{algorithm2e}
\renewcommand{\algorithmcfname}{ALGORITHM}
\SetAlFnt{\small}
\SetAlCapFnt{\small}
\SetAlCapNameFnt{\small}
\SetAlCapHSkip{0pt}
\IncMargin{-\parindent}

% Metadata Information
\acmVolume{9}
\acmNumber{4}
\acmArticle{39}
\acmYear{2016}
\acmMonth{3}

% Copyright
%\setcopyright{acmcopyright}
\setcopyright{acmlicensed}
%\setcopyright{rightsretained}
%\setcopyright{usgov}
%\setcopyright{usgovmixed}
%\setcopyright{cagov}
%\setcopyright{cagovmixed}

% DOI
\doi{0000001.0000001}

%ISSN
\issn{1234-56789}


%%%%%%%%%%%%%%%%%%%%%%%%%
%% Document and Layout %%
%%%%%%%%%%%%%%%%%%%%%%%%%

% Fix for multiple "No room for a new \dimen" errors.
%
% See: http://tex.stackexchange.com/questions/38607/no-room-for-a-new-dimen
%
\usepackage{etex}

\usepackage[utf8]{inputenc}

% Fix for "'babel/polyglossia' detected but 'csquotes' missing"
% warning. NOTE: Include after inputenc.
%
\usepackage{csquotes}

% Make internal macro definitions accessible,
% e.g. \@title, \@date \@author.
\makeatletter

% Multi-column support.
\usepackage{multicol}

% A useful package which includes macros like \ifdef{}{}{}:
%
\usepackage{etoolbox}

% Uncomment the following line to remove column separation:
%
%\setlength{\columnsep}{5mm}

% Allow user-defined warning and error filters.
%
\usepackage{silence}

\usepackage{adjustbox}


%%%%%%%%%%%%%%%%%%%%%
% Table of Contents %
%%%%%%%%%%%%%%%%%%%%%

% % Set chapter and section numbering depth:
% %
% \setcounter{secnumdepth}{2}


%%%%%%%%%%%%%%%%
% Bibliography %
%%%%%%%%%%%%%%%%
% \usepackage[%
%     backend=biber,
%     style=ieee,
%     % style=numeric-comp,
%     % style=numeric-comp,  % numerical-compressed
%     sorting=none,        % nty,nyt,nyvt,anyt,anyvt,ynt,ydnt,none
%     sortcites=true,      % sort \cite{b a d c}: true,false
%     block=none,          % space between blocks: none,space,par,nbpar,ragged
%     indexing=false,      % indexing options: true,false,cite,bib
%     citereset=none,      % don't reset cites
%     isbn=false,          % print ISBN?
%     url=true,            % print URL?
%     doi=false,           % print DOI?
%     natbib=true,         % natbib compatability
%    ]{biblatex}

% \usepackage{natbib}

% % Filter annoying and unavoidable biblatex warning:
% \WarningFilter{biblatex}{Patching footnotes failed}

% Reduce the font size of the bibliography:
% \renewcommand{\bibfont}{\normalfont\scriptsize}

% Determine which BibTeX file to use:
%
% If available, use my Mendeley BibTex library, located in the home
% directory. Note that this is a relative path and will break if
% either this file or the BibTex library are moved. If the library is
% not present, use the local refs.bib file.
% \newcommand{\BibResourceGlobal}{../../../library.bib}
% \newcommand{\BibResourceLocal}{refs.bib}

% \IfFileExists{\BibResourceGlobal}
%   {\newcommand{\BibResource}{\BibResourceGlobal}}
%   {\newcommand{\BibResource}{\BibResourceLocal}}

% \addbibresource{\BibResource}


%%%%%%%%%%%%%%
% Appendices %
%%%%%%%%%%%%%%

% Appendix package. Documentation:
%
%  http://mirror.ox.ac.uk/sites/ctan.org/macros/latex/contrib/appendix/appendix.pdf
%
% Package options:
%
% toc      - Put a header (e.g., `Appendices') into the Table of Contents
%            (the ToC) before listing the appendices. (This is done by
%            calling the \addappheadtotoc command.)
% page     - Puts a title (e.g., `Appendices') into the document at the
%            point where the appendices environment is begun. (This is
%            done by calling the \appendixpage command.)
% title    - Adds a name (e.g., `Appendix') before each appendix title in
%            the body of the document. The name is given by the value
%            of \appendixname. Note that this is the default behaviour
%            for classes that have chapters.
% titletoc - Adds a name (e.g., `Appendix') before each appendix listed
%            in the ToC. The name is given by the value
%            of \appendixname.
% header   - Adds a name (e.g., `Appendix') before each appendix in page
%            headers.  The name is given by the value
%            of \appendixname. Note that this is the default behaviour
%            for classes that have chapters.
\usepackage[title, titletoc]{appendix}

% pre-requisites for rendering upquotes in listings package.
\usepackage[T1]{fontenc}
\usepackage{lmodern}
\usepackage{textcomp}

% code listings.
\usepackage{listings}

% set \ttfamily to use courier fonts.
%
% See: http://tex.stackexchange.com/a/33686
%
\usepackage{courier}

\lstset{frame=bt,                    % Add top and bottom frame lines
breaklines=true,             % Force line wrapping
captionpos=b,                % Place caption below listing
numbers=left,                % Add left-side line numbers
basicstyle=\scriptsize\ttfamily, % Set font size and type
showstringspaces=false,      % Don't show visible whitespace
numberstyle=\tiny,
upquote=true,                % Use upright quotes, not curly
commentstyle=\bfseries}      % Embolden comments

% Use (*@ @*) to escape LaTeX commands within listings.
\lstset{escapeinside={(*@}{@*)}}

% Add 10pt space between chapters in TOC listings entries:
%\let\Chapter\chapter
%\def\chapter{\addtocontents{lol}{\protect\addvspace{10pt}}\Chapter}


%%%%%%%%%%%%%%%%%%%%%%%%
%% Graphics and maths %%
%%%%%%%%%%%%%%%%%%%%%%%%
\usepackage{amsmath}

% Vector notation, e.g. \vv{x}:
%
\usepackage{esvect}

% Additional amsmath symbols, see:
%
% http://texblog.org/2007/08/27/number-sets-prime-natural-integer-rational-real-and-complex-in-latex/
%
\usepackage{amsfonts}
\usepackage{amssymb}

\usepackage{graphicx}
\usepackage{mathtools}
\usepackage{tikz}
\usepackage{tikz-qtree}

% Provide bold font face in maths.
\usepackage{bm}

\usepackage{subcaption}
\expandafter\def\csname ver@subfig.sty\endcsname{}

% Define an 'myalignat' command which behave as 'alignat' without the
% vertical top and bottom padding. See:
%     http://www.latex-community.org/forum/viewtopic.php?f=5&t=1890
\newenvironment{myalignat}[1]{%
\setlength{\abovedisplayskip}{-.7\baselineskip}%
\setlength{\abovedisplayshortskip}{\abovedisplayskip}%
\start@align\z@\st@rredtrue#1
}%
{\endalign}

% Define additional operators:
\DeclareMathOperator*{\argmin}{arg\,min}
\DeclareMathOperator*{\argmax}{arg\,max}

\DeclareMathOperator*{\gain}{Gain}

% Skeleton operators.
\DeclareMathOperator*{\map}{Map}
\DeclareMathOperator*{\reduce}{Reduce}
\DeclareMathOperator*{\scan}{Scan}
\DeclareMathOperator*{\stencil}{Stencil}
\DeclareMathOperator*{\zip}{Zip}
\DeclareMathOperator*{\allpairs}{All\,Pairs}

% Maths plots using pgfplots, see:
%
%     http://pgfplots.sourceforge.net/pgfplots.pdf
%
\usepackage{pgfplots}

% Disable compatability mode.
%
\pgfplotsset{compat=1.12}

% Gantt charts using pgfgantt, see:
%
%     http://www.ctan.org/pkg/pgfgantt
%
\usepackage{pgfgantt}

% Fix milestone aspect ratio by defining a custom element.
\newganttchartelement*{mymilestone}{
mymilestone/.style={
shape=diamond,
inner sep=2pt,
draw=black,
top color=black,
bottom color=black,
}
}

% Tikz flowchart configuration.
\usetikzlibrary{shapes,arrows,shadows,fit,backgrounds}
\tikzstyle{decision} = [diamond,
draw,
text width=4.5em,
text badly centered,
node distance=3cm,
inner sep=0pt]
\tikzstyle{block}    = [rectangle,
draw,
text width=5em,
text centered,
node distance=3cm,
minimum height=4em,
inner sep=.2cm]
\tikzstyle{line}     = [draw, -latex']

% Add dirtree picture style, see:
%
%     http://tex.stackexchange.com/a/34268
%
\newcount\dirtree@lvl
\newcount\dirtree@plvl
\newcount\dirtree@clvl
\def\dirtree@growth{%
\ifnum\tikznumberofcurrentchild=1\relax
\global\advance\dirtree@plvl by 1
\expandafter\xdef\csname dirtree@p@\the\dirtree@plvl\endcsname{\the\dirtree@lvl}
\fi
\global\advance\dirtree@lvl by 1\relax
\dirtree@clvl=\dirtree@lvl
\advance\dirtree@clvl by -\csname dirtree@p@\the\dirtree@plvl\endcsname
\pgf@xa=0.33cm\relax
\pgf@ya=-\baselineskip\relax
\pgf@ya=\dirtree@clvl\pgf@ya
\pgftransformshift{\pgfqpoint{\the\pgf@xa}{\the\pgf@ya}}%
\ifnum\tikznumberofcurrentchild=\tikznumberofchildren
\global\advance\dirtree@plvl by -1
\fi
}
\tikzset{
dirtree/.style={
growth function=\dirtree@growth,
every node/.style={anchor=north},
every child node/.style={anchor=west},
edge from parent path={(\tikzparentnode\tikzparentanchor) |- (\tikzchildnode\tikzchildanchor)}
}
}

% UML sequence diagram macros, see:
%
%     https://code.google.com/p/pgf-umlsd/
%
% Options:
%
%     underline - Underline object names
%
\usepackage[underline=false]{pgf-umlsd}

% Support for SVG graphics.
%
% NOTE that you must pass the "--shell-escape" argument to pdflatex to
% compile. NOTE also that images *MUST* be placed within the graphics
% path.
\usepackage{svg}
\graphicspath{{img/}}

%%%%%%%%%%%%%%%%%%%%%%
%% Tables and lists %%
%%%%%%%%%%%%%%%%%%%%%%

% Required to use labm8 exported tables.
%
\usepackage{booktabs}

% Required for full page-width tables.
\usepackage{tabularx}

%\usepackage{enumitem}
%\setenumerate{itemsep=0pt}

% Use no left margin for lists:
%\setlist{leftmargin=*}

\usepackage{longtable}

% Define column types L, C, R with known text justification and fixed
% widths:
\usepackage{array}
\newcolumntype{L}[1]{>{\raggedright\let\newline\\\arraybackslash\hspace{0pt}}m{#1}}
\newcolumntype{C}[1]{>{\centering\let\newline\\\arraybackslash\hspace{0pt}}m{#1}}
\newcolumntype{R}[1]{>{\raggedleft\let\newline\\\arraybackslash\hspace{0pt}}m{#1}}


%%%%%%%%%%%%%%%%%%%%%%%%%%%%%
%% Typesetting and symbols %%
%%%%%%%%%%%%%%%%%%%%%%%%%%%%%

% Adjustable font sizes in \Verbatim{}
\usepackage{fancyvrb}

%\usepackage{titlesec}
% Set section and paragraph heading fonts:
%\titleformat*{\section}{\Large\bfseries}
%\titleformat*{\subsection}{\normalsize\bfseries}
%\titleformat*{\subsubsection}{\normalsize}
%\titleformat*{\paragraph}{\large\bfseries}
%\titleformat*{\subparagraph}{\large\bfseries}

% Set section heading margins. Usage:
% \titlespacing*{<command>}{<left>}{<before>}{<after>}
%\titlespacing*{\section}{0pt}{.6em}{.3em}
%\titlespacing*{\subsection}{0pt}{.6em}{.2em}

% Set paragraph indentation size. Default is 15pt.
%\setlength{\parindent}{10pt}

% The line spacing can be globally set using \linespread:
%
% \linespread{1.2}

% Add a command \hr{} which will draw a horizontal rule the width of
% the text.
%
\newcommand{\hr}{\noindent\makebox[\linewidth]{\rule{\textwidth}{0.2pt}}}

% Add a command \br{} which will create a horizontal space of exactly
% one line height.
%
\newcommand{\br}{\hspace{\baselineskip}}

% Define a command to allow word breaking.
\newcommand*\wrapletters[1]{\wr@pletters#1\@nil}
\def\wr@pletters#1#2\@nil{#1\allowbreak\if&#2&\else\wr@pletters#2\@nil\fi}

% Define a command to create centred page titles.
\newcommand{\centredtitle}[1]{
\begin{center}
  \large
  \vspace{0.9cm}
  \textbf{#1}
\end{center}}

% Support hyperlinks using the \hyperref, \url and \href
% macros. Usage:
%
%    \hyperref[label_name]{''link text''}
%
%    \url{<my_url>}
%
%    \href{<my_url>}{<description>}
%
\usepackage{hyperref}

% Disable colored borders of links, cross-references etc in PDF output
\hypersetup{pdfborder={0 0 0}}

% Provide generic commands \degree, \celsius, \perthousand, \micro
% and \ohm which work both in text and maths mode.
\usepackage{gensymb}

%%%%%%%%%%%%%%%%%%%%%%%%%%%%%%%%%
%% Placeholder text generation %%
%%%%%%%%%%%%%%%%%%%%%%%%%%%%%%%%%

% Use either \blindtext or \libpsum to generate placeholder text. Also
% note the macros \blinditemize, \blindenumerate, \blinddescription.
\usepackage[english]{babel}
\usepackage{blindtext}
\usepackage{lipsum}


\let\endtitlepage\relax  % no page break after title

% Document starts
\begin{document}

\begin{titlepage}
  \begin{center}
    {\LARGE Deep Learning for Compilers \par}
    \vskip 2em
    First Year Review Document \\
    {\tiny by} \\
    Chris Cummins \\
    \vskip 2em
    Supervisors: \par
    Hugh Leather, Pavlos Petoumenos, Richard Mayr \par
    {\Large \makebox[3in]{\hrulefill} \par}
    \vskip 1em
    {\small
      \today \\
      Institute for Computing Systems Architecture,\\ School of Informatics,\\ University of Edinburgh}
  \end{center}
  \par
\end{titlepage}

\begin{abstract}
Continued advancements in machine learning have increasingly extended the state-of-the art in language modelling for natural language processing. Coupled with the increasing popularity of websites such as GitHub for hosting software projects, this raises the potential for large scale language modelling over open source code to build probabilistic models which capture both the semantics of a programming language and its common usage in real world applications. This document describes my work towards the development of systems for automatically generating programs in a given programming language. The aim of this work is improvements to predictive modelling for compiler optimisation and compiler testing. In my first year I have applied LSTMs to large corpuses of code mined from open source in order to generate executable OpenCL kernels. These generated programs have been shown to improve the performance of state-of-the-art predictive models; though the programs are typically short and limited to operating only on scalars and vectors of numerical values. This document describes the plans for future work to extend this initial proof-of-concept through the use of formal grammars to generate programs in arbitrary languages, capable of satisfying arbitrary properties of interest. This will enable the automatic generation of programs in any language for a which a large corpus of real world codes is available, with a range of applications including exploration of unknown parts of program feature spaces, and identifying bugs in compilers.
\end{abstract}

\newpage
\tableofcontents
\newpage

\section{Introduction}

This chapter reviews research in areas relevant to this thesis.

The chapter is structured as follows: Section~\ref{sec:related-work-iterative-compilation} reviews the literature of iterative compilation and machine learning for compilers. Section~\ref{sec:related-work-machine-learning-for-pl} describes related work in machine learning over programming languages. Section~\ref{sec:related-work-program-generation} reviews the relevant literature of program generation. Finally Section~\ref{sec:related-work-summary} concludes.
\section{Literature survey}

\subsection{Deep learning}

Deep learning is a nascent branch of machine learning in which deep or multi-level graphs of processing layers are used to detect patterns in natural data~\cite{Buduma2015,LeCun2015}. It is proving especially successful for its ability to ability to process natural data in its raw form. This overcomes the traditionally laborious and time-consuming practise of engineering feature extractors to process raw data into an internal representation or feature vector. Deep learning has successfully discovered structures in high-dimensional data, and is responsible for many breakthrough achievements in machine learning, including human parity in conversational speech recognition~\cite{Xiong2016}; professional level performance in video games~\cite{Mnih2015}; and autonomous vehicle control~\cite{Lozano-Perez2012}.

In past work I used the Long Short-Term Memory (LSTM) architecture of Recurrent Neural Network (RNN)~\cite{Sundermeyer2012,Mikolov2015} to generate sequences of OpenCL code (Appendix~\ref{app:cgo}). The LSTM network architecture comprises recurrent layers of \emph{memory cells}, each consisting of an input, output, and forget gate, and an output layer providing normalized probability values from a 1-of-K coded vocabulary~\cite{Graves,Graves2013}. Although this is the first application of deep learning for generating executable programs, RNNs have been successfully applied to a variety of other generative tasks, including image captioning~\cite{Vinyals}, colourising black and white photographs~\cite{Zhang2016}, artistic style~\cite{Gatys2015}, and image generation~\cite{Gregor2014}.

The proficiency of LSTMs for sequence modeling is demonstrated in~\cite{Sutskever2014}. \citeauthor{Sutskever2014} apply two LSTM networks to translate first a sequence into a fixed length vector, then to decode the vector into an output sequence. This architecture achieves state-of-the-art performance in machine translation. The authors find that reversing the order of the input sequences markedly improves translation performance by introducing new short term dependencies between input and output sequences. Such sequence transformations should be considered for the purpose of program generation.

The application of language modeling for generating executable programs is novel. In training on large corpuses of hand-written code, the language model learns the human biases which are present in common codes~\cite{Caliskan-islam2016}. While such human-induced basiases can prove controverial in social domains~\cite{Bolukbasi2016,Joseph2017}, this enables the generation of programs which, unlike other approaches to program generation, are representative of real workloads.

Neural networks are computationally expensive, though their implementations can be generic and parallelised. Library implementations are available in Torch~\cite{Collobert2011}, Caffe~\cite{Jia2014}, and TensorFlow~\cite{Abadi}. The increasing size and depth of computation graphs in deep learning has challenged the ability to compute results in reasonable time. Possible methods for reducing computational overhead involve fusing operations across layers in the graph using domain specific languages~\cite{Truong2016,Ashari2015a,Potter2015}; decoupling interfaces between layers using small networks to synthesise learning gradients during training~\cite{Jaderberg2016a}; and specialising hardware for computing data parallel workloads using FPGAs and ASICs~\cite{Misra2010}.


\paragraph{Software engineering} Machine learning has been applied to source code to aid software engineering. Naturalize employs techniques developed in the natural language processing domain to model coding conventions~\cite{Allamanis2014a}. JSNice leverages probabilistic graphical models to predict program properties such as identifier names for Javascript~\cite{Raychev}. \citeauthor{Zhang2015a} use deep learning to generate example code for APIs as responses to natural language queries~\cite{Zhang2015a}. \citeauthor{Allamanis2016} use attentional neural networks to generate summaries of source code~\cite{Allamanis2016}. \citeauthor{Wong2013} mines Q\&A site StackOverflow to automatically generate code comments~\cite{Wong2013}. \citeauthor{Raychev2014} use statistical models to provide contextual code completion~\cite{Raychev2014}.

There is an increasing interest in mining source code repositories at large scale~\cite{Allamanis2013a,White2015a,Bird2009}. Previous studies have involved data mining of GitHub to analyze software engineering practices~\cite{Wu2014,Guzman2014,Baishakhi2014a,Vasilescu2015}; however, no work so far has exploited mined source code for program generation.


\subsection{Iterative compilation}

Iterative compilation is the method of performance tuning in which a program is compiled and profiled using multiple different configurations of optimisations in order to find the configuration which maximises performance. One of the the first formalised publications of the technique appeared in \citeyear{Bodin1998} by \citeauthor{Bodin1998}~\cite{Bodin1998}. Iterative compilation has since been demonstrated to be a highly effective form of empirical performance tuning for selecting compiler optimisations.

An enumeration of the optimisation space of Intel Thread Building Blocks in~\cite{Contreras2008} shows that runtime knowledge of the available parallel hardware can have a significant impact on program performance. \citeauthor{Collins2012} exploit this in~\cite{Collins2012}, first using Principle Components Analysis to reduce the dimensionality of the optimisation space, followed by a search of parameter values to improve program performance by a factor of $1.6\times$ over values chosen by a human expert. In~\cite{Collins2013}, they extend this using static feature extraction and nearest neighbour classification to further prune the search space, achieving an average 89\% of the oracle performance after evaluating 45 parameters.

Frameworks for iterative compilation offer mechanisms for abstracting the iterative compilation process from the optimisation space. \emph{OpenTuner} presents a generic framework for optimisation space exploration~\cite{Ansel2013}. \emph{CLTune} is a generic autotuner for OpenCL kernels~\cite{Nugteren2015}. Both frameworks implement \emph{search}, however, the huge number of possible compiler optimisations makes such a search expensive to perform for every new configuration of program, architecture and dataset.

\paragraph{Machine learning} Machine learning has been used to guide iterative compilation and predict optimisations for code. \citeauthor{Stephenson2003} use ``meta optimisation'' to tune compiler heuristics through an evolutionary algorithm to automate the search of the optimisation space~\cite{Stephenson2003}. \citeauthor{Fursin2011} continued this with Milepost GCC, the first machine learning-enabled self-tuning compiler~\cite{Fursin2011}. A survey of machine learning heuristics quality concludes that the automatic \emph{generation} of self-tuning heuristics is an ongoing research challenge that offers the greatest generalisation benefits~\cite{Burke2013}.

\citeauthor{Dastgeer2011} developed a machine learning based autotuner for the SkePU skeleton library in~\cite{Dastgeer2011}. Training data is used to predict the optimal execution device (i.e.\ CPU, GPU) for a given program by predicting execution time and memory copy overhead based on problem size. The autotuner only supports vector operations, and there is limited cross-architecture evaluation. In~\cite{Dastgeer2015a}, the authors extend SkePU to improve the data consistency and transfer overhead of container types, reporting up to a $33.4\times$ speedup over the previous implementation.

\citeauthor{Ogilvie2015} use active learning to reduce the cost of iterative compilation by searching for points in the optimisation space which are close to decision boundaries~\cite{Ogilvie2015}. This reduces the cost of training compared to a random search. \citeauthor{Wahib2015a} use machine learning to automate the selection of GPU kernel transformations~\cite{Wahib2015a}.

PetaBricks is a language and compiler for algorithmic choice~\cite{Ansel2009a}. Users provide multiple implementations of algorithms, optimised for different parameters or use cases. This creates a search space of possible execution paths for a given program. This has been combined with autotuning techniques for enabling optimised multigrid programs~\cite{Chan2009}, with the wider ambition that these autotuning techniques may be applied to all algorithmic choice programs~\cite{Ansel2014}. While this helps produce efficient programs, it places a great burden on the developer, requiring them to provide enough contrasting implementations to make a search of the optimisation space fruitful.

In~\cite{Saclay2010,Memon2013,Fursin2014}, \citeauthor{Fursin2014} advocate a ``big data'' driven approach to autotuning, arguing that the challenges facing widespread adoption of autotuning and machine learning methodologies can be attributed to: a lack of common, diverse benchmarks and datasets; a lack of common experimental methodology; problems with continuously changing hardware and software stacks; and the difficulty to validate techniques due to a lack of sharing in publications. They propose a system for addressing these concerns, the Collective Mind knowledge system, which provides a modular infrastructure for sharing autotuning performance data and related artifacts across the internet.


\paragraph{Dynamic optimisers} Iterative compilation typically involves searching the optimisation space offline --- dynamic optimisers perform this optimisation space exploration at runtime, allowing optimisations tailored to dynamic feature values. This is a challenging task, as a random search of an optimisation space may result in many configurations with suboptimal performance. In a real world system, evaluating many suboptimal configurations will cause significant slowdowns to a program. A resulting requirement of dynamic optimisers is that convergence time towards optimal parameters is minimised.

Existing dynamic optimisation research has typically taken a low level approach to performing optimisations. Dynamo is a dynamic optimiser which performs binary level transformations of programs using information gathered from runtime profiling and tracing~\cite{Bala2000}. While this provides the ability to respond to dynamic features, it restricts the range of optimisations that can be applied to binary transformations. These low level transformations cannot match the performance gains that higher level parameter tuning produces.

\paragraph{Superoptimisers} In~\cite{Massalin1987}, the smallest possible program which performs a specific function is found through an iterative enumeration of the entire instruction set. Starting with a program of a single instruction, the superoptimiser tests to see if any possible instruction passes a set of conformity tests. If not, the program length is increased by a single instruction and the process repeats. The exponential growth in the size of the search space is far too expensive for all but the smallest of hot paths, typically less than 13 instructions. The optimiser is limited to register to register memory transfers, with no support for pointers, a limitation which is addressed in~\cite{Joshi2002}. Denali is a superoptimiser which uses constraint satisfaction and rewrite rules to generate programs which are \emph{provably} optimal, instead of searching for the optimal configuration through empirical testing. Denali first translates a low level machine code into guarded multi-assignment form, then uses a matching algorithm to build a graph of all of a set of logical axioms which match parts of the graph, before using boolean satisfiability to disprove the conjecture that a program cannot be written in $n$ instructions. If the conjecture cannot be disproved, the size of $n$ is increased and the process repeats.

\paragraph{GPUs} Performant GPGPU programs require careful attention from the developer to properly manage data layout in DRAM, caching, diverging control flow, and thread communication. The performance of programs depends heavily on fully utilising zero-overhead thread scheduling, memory bandwidth, and thread grouping. \citeauthor{Ryoo2008a} illustrate the importance of these factors by demonstrating speedups of up to $432\times$ for matrix multiplication in CUDA by appropriate use of tiling and loop unrolling~\cite{Ryoo2008a}. The importance of proper exploitation of local shared memory and synchronisation costs is explored in~\cite{Lee2010}.

In~\cite{Chen2014}, data locality optimisations are automated using a description of the hardware and a memory-placement-agnostic compiler. The authors demonstrate speedups of up to $2.08\times$, although at the cost of requiring accurate memory hierarchy descriptor files for all targeted hardware. The descriptor files must be hand generated, requiring expert knowledge of the underlying hardware in order to properly exploit memory locality.

Data locality for nested parallel patterns is explored in~\cite{Lee}. The authors use an automatic mapping strategy for nested parallel skeletons on GPUs, which uses a custom intermediate representation and a CUDA code generator, achieving $1.24\times$ speedup over hand optimised code on 7 of 8 Rodinia benchmarks.

Reduction of the GPGPU optimisation space is demonstrated in~\cite{Ryoo2008}, using the common subset of optimal configurations across a set of training examples. This technique reduces the search space by 98\%, although it does not guarantee that for a new program, the reduced search space will include the optimal configuration.

\citeauthor{Magni2014} demonstrated that thread coarsening of OpenCL kernels can lead to speedups in program performance between $1.11\times$ and $1.33\times$ in~\cite{Magni2014}. The authors achieve this using a machine learning model to predict optimal thread coarsening factors based on the static features of kernels, and an LLVM function pass to perform the required code transformations.

A framework for the automatic generation of OpenCL kernels from high-level programming concepts is described in~\cite{Steuwer2015}. A set of rewrite rules is used to transform high-level expressions to OpenCL code, creating a space of possible implementations. This approach is ideologically similar to that of PetaBricks, in that optimisations are made through algorithmic choice, although in this case the transformations are performed automatically at the compiler level. The authors report performance on a par with that of hand written OpenCL kernels.

\paragraph{Stencils} Stencil codes have a variety of computationally expensive uses from fluid dynamics to quantum mechanics. Efficient, tuned stencil kernels are highly sought after, with early work in \citeyear{Bolz2003} by \citeauthor{Bolz2003} demonstrating the capability of GPUs for massively parallel stencil operations~\cite{Bolz2003}. In the resulting years, stencil codes have received much attention from the performance tuning research community.

\citeauthor{Ganapathi2009} demonstrated early attempts at autotuning multicore stencil codes in~\cite{Ganapathi2009}, drawing upon the successes of statistical machine learning techniques in the compiler community. They present an autotuner which can achieve performance up to 18\% better than that of a human expert. From a space of 10 million configurations, they evaluate the performance of a randomly selected 1500 combinations, using Kernel Canonical Correlation Analysis to build correlations between tunable parameter values and measured performance targets. Performance targets are L1 cache misses, TLB misses, cycles per thread, and power consumption. The use of KCAA restricts the scalability of their system as the complexity of model building grows exponentially with the number of features. In their evaluation, the system requires two hours of compute time to build the KCAA model for only 400 seconds of benchmark data. They present a compelling argument for the use of energy efficiency as an optimisation target in addition to runtime, citing that it was the power wall that lead to the multicore revolution in the first place. Their choice of only 2 benchmarks and 2 platforms makes the evaluation of their autotuner somewhat limited.

\citeauthor{Berkeley2009} targeted 3D stencils code performance in~\cite{Berkeley2009}. Stencils are decomposed into core blocks, sufficiently small to avoid last level cache capacity misses. These are then further decomposed to thread blocks, designed to exploit common locality threads may have within a shared cache or local memory. Thread blocks are divided into register blocks to take advantage of data level parallelism provided by the available registers. Data allocation is optimised on NUMA systems. The performance evaluation considers speedups of various optimisations with and without consideration for host/device transfer overhead.

\citeauthor{Kamil2010} present an autotuning framework in~\cite{Kamil2010} which accepts as input a Fortran 95 stencil expression and generates tuned shared-memory parallel implementations in Fortan, C, or CUDA. The system uses an IR to explore autotuning transformations, enumerating a subset of the optimisation space and recording only a single execution time for each configuration, reporting the fastest. They demonstrate their system on 4 architectures using 3 benchmarks, with speedups of up to $22\times$ compared to serial implementations. The CUDA code generator does not optimise for the GPU memory hierarchy, using only global memory. As demonstrated in this thesis, improper utilisation of local memory can hinder program performance by two orders of magnitude. There is no directed search or cross-program learning.

In~\cite{Zhang2013a}, \citeauthor{Zhang2013a} present a code generator and autotuner for 3D Jacobi stencil codes. Using a DSL to express kernel functions, the code generator performs substitution from one of two CUDA templates to create programs for execution on GPUs. GPU programs are parameterised and tuned for block size, block dimensions, and whether input data is stored in read only texture memory. This creates an optimisation space of up to 200 configurations. In an evaluation of 4 benchmarks, the authors report performance that is comparable with previous implementations of iterative Jacobi stencils on GPUs~\cite{Holewinski2012, Phillips2010}. The dominating parameter is shown to be block dimensions, followed by block size, then read only memory. The DSL presented in the paper is limited to expressing only Jacobi Stencils applications. Their autotuner requires a full enumeration of the parameter space for each program, which may be impractical for the needs of general purpose stencil computing. Previous work (Appendix~\ref{app:adapt}) overcomes this drawback by learning parameter values which transfer to unseen stencils, without the need for an expensive tuning phase for each program and architecture.

In~\cite{Christen2011}, \citeauthor{Christen2011} presents a DSL for expressing stencil codes, a C code generator, and an autotuner for exploring the optimisation space of blocking and vectorisation strategies. The DSL supports stencil operations on arbitrarily high-dimensional grids. The autotuner performs either an exhaustive, multi-run Powell search, Nelder Mead, or evolutionary search to find optimal parameter values. They evaluate their system on two CPUS and one GPU using 6 benchmarks. The authors do not present a ratio of the available performance that their system achieves, or how the performance of optimisations vary across benchmarks or devices.

A stencil grid can be decomposed into smaller subsections so that multiple GPUs can operate on each subsection independently. This requires a small overlapping region where each subsection meets --- the halo region --- to be shared between devices. For iterative stencils, values in the halo region must be synchronised between devices after each iteration, leading to costly communication overheads. One possible optimisation is to deliberately increase the size of the halo region, allowing each device to compute updated values for the halo region, instead of requiring a synchronisation of shared state. This reduces the communication costs between GPUs, at the expense of introducing redundant computation. Tuning the size of this halo region is the goal of PARTANS~\cite{Lutz2013}, an autotuning framework for multi-GPU stencil computations. \citeauthor{Lutz2013} explore the effect of varying the size of the halo regions using six benchmark applications, finding that the optimal halo size depends on the size of the grid, the number of partitions, and the connection mechanism (i.e.\ PCI express). The authors present an autotuner which determines problem decomposition and swapping strategy offline, and performs an online search for the optimal halo size. The selection of overlapping halo region size compliments the selection of workgroup size which is the subject of previous work (Appendix~\ref{app:adapt}).

In applications of machine learning for iterative compilation, a limiting factor of the effectiveness of learned models is the number of benchmarks used. The development of automatic program generation alleviates this problem by allowing an unbounded number of programs to enumerate the feature space at an increasingly granular scale.


\subsection{Program generation}

\emph{GENESIS}~\cite{Chiu2015} is a language for generating synthetic training programs. The essence of the approach is to construct a probabilistic grammar with embedded semantic actions that defines a language of possible programs. New programs may be created by sampling the grammar and, through setting probabilities on the grammar productions, the sampling is biased towards producing programs from one part of the space or another. This technique is potentially completely general, since a grammar can theoretically be constructed to match any desired program domain. However, despite being theoretically possible, it is not easy to construct grammars which are both suitably general and also produce programs that are in any way similar to human written programs. It has been shown to be successful over a highly restricted space of stencil benchmarks with little control flow or program variability~\cite{Garvey2015b,Falch2015} (Appendix~\ref{app:adapt}). But, it is not clear how much effort it will take, or even if it is possible for human experts to define grammars capable of producing human like programs in more complex domains. By contrast, learning from a corpus provides \emph{general-purpose} program generation over unknown domains, in which the statistical distribution of generated programs is automatically inferred from real world code.

Random program generation is an effective method for software testing. Grammar-based \emph{fuzz testers} have been developed for C~\cite{Yang2012} and OpenCL~\cite{Lidbury2015a}. Programs generated by grammar-based approaches are often unlike real handwritten code, and are typically very large. As such, once a bug has been identified, test case reduction~\cite{Regehr2012a} is required to minimise the size of the program and isolate the code of interest. A mutation-based approach for differential testing the Java Virtual Machine is demonstrated in~\cite{Chena}.

Goal-directed program generators have been used for a variety of domains, including generating linear transforms~\cite{Voronenko2009}, MapReduce programs~\cite{Smith}, and data structure implementations~\cite{Loncaric2016}.


%\subsection{Parallelism}
%
%Programming languages have taken a variety of approaches in adopting parallelism. In C, a threading model was retrofitted using the \textit{POSIX pthreads} standard~\cite{Sura2005}. \citeauthor{Boehm2005} contends this approach in~\cite{Boehm2005}, describing issues in which a compiler developed independently of threading issues cannot be guaranteed to produce correct code. These issues were circumvented in C++11 by the introduction of a concurrency model~\cite{Boehm2008}, based off the success of the Java memory model~\cite{Bash2015a}. C++ and Java are two of the most popular imperative programming languages, and in both cases, the languages specify memory models which guarantee sequential consistency across threads, subject to some restrictions. A dialect of the Java programming language, Titanium, extends the Java feature set, adding multidimensional arrays, immutable classes, and a PGAS programming model~\cite{Yelick1998}. In contrast, Deterministic Parallel Java adds a type and effect system to in order to provide a provably sound language core~\cite{Bocchino2009}.
%
%
%%
%%% \paragraph{Limits of static analysis}
%%Sequential consistency in shared-memory parallel programming~\cite{Krishnamurthy1995,Shasha1988,Sura2005}. Sequential consistency and caches~\cite{Goodman}.
%%
%%Sequential consistency for Java incurs a 10\% slowdown~\cite{Sura2005}.
%%
%
%%
%%Parallelism in the modern functional languages: Haskell, Clojure and Erlang~\cite{Pierro2012}.
%%
%%Scala actors~\cite{Haller2009a}, and~\cite{Haller2012}.
%%
%%
%%\paragraph{Library level parallelism} Optimising MapReduce for multi-core processors~\cite{Kaashoek2010}.
%%TODO: OpenMP, MPI
%
%
%\paragraph{Automatic parallelisation} The goal of automatic parallelisation is to transform arbitrary sequential code into parallelised code. This is a well studied subject, with the typical approach being to perform these code transformations at the compilation stage. In \citeauthor{Banerjee1993}'s thorough review~\cite{Banerjee1993} of the subject, they outline the key challenges of automatic parallelisation: determining whether sequential code can be legally transformed for parallel execution; and identifying the transformation which will provide the highest performance improvement for a given piece of code. Both challenges are hard to tackle. For the former, the difficulties lie in performing accurate analysis of code behaviour. Obtaining accurate dynamic dependency analysis at compile time is an unsolved problem, as is resolving pointers and points-to analysis~\cite{Atkin-granville2013, Hind2001,Ghiya2001}.
%
%The result of these challenges is that reliably performant, automatic parallelisation of arbitrary programs remains a far reaching goal; however, there are noteworthy variations on which have been able to achieve some measure of success. One such example is speculative parallelism, which circumvents the issue of having incomplete dependency information by speculatively executing code regions in parallel while performing dependency tests at runtime, with the possibility to fall back to ``safe'' sequential execution if correctness guarantees are not met~\cite{Prabhu2010,Trachsel2010}.  In~\cite{Jimborean2014}, \citeauthor{Jimborean2014} present a system which combines polyhedral transformations of user code with binary algorithmic skeleton implementations for speculative parallelisation, reporting speedups over sequential code of up to $15.62\times$ on a 24 core processor.
%
%Annotation-driven parallelism takes a similar approach. The user annotates their code to provide ``hints'' to the compiler, which can then perform parallelising transformations. A popular example of this is OpenMP, which uses compiler pragmas to mark code sections for parallel or vectorised execution~\cite{Dagum1998}. Previous work has demonstrated code generators for translating OpenMP to OpenCL~\cite{Grewe2013} and CUDA~\cite{Lee2009}. Again, annotation-driven parallelism suffers from placing a burden on the developer to identify the potential areas for parallelism, and lacks the structure that algorithmic skeletons provide.


%
%
%\paragraph{Debugging}
%General purpose, platform-specific debuggers: Intel Debugger for Linux~\cite{Blair-chappell}.
%
%
%% \paragraph{Debugging Parallel Frameworks and Libraries}
%
%Allinea DDT~\cite{K2010}, a scalable parallel debugger.
%
%TotalView~\cite{Cownie2000}, an OpenMP debugger.
%
%Jumbune~\cite{ImpetusTechnologies}, an open source debugger for Hadoop.
%
%The parallel pipeline library FlumeJava~\cite{Chambers2010} provides debugging support through regular tools using a sequential execution mode,r and routing of exceptions and output from remote workers back to a central host.
%
%
%% \paragraph{Debugging GPGPU programs}
%
%CUDA-GDB~\cite{NVIDIA2016}, Intel OpenCL Debugger for Linux
%OS~\cite{IntelCorporation2016}.
%
%Oclgrind~\cite{Price2015}.
%
%GPUverify~\cite{Betts2012}.
%
%
%% \paragraph{Debugging Algorithmic Skeletons}
%SkIE~\cite{Bacci1999} is based on a coordination language, but provides advanced features such as debugging tools, performance analysis, visualization, and graphical user interface. Instead of directly using the coordination language, programmers interact with a graphical tool, where parallel modules based on skeletons can be composed
%
%SKiPPER~\cite{Serot1999} supports sequential execution for debugging.
%
%
%\paragraph{Profiling}
%Profiling sequential programs~\cite{Ball1994}.
%
%% \paragraph{Profiling parallel programs}
%Profiling parallel programs using KOJAK~\cite{Mohr2003}.
%
%


%\subsection{Performance Benchmarking}
%
%Rodinia~\cite{Che2009,Che2010},
%NPB (SNU~\cite{Seo2011}),
%Parboil~\cite{Stratton2012},
%PolyBench~\cite{Grauer-Gray2012},
%SHOC~\cite{Danalis2010}.
%
%Evaluating GPGPU benchmarks~\cite{Ryoo2015}.
%
%Benchmarking parallel computing systems~\cite{Belli2015}.
%
%\paragraph{Methodology} Use geometric mean for speedups~\cite{Fleming1986}. Statistical rigour\cite{Georges2007}. Execution times and variance~\cite{Box}.
%
%\subsection{Self-Tuning Software}
%

\section{Summary of progress}

Progress in my first year has been focused on developing the initial proof-of-concept of this novel approach to program generation, and in extending and publishing previous work on predictive modeling for optimising GPU stencil computations.

\newpage
\subsection{Research outputs}

\paragraph{Publications}
\begin{enumerate}
\item Cummins, C, Petoumenos, P, Steuwer, M \& Leather, H, ``Autotuning OpenCL Workgroup Size for Stencil Patterns''. In International Workshop on Adaptive Self-tuning Computing Systems (ADAPT). HiPEAC, Prague, Czech Republic, 18 January 2016 [\textbf{Appendix~\ref{app:adapt}}];
\item Cummins, C, Petoumenos, P, Steuwer, M \& Leather, H , ``Towards Collaborative Performance Tuning of Algorithmic Skeletons''. In Workshop on High-Level Programming for Heterogeneous \& Hierarchical Parallel Systems (HLPGPU). HiPEAC, Prague, Czech Republic, 19 January 2016 [\textbf{Appendix~\ref{app:hlpgpu}}];
\item Cummins, C, Petoumenos, P, Wang, Z \& Leather, H, ``Synthesizing Benchmarks for Predictive Modeling''. To appear in International Symposium on Code Generation and Optimization (CGO). Austin, TX, USA, 4--8 February 2017 [\textbf{Appendix~\ref{app:cgo}}].
\end{enumerate}

\paragraph{Posters}
\begin{enumerate}
  \item Cummins, C, Petoumenous, P, Steuwer, M, Leather, H, ``Humans Need Not Apply'', Google PhD Student Summit on Compiler \& Programming Technology, Munich, Germany, 7--9 December 2015;
  \item Cummins, C, Petoumenos, P, Steuwer, M \& Leather, H, ``Autotuning OpenCL Workgroup Sizes'', HiPEAC, Prague, Czech Republic, 18--20 January 2016.
  \item Cummins, C, Petoumenos, P, Steuwer, M \& Leather, H, ``Autotuning OpenCL Workgroup Sizes'', 37th ACM SIGPLAN conference on Programming Language Design \& Implementation (PLDI), Santa Barbara, California, 13--17 June 2016.
  \item Cummins, C, Petoumenos, P, Steuwer, M \& Leather, H, ``Autotuning OpenCL Workgroup Sizes'', 12th International Summer School on Advanced Computer Architecture and Compilation for High-Performance and Embedded Systems (ACACES), Fiuggi, Italy, 10--16 July 2016.
\end{enumerate}

\paragraph{Talks}
\begin{enumerate}
  \item Cummins, C, ``All the OpenCL on GitHub: Teaching an AI to code, one character at a time''. Amazon Development Centre, Edinburgh, UK, 19 May 2016;
  \item Cummins, C, ``Building an AI that Codes'', Ocado Technology, Hatfield, UK, 22 July 2016;
  \item Cummins, C, ``Machine Learning \& Compilers'', Codeplay Software, Edinburgh, UK, 9 September 2016.
\end{enumerate}


\subsection{Codeplay internship}

From April through September I interned at Codeplay Software in Edinburgh. My role within Codeplay was as a member of the SYCL demos team, in which I contributed to SYCL support for TensorFlow and Eigen; developed a Python frontend for the C++ template library VisionCpp; and developed a toolset of metaprogramming utilities for SYCL.

Eigen\footnote{\url{http://eigen.tuxfamily.org/}} is a C++ template library which provides linear math operations on $n$-dimensional tensors. It is a key dependency of TensorFlow~\ref{Abadi}, a popular library for distributed and parallelised machine learning. During my internship I imlpemented GPU memory management for tensors using SYCL, and support for broadcast operations.

SYCL is a single-source specification for heterogeneous parallelism in C++~\cite{Rovatsou2015}. Codeplay is developing a compiler for this standard, ComputeCpp. Compiling a SYCL application with ComputeCpp is a two pass process. In the first pass, the device compiler produces SPIR code for execution on parallel devices. SPIR is an intermediate language which extends LLVM bytecode for parallel compute and graphics~\cite{Portable2014}. The second compiler pass uses a host compiler and links against the generated SPIR bytecode. At runtime, the SYCL runtime schedules kernels for execution on parallel devices, and OpenCL platforms compile the SPIR bytecode to executable device code.

VisionCPP~\cite{Goli2016a} is a C++ template library for performance-portable vision processing using SYCL. Users declare trees of VisionCpp expressions, which at compile time may be fused into a single kernel for efficient execution on GPUs~\cite{Potter2015}. While working at Codeplay I implemented a Python interface for VisionCpp, which allows for simple construction of expression trees using the python object interface. When evaluated, python expression trees are lazily evaluated to a sequence of nodes, from which C++ code is generated. The ComputeCpp compiler is then invoked to generate a native binary for the expression tree, which is linked and loaded by the python runtime and called. Inputs and outputs are transferred between python and the native binary, allowing for a seemless interface of high level scripting language and effecitient native GPU-accelerated code. This is a research project with the potential for publication in a high level GPU programming workshop.

\newpage
\section{Proposal}

%The proposed topic for this PhD is the exploitation of deep learning techniques for compiler research. This broad remit provides scope for contributions in numerous areas, but is focused on the project objectives outlined in Section~\ref{sec:objectives}.

\subsection{Thesis outline}

\paragraph{Chapter 1: Introduction.} An introduction to the main topic, summary of contributions, and motivating examples.

\paragraph{Chapter 2: Literature survey.} A thorough exposition of the relevant literature in the fields of compiler research and machine learning.

\paragraph{Chapter 3: Language modelling for program code.} Defining machine learning systems for language modelling over corpuses of program source code. Background and application of language modelling approaches to programming languages. Description of source code transformations for machine learning, and an empirical evaluation of the effects of transformations on the effectiveness of learned LSTM models. This chapter will provide the foundational methodology for building probabilistic models of programs in a given programming languages.

\paragraph{Chapter 4: Automatic program generation through deep learning.}  Introducing novel techniques for generating and evaluating programs and their inputs, as described in the first project objective. Empirical evaluations for two use cases: compiler autotuning and differential testing. For each evaluation: experimental setup, methodology, results, and analysis. The basis of this chapter will be previous work for synthesising OpenCL kernels [Appendix~\ref{app:cgo}], with extensions for a grammar-based approach to program generation.

\paragraph{Chapter 5: Searching the program feature space.} A methodology for performing directed searches of program feature spaces using deep learning program synthesis, satisfying the second project objective. This may be an iterative process of mutating code to converge on the goal features, or using a model-based approach which guarantees program synthesis with specific features by considering only well-formed sequences which result in the target feature values. This section will contain an empirical evaluation of convergence time and possible coverage of feature spaces.

\paragraph{Chapter 6: Learning the compiler optimisation pipeline.} Applying an agent based approach to the compiler optimisation pipeline, the third project objective. This would replace the fixed ordering of optimisation passes in present compilers, and would use reinforcement learning combined with automatic program generation to explore the space of pass orderings and selection. The agent will be trained using synthesised programs, and empirically evaluated on benchmark suites using LLVM.

\paragraph{Chapter 7: Conclusions.} Summary of contributions and research impact. Future research questions and directions for extended work.


\newpage
\subsection{Work plan}

\paragraph{Year 2} The year will begin with extensions to previous work on CLgen\footnote{\url{http://chriscummins.cc/clgen/}}: 
%
\begin{itemize}
        \item adding support for alternate vocabularies, to enable learning models at the token or AST-level, as opposed to the current character-based approach;
        \item adding support for alternative encodings, enabling additional corpus transformations, for example, by reversing the character sequence, or interleaving characters from the start and end of sequences.
        \item implementing an iterative hill climb search of the program feature space using mutations to model seed text and random number generator state;
\end{itemize}
%
The goal of these extensions is incremental improvements to my current work on deep learning program synthesis (Appendix~\ref{app:cgo}), by demonstrating the ability to enumerate feature spaces, and reducing the rejection rate of synthesised programs. The deliverable for this work would be either an extended journal publication of Appendix~\ref{app:cgo}, or a new publication \emph{``Directed exploration of program feature spaces''}. This outcome would depend on the extend to which these changes improve CLgen. At a minimum, the rejection rate would be measurably decreased so that programs can be generated a faster rate, and an iterative process of sample rejection would be shown to produce programs which converge towards specific feature values.

Upon completion of these first goals by the end of February 2017, a short period of time will be dedicated to analysis of the language models learned by CLgen. By modelling the syntax and semantics of source codes in a given language, it may be possible to extract the learned grammar of the language from the model's intermediate layers. The ability to automatically generate a formal grammar for a program language would have benefits for program synthesis, allowing the development of a grammar based system which operates on well-formed syntax trees, replacing the need for rejection tests to validate that a synthesised program is syntactically and semantically correct.

Should this brief exploratory analysis fail at extracting formal language grammars from learned models, then an alternative approach to reducing the rejection rate of generated programs would be the development of a system to iteratively and incrementally repair issues with rejected programs. This would extend the functionality of existing systems for statically verifying correctness of programs, allowing generated programs with small syntactic and semantic errors to be recovered.

The latter 3 months of the year will be used to test improvements on corpus generation. This will involve an empirical evaluation of the program rejection rate when sampling a model trained on previously generated programs. If the performance of the system improves under the these conditions, then it will allow for training language models on incrementally larger corpuses, and provide a mechanism for a program generator which is self-improving as the number of programs it generates increases. The qualitative evaluation of synthesised programs described in Appendix~\ref{app:cgo} will be repeated for models trained on generated programs.

\paragraph{Year 3} The first three months of my third year will be used to develop a formal grammar based approach to automatic program synthesis. Given a corpus of example programs in a given programming language, and a tool which verifies or rejects a program in the given language, this approach would instantiate the grammar probabilistically, using a language model trained on the corpus to determine the probability of a given production rule based on its probability distribution within the corpus.

By operating only on sequences of well-formed syntax trees, a grammar based approach has the potential to significantly improve the rate of program generation by pruning the space of possible output sequences to only those which produce correct programs. This would allow more strict checks to be imposed on generated programs, allowing focus on generating programs which are free from undefined behaviour.

The successul development of this system will be validated in a set of experiments to generate programs in three programming languages: OpenCL, C, and Java. Generated programs will be tested on multiple compilers, and the computed results of each compared. If the computed results differ and the programs are free from undefined behaviour, then a bug has been exposed in at least one of the disagreeing compilers. The intended deliverable for this work is a publication \emph{``Differential testing compilers through deep learning''}.

Months 4 through 6 of year 3 will be used to develop an agent-based approach to compiler pass selection and ordering. This will involve extensions to the LLVM pass manager to support reinforcement learning, and use the automatic program generator developed previously to explore the program feature space. The implementation work for this project is minimal, with the majority of time required for running empirical evaluations of the system on benchmark suites. The second half of the year will be dedicated to thesis write up.

\section{Conclusions}\label{sec:conclusion}

We extend the state-of-the-art in a number of areas. First, we present a generative model for programming languages which [is in some way more capable than prior];
%
our method achieves X-fold increase in production efficiency compared to syntactic generative models (if grammar based approach);
%
and we generate more human-like test-cases for compilers than prior, in (some/most) cases eradicating the need for test-case reduction.

\newpage
\label{bibliography}
\printbibliography

\begin{appendices}
% 2-8 pages, 10pt font
%
% Topics:
%
% * Machine learning based autotuning.
% * Representative benchmarking.
% * Automatic fault tolerance.
% * Run-time adaption.
%

% The following \documentclass options may be useful:
%
% preprint      Remove this option only once the paper is in final form.
% 10pt          To set in 10-point type instead of 9-point.
% 11pt          To set in 11-point type instead of 9-point.
% authoryear    To obtain author/year citation style instead of numeric.
\documentclass[preprint,nonatbib,10pt]{sigplanconf}

\documentclass[prodmode,acmtaco]{acmsmall}

% Package to generate and customize Algorithm as per ACM style
\usepackage[ruled]{algorithm2e}
\renewcommand{\algorithmcfname}{ALGORITHM}
\SetAlFnt{\small}
\SetAlCapFnt{\small}
\SetAlCapNameFnt{\small}
\SetAlCapHSkip{0pt}
\IncMargin{-\parindent}

% Metadata Information
\acmVolume{9}
\acmNumber{4}
\acmArticle{39}
\acmYear{2016}
\acmMonth{3}

% Copyright
%\setcopyright{acmcopyright}
\setcopyright{acmlicensed}
%\setcopyright{rightsretained}
%\setcopyright{usgov}
%\setcopyright{usgovmixed}
%\setcopyright{cagov}
%\setcopyright{cagovmixed}

% DOI
\doi{0000001.0000001}

%ISSN
\issn{1234-56789}


%%%%%%%%%%%%%%%%%%%%%%%%%
%% Document and Layout %%
%%%%%%%%%%%%%%%%%%%%%%%%%

% Fix for multiple "No room for a new \dimen" errors.
%
% See: http://tex.stackexchange.com/questions/38607/no-room-for-a-new-dimen
%
\usepackage{etex}

\usepackage[utf8]{inputenc}

% Fix for "'babel/polyglossia' detected but 'csquotes' missing"
% warning. NOTE: Include after inputenc.
%
\usepackage{csquotes}

% Make internal macro definitions accessible,
% e.g. \@title, \@date \@author.
\makeatletter

% Multi-column support.
\usepackage{multicol}

% A useful package which includes macros like \ifdef{}{}{}:
%
\usepackage{etoolbox}

% Uncomment the following line to remove column separation:
%
%\setlength{\columnsep}{5mm}

% Allow user-defined warning and error filters.
%
\usepackage{silence}

\usepackage{adjustbox}


%%%%%%%%%%%%%%%%%%%%%
% Table of Contents %
%%%%%%%%%%%%%%%%%%%%%

% % Set chapter and section numbering depth:
% %
% \setcounter{secnumdepth}{2}


%%%%%%%%%%%%%%%%
% Bibliography %
%%%%%%%%%%%%%%%%
% \usepackage[%
%     backend=biber,
%     style=ieee,
%     % style=numeric-comp,
%     % style=numeric-comp,  % numerical-compressed
%     sorting=none,        % nty,nyt,nyvt,anyt,anyvt,ynt,ydnt,none
%     sortcites=true,      % sort \cite{b a d c}: true,false
%     block=none,          % space between blocks: none,space,par,nbpar,ragged
%     indexing=false,      % indexing options: true,false,cite,bib
%     citereset=none,      % don't reset cites
%     isbn=false,          % print ISBN?
%     url=true,            % print URL?
%     doi=false,           % print DOI?
%     natbib=true,         % natbib compatability
%    ]{biblatex}

% \usepackage{natbib}

% % Filter annoying and unavoidable biblatex warning:
% \WarningFilter{biblatex}{Patching footnotes failed}

% Reduce the font size of the bibliography:
% \renewcommand{\bibfont}{\normalfont\scriptsize}

% Determine which BibTeX file to use:
%
% If available, use my Mendeley BibTex library, located in the home
% directory. Note that this is a relative path and will break if
% either this file or the BibTex library are moved. If the library is
% not present, use the local refs.bib file.
% \newcommand{\BibResourceGlobal}{../../../library.bib}
% \newcommand{\BibResourceLocal}{refs.bib}

% \IfFileExists{\BibResourceGlobal}
%   {\newcommand{\BibResource}{\BibResourceGlobal}}
%   {\newcommand{\BibResource}{\BibResourceLocal}}

% \addbibresource{\BibResource}


%%%%%%%%%%%%%%
% Appendices %
%%%%%%%%%%%%%%

% Appendix package. Documentation:
%
%  http://mirror.ox.ac.uk/sites/ctan.org/macros/latex/contrib/appendix/appendix.pdf
%
% Package options:
%
% toc      - Put a header (e.g., `Appendices') into the Table of Contents
%            (the ToC) before listing the appendices. (This is done by
%            calling the \addappheadtotoc command.)
% page     - Puts a title (e.g., `Appendices') into the document at the
%            point where the appendices environment is begun. (This is
%            done by calling the \appendixpage command.)
% title    - Adds a name (e.g., `Appendix') before each appendix title in
%            the body of the document. The name is given by the value
%            of \appendixname. Note that this is the default behaviour
%            for classes that have chapters.
% titletoc - Adds a name (e.g., `Appendix') before each appendix listed
%            in the ToC. The name is given by the value
%            of \appendixname.
% header   - Adds a name (e.g., `Appendix') before each appendix in page
%            headers.  The name is given by the value
%            of \appendixname. Note that this is the default behaviour
%            for classes that have chapters.
\usepackage[title, titletoc]{appendix}

% pre-requisites for rendering upquotes in listings package.
\usepackage[T1]{fontenc}
\usepackage{lmodern}
\usepackage{textcomp}

% code listings.
\usepackage{listings}

% set \ttfamily to use courier fonts.
%
% See: http://tex.stackexchange.com/a/33686
%
\usepackage{courier}

\lstset{frame=bt,                    % Add top and bottom frame lines
breaklines=true,             % Force line wrapping
captionpos=b,                % Place caption below listing
numbers=left,                % Add left-side line numbers
basicstyle=\scriptsize\ttfamily, % Set font size and type
showstringspaces=false,      % Don't show visible whitespace
numberstyle=\tiny,
upquote=true,                % Use upright quotes, not curly
commentstyle=\bfseries}      % Embolden comments

% Use (*@ @*) to escape LaTeX commands within listings.
\lstset{escapeinside={(*@}{@*)}}

% Add 10pt space between chapters in TOC listings entries:
%\let\Chapter\chapter
%\def\chapter{\addtocontents{lol}{\protect\addvspace{10pt}}\Chapter}


%%%%%%%%%%%%%%%%%%%%%%%%
%% Graphics and maths %%
%%%%%%%%%%%%%%%%%%%%%%%%
\usepackage{amsmath}

% Vector notation, e.g. \vv{x}:
%
\usepackage{esvect}

% Additional amsmath symbols, see:
%
% http://texblog.org/2007/08/27/number-sets-prime-natural-integer-rational-real-and-complex-in-latex/
%
\usepackage{amsfonts}
\usepackage{amssymb}

\usepackage{graphicx}
\usepackage{mathtools}
\usepackage{tikz}
\usepackage{tikz-qtree}

% Provide bold font face in maths.
\usepackage{bm}

\usepackage{subcaption}
\expandafter\def\csname ver@subfig.sty\endcsname{}

% Define an 'myalignat' command which behave as 'alignat' without the
% vertical top and bottom padding. See:
%     http://www.latex-community.org/forum/viewtopic.php?f=5&t=1890
\newenvironment{myalignat}[1]{%
\setlength{\abovedisplayskip}{-.7\baselineskip}%
\setlength{\abovedisplayshortskip}{\abovedisplayskip}%
\start@align\z@\st@rredtrue#1
}%
{\endalign}

% Define additional operators:
\DeclareMathOperator*{\argmin}{arg\,min}
\DeclareMathOperator*{\argmax}{arg\,max}

\DeclareMathOperator*{\gain}{Gain}

% Skeleton operators.
\DeclareMathOperator*{\map}{Map}
\DeclareMathOperator*{\reduce}{Reduce}
\DeclareMathOperator*{\scan}{Scan}
\DeclareMathOperator*{\stencil}{Stencil}
\DeclareMathOperator*{\zip}{Zip}
\DeclareMathOperator*{\allpairs}{All\,Pairs}

% Maths plots using pgfplots, see:
%
%     http://pgfplots.sourceforge.net/pgfplots.pdf
%
\usepackage{pgfplots}

% Disable compatability mode.
%
\pgfplotsset{compat=1.12}

% Gantt charts using pgfgantt, see:
%
%     http://www.ctan.org/pkg/pgfgantt
%
\usepackage{pgfgantt}

% Fix milestone aspect ratio by defining a custom element.
\newganttchartelement*{mymilestone}{
mymilestone/.style={
shape=diamond,
inner sep=2pt,
draw=black,
top color=black,
bottom color=black,
}
}

% Tikz flowchart configuration.
\usetikzlibrary{shapes,arrows,shadows,fit,backgrounds}
\tikzstyle{decision} = [diamond,
draw,
text width=4.5em,
text badly centered,
node distance=3cm,
inner sep=0pt]
\tikzstyle{block}    = [rectangle,
draw,
text width=5em,
text centered,
node distance=3cm,
minimum height=4em,
inner sep=.2cm]
\tikzstyle{line}     = [draw, -latex']

% Add dirtree picture style, see:
%
%     http://tex.stackexchange.com/a/34268
%
\newcount\dirtree@lvl
\newcount\dirtree@plvl
\newcount\dirtree@clvl
\def\dirtree@growth{%
\ifnum\tikznumberofcurrentchild=1\relax
\global\advance\dirtree@plvl by 1
\expandafter\xdef\csname dirtree@p@\the\dirtree@plvl\endcsname{\the\dirtree@lvl}
\fi
\global\advance\dirtree@lvl by 1\relax
\dirtree@clvl=\dirtree@lvl
\advance\dirtree@clvl by -\csname dirtree@p@\the\dirtree@plvl\endcsname
\pgf@xa=0.33cm\relax
\pgf@ya=-\baselineskip\relax
\pgf@ya=\dirtree@clvl\pgf@ya
\pgftransformshift{\pgfqpoint{\the\pgf@xa}{\the\pgf@ya}}%
\ifnum\tikznumberofcurrentchild=\tikznumberofchildren
\global\advance\dirtree@plvl by -1
\fi
}
\tikzset{
dirtree/.style={
growth function=\dirtree@growth,
every node/.style={anchor=north},
every child node/.style={anchor=west},
edge from parent path={(\tikzparentnode\tikzparentanchor) |- (\tikzchildnode\tikzchildanchor)}
}
}

% UML sequence diagram macros, see:
%
%     https://code.google.com/p/pgf-umlsd/
%
% Options:
%
%     underline - Underline object names
%
\usepackage[underline=false]{pgf-umlsd}

% Support for SVG graphics.
%
% NOTE that you must pass the "--shell-escape" argument to pdflatex to
% compile. NOTE also that images *MUST* be placed within the graphics
% path.
\usepackage{svg}
\graphicspath{{img/}}

%%%%%%%%%%%%%%%%%%%%%%
%% Tables and lists %%
%%%%%%%%%%%%%%%%%%%%%%

% Required to use labm8 exported tables.
%
\usepackage{booktabs}

% Required for full page-width tables.
\usepackage{tabularx}

%\usepackage{enumitem}
%\setenumerate{itemsep=0pt}

% Use no left margin for lists:
%\setlist{leftmargin=*}

\usepackage{longtable}

% Define column types L, C, R with known text justification and fixed
% widths:
\usepackage{array}
\newcolumntype{L}[1]{>{\raggedright\let\newline\\\arraybackslash\hspace{0pt}}m{#1}}
\newcolumntype{C}[1]{>{\centering\let\newline\\\arraybackslash\hspace{0pt}}m{#1}}
\newcolumntype{R}[1]{>{\raggedleft\let\newline\\\arraybackslash\hspace{0pt}}m{#1}}


%%%%%%%%%%%%%%%%%%%%%%%%%%%%%
%% Typesetting and symbols %%
%%%%%%%%%%%%%%%%%%%%%%%%%%%%%

% Adjustable font sizes in \Verbatim{}
\usepackage{fancyvrb}

%\usepackage{titlesec}
% Set section and paragraph heading fonts:
%\titleformat*{\section}{\Large\bfseries}
%\titleformat*{\subsection}{\normalsize\bfseries}
%\titleformat*{\subsubsection}{\normalsize}
%\titleformat*{\paragraph}{\large\bfseries}
%\titleformat*{\subparagraph}{\large\bfseries}

% Set section heading margins. Usage:
% \titlespacing*{<command>}{<left>}{<before>}{<after>}
%\titlespacing*{\section}{0pt}{.6em}{.3em}
%\titlespacing*{\subsection}{0pt}{.6em}{.2em}

% Set paragraph indentation size. Default is 15pt.
%\setlength{\parindent}{10pt}

% The line spacing can be globally set using \linespread:
%
% \linespread{1.2}

% Add a command \hr{} which will draw a horizontal rule the width of
% the text.
%
\newcommand{\hr}{\noindent\makebox[\linewidth]{\rule{\textwidth}{0.2pt}}}

% Add a command \br{} which will create a horizontal space of exactly
% one line height.
%
\newcommand{\br}{\hspace{\baselineskip}}

% Define a command to allow word breaking.
\newcommand*\wrapletters[1]{\wr@pletters#1\@nil}
\def\wr@pletters#1#2\@nil{#1\allowbreak\if&#2&\else\wr@pletters#2\@nil\fi}

% Define a command to create centred page titles.
\newcommand{\centredtitle}[1]{
\begin{center}
  \large
  \vspace{0.9cm}
  \textbf{#1}
\end{center}}

% Support hyperlinks using the \hyperref, \url and \href
% macros. Usage:
%
%    \hyperref[label_name]{''link text''}
%
%    \url{<my_url>}
%
%    \href{<my_url>}{<description>}
%
\usepackage{hyperref}

% Disable colored borders of links, cross-references etc in PDF output
\hypersetup{pdfborder={0 0 0}}

% Provide generic commands \degree, \celsius, \perthousand, \micro
% and \ohm which work both in text and maths mode.
\usepackage{gensymb}

%%%%%%%%%%%%%%%%%%%%%%%%%%%%%%%%%
%% Placeholder text generation %%
%%%%%%%%%%%%%%%%%%%%%%%%%%%%%%%%%

% Use either \blindtext or \libpsum to generate placeholder text. Also
% note the macros \blinditemize, \blindenumerate, \blinddescription.
\usepackage[english]{babel}
\usepackage{blindtext}
\usepackage{lipsum}


\begin{document}

\special{papersize=8.5in,11in}
\setlength{\pdfpageheight}{\paperheight}
\setlength{\pdfpagewidth}{\paperwidth}

\conferenceinfo{ADAPT '16}{Month d--d, 20yy, City, ST, Country}
\copyrightyear{2016}
\copyrightdata{978-1-nnnn-nnnn-n/yy/mm}
\doi{nnnnnnn.nnnnnnn}

% Uncomment one of the following two, if you are not going for the
% traditional copyright transfer agreement.

%\exclusivelicense                % ACM gets exclusive license to publish,
                                  % you retain copyright

%\permissiontopublish             % ACM gets nonexclusive license to publish
                                  % (paid open-access papers,
                                  % short abstracts)

% \titlebanner{banner above paper title}        % These are ignored unless
\preprintfooter{ADAPT workshop '16}   % 'preprint' option specified.

% \title{Autotuning Stencil Codes using Synthetic Benchmarks}
% \title{Autotuning OpenCL Workgroup Sizes for Stencil Codes}
% \title{Machine learning for OpenCL Workgroup Sizes of Stencil Codes}
% \title{Machine learning for OpenCL Workgroup Sizes of Stencil Codes}
% \title{Methods for Autotuning Workgroup Size of OpenCL Stencil Codes}
\title{Autotuning OpenCL Workgroup Size for Stencil Patterns}

% Comparison of multiple approaches to autotuning stencil patterns:
%
% * Compare classifiers with regressors
% * Compare synthetic vs real training

% \subtitle{Subtitle Text, if any}

\authorinfo{Chris Cummins\and Pavlos Petoumenos\and Hugh Leather}
           {University of Edinburgh}
           {c.cummins@ed.ac.uk,\{ppetoume,hleather\}@inf.ed.ac.uk}

\maketitle

\begin{abstract}
  Selecting the appropriate workgroup size for OpenCL kernels requires
  knowledge of the underlying hardware, the data being operated on,
  and properties of the kernel. This makes portable performance tuning
  a difficult task, and simple heuristics and statically chosen values
  fail to exploit the available performance. To address this, we
  propose the use of machine learning-enabled autotuning to predict
  workgroup sizes for stencil patterns on CPUs and multi-GPUs.

  We present three methodologies for predicting workgroup sizes. The
  first, using classifiers to select the optimal workgroup size. The
  second and third proposed methodologies employ the novel use of
  regressors for performing classification by predicting the runtime
  of kernels and relative performance of different workgroup sizes,
  respectively. We evaluate the effectiveness of each technique in an
  empirical study of 429 combinations of architecture, kernel, and
  dataset, comparing an average of 629 unique workgroup sizes for
  each. We find that auotuning provides a median $3.79\times$ speedup
  over the best possible performance which can be achieved statically,
  achieving 94\% of the available performance.
\end{abstract}

% \category{CR-number}{subcategory}{third-level}

% % general terms are not compulsory anymore,
% % you may leave them out
% \terms
% term1, term2

% \keywords
% keyword1, keyword2

\section{Introduction}\label{sec:introduction}

Stencil codes have a variety of computationally demanding uses from
fluid dynamics to quantum mechanics. Efficient, tuned stencil
implementations are highly sought after, with early work in
\citeyear{Bolz2003} by \citeauthor{Bolz2003} demonstrating the
capability of GPUs for massively parallel stencil
operations~\cite{Bolz2003}. Since then, the introduction of the OpenCL
standard has introduced greater programmability of heterogeneous
devices by providing a vendor-independent layer of abstraction for
data parallel programming of CPUs, GPUs, DSPs, and other
devices~\cite{Stone2010}. However, achieving portable performance of
OpenCL programs is a hard task --- such programs are sensitive to
properties of the underlying hardware, to the program being executed,
and even to the \emph{dataset} that is operated upon. This forces
developers to laboriously hand tune performance on a per-program
basis, since simple heuristics fail to exploit the available
performance. (TODO: reference)

In this paper, we implement machine learning-enabled autotuning for
one such optimisation parameter of OpenCL programs --- that of
workgroup size selection. The 2D optimisation space of OpenCL kernel
workgroup sizes is large, complex and non-linear. Successfully
applying machine learning to such a space requires plentiful training
data, the careful selection of features, and an appropriate
classification approach. The main contributions of this paper are:
%
\begin{itemize}
\item TODO
\end{itemize}

\section{The SkelCL Stencil Pattern}

\begin{figure}
\centering
\includegraphics[width=.75\columnwidth]{img/stencil}
\caption[Stencil border region]{%
  The components of a stencil: an input matrix is decomposed into
  workgroups, consisting of $w_r \times w_c$ elements. Each element is
  mapped to a work-item. Each work-item operates its corresponding
  element and a surrounding border region $S$, consisting of the four
  independent components describing the number of elements north
  $S_n$, east $S_e$, west $S_w$, and south $S_s$ (in this example, 1
  element to the south, and 2 elements in all other directions). Each
  tile is allocated in local memory for fast access of repeated read
  operations.%
}
\label{fig:stencil-shape}
\end{figure}

Introduced in~\cite{Steuwer2011}, SkelCL is an Algorithmic Skeleton
library which provides OpenCL implementations of data parallel
patterns for heterogeneous parallelism using CPUs and
multi-GPUs. SkelCL provides a stencil pattern~\cite{Breuer2014a} in
which a user-provided \emph{customising function} is applied to each
element of a 2D matrix. The value of each element is updated based on
its current value and the value of one or more neighbouring elements,
called the \emph{border region}. The border region is described by a
\emph{stencil shape}, which defines an $i \times j$ rectangular region
about each cell which is used to update the cell value. Stencil shapes
may be asymmetrical, and are defined in terms of the number of cells
in the border region to the north, east, south, and west of each cell,
as shown in Figure~\ref{fig:stencil-shape}. Where elements of a border
region fall outside of the matrix bounds, values are substituted from
either a predefined padding value, or the value of the nearest cell
within the matrix, determined by the user.

When a SkelCL stencil pattern is executed, each of the elements in the
matrix are mapped to OpenCL work-items; and this collection of
work-items is divided into \emph{workgroups} for execution on the
target hardware. A work-item reads the value of its corresponding
matrix element and the surrounding elements defined by the border
region. Since the border regions of neighbouring elements overlap, the
value of all elements within a workgroup are stored in a \emph{tile},
allocated as a contiguous block of local memory. This greatly reduces
the latency of the repeated memory accessed performed by
work-items. Changing the workgroup size thus affects the amount of
local memory required for each workgroup, which in turn affects the
number of workgroups which may be simultaneously active. While the
user defines the size, type, and border region of the matrix being
operated upon, it is the responsibility of the SkelCL stencil
implementation to select an appropriate workgroup size to use.


\section{Autotuning Workgroup Size}

Selecting the appropriate workgroup size for an OpenCL kernel depends
on the properties of the kernel itself, underlying architecture, and
dataset. For a given \emph{scenario} (that is, a combination of
kernel, architecture, and dataset), the goal of this work is to
harness machine learning to \emph{predict} a performant workgroup size
to use, based on some prior knowledge of the performance of workgroup
sizes for other scenarios. In this section, we describe the
optimisation space and the steps required to apply machine
learning. The autotuning algorithms are described in
Section~\ref{sec:ml}.

\subsection{Constraints}

The space of possible workgroup sizes $W$ is constrained by properties
of both the architecture and kernel. Each OpenCL device imposes a
maximum workgroup size which can be statically checked through the
OpenCL Device API. This constraint reflects architectural limitations
of how code is mapped to the underlying execution hardware. Typical
values are powers of two, e.g.\ 1024, 4096, 8192. Additionally,
kernels enforce a maximum workgroup size. This value can be queries at
runtime once a program has been compiled for a specific execution
device. Factors which affect a kernel's maximum workgroup size include
the number of registers required, and the available number of SIMD
execution units for each type of executable instruction.

While in theory, any workgroup size which satisfies the device and
kernel workgroup size constraints should provide a functioning
program, in practise we find that some combinations of scenario and
workgroup size cause an \texttt{CL\_OUT\_OF\_RESOURCES} error to be
thrown when the kernel is enqueued. Note that in many OpenCL
implementations, this error type acts as a generic placeholder and may
not necessarily indicate that the underlying cause of the error was
due to finite resources constraints. Further discussion on the
possible causes and effects of refused parameters is contained in
Section~\ref{sec:results}, but for the purposes of autotuning we
define \emph{refused parameters} as workgroup sizes which satisfy the
kernel and architectural constraints, yet cause a
\texttt{CL\_OUT\_OF\_RESOURCES} error to be thrown when the kernel is
enqueued. We define the space of \emph{legal} workgroup sizes for a
given scenario $s$ as those which satisfy the architectural and kernel
constraints, and are not refused:
%
\begin{equation}
  \footnotesize
  W_{legal}(s) = \left\{w | w \in W, w < W_{\max}(s) \right\} - W_{refused}(s)
\end{equation}
%
Where $W_{\max}(s)$ can be determined at runtime prior to the kernels
execution, but the set $W_{refused}(s)$ can only be discovered
emergently. The set of \emph{safe} parameters are those which are
legal for all scenarios:
%
\begin{equation}
  % \footnotesize
  W_{safe} = \cap \left\{ W_{legal}(s) | s \in S \right\}
\end{equation}


\subsection{Stencil Features}

Since properties of the architecture, program, and dataset all
contribute to the performance of a workgroup size, the success of a
machine learning system depends on the ability to translate these
properties into meaningful explanatory variables ---
\emph{features}. For each scenario, 102 features are extracted
describing the architecture, kernel, and dataset.

Architecture features are extracted using the OpenCL Device API to
query properties such as the size of local memory, maximum work group
size, and number of compute units. Kernel features are extracted from
the source code stencil kernels by compiling first to LLVM IR bitcode,
and using statistics passes to obtain static instruction counts for
each type of instruction present in the kernel, as well as the total
number of instructions. These instruction counts are divided by the
total number of instructions to produce instruction
\emph{densities}. Dataset features include the input and output data
types, and the 2D matrix dimensions.


\subsection{Training Data}\label{subsec:training}

Training data are collected by measuring the runtimes of stencil
programs using different workgroup sizes. These stencil programs are
generated synthetically using a parameterised template substitution
engine. A stencil template is parameterised first by stencil shape
(one parameter for each of the four directions), input and output data
types (either integers, or single or double precision floating
points), and \emph{complexity} --- a simple boolean metric for
indicating the desired number of memory accesses and instructions per
iteration, reflecting the relatively bi-modal nature of stencil codes,
either compute intensive (e.g.\ finite difference time domain and
other PDE solvers), or lightweight (e.g.\ Game of Life and Gaussian
blur).


\section{Machine Learning Methods}\label{sec:ml}

The aim of this work is to design a system which, given a set of prior
observations of the empirical performance of stencil codes with
different workgroup sizes, predicts workgroup sizes for \emph{unseen}
scenarios which maximise the performance. This section presents three
contrasting methods for achieving this goal.


\subsection{Predicting Oracle Workgroup Sizes}

\begin{algorithm}[t]
\begin{algorithmic}[1]
\Require scenario $s$
\Ensure workgroup size $w$

\Procedure{Baseline}{s}
% \Comment Select the best $w$ from $W_{safe}$.
\State $w \leftarrow \text{classify}(f(s))$
\If{$w \in W_{legal}(s)$}
    \State \textbf{return} $w$
\Else
  \State \textbf{return} $\underset{w \in W_{safe}}{\argmax}
\left(
  \prod_{s \in S_{training}} p(s, w)
\right)^{1/|S_{training}|}$
\EndIf
\EndProcedure
\item[] % line break

\Procedure{Random}{s}
% \Comment Select a random workgroup size.
\State $w \leftarrow \text{classify}(f(s))$
\While{$w \not\in W_{legal}(s)$}
  \State $W \leftarrow \left\{ w | w < W_{max}(s), w \not\in W_{refused}(s) \right\}$
  \State $w \leftarrow $ random selection $w \in W$
\EndWhile
\State \textbf{return} $w$
\EndProcedure
\item[] % line break

\Procedure{NearestNeighbour}{s}
% \Comment Select the closest workgroup size to prediction.
\State $w \leftarrow \text{classify}(f(s))$
\While{$w \not\in W_{legal}(s)$}
  \State $d_{min} \leftarrow \infty$
  \State $w_{closest} \leftarrow \text{null}$
  \For{$c \in \left\{ w | w < W_{\max}(s), w \not\in W_{refused}(s) \right\}$}
    \State $d \leftarrow \sqrt{\left(c_r - w_r\right)^2 + \left(c_c - w_c\right)^2}$
    \If{$d < d_{min}$}
      \State $d_{min} \leftarrow d$
      \State $w_{closest} \leftarrow c$
    \EndIf
  \EndFor
  \State $w \leftarrow w_{closest}$
\EndWhile
\State \textbf{return} $w$
\EndProcedure
\end{algorithmic}
\caption{Prediction using classifiers}
\label{alg:autotune-classification}
\end{algorithm}

The first approach to predicting workgroup sizes is to consider the
set of possible workgroup sizes as a hypothesis space and to use a
classifier to predict, for a given set of features, the \emph{oracle}
workgroup size. The oracle workgroup size $\Omega(s)$ is the workgroup
size which provides the lowest mean runtime $t(s,w)$:
%
\begin{equation}
  \Omega(s) = \argmin_{w \in W_{legal}(s)} t(s,w)
\end{equation}
%
Training a classifier for this purpose requires pairs of stencil
features $f(s)$ labelled with their oracle workgroup size for a set of
training scenarios $S_{training}$:
%
\begin{equation}
  D_{training} = \left\{ \left(f(s), \Omega(s)\right) | s \in S_{training} \right\}
\end{equation}
%
After training, the classifier predicts workgroup sizes for unseen
scenarios from the set of oracle workgroup sizes from the training
set. This is a common and intuitive approach to autotuning, in that a
classifier predicts the best parameter value based on the best
parameter values for previous scenarios. However, given the
constrained space of workgroup sizes, this presents the problem that
future scenarios may have different sets of legal workgroup sizes to
that of the training data:
%
\begin{equation}
  \bigcup_{\forall s \in S_{future}} W_{legal}(s) \nsubseteq \left\{ \Omega(s) | s \in S_{training} \right\}
\end{equation}
%
This results in an autotuner which may predict workgroup sizes that
are not legal for scenarios, either because they exceed $W_{\max}(s)$,
or because parameters are refused, $w \in W_{refused}(s)$. For these
cases, we evaluate the effectiveness of three \emph{fallback
  strategies}:
%
\begin{enumerate}
\item \emph{Baseline} --- select the workgroup size which provides the
  highest average case performance from the set of safe workgroup sizes.
\item \emph{Random} --- select a random workgroup size which is
  expected from prior observations to be legal.
\item \emph{Nearest Neighbour} --- select the workgroup size which
  from prior observations is expected to be legal, and has the lowest
  Euclidian distance to the prediction.
\end{enumerate}
%
See Algorithm~\ref{alg:autotune-classification} for definitions.


\subsection{Predicting Kernel Runtimes}

\begin{algorithm}[t]
\begin{algorithmic}[1]
\Require scenario $s$, regressor $R(x, w)$, fitness function $\Delta(x)$
\Ensure workgroup size $w$

\State $W \leftarrow \left\{ w | w < W_{\max}(s) \right\} -
W_{refused}(s)$
\Comment Candidates.
\State $w \leftarrow \underset{w \in W}{\argmax} \Delta(R(f(s), w))$
\Comment Select best candidate.
\While{$w \not\in W_{legal}(s)$}
  \State $W_{refused}(s) = W_{refused}(s) + \{w\}$
  \State $W \leftarrow W - \left\{ w \right\}$
  \Comment Remove candidate from selection.
  \State $w \leftarrow \underset{w \in W}{\argmax} \Delta(R(f(s), w))$
  \Comment Select best candidate.
\EndWhile
\State \textbf{return} $w$
\end{algorithmic}
\caption{Prediction using regressors}
\label{alg:autotune-regression}
\end{algorithm}

A problem of predicting oracle workgroup sizes is that, for each
training instance, an exhaustive search of the optimisation space must
be performed in order to find the oracle workgroup size. An
alternative approach is to instead predict the expected \emph{runtime}
of a kernel given a specific workgroup size. Given training data
consisting of $(f(s),w,t)$ tuples, where $f(s)$ are scenario features,
$w$ is the workgroup size, and $t$ is the observed runtime, we train a
regressor $R(f(s), w)$ which predicts the runtime of scenario and
workgroup size combinations. The selected workgroup size
$\bar{\Omega}(s)$ is then the workgroup size from a pool of candidates
which minimises the output of the regressor, as shown in
Algorithm~\ref{alg:autotune-regression}. The fitness function
$\Delta(x)$ is computes the reciprocal of the predicted runtime, so as
to favour shorter over longer runtimes. Note that the algorithm is
self correcting in the presence of refused parameters --- if a
workgroup size is refused, it is removed from the candidate pool, and
the next best candidate is chosen. This removes the need for fallback
handlers, and the technique allows for training on data for which the
oracle workgroup size is unknown, and has the secondary advantage that
this allows for an additional training instance to be gathered every
time a kernel is evaluated.


\subsection{Predicting Relative Performance}

Accurately predicting the runtime of arbitrary stencil codes is a
difficult problem due to the impacts of flow control. In such cases,
it may be more effective to instead predict the \emph{relative}
performance of two different workgroup sizes for the same kernel. To
do this, we predict the \emph{speedup} of a workgroup size over a
baseline. This baseline is the workgroup which provides the best
average case performance across all scenarios and is known to be
safe. Such a baseline value represents the \emph{best} possible
performance which can be achieved using a single, statically chosen
workgroup size. We train a regressor $R(f(s), w)$ to predict the
relative performance of workgroup sizes over this baseline parameter,
and apply the same algorithm as for predicting runtimes. The fitness
function returns the predicted speedup of a workgroup size over the
baseline, so the selected workgroup size $\bar{\Omega}(s)$ is the
workgroup size from a pool of candidates which maximises the output of
the regressor. This has the same advantageous properties as predicting
runtimes, but by training using relative performance, we minimise the
risk of control flow leading to inaccurate predictions.


\section{Experimental Setup}

\begin{table*}
\scriptsize
\centering
\begin{tabular}{l l | l l l l l l}
\toprule
Host & Host Memory &  OpenCL Device &  Compute units & Frequency & Local Memory & Global Cache & Global Memory \\
\midrule
Intel i5-2430M & 8 GB  & CPU              &              4 &   2400 Hz &        32 KB &       256 KB &       7937 MB \\
Intel i5-4570  & 8 GB  & CPU              &              4 &   3200 Hz &        32 KB &       256 KB &       7901 MB \\
Intel i7-3820  & 8 GB  & CPU              &              8 &   1200 Hz &        32 KB &       256 KB &       7944 MB \\
Intel i7-3820  & 8 GB  & AMD Tahiti 7970  &             32 &   1000 Hz &        32 KB &        16 KB &       2959 MB \\
Intel i7-3820  & 8 GB  & Nvidia GTX 590   &              1 &   1215 Hz &        48 KB &       256 KB &       1536 MB \\
Intel i7-2600K & 16 GB & Nvidia GTX 690   &              8 &   1019 Hz &        48 KB &       128 KB &       2048 MB \\
Intel i7-2600  & 8 GB  & Nvidia GTX TITAN &             14 &    980 Hz &        48 KB &       224 KB &       6144 MB \\
\bottomrule
\end{tabular}
\caption{Specification of experimental platforms and OpenCL devices.}
\label{tab:hw}
\end{table*}

To evaluate the performance of the presented autotuning techniques, an
exhaustive enumeration of the workgroup size optimisation space for
429 combinations of architecture, program, and dataset was performed.

Table~\ref{tab:hw} describes the experimental platforms and OpenCL
devices used. Each platform was unloaded, frequency governors were
disabled, and the benchmark processes were set to the highest priority
available to the task scheduler. Datasets and programs were stored in
an in-memory file system. All runtimes were recorded with millisecond
precision using OpenCL's Profiling API to record the kernel execution
time. A minimum of 30 samples were recorded for each workgroup size in
multiples of 2 up to the maximum allowed for each scenario.

\begin{table}
\scriptsize
\centering
\begin{tabular}{lrrrrp{1.3cm}}
\toprule
      Name &  North &  South &  East &  West &  Instruction Count \\
\midrule
   synthetic-a & 1--30 & 1--30 & 1--30 & 1--30 & 67--137\\
   synthetic-b & 1--30 & 1--30 & 1--30 & 1--30 & 592--706\\
   gaussian    & 1--10 & 1--10 & 1--10 & 1--10 & 82--83 \\
   gol         &      1 &      1 &     1 &     1 &                190 \\
   he          &      1 &      1 &     1 &     1 &                113 \\
   nms         &      1 &      1 &     1 &     1 &                224 \\
   sobel       &      1 &      1 &     1 &     1 &                246 \\
   threshold   &      0 &      0 &     0 &     0 &                 46 \\
\bottomrule
\end{tabular}
\caption{%
  Stencil kernels, border sizes (north, south, east, and west),
  and static instruction counts.
}
\label{tab:kernels}
\end{table}

In addition to the synthetic stencil benchmarks described in
Section~\ref{subsec:training}, six stencil kernels taken from four
reference implementations of standard stencil applications from the
fields of image processing, cellular automata, and partial
differential equation solvers are used: Canny Edge Detection, Conway's
Game of Life, Heat Equation, and Gaussian
Blur. Table~\ref{tab:kernels} shows details of the stencil kernels for
these reference applications and the synthetic training benchmarks
used. For each program, dataset sizes of size $512\times512$,
$1024\times1024$, $2048\times2048$, and $4096\times4096$ were used.

Program behaviour is validated by comparing program output against a
gold standard output collected by executing each of the real-world
benchmarks programs using the baseline workgroup size. The output of
real-world benchmarks with other workgroup sizes is compared to this
gold standard output to test for correct program execution.

Five different classification algorithms are used to predict oracle
workgroup sizes, chosen for their contrasting properties: Naive Bayes,
SMO, Logistic Regression, J48 Decision tree, and Random Forest. For
regression, a Random Forest with regression trees is used, chosen
because of its efficient handling of large feature sets compared to
linear models. The autotuning system is implemented as system daemon
in Python. SkelCL stencil programs request workgroup sizes from this
daemon, which performs feature extraction and classification.

% Feature extraction (particularly compilation to LLVM IR) introduces a
% runtime overhead to the classification process. To minimise this, a
% relational database stores lookup tables for device and dataset
% features, indexed by device IDs and checksums of kernel source codes,
% respectively. During autotuning, before feature extraction for either
% occurs a lookup is performed in the relevant table, meaning that the
% cost of feature extraction is amortised over time.


\section{Results}\label{sec:results}

This section describes the results of enumerating the workgroup size
optimisation space. The effectiveness of autotuning techniques for
exploiting this space are examined in
Section~\ref{sec:evaluation}. The experimental results consist of
measured runtimes for a set of \emph{test cases}, where a test case
$\tau_i$ consists of a scenario, workgroup size pair
$\tau_i = (s_i,w_i)$, and is associated with a \emph{sample} of
observed runtimes of the program. A total of 269813 test cases were
evaluated, which represents an exhaustive enumeration of the workgroup
size optimisation space for 429 scenarios. For each scenario, runtimes
for an average of 629 (max 7260) unique workgroup sizes were
measured. The average sample size for each test case is 83 (min 33,
total 16917118).

\begin{figure}
  \centering
  \includegraphics[width=\columnwidth]{img/oracle_param_space.pdf}
  \caption{%
    Oracle frequency counts for a subset of the workgroup sizes,
    $w_c \le 100, w_r \le 100$. There are 135 unique oracle workgroup
    sizes. The most common oracle workgroup size is
    $w_{(64 \times 4)}$, optimal for 15\% of scenarios.%
  }
\label{fig:oracle-wgsizes}
\end{figure}

The workgroup size optimisation space is non-linear and
complex. Across the 429 different scenarios, there are 135 unique
oracle workgroup sizes, with 31.5\% of scenarios having a unique
oracle workgroup size. Figure~\ref{fig:oracle-wgsizes} shows the
distribution of these oracle workgroup sizes. The average speedup of
the oracle workgroup size over the worst workgroup size for each
scenario is $15.14\times$ (min $1.03\times$, max $207.72\times$).

Of the 8504 unique workgroup sizes tested, 11.4\% were refused in one
or more test cases, with an average of 5.5\% test cases leading to
refused parameters. While there are certain patterns (for example,
workgroup sizes which contain $w_c$ and $w_r$ values which are
multiples of eight are less frequently refused, which is a common
width of SIMD vector operations~\cite{IntelCorporation2012}), a
refused parameter is an obvious inconvenience to the user, as one
would expect that any workgroup size within the specified maximum
should behave \emph{correctly}, if not efficiently.

\begin{figure}
  \centering
  \centering
  \includegraphics[width=.6\columnwidth]{img/refused_params_by_device}
  \caption{%
    The ratio of test cases with refused workgroup sizes, grouped by
    OpenCL device ID. No parameters were refused by the AMD device.%
  }
\label{fig:refused-params}
\end{figure}

Experimental results suggest that the problem is vendor --- or at
least device --- specific. Figure~\ref{fig:refused-params} shows the
ratio of refused test cases, grouped by device. We see a much greater
quantity of refused parameters for test cases on Intel CPU devices
than any other type, while no workgroup sizes were refused by the AMD
GPU. The exact underlying cause for these refused parameters is
unknown, but can likely by explained by inconsistencies or errors in
specific OpenCL driver implementations. As these OpenCL
implementations are still in active development, it is anticipated
that errors caused by unexpected behaviour will become more infrequent
as the technology matures. Note that the ratio of refused parameters
decreases across the three generations of Nvidia GPUs: GTX 590 (2011),
GTX 690 (2012), and GTX TITAN (2013). For now, it is imperative that
any autotuning system is capable of adapting to these refused
parameters by suggesting alternatives when they occur.

The baseline parameter $\bar{w}$ is the workgroup size which provides
the best overall performance while being legal for all
scenarios. Because of refused parameters, only a \emph{single}
workgroup size $w_{(4 \times 4)}$ from the set of experimental results
is found to have a legality of 100\%, suggesting that an adaptive
approach to setting workgroup size is necessary not just for the sake
of maximising performance, but also for guaranteeing program
correctness. The utility of the baseline parameter is that it
represents the best performance that can be achieved through static
tuning of the workgroup size parameter; however, compared to the
oracle workgroup size for each scenario, the baseline parameter
achieves only 24\% of the available performance.


\section{Evaluation}\label{sec:evaluation}

In this section we evaluate the effectiveness of the three proposed
autotuning techniques for predicting performant workgroup sizes. For
each autotuning technique, we partition the experimental data into
training and testing sets. Three strategies for partitioning the data
are used: the first is a 10-fold cross-validation; the second is to
divide the data such that only data collected from synthetic
benchmarks are used for training and only data collected from the
real-world benchmarks are used for testing; the third approach is to
create leave-one-out partitions for each unique device, kernel, and
dataset. For each combination of autotuning technique and testing
dataset, we evaluate each of the workgroup sizes predicted for the
testing data using the following metrics:
%
\begin{itemize}
\item time (real) --- the time taken to make the autotuning
  prediction. This includes classification time and any communication
  overheads.
\item accuracy (binary) --- whether the predicted workgroup size is
  the true oracle, $w = \Omega(s)$.
\item validity (binary) --- whether the predicted workgroup size
  satisfies the workgroup size constraints constraints,
  $w < W_{\max}(s)$.
\item refused (binary) --- whether the predicted workgroup size is
  refused, $w \in W_{refused}(s)$.
\item performance (real) --- the performance of the predicted
  workgroup size relative to the oracle for that scenario.
\item speedups (real) --- the relative performance of the predicted
  workgroup size relative to the baseline workgroup size
  $w_{(4 \times 4)}$, and human expert workgroup size
  $w_{(32 \times 4)}$ (where applicable).
\end{itemize}
%
The \emph{validty} and \emph{refused} metrics measure how often
fallback strategies are required to select a legal workgroup size
$w \in W_{legal}(s)$. This is only required for the classification
approach to autotuning, since the process of selecting workgroup sizes
using regressors respects workgroup size constraints.

\begin{figure}
\centering
\includegraphics[width=\columnwidth]{img/classification-syn-real}
\caption{%
  Autotuning performance using classifiers and synthetic benchmarks. Each
  classifier is trained on data collected from synthetic stencil
  applications, and tested for prediction quality using data from 6
  real-world benchmarks. Each of the different values correspond to a
  different data partitioning strategy, e.g.\ cross-kernel
  partitioning, 10-fold validation, etc. 95\% confidence intervals are
  shown where appropriate.%
}
\label{fig:class-syn}
\end{figure}


\begin{figure}
\centering
\begin{subfigure}[h]{.48\columnwidth}
\centering
\includegraphics[width=\columnwidth]{img/runtime-class-xval}
\caption{}
\label{fig:runtime-class-xval}
\end{subfigure}
\begin{subfigure}[h]{.48\columnwidth}
\centering
\includegraphics[width=\columnwidth]{img/speedup-class-xval}
\caption{}
\label{fig:speedup-class-xval}
\end{subfigure}
\caption{%
  Autotuning performance for each type of test dataset using
  regressors to predict: (\subref{fig:runtime-class-xval}) kernel
  runtimes, and (\subref{fig:speedup-class-xval}) relative performance
  of workgroup sizes.%
}
\label{fig:regression-class}
\end{figure}

\subsection{Predicting Oracle Workgroup Size}

With the exception of the ZeroR, which is a classifier that predicts
\emph{only} the baseline workgroup size
$w_{\left( 4 \times 4 \right)}$, the classifiers achieve good speedups
over the baseline, ranging from $4.61\times$ to $5.05\times$ when
averaged across all test sets. Figure~\ref{fig:class-syn} shows the
results when classifiers are trained using data from synthetic
benchmarks and tested using real-world benchmarks. The highest average
speedup is achieved by the SMO classifier, and the lowest by Naive
Bayes. The difference between average speedups is not significant
between the types of classifier, with the exception of SimpleLogistic,
which performs poorly when trained with synthetic benchmarks and
tested against real-world programs. This suggests the model
over-fitting to features of the synthetic benchmarks which are not
shared by the real-world tests.

Of the three approaches to handling invalid predictions, the fallback
handler with the best average case performance is
\textsc{NearestNeighbour}, achieving an average speedup across all
classifiers and validation sets of $5.26\times$. The speedup of
\textsc{Random} fallback handler is $3.69\times$, and $1.0\times$ for
\textsc{Baseline}. Interestingly, both the lowest and highest speedups
are achieved by the \textsc{Random} fallback handler, since it
essentially performs a random exploration of the optimisation
space. However, the \textsc{NearestNeighbour} fallback handler
provides consistently greater speedups for the majority of test cases,
indicating that it successfully exploits structure in the optimisation
spaces.


\subsection{Predicting Runtimes and Speedups}

Figures~\ref{fig:runtime-class-xval} and~\ref{fig:speedup-class-xval}
show a summary of results for autotuning using regressors to predict
program runtimes and speedups, respectively. Of the two regression
techniques, predicting the \emph{speedup} of workgroup sizes is much
more successful than predicting the \emph{runtime}. This is most
likely caused by the inherent difficulty in predicting the runtime of
arbitrary code, where dynamic factors such as flow control and loop
bounds are not captured by static instruction counts and densities
which are used as features by the machine learning models. The average
speedup achieved by predicting runtimes is $4.14\times$. For
predicting speedups, the average is $5.57\times$.

The prediction times using regressors are significantly greater than
using classifiers. This is because, while a classifier makes a single
prediction, the number of predictions required of a regressor grows
with the size of $W_{\max}(s)$, since classification with regression
requires making predictions for all
$w \in \left\{ w | w < W_{\max}(s) \right\}$. The fastest classifier
is J48, due to the it's simplicity --- it can be implemented as a
sequence of nested \texttt{if} and \texttt{else} statements.


\subsection{Comparison with Human Expert}

\begin{figure}
\centering
\includegraphics[width=\columnwidth]{img/speedup-distributions}
\caption[Speedup results over human expert]{%
  Distributions of speedups over \emph{human expert}, ignoring cases
  where the workgroup size selected by human experts is
  invalid. Classifiers are using \textsc{NearestNeighbour} fallback
  handlers. The speedup axis is fixed to the range 0--2.5 to highlight
  the IQRs, which results in some outlier speedups > 2.5 being
  clipped.%
}
\label{fig:speedup-distributions}
\end{figure}

In the original implementation of the SkelCL stencil
pattern~\cite{Breuer2014a}, \citeauthor{Breuer2014a} selected a
workgroup size of $w_{(32 \times 4)}$ in an evaluation of 4 stencil
operations on a Tesla S1070 system. In our evaluation of 429
combinations of kernel, architecture, and dataset, we found that this
workgroup size is refused by 2.6\% of scenarios, making it unsuitable
for use as a baseline. However, if we remove the scenarios for which
$w_{(32 \times 4)}$ is \emph{not} a legal workgroup size, we can
directly compare the performance against the autotuning predictions.

Figure~\ref{fig:speedup-distributions} plots the distribution of
speedups of all test instances over the human expert parameter for
each autotuning technique. The speedup distributions show consistent
classification results for the five classification techniques, with
the speedup at the lower quartile for all classifiers being
$\ge 1.0\times$. The IQR for all classifiers is $< 0.5$, but there are
outliers with speedups both well below $1.0\times$ and well above
$2.0\times$. In contrast, the speedups achieved using regressors to
predict runtimes have a lower range, but also a lower median and a
larger IQR. Clearly, this approach is the least effective of the
evaluated autotuning techniques. Using regressors to predict relative
performance is more successful, with the highest median speedup of all
the techniques. However, it also has a large IQR and the lower
quartile has a speedup value well below 1, meaning that for more than
25\% of test instances, the workgroup size selected did not perform as
well as the human expert selected workgroup size.


% \begin{figure}
% \centering
% \begin{subfigure}[t]{0.48\columnwidth}
% \centering
% \includegraphics[width=\columnwidth]{img/heatmap_1}
% \vspace{-1.5em} % Shrink vertical padding
% \caption{}
% \label{fig:class-hmaps-1}
% \end{subfigure}
% \begin{subfigure}[t]{0.48\columnwidth}
% \centering
% \includegraphics[width=\columnwidth]{img/heatmap_2}
% \vspace{-1.5em} % Shrink vertical padding
% \caption{}
% \label{fig:class-hmaps-2}
% \end{subfigure}
% \\
% \begin{subfigure}[t]{0.48\columnwidth}
% \centering
% \includegraphics[width=\columnwidth]{img/heatmap_3}
% \vspace{-1.5em} % Shrink vertical padding
% \caption{}
% \label{fig:class-hmaps-3}
% \end{subfigure}
% \begin{subfigure}[t]{0.48\columnwidth}
% \centering
% \includegraphics[width=\columnwidth]{img/heatmap_5}
% \vspace{-1.5em} % Shrink vertical padding
% \caption{}
% \label{fig:class-hmaps-4}
% \end{subfigure}
% \\
% \begin{subfigure}[t]{0.48\columnwidth}
% \centering
% \includegraphics[width=\columnwidth]{img/reg_runtime_heatmap}
% \vspace{-1.5em} % Shrink vertical padding
% \caption{}
% \label{fig:class-hmaps-5}
% \end{subfigure}
% \begin{subfigure}[t]{0.48\columnwidth}
% \centering
% \includegraphics[width=\columnwidth]{img/reg_speedup_heatmap}
% \vspace{-1.5em} % Shrink vertical padding
% \caption{}
% \label{fig:class-hmaps-6}
% \end{subfigure}
% \caption[Classification error heatmaps]{%
%   Heatmaps of classification errors for 10-fold cross-validation,
%   showing a subset of the optimisation space. The shading in each
%   cells indicates if it is predicted less frequently (blue), ore more
%   frequently (red) than it is optimal. Colour gradients are normalised
%   across plots.%
% }
% \label{fig:class-hmaps}
% \end{figure}

% Figure~\ref{fig:class-hmaps} visualises the classification errors of
% each of the autotuning techniques. It shows that while the performance
% of all of the classifiers is comparable, the distribution of
% predictions is not. Only the NaiveBayes and RandomForest classifiers
% predicted the human expert selected workgroup size of
% $w_{(32 \times 4)}$ as frequently, or more frequently, than it was
% optimal. The two regression techniques were the least accurate of all
% of the autotuning techniques.


\section{Related Work}\label{sec:related}

% Iterative compilation is the method of performance tuning in which a
% program is compiled and profiled using multiple different
% configurations of optimisations in order to find the configuration
% which maximises performance. One of the the first formalised
% publications of the technique appeared in \citeyear{Bodin1998} by
% \citeauthor{Bodin1998}~\cite{Bodin1998}.  Iterative compilation has
% since been demonstrated to be a highly effective form of empirical
% performance tuning for selecting compiler optimisations.

% Given the huge number of possible compiler optimisations (there are
% 207 flags and parameters to control optimisations in GCC v4.9), it
% is often unfeasible to perform an exhaustive search of the entire
% optimisation space, leading to the development of methods for
% reducing the cost of evaluating configurations. These methods reduce
% evaluation costs either by shrinking the dimensionality or size of
% the optimisation space, or by guiding a directed search to traverse
% a subset of the space.

% Machine learning has been successful applied to this problem,
% in~\cite{Stephenson2003}, using ``meta optimisation'' to tune
% compiler heuristics through an evolutionary algorithm to automate
% the search of the optimisation space. \citeauthor{Fursin2011}
% continued this with Milepost GCC, the first machine learning-enabled
% self-tuning compiler~\cite{Fursin2011}. A recent survey of the use
% of machine learning to improve heuristics quality by
% \citeauthor{Burke2013} concludes that the automatic
% \emph{generation} of these self-tuning heuristics but is an ongoing
% research challenge that offers the greatest generalisation
% benefits~\cite{Burke2013}.

\citeauthor{Ganapathi2009} demonstrated early attempts at autotuning
multicore stencil codes in~\cite{Ganapathi2009}, presenting an
autotuner which evaluates a random 1500 selections from an space of 10
million optimisations, achieving 18\% better than that of a human
expert. The Kernel Canonical Correlation Analysis used in their
autotuner restricts the scalability of their system, as the complexity
of model building grows exponentially with the number of features. In
their evaluation, the system requires two hours of compute time to
build the model for only 400 seconds of benchmark data.

% \citeauthor{Berkeley2009} targeted 3D stencils code performance
% in~\cite{Berkeley2009}. Stencils are decomposed into core blocks,
% sufficiently small to avoid last level cache capacity misses. These
% are then further decomposed to thread blocks, designed to exploit
% common locality threads may have within a shared cache or local
% memory. Thread blocks are divided into register blocks in order to
% take advantage of data level parallelism provided by the available
% registers. Data allocation is optimised on NUMA systems. The
% performance evaluation considers speedups of various optimisations
% with and without consideration for host/device transfer overhead.

\citeauthor{Kamil2010} present an autotuning framework
in~\cite{Kamil2010} which accepts as input a Fortran 95 stencil
expression and generates tuned shared-memory parallel implementations
in Fortan, C, or CUDA. The system uses an IR to explore autotuning
transformations, enumerating a subset of the optimisation space and
recording only a single execution time for each configuration,
reporting the fastest. They demonstrate their system on 4
architectures using 3 benchmarks, with speedups of up to $22\times$
compared to serial implementations. The CUDA code generator does not
optimise for the GPU memory hierarchy, using only global memory. As
demonstrated in this work, improper utilisation of local memory can
hinder program performance by two orders of magnitude.
% There is no directed search or cross-program learning.

% In~\cite{Zhang2013a}, \citeauthor{Zhang2013a} present a code generator
% and autotuner for 3D Jacobi stencil codes. Using a DSL to express
% kernel functions, the code generator performs substitution from one of
% two CUDA templates to create programs for execution on GPUs. GPU
% programs are parameterised and tuned for block size, block dimensions,
% and whether input data is stored in read only texture memory. This
% creates an optimisation space of up to 200 configurations. In an
% evaluation of 4 benchmarks, the authors report impressive performance
% that is comparable with previous implementations of iterative Jacobi
% stencils on GPUs~\cite{Holewinski2012, Phillips2010}. The dominating
% parameter is shown to be block dimensions, followed by block size,
% then read only memory. The DSL presented in the paper is limited to
% expressing only Jacobi Stencils applications. Critically, their
% autotuner requires a full enumeration of the parameter space for each
% program. Since there is no indication of the compute time required to
% gather this data, it gives the impression that the system would be
% impractical for the needs of general purpose stencil computing. The
% autotuner presented in this thesis overcomes this drawback by learning
% parameter values which transfer to unseen stencils, without the need
% for an expensive tuning phase for each program and architecture.

% In~\cite{Christen2011}, \citeauthor{Christen2011} presentf a DSL for
% expressing stencil codes, a C code generator, and an autotuner for
% exploring the optimisation space of blocking and vectorisation
% strategies. The DSL supports stencil operations on arbitrarily
% high-dimensional grids. The autotuner performs either an exhaustive,
% multi-run Powell search, Nelder Mead, or evolutionary search to find
% optimal parameter values. They evaluate their system on two CPUS and
% one GPU using 6 benchmarks. A comparison of tuning results between
% different GPU architectures would have been welcome, as the results
% presented in this thesis show that devices have different responses to
% optimisation parameters. The authors do not present a ratio of the
% available performance that their system achieves, or how the
% performance of optimisations vary across benchmarks or devices.

PARTANS is an autotuning framework which targets the size of the halo
region for multi-GPU stencils~\cite{Lutz2013}. \citeauthor{Lutz2013}
explore the effect of varying the optimisation space using six
benchmark applications, finding that the optimal halo size depends on
the size of the grid, the number of partitions, and the connection
mechanism. The authors present an autotuner which determines problem
decomposition and swapping strategy offline, and performs an online
search for the optimal halo size. The selection of overlapping halo
region size compliments the selection of workgroup size which is the
subject of this thesis. However, PARTANS uses a custom DSL rather than
the generic interface provided by SkelCL, and PARTANS does not learn
the results of tuning across programs, or across multiple runs of the
same program.

\citeauthor{Collins2012} autotune Algorithmic Skeletons
in~\cite{Collins2012}, first using Principle Components Analysis to
reduce the dimensionality of the optimisation space, followed by a
search of parameter values to optimise program performance by a factor
of $1.6\times$ over values chosen by a human
expert. In~\cite{Collins2013}, they extend this using static feature
extraction and nearest neighbour classification to further prune the
search space, achieving an average 89\% of the oracle performance
after evaluating 45 parameters.

A method for the automatic generation of synthetic benchmarks for the
purpose of performance tuning is presented in~\cite{Chiu2015}, using
parameterised template substitution over a user-defined range of
values to generate training programs. The authors describe an
application of their tool for generating OpenCL stencil kernels, but
do not report any performance results.

% Performant GPGPU programs require careful attention from the developer
% to properly manage data layout in DRAM, caching, diverging control
% flow, and thread communication. The importance of proper exploitation
% of local shared memory and synchronisation costs is explored
% in~\cite{Lee2010}. In~\cite{Chen2014}, data locality optimisations are
% automated using a description of the hardware and a
% memory-placement-agnostic compiler. The authors demonstrate impressive
% speedups of up to $2.08\times$, although at the cost of requiring
% accurate memory hierarchy descriptor files for all targeted
% hardware. The descriptor files must be hand generated, requiring
% expert knowledge of the underlying hardware in order to properly
% exploit memory locality.


\section{Conclusions}\label{sec:conclusions}

To the best of our knowledge, the autotuning methodologies presented
in this work constitute the first attempt to autotune the workgroup
size of high-level stencil patterns. The autotuning techniques
proposed in this paper achieve up to 94\% of the available
performance, providing speedups of $5.57\times$ over static tuning,
while providing robust fallbacks in the presence of unexpected
behaviour of OpenCL driver implementations. Of the techniques
proposed, predicting the relative performances of workgroup sizes
using regressors provides the highest median speedup, whilst
predicting the oracle workgroup size using decision tree classifiers
requires the lowest overhead. This presents a trade-off between
classification time and training time, which could be explored in
future work, possibly using a hybrid combination of the techniques
presented in this paper.

In the future, we will expand the autotuner to accommodate additional
optimisation parameters, as well as further exploring the transition
towards online machine learning which is enabled by using regressors
to predict relative performance or kernel runtimes. This could be
combined with the use of adaptive sampling plans to minimise the
number of observations required to distinguish bad from good parameter
values, such as presented in~\cite{Leather2009}. The use of dynamic
profiling could be used to increase the prediction accuracy of kernel
runtimes.


\acks

This work was supported by grant EP/L01503X/1 for the University of
Edinburgh School of Informatics Centre for Doctoral Training in
Pervasive Parallelism
(\url{http://pervasiveparallelism.inf.ed.ac.uk/}) from the UK
Engineering and Physical Sciences Research Council (EPSRC).

% We recommend abbrvnat bibliography style.

\label{bibliography}
\printbibliography


\end{document}

% 6-10 pages, 9pt font
%
% Topics:
%
% * Machine learning based autotuning.
% * Representative benchmarking.
% * Automatic fault tolerance.
% * Run-time adaption.
%

% The following \documentclass options may be useful:
%
% preprint      Remove this option only once the paper is in final form.
% 10pt          To set in 10-point type instead of 9-point.
% 11pt          To set in 11-point type instead of 9-point.
% authoryear    To obtain author/year citation style instead of numeric.
\documentclass[nonatbib,preprint,9pt]{sigplanconf}

\documentclass[prodmode,acmtaco]{acmsmall}

% Package to generate and customize Algorithm as per ACM style
\usepackage[ruled]{algorithm2e}
\renewcommand{\algorithmcfname}{ALGORITHM}
\SetAlFnt{\small}
\SetAlCapFnt{\small}
\SetAlCapNameFnt{\small}
\SetAlCapHSkip{0pt}
\IncMargin{-\parindent}

% Metadata Information
\acmVolume{9}
\acmNumber{4}
\acmArticle{39}
\acmYear{2016}
\acmMonth{3}

% Copyright
%\setcopyright{acmcopyright}
\setcopyright{acmlicensed}
%\setcopyright{rightsretained}
%\setcopyright{usgov}
%\setcopyright{usgovmixed}
%\setcopyright{cagov}
%\setcopyright{cagovmixed}

% DOI
\doi{0000001.0000001}

%ISSN
\issn{1234-56789}


%%%%%%%%%%%%%%%%%%%%%%%%%
%% Document and Layout %%
%%%%%%%%%%%%%%%%%%%%%%%%%

% Fix for multiple "No room for a new \dimen" errors.
%
% See: http://tex.stackexchange.com/questions/38607/no-room-for-a-new-dimen
%
\usepackage{etex}

\usepackage[utf8]{inputenc}

% Fix for "'babel/polyglossia' detected but 'csquotes' missing"
% warning. NOTE: Include after inputenc.
%
\usepackage{csquotes}

% Make internal macro definitions accessible,
% e.g. \@title, \@date \@author.
\makeatletter

% Multi-column support.
\usepackage{multicol}

% A useful package which includes macros like \ifdef{}{}{}:
%
\usepackage{etoolbox}

% Uncomment the following line to remove column separation:
%
%\setlength{\columnsep}{5mm}

% Allow user-defined warning and error filters.
%
\usepackage{silence}

\usepackage{adjustbox}


%%%%%%%%%%%%%%%%%%%%%
% Table of Contents %
%%%%%%%%%%%%%%%%%%%%%

% % Set chapter and section numbering depth:
% %
% \setcounter{secnumdepth}{2}


%%%%%%%%%%%%%%%%
% Bibliography %
%%%%%%%%%%%%%%%%
% \usepackage[%
%     backend=biber,
%     style=ieee,
%     % style=numeric-comp,
%     % style=numeric-comp,  % numerical-compressed
%     sorting=none,        % nty,nyt,nyvt,anyt,anyvt,ynt,ydnt,none
%     sortcites=true,      % sort \cite{b a d c}: true,false
%     block=none,          % space between blocks: none,space,par,nbpar,ragged
%     indexing=false,      % indexing options: true,false,cite,bib
%     citereset=none,      % don't reset cites
%     isbn=false,          % print ISBN?
%     url=true,            % print URL?
%     doi=false,           % print DOI?
%     natbib=true,         % natbib compatability
%    ]{biblatex}

% \usepackage{natbib}

% % Filter annoying and unavoidable biblatex warning:
% \WarningFilter{biblatex}{Patching footnotes failed}

% Reduce the font size of the bibliography:
% \renewcommand{\bibfont}{\normalfont\scriptsize}

% Determine which BibTeX file to use:
%
% If available, use my Mendeley BibTex library, located in the home
% directory. Note that this is a relative path and will break if
% either this file or the BibTex library are moved. If the library is
% not present, use the local refs.bib file.
% \newcommand{\BibResourceGlobal}{../../../library.bib}
% \newcommand{\BibResourceLocal}{refs.bib}

% \IfFileExists{\BibResourceGlobal}
%   {\newcommand{\BibResource}{\BibResourceGlobal}}
%   {\newcommand{\BibResource}{\BibResourceLocal}}

% \addbibresource{\BibResource}


%%%%%%%%%%%%%%
% Appendices %
%%%%%%%%%%%%%%

% Appendix package. Documentation:
%
%  http://mirror.ox.ac.uk/sites/ctan.org/macros/latex/contrib/appendix/appendix.pdf
%
% Package options:
%
% toc      - Put a header (e.g., `Appendices') into the Table of Contents
%            (the ToC) before listing the appendices. (This is done by
%            calling the \addappheadtotoc command.)
% page     - Puts a title (e.g., `Appendices') into the document at the
%            point where the appendices environment is begun. (This is
%            done by calling the \appendixpage command.)
% title    - Adds a name (e.g., `Appendix') before each appendix title in
%            the body of the document. The name is given by the value
%            of \appendixname. Note that this is the default behaviour
%            for classes that have chapters.
% titletoc - Adds a name (e.g., `Appendix') before each appendix listed
%            in the ToC. The name is given by the value
%            of \appendixname.
% header   - Adds a name (e.g., `Appendix') before each appendix in page
%            headers.  The name is given by the value
%            of \appendixname. Note that this is the default behaviour
%            for classes that have chapters.
\usepackage[title, titletoc]{appendix}

% pre-requisites for rendering upquotes in listings package.
\usepackage[T1]{fontenc}
\usepackage{lmodern}
\usepackage{textcomp}

% code listings.
\usepackage{listings}

% set \ttfamily to use courier fonts.
%
% See: http://tex.stackexchange.com/a/33686
%
\usepackage{courier}

\lstset{frame=bt,                    % Add top and bottom frame lines
breaklines=true,             % Force line wrapping
captionpos=b,                % Place caption below listing
numbers=left,                % Add left-side line numbers
basicstyle=\scriptsize\ttfamily, % Set font size and type
showstringspaces=false,      % Don't show visible whitespace
numberstyle=\tiny,
upquote=true,                % Use upright quotes, not curly
commentstyle=\bfseries}      % Embolden comments

% Use (*@ @*) to escape LaTeX commands within listings.
\lstset{escapeinside={(*@}{@*)}}

% Add 10pt space between chapters in TOC listings entries:
%\let\Chapter\chapter
%\def\chapter{\addtocontents{lol}{\protect\addvspace{10pt}}\Chapter}


%%%%%%%%%%%%%%%%%%%%%%%%
%% Graphics and maths %%
%%%%%%%%%%%%%%%%%%%%%%%%
\usepackage{amsmath}

% Vector notation, e.g. \vv{x}:
%
\usepackage{esvect}

% Additional amsmath symbols, see:
%
% http://texblog.org/2007/08/27/number-sets-prime-natural-integer-rational-real-and-complex-in-latex/
%
\usepackage{amsfonts}
\usepackage{amssymb}

\usepackage{graphicx}
\usepackage{mathtools}
\usepackage{tikz}
\usepackage{tikz-qtree}

% Provide bold font face in maths.
\usepackage{bm}

\usepackage{subcaption}
\expandafter\def\csname ver@subfig.sty\endcsname{}

% Define an 'myalignat' command which behave as 'alignat' without the
% vertical top and bottom padding. See:
%     http://www.latex-community.org/forum/viewtopic.php?f=5&t=1890
\newenvironment{myalignat}[1]{%
\setlength{\abovedisplayskip}{-.7\baselineskip}%
\setlength{\abovedisplayshortskip}{\abovedisplayskip}%
\start@align\z@\st@rredtrue#1
}%
{\endalign}

% Define additional operators:
\DeclareMathOperator*{\argmin}{arg\,min}
\DeclareMathOperator*{\argmax}{arg\,max}

\DeclareMathOperator*{\gain}{Gain}

% Skeleton operators.
\DeclareMathOperator*{\map}{Map}
\DeclareMathOperator*{\reduce}{Reduce}
\DeclareMathOperator*{\scan}{Scan}
\DeclareMathOperator*{\stencil}{Stencil}
\DeclareMathOperator*{\zip}{Zip}
\DeclareMathOperator*{\allpairs}{All\,Pairs}

% Maths plots using pgfplots, see:
%
%     http://pgfplots.sourceforge.net/pgfplots.pdf
%
\usepackage{pgfplots}

% Disable compatability mode.
%
\pgfplotsset{compat=1.12}

% Gantt charts using pgfgantt, see:
%
%     http://www.ctan.org/pkg/pgfgantt
%
\usepackage{pgfgantt}

% Fix milestone aspect ratio by defining a custom element.
\newganttchartelement*{mymilestone}{
mymilestone/.style={
shape=diamond,
inner sep=2pt,
draw=black,
top color=black,
bottom color=black,
}
}

% Tikz flowchart configuration.
\usetikzlibrary{shapes,arrows,shadows,fit,backgrounds}
\tikzstyle{decision} = [diamond,
draw,
text width=4.5em,
text badly centered,
node distance=3cm,
inner sep=0pt]
\tikzstyle{block}    = [rectangle,
draw,
text width=5em,
text centered,
node distance=3cm,
minimum height=4em,
inner sep=.2cm]
\tikzstyle{line}     = [draw, -latex']

% Add dirtree picture style, see:
%
%     http://tex.stackexchange.com/a/34268
%
\newcount\dirtree@lvl
\newcount\dirtree@plvl
\newcount\dirtree@clvl
\def\dirtree@growth{%
\ifnum\tikznumberofcurrentchild=1\relax
\global\advance\dirtree@plvl by 1
\expandafter\xdef\csname dirtree@p@\the\dirtree@plvl\endcsname{\the\dirtree@lvl}
\fi
\global\advance\dirtree@lvl by 1\relax
\dirtree@clvl=\dirtree@lvl
\advance\dirtree@clvl by -\csname dirtree@p@\the\dirtree@plvl\endcsname
\pgf@xa=0.33cm\relax
\pgf@ya=-\baselineskip\relax
\pgf@ya=\dirtree@clvl\pgf@ya
\pgftransformshift{\pgfqpoint{\the\pgf@xa}{\the\pgf@ya}}%
\ifnum\tikznumberofcurrentchild=\tikznumberofchildren
\global\advance\dirtree@plvl by -1
\fi
}
\tikzset{
dirtree/.style={
growth function=\dirtree@growth,
every node/.style={anchor=north},
every child node/.style={anchor=west},
edge from parent path={(\tikzparentnode\tikzparentanchor) |- (\tikzchildnode\tikzchildanchor)}
}
}

% UML sequence diagram macros, see:
%
%     https://code.google.com/p/pgf-umlsd/
%
% Options:
%
%     underline - Underline object names
%
\usepackage[underline=false]{pgf-umlsd}

% Support for SVG graphics.
%
% NOTE that you must pass the "--shell-escape" argument to pdflatex to
% compile. NOTE also that images *MUST* be placed within the graphics
% path.
\usepackage{svg}
\graphicspath{{img/}}

%%%%%%%%%%%%%%%%%%%%%%
%% Tables and lists %%
%%%%%%%%%%%%%%%%%%%%%%

% Required to use labm8 exported tables.
%
\usepackage{booktabs}

% Required for full page-width tables.
\usepackage{tabularx}

%\usepackage{enumitem}
%\setenumerate{itemsep=0pt}

% Use no left margin for lists:
%\setlist{leftmargin=*}

\usepackage{longtable}

% Define column types L, C, R with known text justification and fixed
% widths:
\usepackage{array}
\newcolumntype{L}[1]{>{\raggedright\let\newline\\\arraybackslash\hspace{0pt}}m{#1}}
\newcolumntype{C}[1]{>{\centering\let\newline\\\arraybackslash\hspace{0pt}}m{#1}}
\newcolumntype{R}[1]{>{\raggedleft\let\newline\\\arraybackslash\hspace{0pt}}m{#1}}


%%%%%%%%%%%%%%%%%%%%%%%%%%%%%
%% Typesetting and symbols %%
%%%%%%%%%%%%%%%%%%%%%%%%%%%%%

% Adjustable font sizes in \Verbatim{}
\usepackage{fancyvrb}

%\usepackage{titlesec}
% Set section and paragraph heading fonts:
%\titleformat*{\section}{\Large\bfseries}
%\titleformat*{\subsection}{\normalsize\bfseries}
%\titleformat*{\subsubsection}{\normalsize}
%\titleformat*{\paragraph}{\large\bfseries}
%\titleformat*{\subparagraph}{\large\bfseries}

% Set section heading margins. Usage:
% \titlespacing*{<command>}{<left>}{<before>}{<after>}
%\titlespacing*{\section}{0pt}{.6em}{.3em}
%\titlespacing*{\subsection}{0pt}{.6em}{.2em}

% Set paragraph indentation size. Default is 15pt.
%\setlength{\parindent}{10pt}

% The line spacing can be globally set using \linespread:
%
% \linespread{1.2}

% Add a command \hr{} which will draw a horizontal rule the width of
% the text.
%
\newcommand{\hr}{\noindent\makebox[\linewidth]{\rule{\textwidth}{0.2pt}}}

% Add a command \br{} which will create a horizontal space of exactly
% one line height.
%
\newcommand{\br}{\hspace{\baselineskip}}

% Define a command to allow word breaking.
\newcommand*\wrapletters[1]{\wr@pletters#1\@nil}
\def\wr@pletters#1#2\@nil{#1\allowbreak\if&#2&\else\wr@pletters#2\@nil\fi}

% Define a command to create centred page titles.
\newcommand{\centredtitle}[1]{
\begin{center}
  \large
  \vspace{0.9cm}
  \textbf{#1}
\end{center}}

% Support hyperlinks using the \hyperref, \url and \href
% macros. Usage:
%
%    \hyperref[label_name]{''link text''}
%
%    \url{<my_url>}
%
%    \href{<my_url>}{<description>}
%
\usepackage{hyperref}

% Disable colored borders of links, cross-references etc in PDF output
\hypersetup{pdfborder={0 0 0}}

% Provide generic commands \degree, \celsius, \perthousand, \micro
% and \ohm which work both in text and maths mode.
\usepackage{gensymb}

%%%%%%%%%%%%%%%%%%%%%%%%%%%%%%%%%
%% Placeholder text generation %%
%%%%%%%%%%%%%%%%%%%%%%%%%%%%%%%%%

% Use either \blindtext or \libpsum to generate placeholder text. Also
% note the macros \blinditemize, \blindenumerate, \blinddescription.
\usepackage[english]{babel}
\usepackage{blindtext}
\usepackage{lipsum}


\begin{document}

\special{papersize=8.5in,11in}
\setlength{\pdfpageheight}{\paperheight}
\setlength{\pdfpagewidth}{\paperwidth}

\conferenceinfo{HLPGPGPU '16}{Month d--d, 20yy, City, ST, Country}
\copyrightyear{2016}
\copyrightdata{978-1-nnnn-nnnn-n/yy/mm}
\doi{nnnnnnn.nnnnnnn}

% Uncomment one of the following two, if you are not going for the
% traditional copyright transfer agreement.

%\exclusivelicense                % ACM gets exclusive license to publish,
                                  % you retain copyright

%\permissiontopublish             % ACM gets nonexclusive license to publish
                                  % (paid open-access papers,
                                  % short abstracts)

% \titlebanner{banner above paper title}        % These are ignored unless
% \preprintfooter{HLPGPGPU workshop '16}   % 'preprint' option specified.

% \title{Robust Autotuning of Stencil codes for GPUs with OmniTune}
% \title{Towards robust cross-architecture GPGPU patterns autotuning}
\title{Towards Collaborative Performance Tuning of Algorithmic Skeletons}

% \subtitle{Subtitle Text, if any}

\authorinfo{Chris Cummins\and Pavlos Petoumenos \and Michel Steuwer \and Hugh Leather}
           {University of Edinburgh}
           {c.cummins@ed.ac.uk, ppetoume@inf.ed.ac.uk, michel.steuwer@ed.ac.uk, hleather@inf.ed.ac.uk}

\maketitle

\begin{abstract}
  The physical limitations of microprocessor design have forced the
  industry towards increasing heterogeneous designs to extract
  performance. This trend has not been matched with adequate software
  tools, leading to a growing disparity between the availability of
  parallelism and the ability for application developers to exploit
  it.

  Algorithmic skeletons simplify parallel programming by providing
  high-level, reusable patterns of computation. Achieving performant
  skeleton implementations is a difficult task; skeleton authors must
  attempt to anticipate and tune for a wide range of architectures and
  use cases. This results in implementations that target the general
  case and cannot provide the performance advantages that are gained
  from tuning low level optimisation parameters. Autotuning combined
  with machine learning offers promising performance benefits in these
  situations, but the high cost of training and lack of available
  tools limits the practicality of autotuning for real world
  programming. We believe that performing autotuning at the level of
  the skeleton library can overcome these issues.

  In this work, we present OmniTune --- an extensible and distributed
  framework for dynamic autotuning of optimisation parameters at
  runtime. OmniTune uses a client-server model with a flexible API to
  support machine learning enabled autotuning. Training data is shared
  across a network of cooperating systems, using a collective approach
  to performance tuning.

  We demonstrate the practicality of OmniTune in a case study using
  the algorithmic skeleton library SkelCL. By automatically tuning the
  workgroup size of OpenCL Stencil skeleton kernels, we show that that
  static tuning across a range of GPUs and programs can achieve only
  $26\%$ of the optimal performance, while OmniTune achieves $92\%$ of
  this maximum, equating to an average $5.65\times$ speedup. This is
  achieved without introducing a significant runtime overhead, and
  enables portable, cross-device and cross-program tuning.
\end{abstract}

% \category{CR-number}{subcategory}{third-level}

% % general terms are not compulsory anymore,
% % you may leave them out
% % \terms
% % term1, term2

% \keywords
% keyword1, keyword2

\section{Introduction}\label{sec:introduction}

General purpose programming with GPUs has been shown to provide huge
parallel throughput, but poses a significant programming challenge,
requiring application developers to master an unfamiliar programming
model (such as provided by CUDA or OpenCL) and architecture (SIMD with
a multi-level memory hierarchy). As a result, GPGPU programming is
often considered beyond the realm of everyday development. If steps
are not taken to increase the accessibility of such parallelism, the
gap between potential and utilised performance will continue to widen
as hardware core counts increases.

Algorithmic skeletons offer a solution to this this
\emph{programmability challenge} by raising the level of
abstraction. This simplifies parallel programming, allowing developers
to focus on solving problems rather than coordinating parallel
resources. Skeletons provide robust parallel implementations of common
patterns of computation which developers parameterise with their
application-specific code~\cite{Gonzalez2010}. This greatly reduces
the challenge of parallel programming, allowing users to structure
their problem-solving logic sequentially, while offloading the
cognitive cost of parallel coordination to the skeleton library
author. The rising number of skeleton frameworks supporting graphics
hardware illustrates the demand for high level abstractions for GPGPU
programming~\cite{Enmyren2010, Marques2013, Nugteren2014a}. The
challenge is in maintaining portable performance across the breadth of
devices in the rapidly developing GPU and heterogeneous architecture
landscape.


\subsection{The Performance Portability Challenge}

The performance of parallel programs is sensitive to low level
\emph{parameters}, and when tuning such parameters for maximising
performance, one size does not fit all. The suitability of parameter
values depends on the program implementation, the target hardware, and
the dataset that is operated upon. % TODO: citation
Iterative compilation and autotuning have been shown to help in these
cases by automating the process of tuning parameter values to match
individual execution environments. % TODO: citation
However, there have been few attempts to develop general mechanisms
for these techniques, and the time taken to develop ad-hoc autotuning
solutions and gather performance data is often prohibitively
expensive.

We believe that by embedding autotuning at the skeletal level, it is
possible to achieve performance with algorithmic skeletons that is
competitive with --- and in some cases, exceeds --- that of hand tuned
parallel implementations which traditionally came at the cost of many
man hours of work from expert programmers to develop.

Incorporating autotuning into algorithmic skeleton libraries has two
key benefits: first, it minimises development effort by requiring only
a modification to the skeleton implementation rather than to every
user program; and second, by targeting a library, it enables a broader
and more substantive range of performance data to be gathered than
with ad-hoc tuning of individual programs.


\subsection{Contributions}

The key contributions of this work are:

\begin{itemize}
\item The design and implementation of a generic toolset for
  autotuning: \emph{OmniTune} is a novel and extensible framework for
  collaborative autotuning of optimisation parameters across the life
  cycle of programs.
\item The integration of OmniTune with an established skeleton library
  for CPU and multi-GPU parallelism, SkelCL~\cite{Steuwer2011}. We
  extend SkelCL to provide autotuning for the selection of OpenCL
  workgroup size for Stencil skeletons.
\item An empirical evaluation of OmniTune across 7 different
  architectures, demonstrating that OmniTune achieves 92\% of the best
  possible performance, providing a median speedup of $5.65\times$
  over the best possible statically chosen workgroup size.
\end{itemize}

\section{Motivation}

\begin{figure}
\centering
\begin{subfigure}[h]{.45\columnwidth}
\centering
\includegraphics[width=1.0\columnwidth]{img/motivation_1}
\vspace{-1.5em} % Shrink vertical padding
\caption{}
\label{fig:motivation-1}
\end{subfigure}
~%
\begin{subfigure}[h]{.45\columnwidth}
\centering
\includegraphics[width=1.0\columnwidth]{img/motivation_2}
\vspace{-1.5em} % Shrink vertical padding
\caption{}
\label{fig:motivation-2}
\end{subfigure}
\caption{%
  The performance of different workgroup sizes for the same stencil
  program on two different devices: (\subref{fig:motivation-1}) Intel
  CPU, (\subref{fig:motivation-2}) NVIDIA GPU. Selecting an
  appropriate workgroup size depends on the execution device.%
}
\label{fig:motivation-arch}
\end{figure}

\begin{figure}
\centering
\begin{subfigure}[h]{.45\columnwidth}
\centering
\includegraphics[width=1.0\columnwidth]{img/motivation_3}
\vspace{-1.5em} % Shrink vertical padding
\caption{}
\label{fig:motivation-3}
\end{subfigure}
~%
\begin{subfigure}[h]{.45\columnwidth}
\centering
\includegraphics[width=1.0\columnwidth]{img/motivation_4}
\vspace{-1.5em} % Shrink vertical padding
\caption{}
\label{fig:motivation-4}
\end{subfigure}
\caption{%
  The performance of different workgroup sizes for two different
  stencil programs on the same execution device. Selecting an
  appropriate workgroup size depends on the program.%
}
\label{fig:motivation-prog}
\end{figure}


In this section we will briefly examine the performance impact of
selecting workgroup size for the SkelCL Stencil skeleton. A full
explanation of SkelCL and the workgroup size parameter space is given
Section~\ref{sec:omnitune-skelcl}.

SkelCL uses OpenCL to parallelise skeleton operations across many
threads. In OpenCL, multiple threads are grouped into
\emph{workgroups}. The shape and size of these groups is known to have
a big impact on performance. For the SkelCL stencil skeleton, the
selection of workgroup size presents a two dimensional parameter
space, consisting of a number of rows and columns ($w_r \times w_c$).
Measuring and plotting the runtime of stencil programs using different
workgroup sizes allows us to compare the performance of different
workgroup sizes for different combinations of architecture and
program. Figure~\ref{fig:motivation-arch} shows this performance
comparison for a single stencil program on two different devices,
demonstrating that a good choice of workgroup size is device
dependent. The optimisation space of the same stencil benchmark on
different devices is radically different --- not only does the optimal
workgroup size change between devices, but the performance of
suboptimal workgroup sizes is also dissimilar. The optimisation space
of~\ref{fig:motivation-1} has a grid-like structure, with clear
performance advantages of workgroup sizes at multiples of 8 for
$w_c$. A developer specifically targeting this device would learn to
select workgroup sizes following this pattern. This domain specific
knowledge clearly does not transfer to the device shown
in~\ref{fig:motivation-2}.

In Figure~~\ref{fig:motivation-prog}, we compare the performance of
two different stencil programs on the \emph{same} device, showing that
workgroup size choice is also program dependent. In each of these four
examples, the optimal workgroup size changes, as does the relative
performance of suboptimal parameters. The average speedup of the best
over the worst workgroup size is $37.0\times$, and the \emph{best}
average performance that can be achieved using a single fixed
workgroup size is only 63\% of the maximum.

SkelCL uses a fixed workgroup size. Since both the execution device
and the user-provided stencil code are not known until runtime,
selection of workgroup size should be made dynamically. To the best of
our knowledge, there are no existing systems for runtime autotuning of
arbitrary parameter values, and autotuners are generally developed
ad-hoc and on a per-case basis.


\section{The OmniTune Framework}\label{sec:autotune}

\begin{figure}
\centering
\includegraphics[width=.98\columnwidth]{img/omnitune-system-overview.pdf}
\caption{%
  OmniTune system architecture, showing the separate components and
  the one to many relationship between servers to client applications,
  and remotes to servers.%
  % \vspace{-2em}%
}
\label{fig:omnitune-system-overview}
\end{figure}

OmniTune is a novel framework for extensible, distributed autotuning
of parameter values at runtime using machine learning. It serves as a
generic platform for developing autotuning solutions, aiming to reduce
both the engineering time required to target new optimisation
parameters, and the time to deploy on new systems.

It emphasises collaborative, online learning of optimisation spaces. A
client-server architecture with clearly delineated separation of
concerns minimises the code footprint in client applications, enabling
quick re-purposing for autotuning targets. OmniTune provides a
lightweight interface for communication between each of the
components, and aims to strike a balance between offering a fully
featured environment for quickly implementing autotuning, while
providing enough flexibility to cater to a wide range of use
cases. First, we describe the overall structure of OmniTune and the
rationale for the design, followed by the interfaces and steps
necessary to apply OmniTune.


\subsection{System Architecture}

Common implementations of autotuning in the literature either embed
the autotuning logic within the each target application, % TODO:
                                % citation
or take a standalone approach in which the autotuner is a program
which must be invoked by the user to tune a target
application. % TODO: citation
Embedding the autotuner within each target application has the
advantage of providing ``always-on'' behaviour, but is infeasible for
complex systems in which the cost of building machine learning models
must be added to each program run. The standalone approach separates
the autotuning logic, at the expense of adding one additional step to
the build process. The approach taken in OmniTune aims to combine the
advantages of both techniques by implementing autotuning \emph{as a
  service}, in which a standalone autotuning server performs the heavy
lifting of managing training data and machine learning models, with a
minimal set of lightweight communication logic to be embedded in
target applications.

OmniTune is built around a three tier client-server model, shown in
Figure~\ref{fig:omnitune-system-overview}. The applications which are
to be autotuned are the \emph{clients}. These clients communicate with
a system-wide \emph{server}, which handles autotuning requests. The
server communicates and caches data sourced from a \emph{remote}
server, which maintains a global store of all autotuning data. There
is a many to one relationship between clients, servers, and remotes,
such that a single remote may handle connections to multiple servers,
which in turn may accept connections from multiple clients. This
design has two primary advantages: the first is that it decouples the
autotuning logic from that of the client program, allowing developers
to easily repurpose the autotuning framework to target additional
optimisation parameters without a significant development overhead for
the target applications; the second advantage is that this enables
collective tuning, in which training data gathered from a range of
devices can be accessed and added to by any OmniTune server.

The OmniTune framework is implemented as a set of Python classes,
which are inherited from or implemented to target specific autotuning
cases. The generic implementation of OmniTune's server and remote
components consists of 8987 lines of Python and MySQL code. No client
logic is provided, since that is use case dependent (See
Section~\ref{sec:omnitune-skelcl} for an example implementation for
SkelCL). Inter-process communication between client programs and the
server uses the D-Bus protocol. D-Bus is cross-platform, and bindings
are available for most major programming languages, allowing
flexibility for use with a range of clients. Communication between
servers and remotes uses TCP/IP (we used an Amazon Web Services
database instance for development).


\subsection{Autotuning Behaviour}

The goal of machine learning enabled autotuning is to build models
from empirical performance data of past programs to select parameter
values for new \emph{unseen} programs. Instead of an iterative process
of trial and improvement, parameter values are \emph{predicted}, by
building correlations between performance, and \emph{features}
(explanatory variables). The data used to build such models is called
training data. OmniTune supports autotuning using a separate offline
training phase, online training, or a mixture of both. For each
autotuning-capable machine, an OmniTune server acts as an intermediary
between training data and the client application, and hosts the
autotuning logic. On launch, a server requests the latest training
data from the remote, which it uses to build the relevant models for
performing prediction of optimisation parameter values. If additional
training data is gathered by the server, this can be uploaded to the
remote.

The autotuning technique is application-specific, and can depend on
the type of parameter being tuned (e.g. a binary flag or one or more
numeric values). The general pattern is that a client application will
request parameter values from an OmniTune server by sending it a set
of explanatory variables. The server will then use machine learning
models to form a prediction for the optimal parameter values and
return these. Crucially, there is a mechanism provided for the client
to \emph{refuse} parameter values. This functionality is provided for
cases where the predicted parameter values are in some way invalid and
do not lead to a functioning program. % TODO: citation

The server contains a library of machine learning tools to perform
parameter prediction, interfacing with the popular datamining software
suite Weka\footnote{\url{http://www.cs.waikato.ac.nz/ml/weka/}} using
a Java Native Interface. The provided tools include classifiers,
regressors, and a selection of meta-learning algorithms.

OmniTune servers may perform additional feature extraction of
explanatory variables supplied by incoming client requests. The reason
for performing feature extraction on the server as opposed to on the
client side is that this allows the results of expensive operations
(for example, analysing source code of target applications) to be
cached for use across the lifespan of client applications. The
contents of these local caches are periodically and asynchronously
synced with the remote to maintain a global store of lookup tables for
expensive operations.


\subsection{Interfaces}

Key design elements of OmniTune are the interfaces exposed by the
server and remote components. Figure~\ref{fig:omnitune-comms} shows an
example communication pattern between the three components of an
OmniTune system using these interfaces. In the example, a server first
requests training data from the remote. A client application then
performs a training phase in which it requests a set of parameters for
training, evaluates the performance of the parameters, and then
submits a measured value, which the server uses to update the
remote. After training, another client program requests a set of
parameters for performance, refuses them, and makes a new request.

\begin{figure}
\centering
\includegraphics[width=1.0\columnwidth]{img/omnitune-comms}
\caption{%
  An example communication pattern between OmniTune components,
  showing an offline training phase.%
}
\label{fig:omnitune-comms}
\end{figure}

\paragraph{Client-Server} An OmniTune server exposes a public
interface over D-Bus with four operations. Client applications invoke
these methods to request parameter values, submit new training
observations, and refuse suggested parameters:
%
\begin{itemize}
\item \textsc{Request}$(x) \to p$\\*Given explanatory variables $x$,
  request the parameter values $p$ which are expected to provide
  maximum performance.
\item \textsc{RequestTraining}$(x) \to p$\\*Given explanatory
  variables $x$, allow the server to select parameter values $p$ for
  evaluating their fitness.
\item \textsc{Submit}$(x, p, y)$\\*Submit an observed measurement of
  fitness $y$ for parameter values $p$, given explanatory variables
  $x$.
\item \textsc{Refuse}$(x, p)$\\*Refuse parameter values $p$, given a
  set of explanatory variables $x$. Once refused, those parameters are
  blacklisted and will not be returned by any subsequent calls to
  \textsc{Request()} or \textsc{RequestTraining()} for the same
  explanatory variables $x$.
\end{itemize}
%
% This set of operations enables the core functionality of an autotuner,
% while providing flexibility for the client to control how and when
% training data is collected.

\paragraph{Server-Remote} The role of the remote is to provide
bookkeeping of training data for machine learning. Remotes allow
shared access to data from multiple servers using a transactional
communication pattern, supported by two methods:
%
\begin{itemize}
\item \textsc{Push}$(\bf{x}, \bf{p}, \bf{y})$\\*Asynchronously submit
  training data as three lists: explanatory variables $\bf{x}$,
  parameter values $\bf{p}$, and observed outcomes $\bf{y}$.
\item \textsc{Pull}$() \to (\bf{x}, \bf{p}, \bf{y})$\\*Request
  training data as three lists: explanatory variables $\bf{x}$,
  parameter values $\bf{p}$, and observed outcomes $\bf{y}$.
\end{itemize}


\subsection{Extensibility}

\begin{figure}
\centering
\includegraphics[width=\columnwidth]{img/omnitune-system-flow.pdf}
\caption[Optimisation parameter selection with OmniTune]{%
  Predicting parameter values and collecting training data with
  OmniTune.%
}
\label{fig:omnitune-system-flow}
\end{figure}

To extend OmniTune to target an optimisation parameter, a developer
extends the server class to implement response handlers for the four
public interface operations, and then inserts client code into the
target application to call these operations. The implementation of
these response handlers and invoking client code dictates the type of
autotuning methods supported. Figure~\ref{fig:omnitune-system-flow}
shows the flow diagram for an example OmniTune implementation, in
which a training observation is recorded for every training parameter
requested. In the next Section, we will detail the steps required to
apply OmniTune to SkelCL.


\section{Integration of OmniTune with SkelCL}\label{sec:omnitune-skelcl}

In this section we demonstrate the practicality of OmniTune by
integrating the framework into an established algorithmic skeleton
library. Introduced in~\cite{Steuwer2011}, SkelCL allows users to
easily harness the power of GPUs and CPUs for data parallel computing,
offering a set of OpenCL implementations of data parallel skeletons in
an object oriented C++ library.

The goal of SkelCL is to enable the transition towards higher-level
programming of GPUs, without requiring users to be intimately
knowledgeable of the concepts unique to OpenCL programming, such as
the memory or execution model. SkelCL has been shown to reduce
programming effort for developing real applications through the use of
robust pattern implementations and automated memory
management. Skeletons are parameterised with user functions which are
compiled into OpenCL kernels for execution on device hardware. SkelCL
supports operations on one or two dimensional arrays of data, with the
Vector and Matrix container types transparently handling lazy
transfers between host and device memory, and supporting partitioning
for multi-GPU execution. SkelCL is freely available and distributed
under dual GPL and academic
licenses\footnote{\url{http://skelcl.uni-muenster.de}}.

\subsection{The Stencil Skeleton}

Stencils are patterns of computation which operate on uniform grids of
data, where the value of each cell is updated based on its current
value and the value of one or more neighbouring elements, called the
\emph{border region}. Figure~\ref{fig:stencil-img} shows the use of a
stencil to apply a Gaussian blur to an image. SkelCL provides a 2D
stencil skeleton which allows users to provide a function which
updates a cell's value, while SkelCL orchestrates the parallel
execution of this function across all cells~\cite{Steuwer2014a}.

The border region is described by a \emph{stencil shape}, which
defines an $i \times j$ rectangular region around each cell which is
used to update the cell value. Stencil shapes may be asymmetrical, and
are defined in terms of the number of cells in the border region to
the north, east, south, and west of each cell. Given a function $f$, a
stencil shape $S$, and an $n \times m$ matrix:
%
\begin{equation}
\scriptsize
% \begin{split}
\stencil \left( f, S,
\begin{bmatrix}
  x_{11} & \cdots & x_{1m} \\
  \vdots & \ddots & \vdots \\
  x_{n1} & \cdots & x_{nm}
\end{bmatrix} \right)
\to
\begin{bmatrix}
  z_{11} & \cdots & z_{1m} \\
  \vdots & \ddots & \vdots \\
  z_{n1} & \cdots & z_{nm}
\end{bmatrix}
% \end{split}
\end{equation}
%
where:
%
\begin{equation}
\scriptsize
z_{ij} = f \left(
\begin{bmatrix}
  x_{i-S_n,j-S_w} & \cdots & x_{i-S_n,j+S_e} \\
  \vdots & \ddots & \vdots \\
  x_{i+S_s,j-S_w} & \cdots & x_{i+S_s,j+S_e}
\end{bmatrix} \right)
\end{equation}
%
For border region elements outside the bounds of the matrix, values
are substituted from either a predefined padding value, or the value
of the nearest element within the matrix, depending on user
preference.

A popular usage of Stencil codes is for iterative problem solving,
whereby a stencil operation is repeated over a range of discrete time
steps $0 \le t \le t_{max}$, and $t \in \mathbb{Z}$. An iterative
stencil operation $g$ accepts a customising function $f$, a Stencil
shape $S$, and a matrix $M$ with initial values $M_{init}$. The value
of an iterative stencil can be defined recursively as:
%
\begin{equation}
\scriptsize
g(f, S, M, t) =
\begin{cases}
  \stencil \left( f, S, g(f, S, M, t-1) \right),& \text{if } t \geq 1\\
  M_{init}, & \text{otherwise}
\end{cases}
\end{equation}
%
Examples of iterative stencils include cellular automata and partial
differential equation solvers. % Another extension of the stencil
% operation accepts an ordered list of customising functions which are
% applied sequentially for each iteration. This has applications for
% multi-stage stencil operations such as Canny Edge Detection, in
% which four distinct stencil operations are performed as a sequence.

In the implementation of the SkelCL stencil skeleton, each element in
the matrix is mapped to a unique thread (known as a \emph{work item}
in OpenCL) which applies the user-specified function. The work items
are then divided into \emph{workgroups} for execution on the target
hardware. Each work-item reads the value of its corresponding matrix
element and the surrounding elements defined by the border
region. Since the border regions of neighbouring elements overlap, the
value of all elements within a workgroup are copied into a
\emph{tile}, allocated as a contiguous region of the fast, but small
local memory. As local memory access times are much faster than that
of global device memory, this greatly reduces the latency of the
border region reads performed by each work item. Changing the size of
workgroups thus affects the amount of local memory required for each
workgroup, and in turn affects the number of workgroups which may be
simultaneously active on the device. While the user defines the data
size and type, the shape of the border region, and the function being
applied to each element, it is the responsibility of the SkelCL
stencil implementation to select an appropriate workgroup size to use.

\subsection{Optimisation Parameters}\label{subsec:op-params}

SkelCL stencil kernels are parameterised by a workgroup size $w$,
which consists of two integer values to denote the number of rows and
columns in a workgroup. The space of optimisation parameter values is
subject to hard constraints, and these constraints cannot conveniently
be statically determined. Contributing factors are architectural
limitations, kernel constraints, and refused parameters.  Each OpenCL
device imposes a maximum workgroup size which can be statically
checked. These are defined by archiectural limitations of how code is
mapped to the underlying execution hardware. Typical values are powers
of two, e.g.\ 1024, 4096, 8192. At runtime, once an OpenCL program has
been compiled to a kernel, users can query the maximum workgroup size
supported by that particular kernel dynamically. This value cannot
easily be obtained statically as there is no mechanism to determine
the maximum workgroup size for a given source code and device without
first compiling it, which in OpenCL does not occur until runtime.

Factors which affect a kernel's maximum workgroup size include the
number registers required for a kernel, and the available number of
SIMD execution units for each type of instructions in a kernel. In
addition to satisfying the constraints of the device and kernel, not
all points in the workgroup size optimisation space are guaranteed to
provide working programs. A \emph{refused parameter} is a workgroup
size which satisfies the kernel and architectural constraints, yet
causes a \texttt{CL\_OUT\_OF\_RESOURCES} error to be thrown when the
kernel is enqueued. Note that in many OpenCL implementations, this
error type acts as a generic placeholder and may not necessarily
indicate that the underlying cause of the error was due to finite
resources constraints. We define a \emph{legal} workgroup size as one
which, for a given \emph{scenario} (a combination of program, device,
and dataset), satisfies the architectural and kernel constraints, and
is not refused. The subset of all possible workgroup sizes
$W_{legal}(s) \subset W$ that are legal for a given scenario $s$ is
then:
%
\begin{equation}
  W_{legal}(s) = \left\{w | w \in W, w < W_{\max}(s) \right\} - W_{refused}(s)
\end{equation}
%
Where $W_{\max}(s)$ can be determined at runtime prior to the kernels
execution, but the set $W_{refused}(s)$ can only be determined
experimentally.


\begin{figure}
\centering
\includegraphics[width=.98\columnwidth]{img/lena-stencil.pdf}
\caption{%
  Application of a Gaussian blur stencil operation to an image, with a
  border region of radius 1. In a Gaussian blur, pixel values are
  interpolated with neighbouring pixels, producing a smoothed effect.%
}
\label{fig:stencil-img}
\end{figure}


The \emph{oracle} workgroup size $\Omega(s) \in W_{legal}(s)$ of a
scenario $s$ is the $w$ value which provides the lowest mean
runtime. The relative performance $p(s,w)$ of a particular workgroup
against the maximum available performance for that scenario, within
the range $0 \le p(s,w) \le 1$, is the ratio of the runtime of a
program with workgroup size $w$ over the oracle workgroup size
$\Omega(s)$. For a given workgroup size, the average performance
$\bar{p}(w)$ across a set of scenarios $S$ can be found using the
geometric mean of performance relative to the oracle:
%
\begin{equation}
  \bar{p}(w) =
  \left(
    \prod_{s \in S} p(s, w)
  \right)^{1/|S|}
\end{equation}
%
The \emph{baseline} workgroup size $\bar{w}$ is the value which
provides the best average case performance across a set of
scenarios. Such a baseline value represents the \emph{best} possible
performance which can be achieved using a single, statically chosen
workgroup size. By defining $W_{safe} \in W$ as the intersection of
legal workgroup sizes, the baseline can be found using:
%
\begin{align}
W_{safe} &= \cap \left\{ W_{legal}(s) | s \in S \right\}\\
\bar{w} &= \argmax_{w \in W_{safe}} \bar{p}(w)
\end{align}


\subsection{Machine Learning}

The optimisation space presented by the workgroup size of OpenCL
kernels is large, complex, and non-linear. The challenge is to design
a system which, given a set of prior observations of the empirical
performance of stencil codes with different workgroup sizes, predict
workgroup sizes for \emph{unseen} stencils which will maximise the
performance. Successfully applying machine learning requires plentiful
training data, the careful selection of features, and appropriate
machine learning methods. For the purpose of this work we use a
\emph{classification} approach, in which a classifier automatically
correlates patterns between explanatory variables (features) and the
workgroup sizes which provide optimal performance. The classifier used
is the popular J48 Decision Tree~\cite{Han2011}, chosen due to its low
classification cost and ability to efficiently handle large
dimensionality training data.

For each scenario, a total of 102 explanatory variables are extracted
to capture information about the device, program, and dataset. Device
features encode the device type (e.g. CPU or GPU, integrated or
external, connection bus), properties about the host (e.g.\ system
memory, maximum clock frequency), and numerous properties about the
execution device (e.g.\ number of compute units, local memory size,
global caches). Program features include per-instruction type
densities, the total number of basic blocks, and the total instruction
count. They are extracted using static instruction count passes over
an LLVM IR compiled version of the user stencil
implementation. Compilation to bitcode is a relatively expensive task,
so lookup tables are used to cache repeated uses of the same stencil
codes, identified by the source code checksum. Dataset features
include the data type and dimensions of the SkelCL container type.

To collect training data, we run multiple iterations of a stencil
program to enumerate the workgroup size optimisation space, and use
the OpenCL's Profiling API to record stencil kernel execution times in
the client application, which are then submitted to the OmniTune
server. The \textsc{RequestTraining}$(x)$ server interface returns a
workgroup size with a randomly selected even number of rows and
columns up to the maximum allowed:
$\left\{ 2x \in \mathbb{Z} | 1 \le x < \frac{W_{max}(s)}{2} \right\}$.

A parameterised template substitution engine is used to generate
synthetic stencil applications for gathering performance
data. Stencils templates are parameterised with a border region size
and \emph{complexity}, a simple metric which broadly dictates the
number of operations in a given stencil code.

Once the performance of a particular workgroup size for a scenario is
assessed, the set of features describing the scenario is labelled with
the oracle workgroup size, which is used as the target class to train
a classifier. A classifier uses these performance results to make
predictions of the oracle workgroup size for new sets of features, by
predicting a workgroup size from the set of oracle workgroup sizes of
the training data.
% TODO: reword? ^^^

This approach presents the problem that after training, there is no
guarantee that the set of workgroup sizes which may be predicted is
within the set of legal workgroup sizes for future scenarios. This may
result in a classifier predicting a workgroup size which is not legal
for a scenario, $w \not\in W_{legal}(s)$, either because it exceeds
$W_{\max}(s)$, or because the parameter is refused. If this occurs, a
\emph{nearest neighbour} approach is used to select the workgroup size
$w$ which is expected to be legal and has the lowest Euclidian
distance to the predicted value $c$:
%
\begin{equation}
  w = \underset{w \in W_{legal(s)}}{\argmin} \sqrt{\left(c_r - w_r\right)^2 + \left(c_c - w_c\right)^2}
\end{equation}
%
This process of selecting alternative parameters will iterate until a
legal parameter is found.

\subsection{Implementation}

The OmniTune framework consists of 8987 lines of Python and MySQL
code. A further 976 lines are required for the SkelCL frontend to
implement the server response handlers and database backend. By
design, the client-server model minimises the impact of number of
modifications that are required to enable autotuning in client
applications. The only modification required to SkelCL is to replace
the hardcoded values for workgroup size with a subroutine to request a
workgroup size from the OmniTune server over a D-Bus connection. To
use the system, a user must download a copy of SkelCL modified with
the OmniTune functionality, and start a local OmniTune server
instance. A configuration file is used to determine the domain address
and authentication details of the remote server. On first launch, the
OmniTune server will fetch the latest training data from the remote.


\section{Experimental Setup}

This section describes an exhaustive enumeration of the workgroup size
optimisation space for 429 combinations of architecture, program, and
dataset. It contains the methodology used to collect empirical
performance data on which to base performance comparisons of different
workgroup sizes, and the steps necessary to obtain repeatable results.

A full enumeration of the workgroup size optimisation spaces was
performed across synthetically generated benchmarks and four reference
stencil benchmarks: Canny Edge Detection, Conway's Game of Life, Heat
Equation, and Gaussian Blur. Performance data was collected from 7
experimental platforms, comprising 4 GPU devices: AMD Tahiti 7970,
Nvidia GTX 590, Nvidia GTX 690, Nvidia GTX TITAN; and 3 CPU devices:
Intel i5-2430M, Intel i5-4570, i7-3820. Each platform was unloaded,
frequency governors disabled, and benchmark processes set to the
highest priority available to the task scheduler. Datasets and
programs were stored in an in-memory file system. For each program,
dataset sizes of size $512\times512$, $1024\times1024$,
$2048\times2048$, and $4096\times4096$ were used. A minimum of 30
samples were recorded for each scenario and workgroup size.

Program behavior was validated by comparing program output against a
gold standard output collected by executing each of the real-world
benchmarks programs using the baseline workgroup size. The output of
real-world benchmarks with other workgroup sizes is compared to this
gold standard output to test for correct program execution.


\section{Evaluation}\label{sec:evaluation}

This section evaluates the performance of OmniTune when tasked with
selecting workgroup sizes for SkelCL stencil codes. First we discuss
measurement noise present in the experimental results, and the methods
used to accommodate for it. Then we examine the observed effect that
workgroup size has on the performance of SkelCL stencils. The
prediction quality of OmniTune is scrutinised for portability across
programs, devices, and datasets.

The experimental results consist of measured runtimes for a set of
\emph{test cases}, collected using the methodology explained in the
previous section. Each test case $\tau_i$ consists of a scenario,
workgroup size pair $\tau_i = (s_i,w_i)$, and is associated with a
\emph{sample} of observed runtimes from multiple runs of the
program. A total of 269813 evaluated, which represents an exhaustive
enumeration of the workgroup size optimisation space for 429
scenarios. For each scenario, runtimes for an average of 629 (max
7260) unique workgroup sizes were measured. The average sample size of
runtimes for each test case is 83 (min 33, total 16917118).


\subsection{Statistical Soundness}

\begin{figure}
\begin{subfigure}[h]{.32\columnwidth}
\centering
\includegraphics[width=\textwidth]{img/runtimes_histogram_1}
\vspace{-1.5em} % Shrink vertical padding
\caption{}
\label{fig:runtimes-histogram-1}
\end{subfigure}
~%
\begin{subfigure}[h]{.32\columnwidth}
\centering
\includegraphics[width=\textwidth]{img/runtimes_histogram_2}
\vspace{-1.5em} % Shrink vertical padding
\caption{}
\label{fig:runtimes-histogram-2}
\end{subfigure}
~%
\begin{subfigure}[h]{.32\columnwidth}
\centering
\includegraphics[width=\textwidth]{img/runtimes_histogram_3}
\vspace{-1.5em} % Shrink vertical padding
\caption{}
\label{fig:runtimes-histogram-3}
\end{subfigure}
\\
\begin{subfigure}[h]{.32\columnwidth}
\centering
\includegraphics[width=\textwidth]{img/runtimes_histogram_4}
\vspace{-1.5em} % Shrink vertical padding
\caption{}
\label{fig:runtimes-histogram-4}
\end{subfigure}
~%
\begin{subfigure}[h]{.32\columnwidth}
\centering
\includegraphics[width=\textwidth]{img/runtimes_histogram_5}
\vspace{-1.5em} % Shrink vertical padding
\caption{}
\label{fig:runtimes-histogram-5}
\end{subfigure}
~%
\begin{subfigure}[h]{.32\columnwidth}
\centering
\includegraphics[width=\textwidth]{img/runtimes_histogram_6}
\vspace{-1.5em} % Shrink vertical padding
\caption{}
\label{fig:runtimes-histogram-6}
\end{subfigure}
\\
\begin{subfigure}[h]{.32\columnwidth}
\centering
\includegraphics[width=\textwidth]{img/runtimes_histogram_7}
\vspace{-1.5em} % Shrink vertical padding
\caption{}
\label{fig:runtimes-histogram-7}
\end{subfigure}
~%
\begin{subfigure}[h]{.32\columnwidth}
\centering
\includegraphics[width=\textwidth]{img/runtimes_histogram_8}
\vspace{-1.5em} % Shrink vertical padding
\caption{}
\label{fig:runtimes-histogram-8}
\end{subfigure}
~%
\begin{subfigure}[h]{.32\columnwidth}
\centering
\includegraphics[width=\textwidth]{img/runtimes_histogram_9}
\vspace{-1.5em} % Shrink vertical padding
\caption{}
\label{fig:runtimes-histogram-9}
\end{subfigure}
\caption[Distribution of stencil code runtimes]{%
  Distribution of runtime samples for test cases from three
  devices. Each plot contains a 35-bin histogram of 1000 samples, and
  a fitted kernel density estimate with bandwidth 0.3. The sample mean
  is shown as a vertical dashed line. The top row are from the Intel
  i5-4570, the second row from the Nvidia GTX 590, and the third row
  from the AMD Tahiti 7970. In some of the plots, the distribution of
  runtimes is bi- or multi-modal, and skewed to the lower end of the
  runtimes range.%
}
\label{fig:runtime-histograms}
\end{figure}

The complex interaction between processes competing for the finite
resources of a system introduces many sources for noise in program
runtime measurements. Before making any judgements about the relative
performance of optimisation configurations, we must establish the
level of noise present in these measurements. To do this, we evaluate
the distribution of runtimes for a randomly selected 1000 test cases,
recording 1000 runtime observations for each. We can then produce
fine-grained histograms of runtimes for individual test
cases. Figure~\ref{fig:runtime-histograms} shows an example nine of
these, for test cases from three devices. The plots show that the
distribution of runtimes is not always Gaussian; rather, it is
sometimes multimodal, and generally skewed to the lower end of the
runtime range, with a long ``tail'' to the right. This fits our
intuition that programs have a hard \emph{minimum} runtime enforced by
the time taken to execute the instructions of a program, and that
noise introduced to the system extends this runtime. For example,
preempting an OpenCL process on a CPU so that another process may run
may cause the very long tail visible in
Figure~\ref{fig:runtimes-histogram-1}.

The central limit theorem allows the assumption of an underlying
Gaussian distribution for samples of size $\ge 30$~\cite{Georges2007}.
Given our minimum sample size of 33, we can use 95\% confidence
intervals to provide statistical confidence that the arithmetic mean
of observed runtimes with respect to the true mean. As the number or
samples increases, we should expect the size of the confidence
interval to shrink. This is illustrated in Figure~\ref{fig:ci-trends},
which plots the average size of 95\% confidence intervals across the
1000 test cases, normalised to their respective means, as a function
of sample size. It shows the diminishing returns that increasing
sample size provides. For example, increasing the sample count from 10
to 30 results in an approximate 50\% reduction in confidence interval
size. Increasing the sample size from 30 to 50 results in only a 25\%
reduction.

\begin{figure}
\centering
\includegraphics[width=\columnwidth]{img/ci_trend}
\caption[Confidence interval size vs.\ sample count]{%
  Ratio of confidence interval to mean as a function of sample
  count. Two dashed lines indicate the confidence intervals at the
  minimum (3.7\%) and mean (2.5\%) number of samples used in the
  experimental dataset.%
}
\label{fig:ci-trends}
\end{figure}

By comparing the average confidence interval at different sample
counts against the full experiment results of 269813 test cases, we
can assert with 95\% confidence that the true mean for each test case
is within 2.5\% of the sample mean (given the average number of
samples per test case), or 3.7\% in the worst case (at the minimum
number of samples). Since the differences between baseline and optimal
workgroup sizes is often well in excess of 100\%, there is no overlap
of confidence intervals between competing workgroup sizes.


\subsection{Workgroup Size Optimisation Space}

In this subsection we explore the impact that the workgroup size
optimisation space has on the performance of SkelCL stencil programs.

\subsubsection{Oracle Workgroup Sizes}

\begin{figure}
\centering
\includegraphics[width=\columnwidth]{img/num_params_oracle.pdf}
\caption[Oracle accuracy vs.\ number of workgroup sizes]{%
  Accuracy compared to the oracle as a function of the number of
  workgroup sizes used. The best accuracy that can be achieved using a
  single statically chosen workgroup size is 15\%. Achieving 50\%
  oracle accuracy requires a minimum of 14 distinct workgroup sizes.%
}
\label{fig:oracle-accuracy}
\end{figure}

For each scenario $s$, the oracle workgroup size $\Omega(s)$ is the
workgroup size which resulted in the lowest mean runtime. If the
performance of stencils were independent of workgroup size, we would
expect that the oracle workgroup size would remain constant across all
scenarios $s \in S$. Instead, we find that there are 135 unique oracle
workgroup sizes, with 31.5\% of scenarios having a unique workgroup
size. This demonstrates the difficult in attempting to tune for
\emph{optimal} parameter values, since 14 distinct workgroup sizes are
needed to achieve just 50\% of the oracle accuracy
(Figure~\ref{fig:oracle-accuracy}), although it is important to make
the distinction that oracle \emph{accuracy} and \emph{performance} are
not equivalent. The workgroup size which is most frequently optimal is
$w_{(64 \times 4)}$, which is optimal for 15\% of scenarios. Note that
this is not adequate to use as a baseline for static tuning, as it
does not respect legality constraints, that is
$w_{(64 \times 4)} \not\in W_{safe}$.


\paragraph{Maximum workgroup sizes}

We define the \emph{coverage} of a workgroup size to be the ratio
$0 \le x \le 1$ between the number of scenarios for which the
workgroup size is less than $W_{\max}(s)$, normalised to the total
number of scenarios. A coverage of 1 implies a workgroup size which is
always legal for all combinations of stencil and architecture. A
workgroup size with a coverage of 0 is never legal. Note that since
$W_{\max}(s)$ defines a hard limit for a given $s$, if statically
selecting a workgroup size, one must limit the optimisation space to
the smallest $W_{\max}(s)$ value, i.e.\ only the workgroup sizes with
a coverage of 1. The observed $W_{\max}(s)$ values range from
256--8192, which results in up to a 97\% reduction in the size of the
optimisation space when $W_{\max}(s) = 8192$, even though only 14\% of
scenarios have the minimum value of $W_{\max}(s) = 256$.

\paragraph{Refused Parameters}

In addition to the hard constraints imposed by the maximum workgroup
size, there are also refused parameters, which are workgroup sizes
which are rejected by the OpenCL runtime and do not provide a
functioning program. Of the 8504 unique workgroup sizes tested, 11.4\%
were refused in one or more test cases. An average of 5.5\% of all
test cases lead to refused parameters. For a workgroup size to be
refused, it must satisfy the architectural and program-specific
constraints which are exposed by OpenCL, but still lead to a
\texttt{CL\_OUT\_OF\_RESOURCES} error when the kernel is
enqueued. While uncommon, a refused parameter is an obvious
inconvenience to the user, as one would expect that any workgroup size
within the specified maximum should behave \emph{correctly}, if not
efficiently. For now, it is imperative that any autotuning system is
capable of adapting to these refused parameters by suggesting
alternatives when they occur.


\subsubsection{Baseline Parameter}

The baseline parameter $\bar{w}$ is the workgroup size which provides
the best overall performance while being legal for all scenarios. It
is the workgroup size $w \in W_{safe}$ which maximises the output of
the performance function $\bar{p}(w)$. As shown in
Table~\ref{tab:highest-legality}, only a \emph{single} workgroup size
$w_{(4 \times 4)}$ from the set of experimental results is found to
have a legality of 100\%, suggesting that an adaptive approach to
setting workgroup size is necessary not just for the sake of
maximising performance, but also for guaranteeing program correctness.

The utility of the baseline parameter is that it represents the best
performance that can be achieved through static tuning of the
workgroup size parameter. We can evaluate the performance of
suboptimal workgroup sizes by calculating the geometric mean of their
\emph{performance} for a particular scenario $p(s, w)$ across all
scenarios, $\bar{p}(w)$. The baseline parameter $\bar{p}(\bar{w})$
achieves only 24\% of the available performance.

Figure~\ref{fig:speedups} shows the speedup of the oracle workgroup
size over the baseline parameter $w_{(4 \times 4)}$ for all
scenarios. If we assume that sufficiently pragmatic developer with
enough time would eventually find this static optimal, then this
provides a reasonable comparison for calculating speedups of an
autotuner for workgroup size. Comparing the runtime of workgroup sizes
relative to the oracle allows us to calculate upper bounds on the
possible performance which can be expected from autotuning.


\subsubsection{Speedup Upper Bounds}

\begin{figure}
  \includegraphics[width=\columnwidth]{img/max_speedups}
  \caption[Workgroup size speedups]{%
    Speedup of oracle workgroup size over: the worst performing
    workgroup size for each scenario (\emph{Max}), the statically
    chosen workgroup size that provides the best overall performance
    ($w_{(4 \times 4)}$), and the human expert selected parameter
    ($w_{(32 \times 4)}$). Note that the human expert parameter is not
    legal for all scenarios.%
  }
\label{fig:speedups}
\end{figure}

For a given scenario $s$, the ratio of the workgroups sizes from
$W_{legal}(s)$ which provide the longest and shortest mean runtimes is
used to calculate an upper bound for the possible performance
influence of workgroup size:
%
\begin{equation}
r_{max}(s) = r(s, \argmax_{w \in W_{legal}(s)} t(s,w), \Omega(s))
\end{equation}
%
When applied to each scenario $s \in S$ of the experimental results,
we find the average of speedup upper bounds to be $15.14\times$ (min
$1.03\times$, max $207.72\times$). This demonstrates the importance of
tuning stencil workgroup sizes --- if chosen incorrectly, the runtime
of stencil programs can be extended by up to $207.72\times$. Note too
that for 5 of the scenarios, the speedup of the best over worst
workgroup sizes is $\le 5\%$. For these scenarios, there is little
benefit to autotuning; however, this represents only 1.1\% of the
tested scenarios. For 50\% of the scenarios, the speedup of the best
over worst workgroup sizes is $\ge 6.19\times$.


\subsubsection{Human Expert}

In the original implementation of the SkelCL stencil
skeleton~\cite{Breuer2013}, \citeauthor{Breuer2013} selected a
workgroup size of $w_{(32 \times 4)}$ in an evaluation of 4 stencil
operations on a Tesla S1070 system. We can use this as an additional
parameter to compare the relative performance of workgroup sizes
against. However, the $w_{(32 \times 4)}$ workgroup size is invalid
for 2.6\% of scenarios, as it is refused in 11 test cases. By device,
those are: 3 on the GTX 690, 6 on the i5-2430M, and 2 on the i5-4570.
For the scenarios where $w_{(32 \times 4)}$ \emph{is} legal, the human
expert chosen workgroup size achieves an impressive geometric mean of
79.2\% of the oracle performance. The average speedup of oracle
workgroup sizes over the workgroup size selected by a human expert is
$1.37\times$ (min $1.0\times$, max $5.17\times$). Since the workgroup
size selected by the human expert is not legal for all scenarios, we
will examine the effectiveness of heuristics for tuning workgroup
size.


\subsubsection{Heuristics}

In this subsection we will consider the effectiveness of instead
selecting workgroup size using two types of heuristics. The first,
using the maximum workgroup size returned by the OpenCL device and
kernel APIs to select the workgroup size adaptively. The second, using
per-device heuristics, in which the workgroup size is selected based
on the specific architecture that a stencil is operating on.


\paragraph{Using maximum legal size}

\begin{figure}
  \begin{subfigure}[h]{\columnwidth}
    \centering
    \includegraphics[width=\columnwidth]{img/performance_max_wgsize}
    \vspace{-1.5em} % Shrink vertical padding
    \caption{}
    \label{fig:performance-max-wgsize}
  \end{subfigure}
  \\
  \begin{subfigure}[h]{.48\columnwidth}
    \centering
    \includegraphics[width=\columnwidth]{img/performance_max_c}
    \vspace{-1.5em} % Shrink vertical padding
    \caption{}
    \label{fig:performance-wg-c}
  \end{subfigure}
  ~%
  \begin{subfigure}[h]{.48\columnwidth}
    \centering
    \includegraphics[width=\columnwidth]{img/performance_max_r}
    \vspace{-1.5em} % Shrink vertical padding
    \caption{}
    \label{fig:performance-wg-r}
  \end{subfigure}

  \caption[Workgroup size performances vs.\ size]{%
    Comparing workgroup performance relative to the oracle as function
    of: (\subref{fig:performance-max-wgsize})~maximum legal size,
    (\subref{fig:performance-wg-c})~number of columns, and
    (\subref{fig:performance-wg-r})~ number of rows. Each workgroup
    size is normalised to the maximum allowed for that scenario, $W_{\max}(s)$.%
  }
  \label{fig:performance-wgsizes}
\end{figure}

\begin{figure}
  \begin{subfigure}[h]{\columnwidth}
    \centering
    \includegraphics[width=\columnwidth]{img/performance_kernels.pdf}
    \vspace{-1.5em} % Shrink vertical padding
    \caption{}
    \label{fig:performance-kernels}
  \end{subfigure}
  \\
  \begin{subfigure}[h]{.48\columnwidth}
    \centering
    \includegraphics[width=\columnwidth]{img/performance_devices.pdf}
    \vspace{-1.5em} % Shrink vertical padding
    \caption{}
    \label{fig:performance-devices}
  \end{subfigure}
  ~%
  \begin{subfigure}[h]{.48\columnwidth}
    \centering
    \includegraphics[width=\columnwidth]{img/performance_datasets.pdf}
    \vspace{-1.5em} % Shrink vertical padding
    \caption{}
    \label{fig:performance-datasets}
  \end{subfigure}
  \caption[Workgroup size performances across device, kernel, and dataset]{%
    Performance relative to the oracle of workgroup sizes across all
    test cases, grouped by: (\subref{fig:performance-kernels})~kernels,
    (\subref{fig:performance-devices})~devices, and
    (\subref{fig:performance-datasets})~datasets.%
  }
  \label{fig:performances}
\end{figure}

A common approach taken by OpenCL developers is to set the workgroup
size for a kernel based on the maximum legal workgroup size queried
from the OpenCL APIs. For example, to set the size of 2D workgroup, a
developer the square root of the (scalar) maximum wgsize to set the
number of columns and rows (since $w_c \cdot w_r$ must be
$< W_{\max}(s)$). To consider the effectiveness of this approach, we
group the workgroup size performances based on the ratio of the
maximum allowed for each scenario. We can also perform this for each
of the two dimensions --- rows and columns --- of the stencil
workgroup size.

Figure~\ref{fig:performance-wgsizes} shows the distribution of
runtimes when grouped this way, demonstrating that the performance of
(legal) workgroup sizes are not correlated with the maximum workgroup
sizes $W_{\max}(s)$. However, when considering individual components,
we observe that the best median workgroup size performances are
achieved with a number of columns that is between 10\% and 20\% of the
maximum, and a number of rows that is between 0\% and 10\% of the
maximum.


\paragraph{Per-device workgroup sizes}

\begin{table}
  \scriptsize
  \centering
  \begin{tabular}{lllp{1cm}p{1cm}}
    \toprule
    Device &         Oracle & Legality & Perf.\ min & Perf.\ avg. \\
    \midrule
    AMD Tahiti 7970 &   $48\times 4$ &      1.0 &       0.54 &        0.91 \\
    Intel i5-2430M &  $64\times 16$ &      0.8 &       0.37 &        0.91 \\
    Intel i5-4570 &   $88\times 8$ &     0.89 &       0.33 &        0.89 \\
    Intel i7-3820 &  $40\times 24$ &     0.95 &       0.76 &        0.97 \\
    NVIDIA GTX 590 &  $12\times 2$ &      0.8 &        0.2 &         0.9 \\
    NVIDIA GTX 690 &   $64\times 4$ &     0.93 &       0.32 &        0.84 \\
    NVIDIA GTX TITAN &   $64\times 4$ &      1.0 &       0.26 &        0.81 \\
    \textbf{CPUs} &   $88\times 8$ &     0.88 &       0.33 &        0.91 \\
    \textbf{GPUs} &   $64\times 4$ &     0.76 &       0.26 &        0.86 \\
    \bottomrule
  \end{tabular}
  \caption[Performance of tuning with a per-device heuristic]{%
    Selecting workgroup size using a per-device heuristic. The mode
    optimal workgroup size for each device type $\bar{w}$ is evaluated
    based on legality, and relative performance to the oracle (minimum
    and average) when legal.%
  }
  \label{tab:heuristic-dev}
\end{table}

One possible technique to selecting workgroup size is to tune
particular values for each targeted execution device. This approach is
sometimes adopted for cases with particularly high requirements for
performance on a single platform, so it produces an interesting
contrast to evaluating a machine learning approach, which attempts to
predict workgroup sizes for unseen platforms without the need for an
expensive exploration phase on the new platform.

Figure~\ref{fig:performances} shows the performance of workgroup sizes
relative to the oracle across scenarios grouped by: kernel, device,
and dataset. When grouped like this, a number of observations can
made. First is that not all of the kernels are sensitive to tuning
workgroup size to the same degree. The \emph{sobel} kernel has the
lowest median performance, indicating that it is the most profitable
to tune, while the \emph{threshold} kernel is the least
profitable. Similarly, the Intel i7-3820 is far less amenable to
tuning than the other devices, while the Intel i5-4570 is the most
sensitive to the workgroup size parameter. Such variances in the
distributions of workgroup sizes suggest that properties underlying
the architecture, kernel, and dataset all contribute towards the
proper selection of workgroup size.

To test the performance of a per-device heuristic for selecting
workgroup size, we group the scenarios by device, and compare the
relative performance of all workgroup sizes for each group of
scenarios. The most frequently optimal workgroup size $\bar{w}$ for
each device is selected, and the legality and performance of each
scenario using that workgroup size is evaluated.
Table~\ref{tab:heuristic-dev} shows the results of this evaluation.
The GTX 690 and GTX TITAN share the same $\bar{w}_{(64 \times 4)}$
value, while every other device has a unique optimum. The general case
performance of these per-device parameters is very good, although
legality is only assured for the GTX TITAN and AMD 7970 (which did not
refuse any parameters). However, the worst case performance of
per-device workgroup sizes is poor for all except the i7-3820 (which
is least sensitive to tuning), suggesting that device alone is not
capable of reliably informing the choice of workgroup size.


\subsubsection{Summary}

In this subsection we have explored the performance impact of the
workgroup size optimisation space. By comparing the relative
performance of an average of 629 workgroup sizes for each of 429
scenarios, the following conclusions can be drawn:

\begin{enumerate}
\item The performance of a workgroup size for a particular scenario
  depends properties of the hardware, software, and dataset.
\item The performance gap between the best and workgroup sizes for a
  particular combination of hardware, software, and dataset is up to
  $207.72\times$.
\item Not all workgroup sizes are legal, and the space of legal
  workgroup sizes cannot statically be determined. Adaptive tuning of
  workgroup size is required to ensure reliable performance.
\item Differing scenarios have wildly different optimal workgroup
  sizes, and the best performance can be achieved using static tuning
  is optimal for only 15\% of scenarios.
\end{enumerate}
%
In the following subsection, we will evaluate the performance of
OmniTune for selecting workgroup sizes.


\subsection{Autotuning Workgroup Sizes}

In this Section we evaluate the performance of OmniTune for predicting
workgroup sizes. To perform an evaluation, we partition the
experimental data into a training set and a test set:
$S_{training} \subset S$ and $S_{testing} = S - S_{training}$. A set
of labelled training data $D_{training}$ is derived from
$S_{training}$, and the prediction quality is testing using the
validation set $D_{testing}$ derived from $S_{training}$. We use
multiple approaches to partitioning test data to evaluate the
prediction quality under different scenarios. The processes for
generating validation sets are:
%
\begin{itemize}
\item 10-fold --- shuffle the set of all data and divide into 10
  validation sets, each containing 10\% of the data. This process is
  repeated for 10 rounds, resulting in 100 validations of 10
  permutations of the data.
\item Synthetic --- divide the training data such that it consists
  solely of data collected from synthetic benchmarks, and use data
  collected from real-world benchmarks to test.
\item Device --- partition the training data into $n$ sets, one for
  each device. Use $n-1$ sets for training, repeating until every
  partition has been used for testing once.
\item Kernel --- partition the training data into $n$ sets, one for
  each kernel. Use $n-1$ sets for training, repeating until every
  partition has been used for testing once.
\item Dataset --- partition the training data into $n$ sets, one for
  each type of dataset. Use $n-1$ sets for training, repeating until
  every partition has been used for testing once.
\end{itemize}
%
Training data consists of pairs of feature vectors $f(s)$ and oracle
workgroup sizes $\Omega(s)$:
%
\begin{equation}
  D_{training} = \left\{ (f(s),\Omega(s)) | s \in S_{training} \right\}
\end{equation}
%
Testing data are not labelled with oracle workgroup sizes:
%
\begin{equation}
  D_{testing} = \left\{ f(s) | s \in S_{testing} \right\}
\end{equation}
%
Workgroup sizes are predicted for each scenario in the testing set,
and the quality of the predicted workgroup size is evaluated using the
following metrics:
%
\begin{itemize}
\item accuracy (binary) --- whether the predicted workgroup size is
  the true oracle, $p(f(s)) = \Omega(s)$.
\item validity (binary) --- whether the classifier predicted a
  workgroup size which satisfies the workgroup size constraints
  constraints, $p(f(s)) < W_{\max}(s)$.
\item refused (binary) --- whether the classifier predicted a
  workgroup size which is refused, $p(f(s)) \in W_{refused}(s)$.
\item performance (real) --- the relative performance of the predicted
  workgroup size relative to the oracle,
  $0 \le r(p(f(s)), \Omega(s)) \le 1$.
\item speedups (real) --- the relative performance of the predicted
  workgroup size relative to the baseline workgroup size
  $w_{(4 \times 4)}$, and human expert workgroup size
  $w_{(32 \times 4)}$ (where applicable).
\item time (real) --- the round trip time required to make the prediction,
  as measured by the OmniTune client. This includes classification
  time and inter-process communication overheads between the client
  and server.
\end{itemize}
%
The \emph{validty} and \emph{refused} metrics measure how often the
nearest neighbour fallback strategy is required to select a legal
workgroup size $w \in W_{legal}(s)$.

The purpose of this evaluation is to test the effectiveness of machine
learning-enabled autotuning for predicting workgroup sizes of SkelCL
stencils codes.

The classifier achieves good speedups over the baseline. Average
classification speedups across all validation sets range between
$4.61\times$ and $5.05\times$. Figures~\ref{fig:class-syn}
and~\ref{fig:class-arch} show a summary of results using 10-fold
cross-validation and cross-device validation, respectively.

By isolating the test cases where an illegal or refused parameter was
predicted, we can analyse the effectiveness of the
\textsc{NearestNeighbour} fallback handler. The handler has an average
speedup across all validation sets of $5.26\times$, indicating that it
successfully exploits the structure of the optimisation spaces.

\begin{table}
\scriptsize
\centering
\begin{tabular}{llll}
\toprule
              Job &    Performance &            Speedup &       Human Expert \\
\midrule
          10-fold &           92\% &       $5.65\times$ &       $1.26\times$ \\
        Synthetic &           92\% &       $4.79\times$ &       $1.13\times$ \\
           Device &           85\% &       $5.23\times$ &       $1.17\times$ \\
           Kernel &           89\% &       $5.43\times$ &       $1.21\times$ \\
          Dataset &           91\% &       $5.63\times$ &       $1.25\times$ \\
 \textbf{Average} &  \textbf{90\%} &  $\bm{5.45\times}$ &  $\bm{1.22\times}$ \\
\bottomrule
\end{tabular}
\caption{Validation results for J48 and \textsc{NearestNeighbour}
  classification.}
\label{tab:class}
\end{table}


\subsection{Classifier selection}

\begin{figure}
\centering
\begin{subfigure}[t]{0.48\columnwidth}
\centering
\includegraphics[width=\columnwidth]{img/heatmap_1}
\vspace{-1.5em} % Shrink vertical padding
\caption{}
\label{fig:class-hmaps-1}
\end{subfigure}
\begin{subfigure}[t]{0.48\columnwidth}
\centering
\includegraphics[width=\columnwidth]{img/heatmap_2}
\vspace{-1.5em} % Shrink vertical padding
\caption{}
\label{fig:class-hmaps-2}
\end{subfigure}
\\
\begin{subfigure}[t]{0.48\columnwidth}
\centering
\includegraphics[width=\columnwidth]{img/heatmap_3}
\vspace{-1.5em} % Shrink vertical padding
\caption{}
\label{fig:class-hmaps-3}
\end{subfigure}
\begin{subfigure}[t]{0.48\columnwidth}
\centering
\includegraphics[width=\columnwidth]{img/heatmap_5}
\vspace{-1.5em} % Shrink vertical padding
\caption{}
\label{fig:class-hmaps-4}
\end{subfigure}
\caption{%
  Heatmaps of classification errors for a subset of the optimisation
  space using four different classifiers. The shading in each cells
  indicates if it is predicted less frequently (blue), ore more
  frequently (red) than it is optimal. Colour gradients are normalised
  across plots.%
}
\label{fig:class-hmaps}
\end{figure}

The fastest classifier is J48, due to the it's simplicity (it can be
implemented as a sequence of nested \texttt{if}/\texttt{else}
statements).

Figure~\ref{fig:class-hmaps} visualises the classification errors of
each of the autotuning techniques. It shows that while the performance
of all of the classifiers is comparable, the distribution of
predictions is not. Only the NaiveBayes and RandomForest classifiers
predicted the human expert selected workgroup size of
$w_{(32 \times 4)}$ as frequently, or more frequently, than it was
optimal.


\subsubsection{Summary}

From an evaluation of 17 different autotuning techniques using 5
different types of validation sets, the following conclusions about
autotuning performance can be drawn:
%
\begin{itemize}
\item The performance of predicted workgroup sizes for unseen devices
  is within 8\% of the performance for known devices.
\end{itemize}


\section{Related Work}\label{sec:related}

Early work in autotuning applied iterative search techniques to the
space of compiler optimisations~\cite{Bodin1998}. Iterative
compilation has been demonstrated to be a highly effective form of
empirical performance tuning for selecting compiler
optimisations. Given the huge size of the compiler optimisation space
(sometimes exceeding $10^{100}$ possible combinations), it is often
unfeasible to perform an exhaustive search of the entire optimisation
space.

Machine learning techniques have been successfully employed to reduce
the cost of iterative compilation. In~\cite{Agakov}, using statistical
models to focus a search of the optimisation
space. In~\cite{Stephenson2003}, using ``meta optimisation'' to tune
compiler heuristics through an evolutionary algorithm to automate the
search of the optimisation space. In~\cite{Fursin2011},
\citeauthor{Fursin2011} incorporated machine learning-enabled
self-tuning into the GCC compiler.

Optimising GPGPU programs presents different challenges to that of
traditional CPU programming. \citeauthor{Ryoo2008a} demonstrated
speedups of up to $432\times$ for matrix multiplication in CUDA
through the appropriate use of zero-overhead thread scheduling, memory
bandwidth, and thread grouping. The importance of proper exploitation
of local shared memory and synchronisation costs is explored
in~\cite{Lee2010}. In~\cite{Chen2014}, data locality optimisations are
automated using a description of the hardware and a
memory-placement-agnostic compiler. The authors demonstrate impressive
speedups of up to $2.08\times$, although at the cost of requiring
accurate memory hierarchy descriptor files for all targeted
hardware. The descriptor files must be hand generated, requiring
expert knowledge of the underlying hardware in order to properly
exploit memory locality. \citeauthor{Magni2014} demonstrate that
thread coarsening of OpenCL kernels can lead to speedups in program
performance between $1.11\times$ and $1.33\times$
in~\cite{Magni2014}. The authors achieve this using a machine learning
model to predict optimal thread coarsening factors based on the static
features of kernels, and an LLVM function pass to perform the required
code transformations.

% The size of the GPGPU optimisation space is reduced
% in~\cite{Ryoo2008} by using a common subset of optimal
% configurations across a set of training examples. This technique
% reduces the search space by 98\%, although it does not guarantee
% that for a new program, the reduced search space will include the
% optimal configuration.

Auotuning for stencil codes is explored in~\cite{Kamil2010}. A code
generator for Fortran 95 stencils expression and generates tuned
shared-memory parallel implementations in Fortan, C, or CUDA. The
system uses an IR to explore autotuning transformations, enumerating a
subset of the optimisation space and recording only a single execution
time for each configuration, reporting the fastest. They demonstrate
their system on 4 architectures using 3 benchmarks, with speedups of
up to $22\times$ compared to serial implementations. The CUDA code
generator does not optimise for the GPU memory hierarchy, using only
global memory. There is no directed search or cross-program
learning. In~\cite{Lutz2013}, \citeauthor{Lutz2013} demonstrate that
optimal swapping strategy for multi-GPU stencils depends on the size
of the grid, the number of partitions, and the connection mechanism
(e.g.\ PCI express)

OpenTuner is a general purpose toolkit for autotuning using iterative
search~\cite{Ansel2013}. It provides interfaces for application
developers to specify a search space and method for collecting
performance performance results. An \emph{ensemble} of search
techniques is then used to explore the space. They demonstrate their
technique using 14 benchmarks across 6 applications. OpenTuner does
not use machine learning, and performance datasets are not shared
across datasets. For these reasons, the burden of developing ad-hoc
autotuning approaches is lifted only from the application developer,
not from the end user. OpenTuner only stores data locally, so
performance data must be gathered (perhaps redundantly) from each new
device that uses it.  Our approach combines machine learning with
distributed training sets so that new users automatically benefit from
the collective tuning experience of other users. This removes the need
for redundant duplication of experiments, while dramatically reducing
the time to deployment.

In~\cite{Saclay2010,Memon2013}, \citeauthor{Saclay2010} advocate a
collaborative and ``big data'' driven approach to autotuning, arguing
that the challenges facing the widespread adoption of autotuning
methodologies can be attributed to a lack of common benchmarks,
datasets, and experimental methodologies. They propose the use of
``Collective optimization'' to leverage training experience across
devices using shared datasets. This idea is refined
in~\cite{Fursin2014}, in which the authors present a system for
sharing autotuning performance data, as well as additional metadata
about experimental setups. In addition to the mechanism for sharing
training datasets, our system provides the capabilities of performing
autotuning at runtime using a lightweight inter-process communication
interface. Collective Mind does not currently support run-time tuning.
Additionally, Collective Mind uses a NoSQL JSON format for storing
datasets. The relational schema used in this work offers greater
scaling performance and lower storage overhead as the amount of
performance data grows.  The authors do not provide an empirical
evaluation of their technique.


\section{Conclusions}\label{sec:conclusions}

As the trend towards higher core counts and increasing parallelism
continues, the need for high level, accessible abstractions to manage
such parallelism will continue to go. Autotuning proves a valuable aid
for achieving these goals, providing the benefits of low level
performance tuning while maintaining ease of use, without burdening
developers with optimisation concerns. As the need for autotuned
parallelism rises, the desire for collaborative techniques for sharing
performance data must be met with systems capable of supporting this
cross-platform learning.

In this work, we have presented an attempt to provide such a system,
by designing a novel framework which has the benefits of fast,
``always-on'' autotuning, while being able to synchronise data with
global repositories of knowledge which others may contribute to. The
framework provides an interface for autotuning which is sufficiently
generic to be easily re-purposed to target a range of optimisation
parameters.

To demonstrate the utility of this framework, we implemented a
frontend for predicting the workgroup size of OpenCL kernels for
SkelCL stencil codes. The publicly available implementation
\footnote{\url{https://github.com/ChrisCummins/omnitune}} predicts
workgroup sizes for OpenCL stencil skeleton kernels in order to
minimise their runtime on CPUs and multi-GPU systems. This
optimisation space is complex, non linear, and critical for the
performance of stencil kernels, with up to a $207.72\times$ slowdown
if an improper value is picked. Selecting the correct workgroup size
is difficult --- requiring a knowledge of the kernel, dataset, and
underlying architecture. The challenge is increased even more so by
inconsistencies in the underlying system which cause some workgroup
sizes to fail completely. Of the 269813 combinations of workgroup
size, device, program, and dataset tested; only a \emph{single}
workgroup size was valid for all test cases, and achieved only 24\% of
the available performance. The value selected by human experts was
invalid for 2.6\% of test cases. Autotuning in this space requires a
system which is resilient these challenges, and several techniques
were implemented to address them.

Runtime performance of autotuned stencil kernels is very promising,
achieving an average 90\% of the available performance with only a 3ms
autotuning overhead. Even ignoring the cases for which the human
expert selected workgroup size is invalid, this provides a
$1.33\times$ speedup, or a $5.57\times$ speedup over the best
performance that can be achieved using static tuning. Classification
performance is comparable when predicting workgroup sizes for both
unseen programs and unseen devices. I believe that the combination of
performance improvements and the collaborative nature of OmniTune
makes for a compelling case for the use of autotuning as a key
component for enabling performant, high level parallel programming.

The cost of offline training with OmniTune could be reduced by
exploring the use of adaptive sampling plans, such as presented
in~\cite{Leather2009}. This could reduce the number of runtime samples
required to distinguish good from bad optimisation parameter values.

Collaborative training --- hive mind for selecting training
parameters, and built-in redundancy checking/validation

TODO: deltas for push and pull, scalability for huge datasets

TODO: involuntary training requests, or non-binary selectors between
training and performance

%
% \appendix
% \section{Appendix Title}
%
% This is the text of the appendix, if you need one.
%

\acks

This work was supported by the UK Engineering and Physical Sciences
Research Council under grants EP/L01503X/1 for the University of
Edinburgh School of Informatics Centre for Doctoral Training in
Pervasive Parallelism
(\url{http://pervasiveparallelism.inf.ed.ac.uk/}),\\* EP/H044752/1
(ALEA), and EP/M015793/1 (DIVIDEND).

% We recommend abbrvnat bibliography style.

\label{bibliography}
\printbibliography


\end{document}

\includepdf[pages=-,addtotoc={1,section,1,CGO Publication,app:cgo}]{app/2017-cgo.pdf}

\end{appendices}

\end{document}
