% Only peer-reviewed publications.

\Title{Publications}


\Description{%
  % Publication date.
  \MarginText{\textit{2018}}%
  % Authors.
  C.\ \textsc{Cummins}, P.\ \textsc{Petoumenos}, A.\ \textsc{Murray}, H.\ \textsc{Leather}. %
  % Title.
  \textbf{Compiler Fuzzing through Deep Learning}. %
  % Conference / Journal.
  ISSTA'18 (28\% acceptance rate), Amsterdam, Netherlands.\@  % 112 submissions, 31 accepted.
  \newline\vspace{-1em}\\ \noindent
  % Description.
  Unsupervised machine learning to derive program generators for compiler fuzz testing. Implemented in $100\times$ less code than state-of-the-art program generator, and $3.03\times$ faster. Found and reported 67 bug reports in OpenCL compilers.%
}


\Description{%
  % Publication date.
  \MarginText{\textit{2017}}%
  % Authors.
  C.\ \textsc{Cummins}, P.\ \textsc{Petoumenos}, Z.\ \textsc{Wang}, H.\ \textsc{Leather}. %
  % Title.
  \textbf{End-to-end Deep Learning of Compiler Heuristics}. %
  % Conference / Journal.
  Best Paper PACT'17 (23\% acceptance rate), Portland, Oregon.\@
  \newline\vspace{-1em}\\ \noindent
  % Description.
  Learning optimization heuristics directly from raw source code, without the need for feature extraction. 12\% and 14\% performance improvements over state-of-the art, with greatly reduced development costs and the ability to transfer learning across heuristics.%  Can transfer knowledge gained from one optimization task to another, even if the learned tasks are dissimilar.
}


\Description{%
  % Publication date.
  \MarginText{2017}%
  % Authors.
  C.\ \textsc{Cummins}, P.\ \textsc{Petoumenos}, Z.\ \textsc{Wang}, H.\ \textsc{Leather}. %
  % Title.
  \textbf{Synthesizing Benchmarks for Predictive Modeling}. %
  % Conference / Journal.
  Best Paper CGO'17 (22\% acceptance rate), Austin, Texas.\@
  \newline\vspace{-1em}\\  \noindent
  % Description.
  Deep learning over massive codebases from GitHub to generate benchmark programs. Automatically synthesizes OpenCL kernels which are indistinguishable from hand-written code, and improves state-of-the-art predictive model performance by $4.30\times$.%
}


\Description{%
  % Publication date.
  \MarginText{2016}%
  % Authors.
  C.\ \textsc{Cummins}, P.\ \textsc{Petoumenos}, M.\ \textsc{Steuwer}, H.\ \textsc{Leather}. %
  % Title.
  \textbf{Autotuning OpenCL Workgroup Sizes} (extended abstract). %
  % Conference / Journal.
  ACACES'16, Fiuggi, Italy.\@ %
  \newline\vspace{-1em}\\ \noindent
  % Description.
  Machine learning-enabled autotuning of multi-GPU OpenCL workgroup sizes. Static tuning achieves only 26\% of the maximum performance, our approach achieves 92\%.%
}


\Description{%
  % Publication date.
  \MarginText{2016}%
  % Authors.
  C.\ \textsc{Cummins}, P.\ \textsc{Petoumenos}, M.\ \textsc{Steuwer}, H.\ \textsc{Leather}. %
  % Title.
  \textbf{Towards Collaborative Performance Tuning of Algorithmic Skeletons}. %
  % Conference / Journal.
  HLPGPU'16, HiPEAC, Prague.\@
  \newline\vspace{-1em}\\ \noindent
  % Description.
  A distributed framework for dynamic prediction of optimisation parameters using machine learning. Automatically exceeds human experts by $1.22\times$.%
}


\Description{%
  % Publication date.
  \MarginText{2016}%
  % Authors.
  C.\ \textsc{Cummins}, P.\ \textsc{Petoumenos}, M.\ \textsc{Steuwer}, H.\ \textsc{Leather}. %
  % Title.
  \textbf{Autotuning OpenCL Workgroup Size for Stencil Patterns}. %
  % Conference / Journal.
  ADAPT'16, HiPEAC, Prague.\@ %
  \newline\vspace{-1em}\\ \noindent
  % Description.
  Three methodologies to autotune stencil patterns using machine learning. Speedups of $3.79\times$ over the best possible static size, 94\% of the maximum performance.%
}


\Description{%
  % Publication date.
  \MarginText{2015}%
  % Authors.
  E.\ \textsc{Bunkute}, C.\ \textsc{Cummins}, F.\ \textsc{Crofts}, G.\ \textsc{Bunce}, I.\ T.\ \textsc{Nabney}, D.\ R.\ \textsc{Flower}.
  % Title.
  \textbf{\href{http://bioinformatics.oxfordjournals.org/content/31/2/295.full?etoc}{PIP-DB: The Protein Isoelectric Point Database}}. %
  % Conference / Journal.
  Bioinformatics, 31(2), 295-296. Chicago. %
  \newline\vspace{-1em}\\ \noindent
  % Description.
  An open source search engine of protein isoelectric points. Provides public access to bioinformatics data from the literature for comparison and benchmarking purposes.%
}
