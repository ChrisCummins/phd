\todo[inline]{Crisis, solution, happiness}

% TODO: Reference the compilers episode of the 50 things podcast for some ideas on how to introduce compilers.

Compilers are an essential ingredient in computer systems. A compiler translates source code --- a formal specification for the behaviour of a program, written by human programmers --- into instructions that a computer executes.

There are two primary requirements for a compiler. First, it must be correct, that is, the translation from human to computer language should not change the meaning of the program being described. Secondly, the generated sequence of computer instructions should be efficient. 

This thesis presents new ways of approaching the problems of compiler correctness and efficiency.

Machine learning is a set of techniques that enable computers to solve tasks not by being explicitly programmed, but by observing prior tasks and their solutions, learning how to map a problem to a solution.

In performing the translation, compilers have many different choices in how to generate the instructions. These choices may affect the performance or energy efficiency of a program. Each of the choices was put there by the engineers that built the compiler.

Compilers are frighteningly complex, they are too large to fit inside the mind of a single human, and the number of available choices during compilation is enormous - many more than the number of atoms in the observable universe.

This thesis presents new techniques to simplify compilers and make them easier to build. The main idea is, rather than have compiler builders expertly craft all of the choices, to let compilers learn for themselves.
