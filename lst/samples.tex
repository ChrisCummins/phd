\newsavebox{\IntelSizetIntReduced}
\begin{lrbox}{\IntelSizetIntReduced}
  \hspace{1.5em}
  \begin{lstlisting}
    kernel void A(global double* a) {
      int b = get_global_id(0);
      if (b < -1)
        a[b] = 1;
    }
  \end{lstlisting}
\end{lrbox}

\newsavebox{\OclgrindRaceSwitch}
\begin{lrbox}{\OclgrindRaceSwitch}
  \hspace{1.5em}
  \begin{lstlisting}
  kernel void A(global int* a, global int* b) {
    switch (get_global_id(0)) {
    case 0:
      a[get_global_id(0)]=b[get_global_id(0)+13];
      break;
    case 2:
      a[get_global_id(0)]=b[get_global_id(0)+11];
      break;
    case 6:
      a[get_global_id(0)]=b[get_global_id(0)+128];
    }
    barrier(2);
  }
  \end{lstlisting}
\end{lrbox}

\newsavebox{\AlmostEverythingCrash}
\begin{lrbox}{\AlmostEverythingCrash}
  \hspace{1.5em}
  \begin{lstlisting}
    kernel void A() {
      __builtin_astype(d, uint4);
    }
  \end{lstlisting}
\end{lrbox}

\newsavebox{\OclgrindSemaAssertion}
\begin{lrbox}{\OclgrindSemaAssertion}
  \hspace{1.5em}
  \begin{lstlisting}
    kernel void A(global unsigned char* a, global unsigned char* b) {
      unsigned long c = get_global_id(0);
      d[0] = (mad24(f, (int)(a[0], b[get_global_id(0)])) % (d * (d + 3)) + (c / 2))] * a[c + 1];
    }
  \end{lstlisting}
\end{lrbox}
% This has been forwarded to the LLVM folks https://bugs.llvm.org/show_bug.cgi?id=33897
% __kernel void A(__global float* b) {
%	(float4)(b);
% }

\newsavebox{\IntelPtrCompilerHang}
\begin{lrbox}{\IntelPtrCompilerHang}
  \hspace{1.5em}
  \begin{lstlisting}
    kernel void A(global ulong* a) {
      a[get_global_id(0)] = (ulong)(a + 1);
    }
  \end{lstlisting}
\end{lrbox}
% Another 3± hang:
% kernel void A(global const char* a, global char* b, global char* c) {
%   int d = get_global_id(0);
%   c[d] = get_global_id(0) + c;
% }

% Another 3± hang:
% __kernel void A(global int* a) {
%   local int b[4][3][4][5];
%   b[1][2][3][3] = b[3];
% }

\newsavebox{\NvidiaOptLoopHang}
\begin{lrbox}{\NvidiaOptLoopHang}
  \hspace{1.5em}
  % CLgenProgram.id = 6992
  \begin{lstlisting}
    kernel void A(global float* a, global float* b,
                  global float* c) {
      int d, e, f;
      d = get_local_id(0);
      for (int g = 0; g < 100000; g++)
        barrier(1);
    }
  \end{lstlisting}
\end{lrbox}

\newsavebox{\XeonPhiSpin}
\begin{lrbox}{\XeonPhiSpin}
  \hspace{1.5em}
  \begin{lstlisting}
    kernel void A(global unsigned char* a,
                  unsigned b) {
      a[get_global_id(0)] %= 42;
      barrier(1);
    }
  \end{lstlisting}
\end{lrbox}
% LLVM ERROR: LLVM2PIL: Cannot yet select: 0x7f72a45a7ed0: i8,i8 = umul_lohi 0x7f72a45a87d0, 0x7f72a45a88d0 [ORD=16] [ID=22]
% 0x7f72a45a87d0: i8 = srl 0x7f72a45a7dd0, 0x7f72a42fff90 [ORD=16] [ID=21]
% 0x7f72a45a7dd0: i8,ch = load 0x7f72a41d9438, 0x7f72a4300790, 0x7f72a4300590<LD1[%scevgep2]> [ORD=15] [ID=20]
% 0x7f72a4300790: i64 = add 0x7f72a45a83d0, 0x7f72a45a80d0 [ORD=14] [ID=17]
% 0x7f72a45a83d0: i64,ch = CopyFromReg 0x7f72a41d9438, 0x7f72a45a86d0 [ORD=14] [ID=13]
% 0x7f72a45a86d0: i64 = Register %vreg1 [ID=1]
% 0x7f72a45a80d0: i64,ch = CopyFromReg 0x7f72a41d9438, 0x7f72a45a81d0 [ORD=14] [ID=14]
% 0x7f72a45a81d0: i64 = Register %vreg2 [ID=2]
% 0x7f72a4300590: i64 = undef [ID=3]
% 0x7f72a42fff90: i8 = Constant<1> [ID=9]
% 0x7f72a45a88d0: i8 = Constant<49> [ID=10]

\newsavebox{\IntelOptLoopHang}
\begin{lrbox}{\IntelOptLoopHang}
  \hspace{1.5em}
  \begin{lstlisting}
    kernel void A(global int* a) {
      int b = get_global_id(0);
      while (b < 512) { }
    }
  \end{lstlisting}
\end{lrbox}
% Another example:
%   kernel void A(long a, long b) {
%     while (a > 273444) { }
%   }


\newsavebox{\NvidiaRecursionSegfault}
\begin{lrbox}{\NvidiaRecursionSegfault}
  \hspace{1.5em}
  \begin{lstlisting}
    kernel void A(float4 a, global float4* b,
                  global float4* c, unsigned int d,
                  global double* e, global int2* f, 
                  global int4* g, constant int* h, 
                  constant int* i) {
      A(a, b, c, d, d, e, f, g, h);
    }
  \end{lstlisting}
\end{lrbox}
% HAND REDUCED:
%
%   kernel void A(constant int2* a, global int* b) { A(b, a); }
\newsavebox{\NvidiaRecursionSegfaultReduced}
\begin{lrbox}{\NvidiaRecursionSegfaultReduced}
  \hspace{1.5em}
  \begin{lstlisting}
    kernel void B(constant int2* a, global int* b) {
      A(b, a);
    }
  \end{lstlisting}
\end{lrbox}

\newsavebox{\BeignetScalarizeInsert}
\begin{lrbox}{\BeignetScalarizeInsert}
  \hspace{1.5em}
  \begin{lstlisting}
    kernel void A(global float4* a) {
      a[get_local_id(0) / 8][get_local_id(0)] = 
          get_local_id(0);
    }
  \end{lstlisting}
\end{lrbox}

\newsavebox{\OclgrindUncorrectedTypos}
\begin{lrbox}{\OclgrindUncorrectedTypos}
  \hspace{1.5em}
  \begin{lstlisting}
    kernel void A(global float* a, global float* b) {
      a[0] = max(a[c], b[2]);
    }
  \end{lstlisting}
\end{lrbox}

\newsavebox{\BeignetPtrIntSpin}
\begin{lrbox}{\BeignetPtrIntSpin}
  \hspace{1.5em}
  \begin{lstlisting}
    kernel void A(global int* a) {
      int b = get_global_id(0);
      a[b] = (6 * 32) + 4 * (32 / 32) + a;
    }
  \end{lstlisting}
\end{lrbox}

\newsavebox{\NvidiaCompileSegfault}
\begin{lrbox}{\NvidiaCompileSegfault}
  \hspace{1.5em}
  \begin{lstlisting}
    kernel void A(global float* a, local float* b, local float* c, int d, int e) {
      int f, g;
      int h = get_local_id(0);
      int i = get_local_id(1);
      int j = get_global_id(0);
      global char* k = c + f * g + f;
      if (f + 1 < h)
        b[f * d + g * h + g] = g * f;
    }
  \end{lstlisting}
\end{lrbox}

\newsavebox{\XeonPhiSegfault}
\begin{lrbox}{\XeonPhiSegfault}
  \hspace{1.5em}
  \begin{lstlisting}
    kernel void A(void) {
      global int* a;
      unsigned int* b;
      b = a[0];
      a[0] = b;
      a[0] = b;
      barrier(1);
      if (get_global_id(0) == 0)
        *a = 0;
      a[get_local_id(0)] = 0;
    }
  \end{lstlisting}
\end{lrbox}

\newsavebox{\IntelVectorizerSegfault}
\begin{lrbox}{\IntelVectorizerSegfault}
  \hspace{1.5em}
  \begin{lstlisting}
    kernel void A() {
      while (true) {
        barrier(1);
      }
    }
  \end{lstlisting}
\end{lrbox}

\newsavebox{\IntelSizetIntUnreduced}
\begin{lrbox}{\IntelSizetIntUnreduced}
  \hspace{1.5em}
  \begin{lstlisting}
    kernel void A(global double* a, global double* b,
                  global double* c, int d, int e) {
      double f;
      int g = get_global_id(0);
      if (g < e - d - 1)
        c[g] = (((e) / d) % 5) % (e + d);
    }
  \end{lstlisting}
\end{lrbox}

\newsavebox{\PoclUndefinedSymbols}
\begin{lrbox}{\PoclUndefinedSymbols}
  \hspace{1.5em}
  \begin{lstlisting}
    kernel void A(local int* a) {
      for (int b = 0; b < 100; b++)
        B(a);
    }
  \end{lstlisting}
\end{lrbox}


\newsavebox{\BeignetCastError}
\begin{lrbox}{\BeignetCastError}
  \hspace{1.5em}
  \begin{lstlisting}
    kernel void A(global float* a, global float* b, global float* c, const int d) {
      int e = get_global_id(0);
      if (e < d)
        c[e] = a[e] + b[e];
      b[e] = (char)(c[e] + d);
    }
  \end{lstlisting}
\end{lrbox}

\newsavebox{\XeonPhiInvalidWrite}
\begin{lrbox}{\XeonPhiInvalidWrite}
  \hspace{1.5em}
  \begin{lstlisting}
    kernel void A(global int* a) {
      a[0] = 1;
      a[-1] = 2;
      a[0] = 3;
    }
  \end{lstlisting}
\end{lrbox}

\newsavebox{\UninitRead}
\begin{lrbox}{\UninitRead}
  \hspace{1.5em}
  \begin{lstlisting}
      kernel void A(global int* a, global int* b) {
        int c[16];
        int d = get_global_id(0);
        a[d] = b[d] + c[d];
      }
  \end{lstlisting}
\end{lrbox}

\newsavebox{\IntelPtrAssertion}
\begin{lrbox}{\IntelPtrAssertion}
  \hspace{1.5em}
  \begin{lstlisting}
    kernel void A(global int* a, global int* b) {
      int c = (int)get_global_id(0);
      a[c] += b;
    }
  \end{lstlisting}
\end{lrbox}

\newsavebox{\IntelScalarAssertion}
\begin{lrbox}{\IntelScalarAssertion}
  \hspace{1.5em}
  \begin{lstlisting}
    kernel void A(global float* a, global float* b, global float* c, local float* d, unsigned int e, unsigned int f) {
      for (unsigned int g = get_local_id(0) + get_local_size(0); g < get_local_size(0); g += get_local_size(0))
        a[2 * get_local_id(0) + 1] = get_local_id(0);
    }
  \end{lstlisting}
\end{lrbox}

\newsavebox{\BeignetTernary}
\begin{lrbox}{\BeignetTernary}
  \hspace{1.5em}
  \begin{lstlisting}
    kernel void A(global int* a, global int* b, 
                  global int* c) {
      c[0] = (a[0] > b[0]) ? a[0] : 0;
      c[2] = (a[3] <= b[3]) ? a[4] : b[5];
      c[4] = (a[4] <= b[5]) ? a[7] : b[7];
      c[7] = (a[7] < b[0]) ? a[0] : (a[0] > b[1]);
    }
  \end{lstlisting}
\end{lrbox}
% HAND REDUCED:
%
%   kernel void A(global int* a, global int* b, global int* c) {
%     c[0] = 100;
%     c[1] = (a[3] <= b[4]) ? a[4] : b[5];
%   }

\newsavebox{\BeignetTernarySmaller}
\begin{lrbox}{\BeignetTernarySmaller}
  \hspace{1.5em}
  \begin{lstlisting}
    kernel void A(global int* a, global int* b, global int* c) {
      c[0] = (a[0] > 0) ? a[0] : b[1];
      c[1] = (a[1] <= b[1]) ? a[0] : b[0];
    }
  \end{lstlisting}
\end{lrbox}

\newsavebox{\IntelGtDoubleAssertion}
\begin{lrbox}{\IntelGtDoubleAssertion}
  \hspace{1.5em}
  \begin{lstlisting}
    kernel void A(read_only image2d_t a, 
                  global double2* b) {
      b[0] = get_global_id(0);
    }
  \end{lstlisting}
\end{lrbox}

\newsavebox{\IntelDeclDoesntDeclareAnything}
\begin{lrbox}{\IntelDeclDoesntDeclareAnything}
  \hspace{1.5em}
  \begin{lstlisting}
    kernel void A() {
      char a;
      int b;
      int const;
    }
  \end{lstlisting}
\end{lrbox}

\newsavebox{\AddressQualifiedAutoVar}
\begin{lrbox}{\AddressQualifiedAutoVar}
  \hspace{1.5em}
  \begin{lstlisting}
    kernel void A(global float4* a, global float4* b, global float4* c, global float4* d, global float4* e, float f) {
      unsigned int g = get_global_id(0);
      int h = get_global_size(0);
      constant sampler_t i = 0x0000 | 0x0004 | 0x0000;
      unsigned int j = g * (1 << ((h % 4) - (2)));
    }
  \end{lstlisting}
\end{lrbox}

\newsavebox{\NvidiaLocalGlobalSegfault}
\begin{lrbox}{\NvidiaLocalGlobalSegfault}
  \hspace{1.5em}
  \begin{lstlisting}
    kernel void A(global int* a) {
      local int b[2][3][4][5];
      if (get_global_id(0) == 0)
        a = b[0];
    }
  \end{lstlisting}
\end{lrbox}

\newsavebox{\NvidiaLocalSegfault}
\begin{lrbox}{\NvidiaLocalSegfault}
  \hspace{1.5em}
  \begin{lstlisting}
    kernel void A(global uchar4* a, const int b) {
      local int c[16];
      local float8* d = a + 133;
      atomic_cmpxchg(c, 10, 13);
    }
  \end{lstlisting}
\end{lrbox}


% Program ID: 35612
\newsavebox{\IntelPostDominanceFrontier}
\begin{lrbox}{\IntelPostDominanceFrontier}
  \hspace{1.5em}
  \begin{lstlisting}
    kernel void A() {
      while (true)
        barrier(1);
    }
  \end{lstlisting}
\end{lrbox}
% Another:
% kernel void A(local int* a) {
%   while (1) {
%   }
% }
%
% And another:
%    kernel void A(global int* a) {
%      bool b;
%      int c;
%      for (b = 0; b < 100; b++)
%        a[c] += a[c];
%    }
%
% One more:
%   kernel void A(global int* a, global int* b) {
%     int c = get_global_id(0);
%     int d = 0;
%     while (d < 1024)
%       a[c] = d;
%   }

\newsavebox{\IntelPredicator}
\begin{lrbox}{\IntelPredicator}
  \hspace{1.5em}
  \begin{lstlisting}
    kernel void A(global int* a) {
      int b = get_global_id(0);
      while (b < *a)
        if (a[0] < 0)
          a[1] = b / b * get_local_id(0);
    }
  \end{lstlisting}
\end{lrbox}

\newsavebox{\IntelPrepareKernelArgs}
\begin{lrbox}{\IntelPrepareKernelArgs}
  \hspace{1.5em}
  \begin{lstlisting}
    kernel void A(int a, global int* b) {
      int c = get_global_id(0);
      int d = work_group_scan_inclusive_max(c);
      b[c] = c;
    }
  \end{lstlisting}
\end{lrbox}

\newsavebox{\IntelDAGInstrSelection}
\begin{lrbox}{\IntelDAGInstrSelection}
  \hspace{1.5em}
  \begin{lstlisting}
    kernel void A(global half* a, global int* b, global bool* c, int d, int e) {
      int f = get_global_id(0);
      int g = get_global_id(1) * e;
      if (f < e)
        a[f] = b[f];
    }
  \end{lstlisting}
\end{lrbox}

\newsavebox{\IntelSPIRMetadata}
\begin{lrbox}{\IntelSPIRMetadata}
  \hspace{1.5em}
  \begin{lstlisting}
    kernel void A() {
      local float a; A(a);
    }
  \end{lstlisting}
\end{lrbox}

\newsavebox{\IntelRemoveDupeBarrier}
\begin{lrbox}{\IntelRemoveDupeBarrier}
  \hspace{1.5em}
  \begin{lstlisting}
    kernel void A() {
      local int a[10];
      local int b[16][16];
      a[1024 + (2 * get_local_id(1) +
        get_local_id(0)) + get_local_id(0)] = 6;
      barrier(b);
    }
  \end{lstlisting}
\end{lrbox}

\newsavebox{\IntelCombineRedundant}
\begin{lrbox}{\IntelCombineRedundant}
  \hspace{1.5em}
  \begin{lstlisting}
    kernel void A(global float* a, global float* b,
                  global float* c, const int d) {
      for (unsigned int e = get_global_id(0);
           e < d; e += get_global_size(0))
        for (unsigned f = 0; f < d; ++f)
          e += a[f];
    }
  \end{lstlisting}
\end{lrbox}


\newsavebox{\BeigPtrAssertion}
\begin{lrbox}{\BeigPtrAssertion}
  \hspace{1.5em}
  \begin{lstlisting}
    kernel void A(global int* a, global int* b,
                  global int* c) {
      a[get_global_id(0)] = a[get_global_id(0)] > b;
    }
  \end{lstlisting}
\end{lrbox}

\newsavebox{\BeigIterAssertion}
\begin{lrbox}{\BeigIterAssertion}
  \hspace{1.5em}
  \begin{lstlisting}
    kernel void A(global int* a) {
      global int* b = ((void*)0);
      b[0] = a;
    }
  \end{lstlisting}
\end{lrbox}

\newsavebox{\ParserFailOne}
\begin{lrbox}{\ParserFailOne}
  \hspace{1.5em}
  \begin{lstlisting}
    void A(){(global a*)()
  \end{lstlisting}
\end{lrbox}

\newsavebox{\ParserFailTwo}
\begin{lrbox}{\ParserFailTwo}
  \hspace{1.5em}
  \begin{lstlisting}
    void A(){void* a; uint4 b=0; b=(b>b)?a:a
  \end{lstlisting}
\end{lrbox}

\newsavebox{\ParserFailThree}
\begin{lrbox}{\ParserFailThree}
  \hspace{1.5em}
  \begin{lstlisting}
    void A(){double2 k; return (float4)(k,k,k,k)
  \end{lstlisting}
\end{lrbox}

\newsavebox{\DagPass}
\begin{lrbox}{\DagPass}
  \hspace{1.5em}
  \begin{lstlisting}
    kernel void A(global half* a) {
      int b = get_global_id(0);
      a[b] = b * b;
    }
  \end{lstlisting}
\end{lrbox}

\newsavebox{\SimplifyTheCFGPass}
\begin{lrbox}{\SimplifyTheCFGPass}
  \hspace{1.5em}
  \begin{lstlisting}
    kernel void A(global float* a, global float* b,
                  const int c) {
      for (int d = 0; d < c; d++)
        for (d = 0; d < a; d += 32)
          b[d] = 0;
    }
  \end{lstlisting}
\end{lrbox}

\newsavebox{\LoopWIAnalysisPass}
\begin{lrbox}{\LoopWIAnalysisPass}
  \hspace{1.5em}
  \begin{lstlisting}
    kernel void A(global int* a, const int b, const int c) {
      int d; int e; int f = 0;
      for (int g = b >> 1; g > 1; g >>= 1) {
        barrier(1);
        for (int h = c; h < g; h += d) {
          int i = f * (2 * h + 1) - 1;
          int j = f * (2 * h + 2) - 1;
          a[j] += a[i];
        }
      }
    }
  \end{lstlisting}
\end{lrbox}