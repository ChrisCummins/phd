\chapter{Introduction}

\lipsum[1-2]

\section{Machine Learning for Compilers}

\section{The Problem}

\section{Contributions}

This thesis presents machine learning methodologies for simplifying compiler construction.

The major contributions of this thesis are as follows:

\textbf{benchmark generator}.

A novel methodology for the \textbf{construction of optimization heuristics} without the use of explicit code features.

An application to \textbf{fuzz testing}.

\section{Structure}

This thesis is organized as follows:

\textbf{Chapter~\ref{chap:background}} defines terminology and describes the  methodologies and techniques used in this thesis.

\textbf{Chapter~\ref{chap:related-work}} provides an overview of previous work on machine learning for compilers, with an emphasis on performance optimization.

\textbf{Chapter~\ref{chap:clgen}} describes a machine learning generator for source codes. The generator is evaluated for its ability to produce OpenCL benchmarks.

\textbf{Chapter~\ref{chap:clgen}} introduces a novel machine learning generator for source codes. The generator is evaluated for its ability to produce OpenCL benchmarks.

\textbf{Chapter~\ref{chap:deepsmith}} presents an application of the machine learning generator for synthesizing compiler test cases. The chapter contains an extensive evaluation of OpenCL compilers using the synthesized test cases.

\textbf{Chapter~\ref{chap:deeptune}} introduces a novel methodology for constructing optimizing compiler heuristics without the need for code features.

\textbf{Chapter~\ref{chap:conclusions}} summarizes the findings and describes potential avenues for future research.


\section{Summary}

This chapter outlines problems encountered