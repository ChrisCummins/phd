\chapter{Conclusions}
\label{chap:conclusions}

\section{Contributions}

This section summarises the main contributions of this thesis for XXX.

\subsection{Workload Characterization}

\subsection{Compiler Optimizations}

\subsection{Compiler Testing}

\section{Critical Analysis}

\subsection{Limitations of Generative Models}

Extensibility to other languages largely untested.

Generating multi-function programs.

% TODO(cec): CGO'17 limitations

Our new approach enables the synthesis of more human-like programs than current state of the art program generators, and without the expert guidance required by template based generators, but it has limitations. Our method of seeding the language models with the start of a function means that we cannot support user defined types, or calls to user-defined functions. This means that we only consider scalars and arrays as inputs; while 6 (2.3\%) of the benchmark kernels from Table~\ref{tab:benchmarks} use irregular data types as inputs. We will address this limitation through recursive program synthesis, whereby a call to a user-defined function or unrecognized type will trigger candidate functions and type definitions to be synthesized. Currently we only run single-kernel benchmarks. We will extend the host driver to explore multi-kernel schedules and interleaving of kernel executions. Our host driver generates datasets from uniform random distributions, as do many of the benchmark suites. For cases where non-uniform inputs are required (e.g. profile-directed feedback), an alternate methodology for generating inputs must be adopted.


\subsection{Limitations of Sequential Classification}

\section{Future Work}
