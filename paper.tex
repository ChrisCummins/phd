\documentclass[prodmode,acmtaco]{acmsmall}

% Package to generate and customize Algorithm as per ACM style
\usepackage[ruled]{algorithm2e}
\renewcommand{\algorithmcfname}{ALGORITHM}
\SetAlFnt{\small}
\SetAlCapFnt{\small}
\SetAlCapNameFnt{\small}
\SetAlCapHSkip{0pt}
\IncMargin{-\parindent}

% Metadata Information
\acmVolume{9}
\acmNumber{4}
\acmArticle{39}
\acmYear{2016}
\acmMonth{3}

% Copyright
%\setcopyright{acmcopyright}
\setcopyright{acmlicensed}
%\setcopyright{rightsretained}
%\setcopyright{usgov}
%\setcopyright{usgovmixed}
%\setcopyright{cagov}
%\setcopyright{cagovmixed}

% DOI
\doi{0000001.0000001}

%ISSN
\issn{1234-56789}


%%%%%%%%%%%%%%%%%%%%%%%%%
%% Document and Layout %%
%%%%%%%%%%%%%%%%%%%%%%%%%

% Fix for multiple "No room for a new \dimen" errors.
%
% See: http://tex.stackexchange.com/questions/38607/no-room-for-a-new-dimen
%
\usepackage{etex}

\usepackage[utf8]{inputenc}

% Fix for "'babel/polyglossia' detected but 'csquotes' missing"
% warning. NOTE: Include after inputenc.
%
\usepackage{csquotes}

% Make internal macro definitions accessible,
% e.g. \@title, \@date \@author.
\makeatletter

% Multi-column support.
\usepackage{multicol}

% A useful package which includes macros like \ifdef{}{}{}:
%
\usepackage{etoolbox}

% Uncomment the following line to remove column separation:
%
%\setlength{\columnsep}{5mm}

% Allow user-defined warning and error filters.
%
\usepackage{silence}

\usepackage{adjustbox}


%%%%%%%%%%%%%%%%%%%%%
% Table of Contents %
%%%%%%%%%%%%%%%%%%%%%

% % Set chapter and section numbering depth:
% %
% \setcounter{secnumdepth}{2}


%%%%%%%%%%%%%%%%
% Bibliography %
%%%%%%%%%%%%%%%%
% \usepackage[%
%     backend=biber,
%     style=ieee,
%     % style=numeric-comp,
%     % style=numeric-comp,  % numerical-compressed
%     sorting=none,        % nty,nyt,nyvt,anyt,anyvt,ynt,ydnt,none
%     sortcites=true,      % sort \cite{b a d c}: true,false
%     block=none,          % space between blocks: none,space,par,nbpar,ragged
%     indexing=false,      % indexing options: true,false,cite,bib
%     citereset=none,      % don't reset cites
%     isbn=false,          % print ISBN?
%     url=true,            % print URL?
%     doi=false,           % print DOI?
%     natbib=true,         % natbib compatability
%    ]{biblatex}

% \usepackage{natbib}

% % Filter annoying and unavoidable biblatex warning:
% \WarningFilter{biblatex}{Patching footnotes failed}

% Reduce the font size of the bibliography:
% \renewcommand{\bibfont}{\normalfont\scriptsize}

% Determine which BibTeX file to use:
%
% If available, use my Mendeley BibTex library, located in the home
% directory. Note that this is a relative path and will break if
% either this file or the BibTex library are moved. If the library is
% not present, use the local refs.bib file.
% \newcommand{\BibResourceGlobal}{../../../library.bib}
% \newcommand{\BibResourceLocal}{refs.bib}

% \IfFileExists{\BibResourceGlobal}
%   {\newcommand{\BibResource}{\BibResourceGlobal}}
%   {\newcommand{\BibResource}{\BibResourceLocal}}

% \addbibresource{\BibResource}


%%%%%%%%%%%%%%
% Appendices %
%%%%%%%%%%%%%%

% Appendix package. Documentation:
%
%  http://mirror.ox.ac.uk/sites/ctan.org/macros/latex/contrib/appendix/appendix.pdf
%
% Package options:
%
% toc      - Put a header (e.g., `Appendices') into the Table of Contents
%            (the ToC) before listing the appendices. (This is done by
%            calling the \addappheadtotoc command.)
% page     - Puts a title (e.g., `Appendices') into the document at the
%            point where the appendices environment is begun. (This is
%            done by calling the \appendixpage command.)
% title    - Adds a name (e.g., `Appendix') before each appendix title in
%            the body of the document. The name is given by the value
%            of \appendixname. Note that this is the default behaviour
%            for classes that have chapters.
% titletoc - Adds a name (e.g., `Appendix') before each appendix listed
%            in the ToC. The name is given by the value
%            of \appendixname.
% header   - Adds a name (e.g., `Appendix') before each appendix in page
%            headers.  The name is given by the value
%            of \appendixname. Note that this is the default behaviour
%            for classes that have chapters.
\usepackage[title, titletoc]{appendix}

% pre-requisites for rendering upquotes in listings package.
\usepackage[T1]{fontenc}
\usepackage{lmodern}
\usepackage{textcomp}

% code listings.
\usepackage{listings}

% set \ttfamily to use courier fonts.
%
% See: http://tex.stackexchange.com/a/33686
%
\usepackage{courier}

\lstset{frame=bt,                    % Add top and bottom frame lines
breaklines=true,             % Force line wrapping
captionpos=b,                % Place caption below listing
numbers=left,                % Add left-side line numbers
basicstyle=\scriptsize\ttfamily, % Set font size and type
showstringspaces=false,      % Don't show visible whitespace
numberstyle=\tiny,
upquote=true,                % Use upright quotes, not curly
commentstyle=\bfseries}      % Embolden comments

% Use (*@ @*) to escape LaTeX commands within listings.
\lstset{escapeinside={(*@}{@*)}}

% Add 10pt space between chapters in TOC listings entries:
%\let\Chapter\chapter
%\def\chapter{\addtocontents{lol}{\protect\addvspace{10pt}}\Chapter}


%%%%%%%%%%%%%%%%%%%%%%%%
%% Graphics and maths %%
%%%%%%%%%%%%%%%%%%%%%%%%
\usepackage{amsmath}

% Vector notation, e.g. \vv{x}:
%
\usepackage{esvect}

% Additional amsmath symbols, see:
%
% http://texblog.org/2007/08/27/number-sets-prime-natural-integer-rational-real-and-complex-in-latex/
%
\usepackage{amsfonts}
\usepackage{amssymb}

\usepackage{graphicx}
\usepackage{mathtools}
\usepackage{tikz}
\usepackage{tikz-qtree}

% Provide bold font face in maths.
\usepackage{bm}

\usepackage{subcaption}
\expandafter\def\csname ver@subfig.sty\endcsname{}

% Define an 'myalignat' command which behave as 'alignat' without the
% vertical top and bottom padding. See:
%     http://www.latex-community.org/forum/viewtopic.php?f=5&t=1890
\newenvironment{myalignat}[1]{%
\setlength{\abovedisplayskip}{-.7\baselineskip}%
\setlength{\abovedisplayshortskip}{\abovedisplayskip}%
\start@align\z@\st@rredtrue#1
}%
{\endalign}

% Define additional operators:
\DeclareMathOperator*{\argmin}{arg\,min}
\DeclareMathOperator*{\argmax}{arg\,max}

\DeclareMathOperator*{\gain}{Gain}

% Skeleton operators.
\DeclareMathOperator*{\map}{Map}
\DeclareMathOperator*{\reduce}{Reduce}
\DeclareMathOperator*{\scan}{Scan}
\DeclareMathOperator*{\stencil}{Stencil}
\DeclareMathOperator*{\zip}{Zip}
\DeclareMathOperator*{\allpairs}{All\,Pairs}

% Maths plots using pgfplots, see:
%
%     http://pgfplots.sourceforge.net/pgfplots.pdf
%
\usepackage{pgfplots}

% Disable compatability mode.
%
\pgfplotsset{compat=1.12}

% Gantt charts using pgfgantt, see:
%
%     http://www.ctan.org/pkg/pgfgantt
%
\usepackage{pgfgantt}

% Fix milestone aspect ratio by defining a custom element.
\newganttchartelement*{mymilestone}{
mymilestone/.style={
shape=diamond,
inner sep=2pt,
draw=black,
top color=black,
bottom color=black,
}
}

% Tikz flowchart configuration.
\usetikzlibrary{shapes,arrows,shadows,fit,backgrounds}
\tikzstyle{decision} = [diamond,
draw,
text width=4.5em,
text badly centered,
node distance=3cm,
inner sep=0pt]
\tikzstyle{block}    = [rectangle,
draw,
text width=5em,
text centered,
node distance=3cm,
minimum height=4em,
inner sep=.2cm]
\tikzstyle{line}     = [draw, -latex']

% Add dirtree picture style, see:
%
%     http://tex.stackexchange.com/a/34268
%
\newcount\dirtree@lvl
\newcount\dirtree@plvl
\newcount\dirtree@clvl
\def\dirtree@growth{%
\ifnum\tikznumberofcurrentchild=1\relax
\global\advance\dirtree@plvl by 1
\expandafter\xdef\csname dirtree@p@\the\dirtree@plvl\endcsname{\the\dirtree@lvl}
\fi
\global\advance\dirtree@lvl by 1\relax
\dirtree@clvl=\dirtree@lvl
\advance\dirtree@clvl by -\csname dirtree@p@\the\dirtree@plvl\endcsname
\pgf@xa=0.33cm\relax
\pgf@ya=-\baselineskip\relax
\pgf@ya=\dirtree@clvl\pgf@ya
\pgftransformshift{\pgfqpoint{\the\pgf@xa}{\the\pgf@ya}}%
\ifnum\tikznumberofcurrentchild=\tikznumberofchildren
\global\advance\dirtree@plvl by -1
\fi
}
\tikzset{
dirtree/.style={
growth function=\dirtree@growth,
every node/.style={anchor=north},
every child node/.style={anchor=west},
edge from parent path={(\tikzparentnode\tikzparentanchor) |- (\tikzchildnode\tikzchildanchor)}
}
}

% UML sequence diagram macros, see:
%
%     https://code.google.com/p/pgf-umlsd/
%
% Options:
%
%     underline - Underline object names
%
\usepackage[underline=false]{pgf-umlsd}

% Support for SVG graphics.
%
% NOTE that you must pass the "--shell-escape" argument to pdflatex to
% compile. NOTE also that images *MUST* be placed within the graphics
% path.
\usepackage{svg}
\graphicspath{{img/}}

%%%%%%%%%%%%%%%%%%%%%%
%% Tables and lists %%
%%%%%%%%%%%%%%%%%%%%%%

% Required to use labm8 exported tables.
%
\usepackage{booktabs}

% Required for full page-width tables.
\usepackage{tabularx}

%\usepackage{enumitem}
%\setenumerate{itemsep=0pt}

% Use no left margin for lists:
%\setlist{leftmargin=*}

\usepackage{longtable}

% Define column types L, C, R with known text justification and fixed
% widths:
\usepackage{array}
\newcolumntype{L}[1]{>{\raggedright\let\newline\\\arraybackslash\hspace{0pt}}m{#1}}
\newcolumntype{C}[1]{>{\centering\let\newline\\\arraybackslash\hspace{0pt}}m{#1}}
\newcolumntype{R}[1]{>{\raggedleft\let\newline\\\arraybackslash\hspace{0pt}}m{#1}}


%%%%%%%%%%%%%%%%%%%%%%%%%%%%%
%% Typesetting and symbols %%
%%%%%%%%%%%%%%%%%%%%%%%%%%%%%

% Adjustable font sizes in \Verbatim{}
\usepackage{fancyvrb}

%\usepackage{titlesec}
% Set section and paragraph heading fonts:
%\titleformat*{\section}{\Large\bfseries}
%\titleformat*{\subsection}{\normalsize\bfseries}
%\titleformat*{\subsubsection}{\normalsize}
%\titleformat*{\paragraph}{\large\bfseries}
%\titleformat*{\subparagraph}{\large\bfseries}

% Set section heading margins. Usage:
% \titlespacing*{<command>}{<left>}{<before>}{<after>}
%\titlespacing*{\section}{0pt}{.6em}{.3em}
%\titlespacing*{\subsection}{0pt}{.6em}{.2em}

% Set paragraph indentation size. Default is 15pt.
%\setlength{\parindent}{10pt}

% The line spacing can be globally set using \linespread:
%
% \linespread{1.2}

% Add a command \hr{} which will draw a horizontal rule the width of
% the text.
%
\newcommand{\hr}{\noindent\makebox[\linewidth]{\rule{\textwidth}{0.2pt}}}

% Add a command \br{} which will create a horizontal space of exactly
% one line height.
%
\newcommand{\br}{\hspace{\baselineskip}}

% Define a command to allow word breaking.
\newcommand*\wrapletters[1]{\wr@pletters#1\@nil}
\def\wr@pletters#1#2\@nil{#1\allowbreak\if&#2&\else\wr@pletters#2\@nil\fi}

% Define a command to create centred page titles.
\newcommand{\centredtitle}[1]{
\begin{center}
  \large
  \vspace{0.9cm}
  \textbf{#1}
\end{center}}

% Support hyperlinks using the \hyperref, \url and \href
% macros. Usage:
%
%    \hyperref[label_name]{''link text''}
%
%    \url{<my_url>}
%
%    \href{<my_url>}{<description>}
%
\usepackage{hyperref}

% Disable colored borders of links, cross-references etc in PDF output
\hypersetup{pdfborder={0 0 0}}

% Provide generic commands \degree, \celsius, \perthousand, \micro
% and \ohm which work both in text and maths mode.
\usepackage{gensymb}

%%%%%%%%%%%%%%%%%%%%%%%%%%%%%%%%%
%% Placeholder text generation %%
%%%%%%%%%%%%%%%%%%%%%%%%%%%%%%%%%

% Use either \blindtext or \libpsum to generate placeholder text. Also
% note the macros \blinditemize, \blindenumerate, \blinddescription.
\usepackage[english]{babel}
\usepackage{blindtext}
\usepackage{lipsum}


\begin{document}

\title{Plausible Test Case Generation for Compiler and Runtime Bug Discovery}

%
% any author declaration will be ignored  when using 'pldi' option (for double blind review)
%
%\authorinfo{Person 1 \and Person 2}
%{\makebox{A Department} \\
%        \makebox{A University}  \\
%        \makebox{A Place, AS 12345}}
%{\{person1,person2\}@cs.auniv.edu}

\maketitle

\diff{Constructing an optimising compiler is an enormous undertaking. Modern compilers are multi-million dollar projects taking many years of development.
%
There are more devices + heterogeneity. Each device requires a new compilers.
%
% LLVM 8.0.1 SLOCount: 
% 1206 unique developers have contributed 307,817 commits, with
% 35,497,808 line additions	and 20,909,608 line deletions. There
% are 5,356,816 lines of code which 103035

% sloccount . --personcost 103035
% average US software developer salary from glassdor
%  https://www.glassdoor.co.uk/Salaries/us-software-engineer-salary-SRCH_IL.0,2_IN1_KO3,20.htm?countryRedirect=true
%
Demand outstrips supply.
%
What is need is better tools to lower the cost of compiler construction.}

% Compilers are a fundamental technology. Their role in translating software to machine code must be performed without error, while maximising the performance and efficiency of the generated code. The precedent for more rigorous validation and improved performance is well established, yet progress is challenging. Compilers comprise thousands of interacting components which must be expertly engineered and tuned, and much of the work of compiler construction has eluded automation. \diff{Furthermore, the rapid transition to heterogeneous parallelism has driven development of broad new range of accelerators which require aggressively-optimising compilers to obtain good performance. For the trend towards heterogeneity to continue, compiler construction must be made cheap.}

% The cost of these shortcomings is wasted energy, poor performance, and buggy software. What is needed is ways to lower the cost of constructing compilers.

This thesis presents new techniques that dramatically lower the cost of compiler construction, while improving robustness and performance. The enabling insight for this research is the leveraging of \emph{recurrent neural networks} to \diff{model the correlations between source code and program behaviour}, enabling tasks which previously required enormous engineering effort to be automated. This is demonstrated in three domains:

% This thesis presents three techniques to simplify and accelerate compiler construction.
% First, a tool for automatic performance characterisation through benchmark generation; second, a low-cost and effective fuzzer for validating correctness; third, a simple technique to address the labour intensive process of optimisation heuristic construction.

First, a generative model for compiler benchmarks is developed. This model is inferred automatically from corpora of readily available open source programs, requiring no grammar or prior knowledge of the programming language. This greatly reduces the cost of development compared to prior approaches, yet the generator produces output of such quality that professional software developers cannot distinguish generated from handwritten code. The efficacy of the generator is demonstrated by supplementing the training data of state-of-the-art predictive models for compiler optimisations. The additional fine-grained exploration of the feature space yields both an automatic improvement in heuristic performance and exposes weaknesses in the prior art which, when corrected, yields further improvements in performance.

Second, this thesis presents techniques that extend the prior approach to the domain of compiler validation. A compiler fuzzer is developed which is far simpler than the state-of-the-art, yet is effective. By learning a generative model rather than engineering a generator from scratch using a grammar, it is implemented in $100\times$ fewer lines of code than the state-of-the-art, and is capable of generating an expressive range of tests that expose bugs that prior techniques cannot. An extensive testing campaign of OpenCL compilers reveals 67 new bugs, many of which have now been fixed.

Finally, this thesis addresses the challenges of machine learning for compiler optimisations, developing methodologies for learning compiler heuristics without the need for code features. Contrasting prior approaches that require features to be expertly engineered and selected, the proposed approach learns directly over the raw textual representation of program code. Doing so outperforms state-of-the-art heuristics in two challenging optimisation domains. Additionally, the methodology permits the novel transfer of information between optimisation problems, enabling a model trained for one task to be adapted to perform another, further improving performance.

\diff{TODO: Conclusions}

\section{Introduction}\label{sec:intro}

\cec{(ideally)} We make the following contributions:
%
\begin{itemize}
\item we present a generative model for programming languages which [is in some way more capable than prior]. Our method achieves X-fold increase in production efficiency compared to syntactic generative models;
\item we generate more human-like test cases for compilers than prior, in (some/most) cases eradicating the need for test case reduction;
\item we identify X bugs in commercial OpenCL drivers and runtimes from Y vendors. Our approach is automatic.
\end{itemize}

\section{Overview of Our Approach and Results}\label{sec:overview}

\paragraph{Test case generation} We use CLgen, our deep learning program synthesis tool, to generate thousands of OpenCL samples.

\paragraph{OpenCL configurations tested} We conducted testing of X different OpenCL configurations, summarized in Table~\ref{tab:platforms}. A \emph{configuration} refers to an OpenCL \emph{<device, driver>} pair.

\begin{table*}[t!]
  \scriptsize %
  \centering %
  % \rowcolors{2}{white}{gray!25}
  \begin{tabular}{ cllllll | rr }
\toprule
\textbf{\#. } & \textbf{Platform} & \textbf{Device} & \textbf{Driver} & \textbf{OpenCL} & 
\textbf{Operating system} & \textbf{Device Type} & \textbf{Testing time} & \textbf{Bugs Submitted} \\
\midrule
1 & NVIDIA CUDA & GeForce GTX 1080 & 375.39 & 1.2 & Ubuntu 16.04 64bit & GPU & 150 hours & 7 \\
2 & NVIDIA CUDA & GeForce GTX 780 & 361.42 & 1.2 & openSUSE  13.1 64bit & GPU & 109 hours & 0 \\
3 & Intel Gen OCL Driver & Intel HD Haswell GT2 & 1.3 & 1.2 & Ubuntu 16.04 64bit & GPU & 100 hours & 11 \\
4 & Intel OpenCL & Intel E5-2620 v4 & 1.2.0.25 & 2.0 & Ubuntu 16.04 64bit & CPU & 129 hours & 5 \\
5 & Intel OpenCL & Intel E5-2650 v2 & 1.2.0.44 & 1.2 & CentOS 7.1 64bit & CPU & 128 hours & 1 \\
6 & Intel OpenCL & Intel i5-4570 & 1.2.0.25 & 1.2 & Ubuntu 16.04 64bit & CPU & 126 hours & 4 \\
7 & Intel OpenCL & Intel Xeon Phi & 1.2 & 1.2 & CentOS 7.1 64bit & Accelerator & 129 hours & 1 \\
8 & POCL & POCL (Intel E5-2620) & 0.14 & 2.0 & Ubuntu 16.04 64bit & CPU & 121 hours & 22 \\
9 & ComputeAorta & ComputeAorta (Intel E5-2620) & 1.14 & 1.2 & Ubuntu 16.04 64bit & CPU & 122 hours & 0 \\
10 & Oclgrind & Oclgrind Simulator & 16.10 & 1.2 & Ubuntu 16.04 64bit & Emulator & 117 hours & 5 \\

\bottomrule
\end{tabular}


  \caption{Experimental platforms we tested.}
  \label{tab:platforms}
\end{table*}


\paragraph{Testing results} All configurations yielded bugs. We found and reported X bugs. In comparing our approach against the state-of-the-art in OpenCL compiler test case generation, we found our approach was able to \cc{\ldots}

\section{Test Case Generation using Deep Learning}

\subsection{CLgen}

Figure~\ref{fig:deeptune} shows the test case generation process.

\begin{itemize}
\item An initial seed corpus is collated from GitHub (w. rejection filter).
\item Corpus is encoded for training.
\item LSTM networks model the vocabulary distribution over the encoded corpus.
\item The trained network is sampled to generate new programs.
\end{itemize}

\noindent In subsequent iterations, synthesized kernels may be used to augment the training corpus.

\begin{itemize}
\item Samples are synthesized by the network and pruned using the rejection filter.
\item The samples are mixed with kernels from the GitHub seed corpus.
\item These combined samples are labeled (synthetic or human), and used to train a classifier for human-or-robot.
\item The classifier is tested on unseen programs, and the accuracy is used as a fitness function for the model's parameters.
\item A GA mutates and evolves the best performing parameters.
\end{itemize}


\begin{figure}
  \centering
  \includegraphics[width=.95\columnwidth]{img/clgen} %
  % \vspace{-2em}%
  \caption{%
    Test-case generation process. A corpus of programs from GitHub is used to seed a generative model for program codes.%
  }%
  \label{fig:deeptune}
\end{figure}

\subsection{cldrive: OpenCL Test Case Execution}

Generate host code stub to compile kernel. If kernel compiles: parse AST, read function arguments, generate data inputs for arguments. Limited subset of data types supported: No structs, no irregular data types (this is just an implementation detail, and can be addressed with more dev time).

\section{Testing Methodologies}\label{sec:methodology}

We describe two types of OpenCL testing using our approach. The first is a direct comparison of our approach to CLSmith, the current state-of-the-art in OpenCL test case generation. The second is results from unstructured, opportunistic testing of OpenCL implementations using CLgen, resulting in XX bug reports.


\subsection{Classifying test results}

We select the mode output across a range of devices. Voting is used to select the oracle output.
%
\begin{enumerate}
	\setcounter{enumi}{-1}
	\item \emph{Invalid test case} Program code and parameter combination is illegal (this is device specific), or some other failure (e.g. program timeout). We simply ignore this class of results.
	\item \emph{Build failure} Online compilation of OpenCL program fails.
	\item \emph{Runtime failure} One or more OpenCL API calls return an error status during the program execution, or the program crashes due to segmentation fault. Note that segmentation faults may be caused by compiler, in which case these outcomes should perhaps be classified as build failures. We do not account for this.
	\item \emph{Wrong code} Program terminates gracefully, but computes a result which differs from the majority.
	\item \emph{Okay} Program terminates gracefully and computes a result which agrees with the majority.
\end{enumerate}


\subsection{Parameters}
\begin{table}[t!]
  \scriptsize %
  \centering %
  % \rowcolors{2}{white}{gray!25}
  \begin{tabular}{llllr}
\toprule
{} &   Global size & Workgroup size & OpenCL Optimizations &  Dataset Size* \\
\midrule
1 &     (1, 1, 1) &      (1, 1, 1) &                  off &            256 \\
2 &     (1, 1, 1) &      (1, 1, 1) &                   on &            256 \\
3 &  (128, 16, 1) &     (32, 1, 1) &                  off &           4096 \\
4 &  (128, 16, 1) &     (32, 1, 1) &                   on &           4096 \\
\bottomrule
\end{tabular}

  \caption{Test case parameters. For the CLSmith driver, dataset size is not required.}
  \label{tab:cldrive-params}
\end{table}

Table~\ref{tab:cldrive-params}. 

\section{Related Work}\label{sec:rw}

\paragraph{Test Case Generation}

The need to test compilers under invalid and unexpected input conditions has been long known~\cite{Boujarwah1997}.

Differential testing using test cases of ranked ``quality'' is presented in~\cite{McKeeman1998}. A lexeme generator and to test C compilers by enumerating token sequences. Similar to DeepSmith, except we weight the generation so that sequences are plausible.

Increasing complexity of generations: RandProg to CSmith. Then EMI testing, and SPE.

The \emph{fuzz taming problem} is addressed in~\cite{Chen2013}, in which a distance metric is used to rank test cases such that diverse, interesting test cases are highly ranked.

Empirical comparion of compiler testing techniques~\cite{Chen2014a}

Analysis of bugs in GCC and LLVM finds 80\% of test-cases to be 45 lines~\cite{Sun2016}.

\cc{TODO}~\cite{Godefroid2008a,Le2015,Sun2016a}~\cite{Kossatchev2005}.

\cc{TODO:} Directed EMI testing~\cite{Le2015}.

``As a matter of implementation quality, a compiler vendor will usually fix a segmentation fault or similar problem even if the crash-inducing test-case, for example, uses a variable without initialization.''~\cite{Regehr2012a}

Skeletal program enumeration~\cite{Zhang2016a}. By enumerating entire program space, provides bounded guarantees of compiler correctness. Same shortcoming as CSmith (well formed programs only). No OpenCL implementation. Probably would take a lot of development effort. \cc{More investigation required.}

CSmith~\cite{Yang2011} and CLSmith~\cite{Lidbury2015a}. Same testing methodology, but very different design goals to our work. The explore the space of \emph{unlikely} programs, by pairing infrequently combined language features. Because Csmith programs are free form undefined behavior, there is only a single interpretation. This allows oracle-less differential testing across compilers, using a voting heuristic to identify erroneous compiler outputs. The ``shape'' of programs generated by Csmith is expert driven. The shape of programs generated by DeepSmith is data driven. The 80 probabilities which control Csmith program generation are extensively hand tuned to produce programs which ``look right''. Our data driven approach does not require this. In fact, our approach is portable across changing usage of a programming language.

The functionality of Csmith is also expert driven. Every language feature supported by Csmith must be laboriously hand crafted, and results in a very complex system of over 40,000 lines of C++ code (which still omits many language features used in real programs, like heap allocation). The language features supported by DeepSmith is bound only by those which have been used on GitHub. DeepSmith has a non-zero probability of generating programs using every language feature found on GitHub. We reason that if a language feature remains unused in the entirety of code on GitHub, then it is reasonable to assume that it is not a feature worth testing. Both Csmith and CLsmith use unrealistic safe-math macros to wrap arithmetic operations. We do not.

Both tools require test-case reduction (\cite{Regehr2012a} and~\cite{Pflanzer2016}, respectively). Work in test-case reduction~\cite{Regehr2012a} reduced Csmith program sizes by 74--594$\times$ while preserving the bug exposing behavior. They median reduced CSmith program size they found was only 20 lines (0.5KB).

\cc{TODO:}

EMI testing~\cite{Le2013a}, Skeletal Program enumeration~\cite{Zhang2017a}.

A mutation-based approach for the Java Virtual Machine is demonstrated in~\cite{Chena}.

TODO~\cite{White2016}.

\paragraph{GPU Testing} GPU Concurrency --- Small \emph{litmus tests}~\cite{Alglave2015}.


\paragraph{Machine Learning} Deep Learning is a nascent field that is responsible for a multitude of breakthroughs in modeling rich, hierarchical datasets. The major milestones are reviewed in~\cite{Wang2017}, and methods in~\cite{Schmidhuber2014}.

MACHINE LEARNING FOR TESTING: Machine learning selectively unsound static analysis~\cite{Heo2017}, Learning a classifier for static analyzers~\cite{Koc2017}.
Transforming program repair ingredients with DL~\cite{White}. Program repair~\cite{Koukoutos2017a}. Learning to prioritize test programs~\cite{Chen2017}. Attention network to identify buffer overruns~\cite{Choi2016}. Localizing bugs from bug reports~\cite{Lam2016,Huo2016}.

There is an increasing interest in mining source code repositories at large scale~\cite{Allamanis2013a,White2015a,Bird2009}. Previous studies have involved data mining of GitHub to analyze software engineering practices~\cite{Wu2014,Guzman2014,Baishakhi2014a,Vasilescu2015}, for example code generation~\cite{Zhang2015a}, code summarization~\cite{Allamanis2016}, comment generation~\cite{Wong2013}, and code completion~\cite{Raychev2014}. Previous applications of deep learning to compilers have involved program synthesis for performance benchmarking~\cite{Cummins2017a} and building optimization heuristics~\cite{Cummins2017b}. No work so far has exploited mined source code for test-case generation. This work is the first to do so.

\section{Conclusions}\label{sec:conclusion}

We extend the state-of-the-art in a number of areas. First, we present a generative model for programming languages which [is in some way more capable than prior];
%
our method achieves X-fold increase in production efficiency compared to syntactic generative models (if grammar based approach);
%
and we generate more human-like test-cases for compilers than prior, in (some/most) cases eradicating the need for test-case reduction.


% \section*{Acknowledgments}

This work was supported by the UK Engineering and Physical Sciences Research
Council under grants EP/L01503X/1 (CDT in Pervasive Parallelism), EP/M01567X/1
(SANDeRs), EP/M015793/1 (DIVIDEND), and EP/P003915/1 (SUMMER). The code and data
for this paper are available at: \url{https://chriscummins.cc/pact17}.
  % double blind review


\printbibliography

\end{document}